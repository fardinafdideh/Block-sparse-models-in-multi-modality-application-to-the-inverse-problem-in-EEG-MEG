\begin{remark}[Block-incoherency]
\label{Rmrk:Basic BMIC} 
It can be seen that for special settings of $d_1 \seq \cdots \seq d_K \seq 1$ and equally-sized blocks, i.e., $d_1 \seq \cdots \seq d_K \seq d$, Property \ref{prp:BMIC-LB} is equivalent to the conventional bound, i.e., $\overbar{M}(\myPhiOne,\myPhiTwo) \sgeq 1/\sqrt{m}$, and the bound of Eldar et al., i.e., $\overbar{M}^{Eldar}_{Inter}(\myPhiOne,\myPhiTwo) \sgeq 1/\sqrt{d \, m}$, respectively.
Generally, for settings of $d_{min} \, d_{max}^{-3/2} \sleq 1$, the lower-bound in Property \ref{prp:BMIC-LB} is less than or equal to the lower-bound in conventional case, which means that in the proposed block-structured scenario, dictionaries can be more \emph{block-incoherent} compared to the conventional case.
Since there is a direct relationship between the sparsity level of recovery condition and block-incoherency, hence improved recovery conditions are obtained in block-structured scenario.
In addition, notice that in the proof of all above basic two orthonormal bases we used the relationships of a dictionary with intra-block orthonormality, because for an orthonormal base $\myPhi \ssin \mathbb{R}^{m \stimes m}$, with $\myPhi^T \myPhi \seq \boldsymbol{I}_{m}$, we have $\myPhi^T [k] \myPhi[k] \seq \boldsymbol{I}_{d_k}$.
\end{remark}