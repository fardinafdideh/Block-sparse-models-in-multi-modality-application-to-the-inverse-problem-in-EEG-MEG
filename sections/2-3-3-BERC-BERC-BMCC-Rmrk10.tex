\begin{remark}
\label{rmrk:BasicBMCC-BasicMCC} 
\myhl{As it is mentioned before, in order to compare the proposed basic block-sparse uncertainty principle in Lemma {\ref{lm:BBUP}} (Basic Block-UP, page {\pageref{lm:BBUP}}), i.e., $(\Vert \mybetao \Vert_{r,0} \spl \Vert \mybetaTwo \Vert_{r,0}) \sgeq 2/(d_{min}^{-1/2} d_{max}^{3/2} \overbar{M}_{q,p/(p-1)} (\myPhiOne,\myPhiTwo))$, with the conventional basic uncertainty principle explained in ({\ref{eq:UP-basic}}) on page {\pageref{eq:UP-basic}}, i.e., $(\Vert \mybetao \Vert_0 \spl \Vert \mybetaTwo \Vert_0) \sgeq 2/\overbar{M} (\myPhiOne,\myPhiTwo)$, the relationship between $d_{min}^{-1/2} d_{max}^{3/2} \overbar{M}_{q,p/(p-1)} (\myPhiOne , \myPhiTwo)$ and $d_{max} \overbar{M}(\myPhiOne,\myPhiTwo)$ should be investigated.
The mentioned comparison is shown in table {\ref{table:BasicBMCC-BasicMCC}}.
Since the two terms are in the denominator of the right-hand side of the uncertainty principle, the smaller values lead to more weakened uncertainty principles, hence, improve the conditions.
Therefore, the values in table {\ref{table:BasicBMCC-BasicMCC}}, which are less than or equal to one, correspond to weakened uncertainty principles due to using block-sparsity.
From table {\ref{table:BasicBMCC-BasicMCC}}, it can be seen that for $\forall (q , p) \ssin \mathbb{R}^2_{\sgeq 2}$ and $d_{min} \seq d_{max}$, the proposed basic block-sparse uncertainty principle in Lemma {\ref{lm:BBUP}} (Basic Block-UP) improves the conventional basic uncertainty principle explained in ({\ref{eq:UP-basic}}).}
\end{remark}