\chapter*{Contributions and perspectives}
\addstarredchapter{Contributions and perspectives}
\markboth{Contributions and perspectives}{Contributions and perspectives}
\label{sec:conclusion}
\minitoc

% ------------------------------------------------------------------------
%\section*{Introduction}
%\addcontentsline{toc}{section}{\protect\numberline{}Introduction}
The contributions of this thesis can be categorized into three main research domains. 
The three classes of research contributions address the three challenges that have been mentioned and discussed in Section Introduction of the dissertation.

In order to explore and address the three mentioned challenges in Introduction, this dissertation is organized in three main parts.
Since the first challenge uses the results of the second challenge, the second challenge is presented in the dissertation before the first challenge, and ultimately in the last part a partial answer to the third challenge is provided through experiments.

First challenge is about the ineffectiveness of some classic methods in high-dimensional problems.
Second challenge raises when classic recovery conditions cannot be established for a new concept of constraint of corresponding optimisation problem, because of materials shortage in new framework. 
Finally, the combination of different information of a same phenomenon is the subject of the third challenge.

In this section, first we briefly recall the challenges and then mention the corresponding contributions to deal with them.

As mentioned in Section Introduction, all the three independent research orientations meet each other in our real-world application, which is distributed EEG/MEG source reconstruction problem, to partially respond to the main research question of the thesis, which is:
\begin{tcolorbox}
%How the diversity of multiple modalities can be effectively exploited in inverse problems with sparsity constraint?
What is the added value of multi-modality, when solving inverse problems?
\end{tcolorbox}
In the following the contributions and perspectives are organized into three sections:
\newpage
% ------------------------------------------------------------------------
\section*{\myhl{High-dimensional problem}}
\addcontentsline{toc}{section}{\protect\numberline{}High-dimensional problem}
\setcounter{challenge}{0}
%As mentioned earlier in Section Introduction, the first challenge to be addressed is:
\begin{tcolorbox}
\begin{challenge}
Many real-world inverse problems are vastly underdetermined, and classical sparse estimation techniques do no longer give acceptable recovery conditions.
How can high-dimensional problems be adapted in favour of the coherence-based notion of conventional conditions?
\end{challenge}
\end{tcolorbox}

%The first challenge as mentioned in Section Introduction is
For vastly underdetermined systems of linear equations, which the columns of the coefficient matrix are great in number, the classical dictionary characterisations such as mutual coherence constant are more likely to be high.
Because the classical dictionary characterisations measure the coherency of the columns of the dictionary.
Naturally, increasing in the number of columns of the dictionary would be equivalent to increasing the probability of coherency between the columns of the dictionary.
On the other hands, due to the inverse relationship between the sparsity level defined in the exact recovery conditions and the classical dictionary characterisation, high coherency leads to low sparsity level.
Therefore, being vastly underdetermined has a negative effect on the exact recovery conditions.

Assuming that the atomic entity in classical dictionary characterisations is \emph{columns} of the dictionary, one possible solution would be to decrease the number of atomic entities by changing the concept of atomic entity from columns to \emph{block of columns} of the dictionary.
Therefore, in Chapter \ref{sec:Clustering} we proposed to cluster the coherent entities (columns or block of columns) of the dictionary.
By the idea of clustering the coherent entities to build new entities, we are improving the exact recovery conditions by gaining two advantages at the same time: (1) reducing the number of entities, (2) building more incoherent entities.

In the EEG/MEG source reconstruction problem, the initial entity is already a fixed-length block of columns (three columns), then we clustered the coherent blocks of columns.
In order to determine the coherency between block of columns we utilised the dictionary characterisation block mutual coherence constant proposed in Chapter \ref{sec:BERC}.
%In Chapter \ref{sec:Clustering}, with the aim of improving the block-sparse exact recovery conditions based on block mutual coherence constant proposed in Chapter \ref{sec:BERC} and consequently the conventional recovery condition, we proposed to cluster the coherent blocks determined by the value of block mutual coherence constant. 

Then, for EEG/MEG source reconstruction problem we profited the idea of clustering the coherent blocks of columns, i.e., brain source lead-fields, to segment the brain source space.
The proposed strategy for brain source space segmentation is in a general case, without being restricted to know any information about the sources activity and sensors measurement, only utilises EEG/MEG lead-field matrix.
%in the domain of brain source space segmentation to without being restricted to have any information about the sources activities and sensors measurements being able to cluster the brain sources utilizing the EEG or MEG leadfield matrix. 
In addition, at the same time by clustering the coherent sources and forming the brain regions, the number of brain regions in which it is ensured to have a unique solution in the EEG or MEG source reconstruction problem is improved.

As perspective in this domain, different standard or customized clustering strategies can be investigated to find an optimum clustering strategy, which leads to improvement in the exact recovery conditions.
One possible strategy in hierarchical clustering algorithm would be to update the similarity matrix at each clustering level, i.e., recompute the new distance of clusters and update the similarity matrix based on new clustering structure.
%there would be the place to more improve the recovery conditions by applying different clustering strategies.
% ------------------------------------------------------------------------
\section*{\myhl{Block-structured problem}}
\addcontentsline{toc}{section}{\protect\numberline{}Block-structured problem}
\begin{tcolorbox}
\begin{challenge}
In classical assumption, the recovery conditions are generated from the columns of coefficient matrix, but clustered coefficient matrix consists of some differently-sized blocks, and not necessarily columns.
Therefore, the initial assumption of classical recovery conditions does no longer hold true. 
How can appropriate recovery conditions be developed for block-structured problems? 
\end{challenge}
\end{tcolorbox}

In Chapter \ref{sec:BERC}, we defined a general framework to cover all types of atomic entity including columns, and equally/differently-sized blocks of columns, in order to propose general theoretical exact recovery conditions.
The proposed theoretical recovery conditions are based on \emph{block-sparsity} constraint, which ensure the uniqueness of the block-sparse solution of corresponding weighted (pseudo-)mixed-norm optimisation problem in an underdetermined system of linear equations.
%In Chapter \ref{sec:BERC}, the sufficient conditions for unique recovery of block-sparse recovery of an arbitrary signal $\boldsymbol{y}$ in a general arbitrary dictionary $\myPhi$ using a general weighted mixed norm $P_{\overbar{p_1},p_2}$ (and also $P_{p_1,p_2}$ in equally-sized blocks) optimisation problems, are proposed. 
The mentioned generality of the framework is in terms of the properties of the underdetermined system of linear equations, extracted characterisations, optimisation problems, and ultimately the recovery conditions.
%In order to propose the recovery conditions, the characterisations and properties were defined in their general case.
The mentioned theoretical exact recovery conditions are categorized in four different groups based on the utilised characterisations and properties, i.e., conditions based on (1) Block-Spark, (2) block null space property, (3) block mutual coherence constant, and (4) cumulative coherence constant.

On the other hand, the proposed framework is consistent with the base findings, since all the materials in the proposed infrastructure are a generalisation of the existing references 
Indeed, we investigated the theoretical relationship between the proposed infrastructure and the classical one, and showed that all the new materials reduce to the conventional ones in specific cases.
%On the other hand, since all the proposed materials are a generalisation of the existing references, they all reduce to the conventional ones in specific cases, i.e. the proposed framework is consistent with the base findings.
We redemonstrated the benefit of block-sparsity assumption compared to conventional sparsity in the improvement of the recovery conditions.
In addition, we theoretically proved the supremacy of the theoretical exact recovery conditions defined in the proposed general infrastructure over existing conditions, which are assuming the same block-sparsity constraint.

%over existing conditions in block-sparsity made by the proposed conditions.
As perspective in this research direction, we could mention to the following subjects:
%the generalisation of the proposed exact recovery conditions to a more realistic stable recovery conditions.
%In addition, in the conditions group of cumulative coherence constant, we are studying on introducing another block-sparse exact recovery condition based on cumulative Block-MCC$_{q,p}$, defined in Definition \ref{def:CBMIC}.
%Future research will focus on:
\begin{itemize}
\item Introducing block-sparse recovery conditions based on the proposed cumulative Block-MCC$_{q,p}$ defined in Definition \ref{def:CBMIC}.
\item Generalising the conventional dictionary characterisation of $\mu$ defined in \cite{Donoho2003}, to establish block-sparse recovery conditions.
\item  Transforming all the previously mentioned block-sparse \emph{exact} recovery conditions to block-sparse \emph{stable} recovery conditions.
In stable or robust recovery conditions we have, $\Vert \boldsymbol{y} \sm \myPhi \hat{\mybeta} \Vert_{2} \sless e$, where, $e$ is a bounded noise.
\item Study on block-sparse optimisation algorithms, and the relationship between the \emph{theoretical} and \emph{algorithmic} block-sparse recovery conditions.
\end{itemize}
% ------------------------------------------------------------------------
\section*{\myhl{Multi-modality}}
\addcontentsline{toc}{section}{\protect\numberline{}Multi-modality}
\begin{tcolorbox}
\begin{challenge}
Is joining multiple modalities always beneficial, knowing that each modality provides us with different properties of the same phenomenon? 
How can the added value of multi-modality be demonstrated?
\end{challenge}
\end{tcolorbox}

In order to approach the above-mentioned important and huge challenge, we simplify and limit the challenge to the conditions of our main problem, i.e., distributed EEG and MEG source reconstruction problem.
In other words, we restrict the mentioned general challenge to have only two modalities, and try to partially address the challenge.

In Chapter \ref{chptr:Multimodality}, we propose a multi-modality framework based on the block structure identification framework proposed in Chapter \ref{sec:Clustering} and block mutual coherence constant proposed in Chapter \ref{sec:BERC}.

To this aim, we applied the block structure identification framework on multi-modal lead-field instead of mono-modal one, to segment brain source space.
To investigate the impact of multi-modality, we defined a lead-field combining strategy, which reduces the impact of other factors such as change in the position and number of sensors.

First, we showed that brain regions resulted from clustering the coherent sources of EEG and MEG lead-field matrices separately, are complementary.
Then, we made use of complementarities of EEG and MEG lead-field matrices to generated a combined EEG and MEG multi-modal lead-field matrix, and it turned out that in multi-modality case the number of clusters determined by the largest distance between adjacent nodes in dendrogram is higher than mono-modal cases. 
In addition, for a fixed number of clusters, the under-sensor brain regions in multi-modal lead-field is smaller than mono-modal lead-field clustering.

%Furthermore, for a fixed number of clusters, the brain regions in multi-modal case are smaller than mono-modal cases.
Therefore, it can be deduced that in multi-modality case, more refined and precise regions appear, hence, the resolution of identifying the active regions increases in comparison to the mono-modal cases.

As perspective in this research area, we could mention to the following points:
\begin{itemize}
\item Study on electromagnetic properties in 2D/3D mediums with different number of boundaries as a general case.
\item Designing more scenarios (other than brain segmentation) to investigate the impact of multi-modality.
\item Designing more methods to combine modalities.
\item The impact of combining more modalities (greater than two).
\item The optimum ratio of the number of measurements from each modality (not necessarily $50\%$ in two modalities case).
\end{itemize}    

%%% Local Variables: 
%%% mode: latex
%%% TeX-master: "../roque-phdthesis"
%%% End: