\begin{property}[$\ell_{q {\to} p}$ operator-norm properties]
\label{prp:OperatorProperties}
Assuming \myhl{$\boldsymbol{A} \ssin \mathbb{R}^{m \stimes n}$, $\boldsymbol{B} \ssin \mathbb{R}^{m \stimes n}$, and $\boldsymbol{C} \ssin \mathbb{R}^{n \stimes l}$,} the $\ell_{q {\to} p}$ operator-norm of a matrix satisfies the following properties:
%$\ell_{q {\to} p}$ operator-norm of a matrix $\boldsymbol{A}$ is a matrix norm as it satisfies the following properties:
\begin{itemize}
\item Nonnegativity: $\forall (q , p) \ssin \mathbb{R}^2_{\sgeq 0} : \Vert \boldsymbol{A} \Vert_{q {\to} p} \sgeq 0$.
\item Positivity: $\forall (q , p) \ssin \mathbb{R}^2_{\sgeq 0} : \Vert \boldsymbol{A} \Vert_{q {\to} p} \seq 0$ if and only if $\boldsymbol{A} \seq \boldsymbol{0}$.
\item Homogeneity: $\forall q \ssin \mathbb{R}_{\sgeq 0} , \forall p \ssin \mathbb{R}_{\sg 0}$, \myhl{$\forall \alpha \ssin \mathbb{R}$} $: \Vert \alpha \boldsymbol{A} \Vert_{q {\to} p} \seq \myabs{\alpha} \Vert \boldsymbol{A} \Vert_{q {\to} p}$.
\item Triangle inequality: $\forall q \ssin \mathbb{R}_{\sgeq 0} , \forall p \ssin \mathbb{R}_{\sgeq 1}$, for $p \seq 0 : \Vert \boldsymbol{A} \spl \boldsymbol{B} \Vert_{q {\to} p} \sleq \Vert \boldsymbol{A} \Vert_{q {\to} p} \spl \Vert \boldsymbol{B} \Vert_{q {\to} p}$.
\end{itemize}
\begin{remark}[Generalised matrix norm]
\label{rmrk:Generalized matrix norm} 
Any matrix norm definition that satisfies the above four properties is called \emph{generalised matrix norm} \cite{HornR.A.2012}.
All the above-mentioned properties hold true $\forall q \ssin \mathbb{R}_{\sgeq 0}$, in contrast to $p$.
Table \ref{table:OperatorNormProperties} summarises the ranges of $p$ in which different properties hold true.
As it can be seen in table \ref{table:OperatorNormProperties}, for $\forall q \ssin \mathbb{R}_{\sgeq 0}$ and $\forall p \ssin \mathbb{R}_{\sgeq 1}$, the $\ell_{q \to p}$ operator-norm satisfies all the above four properties, hence it is a generalised matrix norm in the mentioned range for $q$ and $p$.
\begin{table*}[bp]
%\begin{adjustbox}{width=\textwidth} % ,totalheight=\textheight,.5
\centering
%\tiny
\begin{tabular}{cccc}
\toprule
$\forall q \ssin \mathbb{R}_{\sgeq 0}$ & \multicolumn{1}{c}{$p=0$}  & \multicolumn{1}{c}{$0 < p < 1$} & \multicolumn{1}{c}{$p \geq 1$} \\ \midrule 
\multicolumn{1}{l}{$\ell_{q \to p}$ properties} & \multicolumn{1}{c}{N, P, T} & \multicolumn{1}{c}{N, P, H} & \multicolumn{1}{c}{N, P, H, T} \\ 
\bottomrule 
\end{tabular}
%\end{adjustbox}
\caption{Properties of $\ell_{q \to p}$ operator-norm for different ranges of $p$, while $\forall q \ssin \mathbb{R}_{\sgeq 0}$, where N, P, H, and T stand for the existence of nonnegativity, positivity, homogeneity, and triangle inequality properties, respectively.}
\label{table:OperatorNormProperties}
\end{table*}
%In the following we continue to investigate the other properties of the $\ell_{q \to p}$ operator-norm.
\end{remark}
\begin{itemize}
\item Submultiplicativity: In general, we have:
\begin{gather*}
\forall (q , p) \in \mathbb{R}^2_{\sg 0}, \qquad \mynorm{\boldsymbol{A} \, \boldsymbol{C}}_{q \to p} \leq \mynorm{\boldsymbol{A}}_{q \to p} \, \mynorm{\boldsymbol{C}}_{q \to p} \max \mybrace{1 , n^{\frac1q - \frac1p}}. 
\end{gather*}
For $q \sgeq p$ we have $\max \{1 , n^{1/q \sm 1/p} \} \seq 1$, and then there exists submultiplicativity property, i.e., for $q \sgeq p \sg 0$, we have $\Vert \boldsymbol{A} \, \boldsymbol{C} \Vert_{q {\to} p} \sleq \Vert \boldsymbol{A} \Vert_{q {\to} p} \, \Vert \boldsymbol{C} \Vert_{q {\to} p}$.
\end{itemize}
\begin{remark}[Matrix norm]
\label{rmrk:Oprtr-nrm-matrx}
Any norm definition that satisfies the above five properties is called \emph{matrix norm} \cite{HornR.A.2012}.
The $\ell_{q \to p}$ operator-norm $\forall q \ssin \mathbb{R}_{\sgeq 0}$ and $\forall p \ssin \mathbb{R}_{\sgeq 1}$ satisfies the first four properties (generalised matrix norm), whereas for $q \sgeq p \sg 0$ satisfies the fifth property.
Then, $\ell_{q \to p}$ operator-norm for $q \sgeq p \sgeq 1$ satisfies all the above five properties, hence it is a matrix norm.
\end{remark}
\begin{itemize}
\item Bounds: We define the following four types of bounds and inequalities for the $\ell_{q {\to} p}$ operator-norm: \\
1) $\forall \myparanthese{q , p , q' , p'} \ssin \mathbb{R}^4_{>0}$, we have:
\begin{gather*}
\begin{aligned}
&\mynorm{\boldsymbol{A}}_{q \to p} \geq \max \mybrace{\min \mybrace{1 , m^{\frac1p - \frac{1}{p'}}} \mynorm{\boldsymbol{A}}_{q \to p'} , \min \mybrace{1 , n^{\frac{1}{q'} - \frac1q}} \mynorm{\boldsymbol{A}}_{q' \to p}}, \\
&\mynorm{\boldsymbol{A}}_{q \to p} \leq \min \mybrace{\max \mybrace{1 , m^{\frac1p - \frac{1}{p'}}} \mynorm{\boldsymbol{A}}_{q \to p'} , \max \mybrace{1 , n^{\frac{1}{q'} - \frac1q}} \mynorm{\boldsymbol{A}}_{q' \to p}},
\end{aligned}
\end{gather*}
which the bounds are based on the operator norm having one of the original domains, either $q$, or $p$.

2) In addition, the bounds can be based on the operator norm having totally new domains, e.g., $q'$ and $p'$, i.e., $\forall \myparanthese{q , p , q' , p'} \ssin \mathbb{R}^4_{>0}$ we have:
\begin{gather*}
\begin{aligned}
&\mynorm{\boldsymbol{A}}_{q \to p} \geq \min \mybrace{1 , m^{\frac1p - \frac{1}{p'}}} \min \mybrace{1 , n^{\frac{1}{q'} - \frac1q}} \mynorm{\boldsymbol{A}}_{q' \to p'}, \\
&\mynorm{\boldsymbol{A}}_{q \to p} \leq \max \mybrace{1 , m^{\frac1p - \frac{1}{p'}}} \max \mybrace{1 , n^{\frac{1}{q'} - \frac1q}} \mynorm{\boldsymbol{A}}_{q' \to p'}.
\end{aligned}
\end{gather*}
\begin{remark}[$\ell_{q {\to} p}$ operator-norm inequalities]
\label{rmrk:operator-norm inequalities} 
The above-mentioned general inequalities for basic tractable $\ell_{q \to p}$ operator-norms based on table \ref{table:OperatorNorm} (page \pageref{table:OperatorNorm}) is shown in table \ref{table:OperatorNormRelations}, which includes the following standard inequalities \cite{Golub2013}:
\begin{gather*}
\begin{aligned}
\mynorm{\boldsymbol{A}}_{1 \to \infty} &\leq \mynorm{\boldsymbol{A}}_{2 \to 2} \leq \sqrt{mn} \mynorm{\boldsymbol{A}}_{1 \to \infty}, \\
\frac{1}{\sqrt{n}}\mynorm{\boldsymbol{A}}_{\infty \to \infty} &\leq \mynorm{\boldsymbol{A}}_{2 \to 2} \leq \sqrt{m} \mynorm{\boldsymbol{A}}_{\infty \to \infty}, \\
\frac{1}{\sqrt{m}}\mynorm{\boldsymbol{A}}_{1 \to 1} &\leq \mynorm{\boldsymbol{A}}_{2 \to 2} \leq \sqrt{n} \mynorm{\boldsymbol{A}}_{1 \to 1}.
\end{aligned}
\end{gather*}
\end{remark}
\begin{table*}[tp]
\begin{adjustbox}{width=\textwidth} % ,totalheight=\textheight,.5
\centering
\begin{tabular}{cccc}
\toprule
$\begin{aligned}
1 &\leq \frac{\mynorm{\boldsymbol{A}}_{1 \to 1}}{\mynorm{\boldsymbol{A}}_{1 \to 2}} \leq m ^{\frac12}, \\
n ^{-1} &\leq \frac{\mynorm{\boldsymbol{A}}_{1 \to 1}}{\mynorm{\boldsymbol{A}}_{\infty \to \infty}} \leq m, \\
n ^{-\frac12} &\leq \frac{\mynorm{\boldsymbol{A}}_{2 \to 2}}{\mynorm{\boldsymbol{A}}_{\infty \to \infty}} \leq m ^{\frac12}, \\
1 &\leq \frac{\mynorm{\boldsymbol{A}}_{1 \to 2}}{\mynorm{\boldsymbol{A}}_{1 \to \infty}} \leq m ^{\frac12}, 
\end{aligned}$ &
$\begin{aligned}
1 &\leq \frac{\mynorm{\boldsymbol{A}}_{1 \to 1}}{\mynorm{\boldsymbol{A}}_{1 \to \infty}} \leq m, \\
1 &\leq \frac{\mynorm{\boldsymbol{A}}_{2 \to 2}}{\mynorm{\boldsymbol{A}}_{1 \to 2}} \leq n ^{\frac12}, \\
m ^{-\frac12} &\leq \frac{\mynorm{\boldsymbol{A}}_{\infty \to \infty}}{\mynorm{\boldsymbol{A}}_{1 \to 2}} \leq n, \\
n ^{-\frac12} &\leq \frac{\mynorm{\boldsymbol{A}}_{1 \to 2}}{\mynorm{\boldsymbol{A}}_{2 \to \infty}} \leq m ^{\frac12},
\end{aligned}$ &
$\begin{aligned}
n ^{-\frac12} &\leq \frac{\mynorm{\boldsymbol{A}}_{1 \to 1}}{\mynorm{\boldsymbol{A}}_{2 \to 2}} \leq m ^{\frac12}, \\
1 &\leq \frac{\mynorm{\boldsymbol{A}}_{2 \to 2}}{\mynorm{\boldsymbol{A}}_{1 \to \infty}} \leq \myparanthese{m \, n} ^{\frac12}, \\
1 &\leq \frac{\mynorm{\boldsymbol{A}}_{\infty \to \infty}}{\mynorm{\boldsymbol{A}}_{1 \to \infty}} \leq n, \\
n ^{-\frac12} &\leq \frac{\mynorm{\boldsymbol{A}}_{1 \to \infty}}{\mynorm{\boldsymbol{A}}_{2 \to \infty}} \leq 1.
\end{aligned}$ &
$\begin{aligned}
n ^{-\frac12} &\leq \frac{\mynorm{\boldsymbol{A}}_{1 \to 1}}{\mynorm{\boldsymbol{A}}_{2 \to \infty}} \leq m, \\
1 &\leq \frac{\mynorm{\boldsymbol{A}}_{2 \to 2}}{\mynorm{\boldsymbol{A}}_{2\to \infty}} \leq m ^{\frac12}, \\
1 &\leq \frac{\mynorm{\boldsymbol{A}}_{\infty \to \infty}}{\mynorm{\boldsymbol{A}}_{2 \to \infty}} \leq n ^{\frac12}, \\
\color{white} n ^{-1} &\color{white}\leq \frac{\mynorm{\boldsymbol{A}}_{1 \to 2}}{\mynorm{\boldsymbol{A}}_{\infty \to \infty}} \color{white}\leq m ^{\frac12}, 
\end{aligned}$
\\ \bottomrule 
\end{tabular}
\end{adjustbox}
\caption{Inequalities of basic tractable $\ell_{q \to p}$ operator-norms based on table \ref{table:OperatorNorm} (page \pageref{table:OperatorNorm}) for a matrix $\boldsymbol{A} \ssin \mathbb{R}^{m \stimes n}$.} % , and $q,p \ssin \{1 , 2 , \infty \}$ corresponding to the tractable 
\label{table:OperatorNormRelations}
\end{table*}
\begin{figure}[!b]
\centering
\includegraphics[width=.4\textwidth,keepaspectratio]{images/OperatorNorm-Inequalities.png} 
\centering
\caption{$\ell_{q {\to} p}$ operator-norm inequalities for common $\ell_{q {\to} p}$ operator-norms according to table \ref{table:OperatorNorm} (page \pageref{table:OperatorNorm}).}
\label{fig:OperatorNorm-Inequalities}
\end{figure}
In addition, in general (tractable and intractable), the $\ell_{q {\to} p}$ operator-norm inequalities for a fixed $p$ or $q$ is shown schematically in figure \ref{fig:OperatorNorm-Inequalities}.

3) Another useful lower- and upper-bound $\forall \myparanthese{q , p , q' , p'} \ssin \mathbb{R}^4_{>0}$ are:
\begin{gather*}
\begin{aligned}
&\mynorm{\boldsymbol{A}}_{q \to p} \geq \frac{\min \mybrace{1 , m^{\frac1p - \frac{1}{p'}}} \min \mybrace{1 , n^{\frac{1}{q'} - \frac1q}} \min \mybrace{1 , m^{\frac{1}{p'} -\frac12}} \min \mybrace{1 , n^{\frac{1}{2} - \frac{1}{q'}}}}{\sqrt{\min \mybrace{m , n}}} \mynorm{\boldsymbol{A}}_F, \\
&\mynorm{\boldsymbol{A}}_{q \to p} \leq \max \mybrace{1 , m^{\frac1p - \frac{1}{p'}}} \max \mybrace{1 , n^{\frac{1}{q'} - \frac1q}} \max \mybrace{1 , m^{\frac{1}{p'} - \frac12}} \max \mybrace{1 , n^{\frac12 - \frac{1}{q'}}} \mynorm{\boldsymbol{A}}_F,
\end{aligned}
\end{gather*}
where, the Frobenius norm is defined as $\Vert \boldsymbol{A} \Vert_F \seq \sqrt{\sum_{i=1}^m \sum_{j=1}^n \vert a_{i,j} \vert^2}$.

\myhl{4) In addition, the $\ell_{q {\to} p}$ operator-norm of any matrix $\boldsymbol{A}$ is less than or equal to the $\ell_{q {\to} p}$ operator-norm of another matrix $\boldsymbol{B}$, in which all the elements are the maximum absolute value of the elements of first matrix.
It also holds true, when all the on-diagonal entries of $\boldsymbol{A}$ and $\boldsymbol{B}$ are set to zero:}
%to compare the $\ell_{q {\to} p}$ operator-norm of two matrices $\boldsymbol{A}$ and $\boldsymbol{B}$ with the same size, where 
%\newpage
\begin{gather*}
\mycolor{\forall i , j, \forall \myparanthese{q , p} \in \mathbb{R}^2_{>0}, \qquad \textrm{if } \myabs{a_{i,j}} \leq b_{i,j} = \max_{i,j} \myabs{a_{i,j}} \Rightarrow \mynorm{\boldsymbol{A}}_{q \to p} \leq \mynorm{\boldsymbol{B}}_{q \to p},}
\end{gather*}
\myhl{and}
\begin{gather*}
\mycolor{\forall i , j, \forall \myparanthese{q , p} \in \mathbb{R}^2_{>0}, \qquad \textrm{if } 
\begin{cases}
\begin{aligned}
  &\myabs{a_{i,j}} \leq b_{i,j} = \max_{i,j} \myabs{a_{i,j}}, \quad &&i \neq j \\
  &a_{i,j} = b_{i,j} = 0, \quad &&i = j
\end{aligned}
\end{cases}
\Rightarrow \mynorm{\boldsymbol{A}}_{q \to p} \leq \mynorm{\boldsymbol{B}}_{q \to p}.}
\end{gather*}
%The above inequality also holds true, when all the on-diagonal entries of $\boldsymbol{A}$ and $\boldsymbol{B}$ are set to zero.
\end{itemize}
\end{property}
See Section \ref{prf:OperatorProperties} (page \pageref{prf:OperatorProperties}) for the proof of Property \ref{prp:OperatorProperties}.