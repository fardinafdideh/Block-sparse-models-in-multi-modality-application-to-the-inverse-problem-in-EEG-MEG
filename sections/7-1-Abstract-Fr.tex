De nombreux phénomènes naturels sont trop complexes pour être pleinement reconnus par un seul instrument de mesure ou par une seule modalité. 
    Par conséquent, le domaine de recherche de la multi-modalité a émergé pour mieux identifier les caractéristiques riches du phénomène naturel de la multi-propriété naturelle, en analysant conjointement les données collectées à partir d’uniques modalités, qui sont en quelque sorte complémentaires.
    Dans notre étude, le phénomène d'intérêt multi-propriétés est l'activité du cerveau humain et nous nous intéressons à mieux la localiser au moyen de ses propriétés électromagnétiques, mesurables de manière non invasive.
    En neurophysiologie, l'électroencéphalographie (EEG) et la magnétoencéphalographie (MEG) constituent un moyen courant de mesurer les propriétés électriques et magnétiques de l'activité cérébrale. 
    Notre application dans le monde réel, à savoir le problème de reconstruction de source EEG / MEG, est un problème fondamental en neurosciences, allant des sciences cognitives à la neuropathologie en passant par la planification chirurgicale.
Considérant que le problème de reconstruction de source EEG / MEG peut être reformulé en un système d'équations linéaires sous-déterminé, la solution (l'activité estimée de la source cérébrale) doit être suffisamment parcimonieuse pour pouvoir être récupérée de manière unique. 
La quantité de parcimonie est déterminée par les conditions dites de récupération. 
Cependant, dans les problèmes de grande dimension, les conditions de récupération conventionnelles sont extrêmement strictes. 
En regroupant les colonnes cohérentes d'un dictionnaire, on pourrait obtenir une structure plus incohérente. 
Cette stratégie a été proposée en tant que cadre d’identification de structure de bloc, ce qui aboutit à la segmentation automatique de l’espace source du cerveau, sans utiliser aucune information sur l’activité des sources du cerveau et les signaux EEG / MEG. 
En dépit du dictionnaire structuré en blocs moins cohérent qui en a résulté, la condition de récupération conventionnelle n’est plus en mesure de calculer la caractérisation de la cohérence. 
Afin de relever le défi mentionné, le cadre général des conditions de récupération exactes par bloc-parcimonie, comprenant trois conditions théoriques et une condition dépendante de l'algorithme, a été proposé. 
Enfin, nous avons étudié la multi-modalité EEG et MEG et montré qu'en combinant les deux modalités, des régions cérébrales plus raffinées sont apparues.