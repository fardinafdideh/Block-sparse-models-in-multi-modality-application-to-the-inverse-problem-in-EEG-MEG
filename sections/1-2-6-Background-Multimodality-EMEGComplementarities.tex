In recent years, the partial independence of bioelectric and biomagnetic measurements has been studied and the added value of their combination under a multi-modal signal processing framework has been the subject of some researches.
In this section, some relative merits of cerebral bioelectromagnetic measurements, i.e., EEG and MEG, are reviewed.
\begin{itemize}
\item \textbf{Dimension of measurement.} MEG measurement is vectorial, whereas EEG measurement is scalar.
\item \textbf{Reference measurement.} MEG measures magnetic field, whereas EEG measures potential \emph{differences}.
In other words, the existence of reference electrode is essential in EEG contrary to MEG (figure \ref{fig:Complementarities}(a)).
\item \textbf{Sensor position.} MEG sensors are located outside but near the head, whereas EEG sensors are placed on the surface of the head.
Therefore, MEG makes a contact-less recording indicating that it has the advantage of very low pre-recording preparation time and the disadvantage of necessity of existence of helmet for localization of the head position (figure \ref{fig:Complementarities}(b)).
\item \textbf{Sensitivity to cerebral currents.} MEG is mostly sensitive to primary currents, whereas EEG is sensitive to secondary currents (figure \ref{fig:Complementarities}(c)).
The cerebral currents were briefly discussed in Section \ref{sec:Physiological_basis}.
\item \textbf{Sensitivity to orientation of dipole.} MEG has very low sensitivity to radial dipoles and is zero for spherical head model \cite{Ahlfors2010} (figure \ref{fig:Complementarities}(d)).
\item \textbf{Sensitivity to depth of dipole.} MEG has low sensitivity to deep sources, because deeper sources become more quasi-radial (figure \ref{fig:Complementarities}(e)).
\item \textbf{Sensitivity to head tissues.} In contrast to MEG, EEG is highly sensitive to the geometry and conductivity of media \cite{Acar2003,Gencer2004}.
Therefore, EEG is distorted while passing through the brain tissues, especially the skull because of its low conductivity, i.e., high resistivity.

On the other hand, MEG is \myhl{far less sensitive} to internal heterogeneities \cite{Hari2017}.
For instance in figure \ref{fig:Complementarities}(f), for a same source model, which is realistic highly-folded cortical sheet, MEG is quite equal for the two head models, while EEG is significantly different.

Because of the transparency of the skull to magnetic fields, MEG is able to measure the cerebral activity of smaller brain regions \cite{Malmivuo2011}.
\item \textbf{Topography.} EEG and MEG topographies for a same source activity are almost orthogonal to each other (figure \ref{fig:Complementarities}(g)).
\end{itemize}
%MEG measures vectorial magnetic field outside but near the head, whereas EEG measures scalar scalp potential differences on the surface of the head, i.e., the existence of reference electrode is essential in EEG contrary to MEG.
%MEG has very low sensitivity to radial dipoles and is zero for spherical head model \cite{Ahlfors2010}, also it has low sensitivity to deep sources, because deeper sources become more quasi-radial (figure \ref{fig:Complementarities}(c)).
%In contrast to MEG, EEG is highly sensitive to the geometry and conductivity of media \cite{Acar2003,Gencer2004}, therefore it is distorted while passing through the brain tissues, especially the skull because of its low conductivity, while MEG is blind to internal heterogeneities (figure \ref{fig:Complementarities}(d)).
\begin{figure}[!b]
\centering
\includegraphics[width=1\textwidth]{images/Complementarities.png} % width=0.5\textwidth  scale=0.49
\caption{(a) MEG measures vectorial magnetic field, whereas EEG measures scalar potential differences, (b) MEG sensors are located outside the head, whereas EEG sensors are placed on the surface of the head, (c) MEG is mostly sensitive to primary currents, whereas EEG is sensitive to secondary currents, (d) MEG has very low sensitivity to radial dipoles, (e) MEG has low sensitivity to deep sources, (f) EEG is highly sensitive to the geometry and conductivity of media, and (g) EEG and MEG topographies for a same source activity are almost orthogonal to each other.
%Nature and requirements of EEG and MEG recordings are different, (b) MEG is sensitive to primary currents, whereas EEG is sensitive to secondary currents, (c) MEG is blind to radial and deep sources, (d) EEG is highly sensitive to the geometry and conductivity of tissues, and (e) EEG and MEG topographies are almost orthogonal
.}
\label{fig:Complementarities}
\end{figure}
\FloatBarrier