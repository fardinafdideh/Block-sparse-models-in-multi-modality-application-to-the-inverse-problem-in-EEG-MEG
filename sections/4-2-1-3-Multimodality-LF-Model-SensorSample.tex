By selecting even or odd rows of USLE, we are assuring that the sensors are uniformly distributed all over the head.
Otherwise, by selecting for example the first half of one mono-modal sensors and the second half of the other mono-modal sensors, although the multi-modal sensors cover whole parts of the head, we would lose the information of one part of brain in mono-modality.
%\section{Combined random dictionaries or EEG and MEG leadfields}

Therefore, considering the three mentioned reasons in multi-modality strategy, we are eliminating or reducing the side effects of increased observation, and optimum sensor position, so we can investigate the impact of multi-modality itself.

As mentioned before, the purpose of this chapter is to investigate the impact of multi-modal lead-field matrix within the block structure identification framework introduced in Chapter \ref{sec:Clustering} with application to brain source space segmentation.
Therefore, from the multi-modal USLE in (\ref{eq:multimodal-model}), we only need to lead-field matrices $\myPhi_{\boldsymbol{\mathrm{EEG}}}$ and $\myPhi_{\boldsymbol{\mathrm{MEG}}}$, or precisely $\myPhi_{\boldsymbol{\mathrm{EEG}}}^O$ and $\myPhi_{\boldsymbol{\mathrm{MEG}}}^E$.
In figure \ref{fig:EMEG-LF}(c), the multi-modal lead-field $\myPhi_{\boldsymbol{\mathrm{EMEG}}}$ is built by concatenating the odd rows of $\myPhi_{\boldsymbol{\mathrm{EEG}}}$ (blue lead-field) and the even rows of $\myPhi_{\boldsymbol{\mathrm{MEG}}}$ (green lead-field).
%the proposed multi-modality framework, we only need the leadfield matrices.
% or dictionaries.
%Therefore, only leadfield matrices $\myPhi_{\boldsymbol{\mathrm{EEG}}}$ and $\myPhi_{\boldsymbol{\mathrm{MEG}}}$ are enough to segment the brain source space.
%The key aim in this chapter is to integrate the complementary information of EEG and MEG modalities within the block structure identification framework introduced in Chapter \ref{sec:Clustering} with application to brain source space segmentation.

As represented graphically in figure \ref{fig:EMEG-LF}, number of multi-modal EMEG sensors in figure \ref{fig:EMEG-LF}(c) is equal to the number of mono-modal EEG and MEG sensors in figure \ref{fig:EMEG-LF}(a) and (b), which is equal to the number of rows of the lead-field matrices, i.e. $m$.

As mentioned before, the MEG sensors in figure \ref{fig:EMEG-LF} are the closest sensors to EEG sensors, e.g., MEG sensor $\#1$ is the closest MEG sensor to EEG sensor $\#1$, and so on.

The minimal change in sensors position in figure \ref{fig:EMEG-LF}(c) in comparison to figure \ref{fig:EMEG-LF}(a) and (b), also the uniform distribution of sensors, would be more evident by increasing the density of sensors.

%On the other hand, in order to combine the electromagnetic properties of the head, latent in the EEG and MEG leadfield matrices, 
% which results in improved source reconstruction problem, 
%one can simply replace one in a row the rows of one modality with the other one, while the total number of the sensors in mono-modal and multi-modal leadfields are equal, as shown in figure \ref{fig:EMEG-LF}(b).


\begin{figure}[!b]
\centering
\includegraphics[width=1\textwidth,keepaspectratio]{images/EMEG-LF.png} % width=0.5\textwidth  scale=0.49
\centering
\caption{The rows of (a) EEG lead-field $\myPhi_{\boldsymbol{\mathrm{EEG}}}$ and (b) MEG lead-field $\myPhi_{\boldsymbol{\mathrm{MEG}}}$ can be combined together to form (c) the new multi-modal lead-field $\myPhi_{\boldsymbol{\mathrm{EMEG}}}$.}
\label{fig:EMEG-LF}
\end{figure}
\FloatBarrier
