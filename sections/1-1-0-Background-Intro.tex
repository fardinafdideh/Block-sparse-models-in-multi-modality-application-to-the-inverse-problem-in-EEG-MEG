%In a distributed \emph{ElectroEncephaloGraphy (EEG)}/\emph{MagnetoEncephaloGraphy (MEG)} source reconstruction, 
%%which is our main problem, 
%two main issues are controversial.
%First, a single modality is insufficient to collect a comprehensive information about the cerebral activity, which is discussed in Section \ref{sec:Multimodality}.
%Secondly, in the resulted Underdetermined System of Linear Equations (USLE) 
%%resulted from the EEG/MEG source reconstruction, 
%\emph{blocks} of coefficients need to be penalized.
%This is because of the nature of the problem that the activity of a group of coefficients is meaningful and not a single coefficient.
This chapter presents a background on the methods to face 
%the mentioned 
two challenges
%, i.e., multimodality and block-sparsity, respectively.
of multi-modality, in Section \ref{sec:Multimodality}, and block-sparsity.
In order to review the recent studies on block-sparsity, we start from the conventional sparsity constraint in finding a unique solution of an USLE\footnote{\emph{Underdetermined System(s) of Linear Equations}} in Section \ref{sec:Sparsity}.
Then, a review of recent studies on block-sparsity is presented in Section \ref{sec:Block-sparsity}.
At last, the role of block-sparsity in multi-modality is discussed in Section \ref{sec:Block-sparsity-multimodality}.