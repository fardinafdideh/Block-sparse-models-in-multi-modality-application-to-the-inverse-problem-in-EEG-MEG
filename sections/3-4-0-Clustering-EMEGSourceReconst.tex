It was explained in Section \ref{sec:Leadfield} that the solution of the  EEG/MEG forward problem is called lead-field, denoted by $\myPhi_{\boldsymbol{\mathrm{{EEG/MEG}}}} \ssin \mathbb{R}^{m \stimes n}$ , which linearly relates the activities in source, denoted by $\mybeta$, and EEG/MEG sensor space, denoted by $\boldsymbol{y_\mathrm{{EEG/MEG}}}$.
In other words, we have $\boldsymbol{y_\mathrm{{EEG/MEG}}} \seq \myPhi_{\boldsymbol{\mathrm{{EEG/MEG}}}} \mybeta$.
 
As mentioned earlier in Section \ref{sec:ForwardInverseProblem}, the inverse problem involves computing the parameters of the brain sources.
Since the number of the sensors, i.e., $m$, is much less than the number of the brain sources, i.e., $K$, there would be infinitely many states of the sources that lead to the same electromagnetic fields, i.e., $\boldsymbol{y_\mathrm{{EEG/MEG}}} \seq \myPhi_{\boldsymbol{\mathrm{{EEG/MEG}}}} \mybeta_i$, where $i \ssin \{\boldsymbol{1}, \cdots , \boldsymbol{\infty} \}$.
Hence, the EEG/MEG source reconstruction problem is ill-posed. 

To make the problem solvable with a unique solution, a dipole source model can be applied to the brain current sources.
In other words, a population of cerebral neurons can be represented by a single dipole.
A dipole can be represented mathematically as a vector.
Considering the 3D Cartesian coordination, for each dipole we have six parameters to be determined.
\begin{itemize}
\item 
Three parameters are related to the dipole position in $\boldsymbol{i}$, $\boldsymbol{j}$, and $\boldsymbol{k}$ direction, which by meshing the brain, these positions can be also limited to some predefined discrete values placed in the vertices of brain mesh.
\item 
Three parameters characterise the power of the dipole in each of three directions, i.e., the magnitudes of the dipole moment in $\boldsymbol{i}$, $\boldsymbol{j}$, and $\boldsymbol{k}$ direction.
\end{itemize}

%The other 
For instance in figure \ref{fig:EEG_MEG_LF}, the dipole located in $s_4$ is related to three consecutive entries in the source activity vector $\mybeta \ssin \mathbb{R}^n$, where $K \seq n {/} 3$.



Assuming only four brain source positions $s_1 , \cdots , s_4$ and three electric or magnetic sensors in a simple case, figure \ref{fig:EEG_MEG_LF} represents EEG and MEG linear models.
%two leadfield matrices for EEG and MEG.

As it can be seen in figure \ref{fig:EEG_MEG_LF}, \myhl{regardless of the head electromagnetic properties, and in terms of the parameters of the dipole source model,} each element of EEG lead-field is a function $f$ of direction vectors $\boldsymbol{i}$, $\boldsymbol{j}$, or $\boldsymbol{k}$ and the distance between the position of corresponding source and sensor, e.g., the EEG lead-field for source $s_1$ is represented in figure \ref{fig:EEG_MEG_LF}.
Each consecutive three elements in a row belongs to a certain sensor and source.
Similarly, %as can be seen in the right part of figure \ref{fig:EEG_MEG_LF}, 
\myhl{in terms of the parameters of the dipole source model,} each element of MEG lead-field is a function $g$ of direction vectors $\boldsymbol{i}$, $\boldsymbol{j}$, or $\boldsymbol{k}$ and the distance between the position of corresponding source and sensor, e.g., the MEG lead-field for source $s_2$ is represented in figure \ref{fig:EEG_MEG_LF}.

Therefore, due to discretising and linearising effect of lead-field and having much less sensors than sources, an EEG/MEG source reconstruction problem can be stated in a form of USLE.
\newpage
\begin{figure}[!b]
\centering
\includegraphics[width=1\textwidth,keepaspectratio]{images/EEG_MEG_LF.png} % width=0.5\textwidth  scale=0.49
\centering
\caption{A simple EEG (left) and MEG (right) linear models for four brain sources $s_1 , \cdots , s_4$, and three electric and magnetic sensors, respectively.
A 3D dipole located in $s_4$ is shown and represented mathematically.} % 
\label{fig:EEG_MEG_LF}
\end{figure}
\FloatBarrier
\newpage
%------------------------------------------------------
\subsection{EEG/MEG source space segmentation} % using clustering of leadfield
\label{sec:EMEG segmentation} 
As mentioned earlier, the EEG/MEG source reconstruction problem can be modelled as an USLE.
In our real-world problem, the dictionary in USLE is called the lead-field matrix.
the whole lead-field matrix, is built by horizontal concatenation of lead-fields of all sources in the source space.
As described before, the lead-field matrix of a single source is a $m$ by 3 matrix, where, $m$ is the number of sensors in the sensor space.
Hence, a whole lead-field matrix is composed of consecutive three-column individual lead-field matrices.

On the other hands, as mentioned before, in the EEG/MEG linear model each of three-column individual lead-field matrices $\myPhi [k]$, $\forall k$, are multiplied to the activity of corresponding dipole in the source vector $\mybeta$ to construct the activity of the sensor space, i.e., $\boldsymbol{y} \seq \sum_{k \seq 1}^K \myPhi [k] \mybeta [k]$.
Then, each block of dictionary, $\myPhi [k]$, $\forall k$, is the direct coefficient of its corresponding source activity.

% each block of length three in the dictionary represents the leadfield matrix of a source in the source space where each source lie into the representation vector $\mybeta$.
Therefore, by clustering the coherent lead-field matrices $\myPhi [k]$ of the whole lead-field matrix $\myPhi$, their corresponding blocks in the source activity vector $\mybeta$ will also be clustered.
Hence, the sources will be grouped and will form some segments in the source space.

To realise the mentioned idea, we propose the general Block-MCC$_{q,p}$ dictionary characterisation, which measures the similarity between the blocks of the dictionary.
Then, Block-MCC$_{q,p}$ can be used in the clustering step to make similarity matrix.

As represented graphically in figure \ref{fig:Source_Segmentation_Schematic}, the six sources in the brain are clustered in different three segments based on the clustering of their corresponding lead-fields in the lead-field matrix.
In other words, clustering of the coherent lead-field matrices $\myPhi [1]$, $\myPhi [4]$, and $\myPhi [6]$ leads to clustering of their corresponding sources $\mybeta [1]$, $\mybeta [2]$, and $\mybeta [3]$, and so on.
\begin{figure}[!b]
\centering
\includegraphics[width=0.8\textwidth,keepaspectratio]{images/Source_Segmentation_Schematic.png} % width=0.5\textwidth  scale=0.49
\centering
\caption{Segmenting the source space of brain using clustering the corresponding lead-fields.}
\label{fig:Source_Segmentation_Schematic}
\end{figure}
\FloatBarrier
