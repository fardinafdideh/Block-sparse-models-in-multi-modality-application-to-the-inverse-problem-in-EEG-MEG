%\iffalse
%Next, we rewrite the bound in Theorem \ref{th:BERC-BMIC}:
%% in two cases of pessimistic and optimistic: 
%%based on the bounds of $\Vert \boldsymbol{x} \Vert_{\overbar{p},1}/ \Vert\boldsymbol{x} \Vert_{\overbar{q},1}$ for $0 \sless q \sless p$:
%\begin{corollary}
%\label{crl:BERC-BMIC}
%%For $0 \sless q \sless p$, 
%The bound of inequality of Block-ERC based on Block-MCC$_{q,p}$ in Theorem \ref{th:BERC-BMIC} is obtained as follows:
%\begin{equation*}
%\begin{aligned}
%\mynorm{\mybetaz}_{r,0} &<
%\min_{\boldsymbol{x}} \frac{1+ \myparanthese{d_{max} M_{q,p}\myparanthese{\myPhi} \frac{\mynorm{\boldsymbol{x}}_{\boldsymbol{w};q,1}}{\mynorm{\boldsymbol{x}}_{\boldsymbol{w};p,1}}}^{-1}}{2} \\
%&= \frac{1+ \myparanthese{d_{max} M_{q,p}\myparanthese{\myPhi} \displaystyle\max_k d_k^{\frac1p - \frac1q} \max \mybrace{1 , d_k^{\frac1q - \frac1p}}}^{-1}}{2} \\
%&\leq \min_{\boldsymbol{x} \in \myKerMath} \frac{1+ \myparanthese{d_{max} M_{q,p}\myparanthese{\myPhi} \frac{\mynorm{\boldsymbol{x}}_{\boldsymbol{w};q,1}}{\mynorm{\boldsymbol{x}}_{\boldsymbol{w};p,1}}}^{-1}}{2}.
%\end{aligned}
%\end{equation*}
%whereas for equally-sized blocks, i.e., $d_1 \seq \cdots \seq d_K \seq d$, we have:
%\begin{equation*}
%\begin{aligned}
%\mynorm{\mybetaz}_{r,0} &<
%\min_{\boldsymbol{x}} \frac{1+ \myparanthese{d^{1 + \frac1p - \frac1q} M_{q,p}\myparanthese{\myPhi} \frac{\mynorm{\boldsymbol{x}}_{q,1}}{\mynorm{\boldsymbol{x}}_{p,1}}}^{-1}}{2} \\
%&= \frac{1+ \myparanthese{d^{1+\frac1p-\frac1q} M_{q,p}\myparanthese{\myPhi} \max \mybrace{1 , d^{\frac1q - \frac1p}}}^{-1}}{2} \\
%&\leq \min_{\boldsymbol{x} \in \myKerMath} \frac{1+ \myparanthese{d^{1 + \frac1p - \frac1q} M_{q,p}\myparanthese{\myPhi} \frac{\mynorm{\boldsymbol{x}}_{q,1}}{\mynorm{\boldsymbol{x}}_{p,1}}}^{-1}}{2}.
%\end{aligned}
%\end{equation*}
%\end{corollary}
%\begin{proof}
%Similar to proof of Block-ERC based on $\myBSpkTxt$ in Theorem \ref{th:BERC-BS}, and using Corollary \ref{crl:BUP-BMIC} instead of block-sparse uncertainty principle based on $\myBSpkTxt$.
%\end{proof}
%
%As it can be seen in Corollary \ref{crl:BERC-BMIC}, the block-sparsity level can be obtained through a minimisation problem, which can be approximated by a closed-form lower-bound, i.e., $(1 \spl (d_{max} M_{q,p}(\myPhi) \max_k d_k^{1/p \sm 1/q} \max \{1 , d_k^{1/q \sm 1/p} \} )^{-1} )/2$ and $(1 \spl (d^{1 \spl 1/p \sm 1/q} M_{q,p}(\myPhi) \max \{1 , d^{1/q \sm 1/p} \} )^{-1} )/ 2$, for differently and equally-sized block structures, respectively.
%\fi
Now Block-ERC of Eldar et al., i.e., $\Vert \mybetaz \Vert_{2,0} \sless (1 \spl (d M^{Eldar}_{Inter}(\myPhi))^{-1} (1 \sm (d \sm 1)M^{Eldar}_{Intra}(\myPhi)))/2$, explained in (\ref{BERC-Eldar}) on page \pageref{BERC-Eldar}, can be compared to our Block-ERC proposed in Theorem \ref{th:BERC-BMIC} (Block-ERC based on Block-MCC$_{q,p}$, page \pageref{th:BERC-BMIC}).
%In order to do that, we consider the proposed most pessimistic closed-form block-sparsity level.
Supposing that all blocks sharing the same block length $d$, it is clear that the relationship between $(M^{Eldar}_{Inter}(\myPhi))^{-1} (1 \sm (d \sm 1)M^{Eldar}_{Intra}(\myPhi))$ and $M_{q,p}^{-1}(\myPhi) \min \{1 , d^{1/q \sm 1/p} \} $ 
%in the most pessimistic case, and $(d^{1/p \sm 1/q} M_{q,p}(\myPhi) \min \{1 , d^{1/q \sm 1/p} \} )^{-1}$ in the most optimistic case 
should be investigated.
%First, in the following lemma, we show the lower-bound of $M_{q,p}^{-1}(\myPhi)$ to $(M^{Eldar}_{Inter}(\myPhi))^{-1} (1 \sm (d \sm 1)M^{Eldar}_{Intra}(\myPhi))$:
\begin{lemma}[\myhl{Eldar et al.\textquotesingle s v.s. proposed $\boldsymbol{\myBSLqpTxt}$}]
\label{lm:Eldar-BMIC} 
%$\forall p \sgeq 1$ 
%\iffalse
%Suppose $M^{Eldar}_{Inter}(\myPhi)$, $M^{Eldar}_{Intra}(\myPhi)$, and $M_{q,p}(\myPhi)$ are inter-block and intra-block coherence of Eldar et al., and our proposed Block-MCC$_{q,p}$, respectively. \\
%1) For a dictionary $\myPhi$ with full column rank blocks, and for $q \sgeq p \sgeq 1$, if: 
%\begin{gather*}
%M^{Eldar}_{Intra}\myparanthese{\myPhi} \leq 
%\frac{1 - \max \mybrace{1 , d^{\frac1p - \frac12}} \max \mybrace{1 , d^{\frac12 - \frac1q}}}{d^{2 - \frac1q} \myparanthese{d - 1}^\frac12 - \myparanthese{d - 1} \max \mybrace{1 , d^{\frac1p - \frac12}} \max \mybrace{1 , d^{\frac12 - \frac1q}}},
%\end{gather*}
%then, the Block-ERC proposed in Theorem \ref{th:BERC-BMIC} improves Block-ERC of Eldar et al., i.e., $d^{1/q \sm 1/p} M_{q,p}^{-1}(\myPhi) \sgeq (M^{Eldar}_{Inter}(\myPhi))^{-1} (1 \sm (d \sm 1)M^{Eldar}_{Intra}(\myPhi))$. \\
%\iffalse
%, whereas if 
%\begin{gather*}
%M^{Eldar}_{Intra}\myparanthese{\myPhi} \leq 
%\min \mybrace{\frac{1}{d^{2 - \frac1q} \myparanthese{d - 1}^\frac12} , \frac{1 - d^{\frac1q - \frac1p} \max \mybrace{1 , d^{\frac1p - \frac12}} \max \mybrace{1 , d^{\frac12 - \frac1q}}}{d^{2 - \frac1q} \myparanthese{d - 1}^\frac12 - \myparanthese{d - 1} d^{\frac1q - \frac1p} \max \mybrace{1 , d^{\frac1p - \frac12}} \max \mybrace{1 , d^{\frac12 - \frac1q}}}}
%\end{gather*}
%then, the Block-ERC proposed in Corollary \ref{crl:BERC-BMIC} in the most optimistic case improves Eldar's Block-ERC, i.e., $M_{q,p}^{-1}(\myPhi) \sgeq (M^{Eldar}_{Inter}(\myPhi))^{-1} (1 \sm (d \sm 1)M^{Eldar}_{Intra}(\myPhi))$.\\
%\fi
%\fi
For a dictionary with intra-block orthonormality, Block-ERC proposed in Theorem \ref{th:BERC-BMIC} (Block-ERC based on Block-MCC$_{q,p}$, page \pageref{th:BERC-BMIC}) for $q \seq p \seq 2$ as the best case is equal to Block-ERC of Eldar et al. (equation (\ref{BERC-Eldar}), page \pageref{BERC-Eldar}). 
\end{lemma}
%\FloatBarrier
The proof of Lemma \ref{lm:Eldar-BMIC} is provided in Section \ref{prf:Eldar-BMIC} (page \pageref{prf:Eldar-BMIC}).
%\iffalse
%\begin{remark}
%\label{Rmrk:Eldar-BMIC-Rmrk} 
%Lemma \ref{lm:Eldar-BMIC} demonstrates that for $q \sgeq p$, if $M^{Eldar}_{Intra}(\myPhi)$ is small enough, then the proposed Block-ERC in Theorem \ref{th:BERC-BMIC} improves Block-ERC of Eldar et al. in (\ref{BERC-Eldar}) through increasing the block-sparsity level.
%\end{remark}
%\fi
\begin{remark}
\label{Rmrk:Eldar-BMIC-equality}
It is worth mentioning that in dictionaries with intra-block orthonormality, by definition of intra-block coherence of Eldar et al., $M^{Eldar}_{Intra}(\myPhi)$ is equal to zero in Block-ERC of Eldar et al. (equation (\ref{BERC-Eldar}), page \pageref{BERC-Eldar}), i.e., $\Vert \mybetaz \Vert_{2,0} \sless (1 \spl (d M^{Eldar}_{Inter}(\myPhi))^{-1} (1 \sm (d \sm 1)M^{Eldar}_{Intra}(\myPhi)))/2 \seq (1 \spl (d M^{Eldar}_{Inter}(\myPhi))^{-1})/2$.
On the other hand, based on lemma \ref{lm:Eldar-BMIC}, the proposed Block-ERC for $q \seq p \seq 2$ is equivalent to the condition of Eldar and her co-workers.
Therefore, our proposed Block-ERC in Theorem \ref{th:BERC-BMIC} (page \pageref{th:BERC-BMIC}) in the following special setting, reduces to Block-ERC of Eldar et al.: 
\begin{itemize}
\item $q \seq p \seq 2$,
\item equally-sized blocks, i.e., $d_1 \seq \cdots \seq d_K \seq d$, and
\item intra-block orthonormality of dictionary, i.e., for $1 \sleq k \sleq K$, $\myPhi^T[k] \myPhi[k] \seq \boldsymbol{I}_d$.
\end{itemize}
%\iffalse
%\begin{itemize}
%\item $q \seq p \seq 2$, 
%\item equally-sized blocks, i.e., $d_1 \seq \cdots \seq d_K \seq d$, and
%\item intra-block orthonormality of dictionary, i.e., for $1 \sleq k \sleq K$, $\myPhi^T[k] \myPhi[k] \seq \boldsymbol{I}_d$.
%\end{itemize}
%\fi
In other words, in the mentioned special setting, \emph{theoretically and independent of the recovery algorithm, the same Block-ERC can be achieved}.
\end{remark}
%\iffalse
%In the above mentioned recovery condition based on the Block-MCC$_{q,p}$ in Theorem \ref{th:BERC-BMIC}, the corresponding optimisation problem $P_{r,0}$ is non-convex.
%Using Block-NSP, we propose the following recovery condition for convex problems:
%%\newpage
%\begin{tcolorbox}
%\begin{theorem}[Block-ERC based on Block-NSP]
%\label{th:BERC-BMIC-Q}
%For any general dictionary $\myPhi$ with Block-MCC$_{q,p}$ $M_{q,p}(\myPhi)$, characterisation $Q_{\boldsymbol{w};p_1,p_2}\myparanthese{S_b\myparanthese{\mybeta},\myPhi}$ defined in Block-NSP, $\forall p \sgeq 1$, $\forall q \sgeq 1$, and $\forall r \sgeq 0$, if 
%\begin{gather*}
%\forall \boldsymbol{x} \in \myKerMath, \qquad \mynorm{\mybeta}_{r,0} < 
%Q_{\boldsymbol{w};q,1}\myparanthese{S_b\myparanthese{\mybeta},\myPhi} +
%\myparanthese{2 \, d_{max} M_{q,p}\myparanthese{\myPhi} \frac{\mynorm{\boldsymbol{x}}_{\boldsymbol{w};q,1}}{\mynorm{\boldsymbol{x}}_{\boldsymbol{w};p,1}}}^{-1},
%\end{gather*}
%then $\mybeta$ is the unique solution to the $P_{\boldsymbol{w};p,1}$ problem.
%Notice that for equally-sized blocks, i.e., $d_1 \seq \cdots \seq d_K \seq d$, if
%\begin{gather*}
%\forall \boldsymbol{x} \in \myKerMath, \qquad \mynorm{\mybeta}_{r,0} < 
%Q_{q,1}\myparanthese{S_b\myparanthese{\mybeta},\myPhi} +
%\myparanthese{2 \, d^{1+\frac1p-\frac1q} M_{q,p}\myparanthese{\myPhi} \frac{\mynorm{\boldsymbol{x}}_{q,1}}{\mynorm{\boldsymbol{x}}_{p,1}}}^{-1},
%\end{gather*}
%then $\mybeta$ is the unique solution to the $P_{p,1}$ problem.
%\end{theorem}
%\end{tcolorbox}
%The proof of Theorem \ref{th:BERC-BMIC-Q} is provided in Section \ref{prf:BERC-BMIC-Q}.
%Again, in the following we rewrite the recovery condition in Theorem \ref{th:BERC-BMIC-Q} in terms of the bound of $\Vert \boldsymbol{x} \Vert_{\boldsymbol{w};p,1}/ \Vert\boldsymbol{x} \Vert_{\boldsymbol{w};q,1}$ and $\Vert \boldsymbol{x} \Vert_{p,1}/ \Vert\boldsymbol{x} \Vert_{q,1}$ introduced in Property \ref{lm:FractionBound}: % for $0 \sless q \sless p$:
%\begin{corollary}
%\label{crl:BERC-BMIC-Q}
%%For $1 \sleq q \sless p$, 
%The bound of recovery condition in Theorem \ref{th:BERC-BMIC-Q} 
%%in the most pessimistic and optimistic cases 
%can be obtained as follows:
%\begin{gather*}
%\begin{aligned}
%\mynorm{\mybetaz}_{r,0} 
%&< \min_{\boldsymbol{x}} Q_{\boldsymbol{w};q,1}\myparanthese{S_b\myparanthese{\mybeta},\myPhi} +
%\myparanthese{2 \, d_{max} M_{q,p}\myparanthese{\myPhi} \frac{\mynorm{\boldsymbol{x}}_{\boldsymbol{w};q,1}}{\mynorm{\boldsymbol{x}}_{\boldsymbol{w};p,1}}}^{-1} \\
%&= Q_{\boldsymbol{w};q,1}\myparanthese{S_b\myparanthese{\mybeta},\myPhi} +
%\myparanthese{2 \, d_{max} M_{q,p}\myparanthese{\myPhi} \max_k d_k^{\frac1p - \frac1q} \max \mybrace{1 , d_k^{\frac1q - \frac1p}}}^{-1} \\
%&\leq \min_{\boldsymbol{x} \in \myKerMath} Q_{\boldsymbol{w};q,1}\myparanthese{S_b\myparanthese{\mybeta},\myPhi} +
%\myparanthese{2 \, d_{max} M_{q,p}\myparanthese{\myPhi} \frac{\mynorm{\boldsymbol{x}}_{\boldsymbol{w};q,1}}{\mynorm{\boldsymbol{x}}_{\boldsymbol{w};p,1}}}^{-1}.
%%\text{Pessimistic} &: \quad \mynorm{\mybetaz}_{r,0} <
%%Q_{\boldsymbol{w};q,1}\myparanthese{S_b\myparanthese{\mybeta},\myPhi} +
%%\myparanthese{2 \, d_{max} M_{q,p}\myparanthese{\myPhi} \max_k d_k^{\frac1p - \frac1q} \max \mybrace{1 , d_k^{\frac1q - \frac1p}}}^{-1}, \\
%%\text{Optimistic} &: \quad  \mynorm{\mybetaz}_{r,0} <
%%Q_{\boldsymbol{w};q,1}\myparanthese{S_b\myparanthese{\mybeta},\myPhi} +
%%\myparanthese{2 \, d_{max} M_{q,p}\myparanthese{\myPhi} \min_k d_k^{\frac1p - \frac1q} \min \mybrace{1 , d_k^{\frac1q - \frac1p}}}^{-1}.
%\end{aligned}
%\end{gather*}
%whereas for equally-sized blocks, i.e., $d_1 \seq \cdots \seq d_K \seq d$, we have:
%%the most pessimistic and optimistic cases reduce to:
%\begin{equation*}
%\begin{aligned}
%\mynorm{\mybetaz}_{r,0} 
%&< \min_{\boldsymbol{x}} Q_{q,1}\myparanthese{S_b\myparanthese{\mybeta},\myPhi} +
%\myparanthese{2 \, d^{1+\frac1p-\frac1q} M_{q,p}\myparanthese{\myPhi} \frac{\mynorm{\boldsymbol{x}}_{q,1}}{\mynorm{\boldsymbol{x}}_{p,1}}}^{-1} \\
%&= Q_{q,1}\myparanthese{S_b\myparanthese{\mybeta},\myPhi} + \myparanthese{2 \, d^{1+\frac1p-\frac1q} M_{q,p}\myparanthese{\myPhi} \max \mybrace{1 , d^{\frac1q - \frac1p}}}^{-1} \\
%&\leq \min_{\boldsymbol{x} \in \myKerMath} Q_{q,1}\myparanthese{S_b\myparanthese{\mybeta},\myPhi} +
%\myparanthese{2 \, d^{1+\frac1p-\frac1q} M_{q,p}\myparanthese{\myPhi} \frac{\mynorm{\boldsymbol{x}}_{q,1}}{\mynorm{\boldsymbol{x}}_{p,1}}}^{-1}
%%\text{Pessimistic} &: \quad \mynorm{\mybetaz}_{r,0} <
%%Q_{q,1}\myparanthese{S_b\myparanthese{\mybeta},\myPhi} +
%%\myparanthese{2 \, d^{1+\frac1p-\frac1q} M_{q,p}\myparanthese{\myPhi} \max \mybrace{1 , d^{\frac1q - \frac1p}}}^{-1}, \\
%%\text{Optimistic} &: \quad \mynorm{\mybetaz}_{r,0} <
%%Q_{q,1}\myparanthese{S_b\myparanthese{\mybeta},\myPhi} +
%%\myparanthese{2 \, d^{1+\frac1p-\frac1q} M_{q,p}\myparanthese{\myPhi} \min \mybrace{1 , d^{\frac1q - \frac1p}}}^{-1}.
%\end{aligned}
%\end{equation*}
%\end{corollary}
%\begin{proof}
%Replacing the bound of $\Vert \boldsymbol{x} \Vert_{\boldsymbol{w};p,1}/\Vert \boldsymbol{x} \Vert_{\boldsymbol{w};q,1}$ and $\Vert \boldsymbol{x} \Vert_{p,1}/\Vert \boldsymbol{x} \Vert_{q,1}$ in Theorem \ref{th:BERC-BMIC-Q} utilising Property \ref{lm:FractionBound}, the proof is done.
%\end{proof}
%\begin{property}
%\label{prp:BERC-BMIC-Q-qq}
%The condition of Theorem \ref{th:BERC-BMIC-Q} in special setting of $q \seq p$ reduces to:
%\begin{gather*}
%\mynorm{\mybeta}_{r,0} < 
%\frac{1 + \myparanthese{d_{max} M_{p,p}\myparanthese{\myPhi}}^{-1}}{2}.
%\end{gather*}
%%whereas for equally-sized blocks, i.e., $d_1 \seq \cdots \seq d_K \seq d$:
%%\begin{gather*}
%%\forall r \geq 0, \quad \mynorm{\mybeta}_{r,0} < 
%%\frac{1 + \myparanthese{d \, M_{p,p}\myparanthese{\myPhi}}^{-1}}{2}.
%%\end{gather*}
%\end{property}
%\begin{proof}
%Continuing from (\ref{eq:DontKnow6}) in the proof of Lemma \ref{lm:Block Spark Inequality}, where, $q \seq p$, we get:
%\begin{gather*}
%\begin{aligned}
%d_{max} M_{p,p}\myparanthese{\myPhi} Q_{\boldsymbol{w};p,1} \mynorm{\boldsymbol{x}}_{\boldsymbol{w};p,1} + Q_{\boldsymbol{w};p,1} \mynorm{\boldsymbol{x}}_{\boldsymbol{w};p,1} &\leq 
% d_{max} M_{p,p}\myparanthese{\myPhi} \mynorm{\boldsymbol{x}}_{\boldsymbol{w};p,1} \mynorm{\mybeta}_{r,0}, \\
%Q_{\boldsymbol{w};p,1} &\leq 
%\frac{d_{max} M_{p,p}\myparanthese{\myPhi} \mynorm{\mybeta}_{r,0}}{d_{max} M_{p,p}\myparanthese{\myPhi} + 1} \\
%&< \frac12.
%\end{aligned}
%\end{gather*}
%The above last inequality comes from the Block-NSP condition in Theorem \ref{th:BNSP} and the proof is done.
%%Similarly, from (\ref{eq:DontKnow7}) the results for equally-sized blocks can be proved.
%\end{proof}
%For $d_1 \seq \cdots \seq d_K \seq 1$, Property \ref{prp:BERC-BMIC-Q-qq} converges to its conventional element-wise counterpart in (\ref{eq:ERC-M}), i.e., $\Vert \mybetaz \Vert_0 \sless (1 \spl M^{-1} (\myPhi))/2$.
%\begin{remark}[Equivalence]
%\label{Rmrk:P1P0-Equivalence} 
%From Corollary \ref{crl:BERC-BMIC} and Property \ref{prp:BERC-BMIC-Q-qq}, it can be deduced that for $1 \sleq q \seq p$ the 
%%both pessimistic and optimistic 
%condition of $\forall r \sgeq 0, \, \Vert \mybeta \Vert_{r,0} \sless (1 \spl d_{max}^{-1} M_{p,p}^{-1}(\myPhi))/2$, guarantees the uniqueness of the solution $\mybeta$ to both non-convex $P_{r,0}$ and convex $P_{\boldsymbol{w};p,1}$ problems.
%\end{remark}
%\fi