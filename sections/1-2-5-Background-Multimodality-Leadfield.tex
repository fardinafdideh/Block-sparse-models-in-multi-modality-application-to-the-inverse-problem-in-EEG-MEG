The transmission of electromagnetic fields from sources space through head tissues towards sensors space can be modelled with quasi-static approximations of Maxwell’s equations \cite{Sarvas1987}.

The quasi-static approximations of Maxwell’s equations, which allows to ignore the time derivatives, can be divided into two groups of: 
\begin{itemize}
\item quasi-electrostatics, which are describing electric field $\boldsymbol{E}$, and
\item quasi-magnetostatics, which are describing magnetic field $\boldsymbol{B}$.
\end{itemize} 

The mentioned quasi-static approximations with electric potential $\boldsymbol{V}$ and magnetic field $\boldsymbol{B}$ in distance $\boldsymbol{r}$ in an infinite homogeneous medium with permittivity $\varepsilon_0$ and permeability $\mu_0$ for a dipole $\boldsymbol{p}$ with electric charge density $\rho$ and electric current density $\boldsymbol{J}$ are shown in figure \ref{fig:quasi-static}, and table \ref{table:quasi-static}.

The quasi-static approximations of Maxwell’s equations in addition to the assumed discrete positions of the brain sources lead to a discretised and linearised relationship between electromagnetic signals and the source activities.
The mentioned relationship is realised through the so-called \emph{lead-field} matrix which contains the electromagnetic and geometrical properties of the head.
\\
\begin{figure}[!h]
\centering
\includegraphics[width=0.6\textwidth,keepaspectratio]{images/Quasi-statics.png} % width=0.5\textwidth  scale=0.49
\centering
\caption{Electromagnetic fields ($\boldsymbol{E}$, $\boldsymbol{B}$) at distance $\boldsymbol{r}$ of dipole $\boldsymbol{p}$.}
\label{fig:quasi-static}
\end{figure}
\begin{table*}[b]
\begin{adjustbox}{width=\textwidth} % ,totalheight=\textheight,.5
\centering
%\tiny
\begin{tabularx}{\textwidth}{Y Y Y Y}% cccc p{3 cm}  *{6}{Y|} 
\toprule
%\cline{2-4}
\multicolumn{2}{c}{Quasi-electrostatics} & \multicolumn{2}{c}{Quasi-magnetostatics} \\ \midrule %\hline
Gauss’s law & $\nabla \cdot \boldsymbol{E} = \frac{\rho}{\varepsilon_0}$ & Gauss’s law & $\nabla \cdot \boldsymbol{B} = 0$ \\ %\cline{1-1}%\hline
Faraday’s law & $\nabla \times \boldsymbol{E} = \boldsymbol{0}$ 
 & Ampère’s law & $\nabla \times \boldsymbol{B} = \mu_0 \boldsymbol{J}$ \\  \midrule%\hline
\multicolumn{2}{c}{$V = \frac{1}{4 \pi \varepsilon_0} \boldsymbol{p} \cdot \frac{\boldsymbol{r}}{\myabs{\boldsymbol{r}}^3}$} & \multicolumn{2}{c}{$\boldsymbol{B} = \frac{\mu_0}{4 \pi} \boldsymbol{p} \times \frac{\boldsymbol{r}}{\myabs{\boldsymbol{r}}^3}$} \\
\bottomrule
\multicolumn{4}{c}{where, $\nabla \cdot$, $\nabla \times$, $\cdot$, and $\times$ are divergence, curl, dot and cross product operators, respectively.} \\ % 
\end{tabularx}
\end{adjustbox}
\caption{The quasi-static approximations of Maxwell’s equations.}
%\caption{The quasi-static approximations of Maxwell’s equations with electric potential $\boldsymbol{V}$ and magnetic field $\boldsymbol{B}$ in an infinite homogeneous medium.}
\label{table:quasi-static}
\end{table*}
\newpage

The lead-field matrix, which is also considered as the solution of the forward problem, is computed when the volume conduction head model, source model, and sensor model is given.
%The forward problem can be solved when the volume conduction head model, source model, and sensor model is given.
%The solution of the forward problem is called \emph{leadfield}, which linearly relates the activities in source and EEG/MEG sensor space.
%This relationship holds true for a fixed and discrete distribution of positions of each source-sensor pair.
%which is a matrix of dimensions sensors' number by three times of the number of sources.

As shown in figure \ref{fig:LF}, knowing the volume conduction head model including scalp, skull, and brain models with $K$ source positions in source space and $m$ sensors in sensor space makes possible the computation of the lead-field matrix of dimension $m$ by $3K$.

Each block of \emph{three columns} in lead-field matrix, consists of the response of all sensors to a specific probing dipole in the source space.
Actually, it is a block of three canonical probing vectors.
%the forward and inverse problems and geometrical models are illustrated.
\begin{figure}[!b]
\centering
\includegraphics[width=1\textwidth]{images/LF.png} % width=0.5\textwidth  scale=0.49
\caption{Forward problem is solved when the head (scalp, skull, and brain layers), source ($K$ source positions), and sensor ($m$ sensors) models are given.
The lead-field matrix $\myPhi$ of dimension $m$ by $3K$ is resulted from concatenation of $K$ three-column matrices corresponding to $K$ source positions in source space.
%Then with the resulted leadfield matrix, the inverse problem can be solved, i.e., given a measurement on sensors, the corresponding active dipole(s) can be estimated.
}
\label{fig:LF}
\end{figure}
\FloatBarrier