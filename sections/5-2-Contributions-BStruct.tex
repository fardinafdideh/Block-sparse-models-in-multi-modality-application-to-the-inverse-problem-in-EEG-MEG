\addcontentsline{toc}{section}{\protect\numberline{}Block-structured problem}
\begin{tcolorbox}
\begin{challenge}
In classical assumption, the recovery conditions are generated from the columns of coefficient matrix, but clustered coefficient matrix consists of some differently-sized blocks, and not necessarily columns.
Therefore, the initial assumption of classical recovery conditions does no longer hold true. 
How can appropriate recovery conditions be developed for block-structured problems? 
\end{challenge}
\end{tcolorbox}

In Chapter \ref{sec:BERC}, we defined a general framework to cover all types of atomic entity including columns, and equally/differently-sized blocks of columns, in order to propose general theoretical exact recovery conditions.
The proposed theoretical recovery conditions are based on \emph{block-sparsity} constraint, which ensure the uniqueness of the block-sparse solution of corresponding weighted (pseudo-)mixed-norm optimisation problem in an underdetermined system of linear equations.
%In Chapter \ref{sec:BERC}, the sufficient conditions for unique recovery of block-sparse recovery of an arbitrary signal $\boldsymbol{y}$ in a general arbitrary dictionary $\myPhi$ using a general weighted mixed norm $P_{\overbar{p_1},p_2}$ (and also $P_{p_1,p_2}$ in equally-sized blocks) optimisation problems, are proposed. 
The mentioned generality of the framework is in terms of the properties of the underdetermined system of linear equations, extracted characterisations, optimisation problems, and ultimately the recovery conditions.
%In order to propose the recovery conditions, the characterisations and properties were defined in their general case.
The mentioned theoretical exact recovery conditions are categorized in four different groups based on the utilised characterisations and properties, i.e., conditions based on (1) Block-Spark, (2) block null space property, (3) block mutual coherence constant, and (4) cumulative coherence constant.

On the other hand, the proposed framework is consistent with the base findings, since all the materials in the proposed infrastructure are a generalisation of the existing references 
Indeed, we investigated the theoretical relationship between the proposed infrastructure and the classical one, and showed that all the new materials reduce to the conventional ones in specific cases.
%On the other hand, since all the proposed materials are a generalisation of the existing references, they all reduce to the conventional ones in specific cases, i.e. the proposed framework is consistent with the base findings.
We redemonstrated the benefit of block-sparsity assumption compared to conventional sparsity in the improvement of the recovery conditions.
In addition, we theoretically proved the supremacy of the theoretical exact recovery conditions defined in the proposed general infrastructure over existing conditions, which are assuming the same block-sparsity constraint.

%over existing conditions in block-sparsity made by the proposed conditions.
As perspective in this research direction, we could mention to the following subjects:
%the generalisation of the proposed exact recovery conditions to a more realistic stable recovery conditions.
%In addition, in the conditions group of cumulative coherence constant, we are studying on introducing another block-sparse exact recovery condition based on cumulative Block-MCC$_{q,p}$, defined in Definition \ref{def:CBMIC}.
%Future research will focus on:
\begin{itemize}
\item Introducing block-sparse recovery conditions based on the proposed cumulative Block-MCC$_{q,p}$ defined in Definition \ref{def:CBMIC}.
\item Generalising the conventional dictionary characterisation of $\mu$ defined in \cite{Donoho2003}, to establish block-sparse recovery conditions.
\item  Transforming all the previously mentioned block-sparse \emph{exact} recovery conditions to block-sparse \emph{stable} recovery conditions.
In stable or robust recovery conditions we have, $\Vert \boldsymbol{y} \sm \myPhi \hat{\mybeta} \Vert_{2} \sless e$, where, $e$ is a bounded noise.
\item Study on block-sparse optimisation algorithms, and the relationship between the \emph{theoretical} and \emph{algorithmic} block-sparse recovery conditions.
\end{itemize}