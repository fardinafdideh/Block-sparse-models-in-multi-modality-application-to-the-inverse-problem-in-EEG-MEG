To define the Block-ERC, it is necessary to introduce some notations.
Except when mentioned explicitly, (1) 
%all of the below characterisations hold true for all $q \ssin \mathbb{R}_{\sg 0}$, $p \ssin \mathbb{R}_{\sgeq 0}$ and 
the block length $d_k \ssin \mathbb{N}$, where, $\sum_{k \seq 1}^K d_k \seq n$, and $d_{max}$ and $d_{min}$ are the maximum and minimum block length, respectively, and (2) the characterisations coincide with the results known in the conventional element-wise case if all blocks are of size $1$, i.e., $d_1 \seq \cdots \seq d_K \seq 1$.
We first give some general definitions, which then will be used to prove uniqueness results.
\begin{Mydefinition}[\myhl{$\boldsymbol{\myBSuppTxt}$ and $\boldsymbol{\myBCardTxt}$}]
\label{Def:Block Support} 
\leftbar
The \emph{Block-Support} and \emph{Block-Cardinality} of a representation vector $\mybeta \ssin \mathbb{R}^{n}$ are defined as:
\begin{gather*}
\label{eq:Block Support}
\begin{aligned}
\myBSuppTxt &\textrm{:} \qquad S_b\myparanthese{\mybeta} && \myeq \mybrace{k: \mynorm{\mybeta\mybracket{k}}_{p} \neq 0, 1 \leq k \leq K} \qquad &&&\forall p \geq 0, \\
%& && \myeq \mybrace{k: \frac{\mynorm{\mybeta\mybracket{k}}_{p}}{d_k^{\frac1p}} \neq 0, 1 \leq k \leq K} \qquad &&&\forall p > 0, \\
\myBCardTxt &\textrm{:} \qquad \myabs{S_b\myparanthese{\mybeta}} && \myeq \mynorm{\mybeta}_{p,0} \qquad \forall p \geq 0. \\
%& \myeq \mynorm{\mybeta}_{\boldsymbol{w};p,0} \qquad &&&\forall p > 0. 
\end{aligned}
\end{gather*}
\endleftbar
\end{Mydefinition}

$\Vert \mybeta \Vert_{p,0}$ simply counts the number of active blocks in the sense of their non-zero $\ell_{p}$ norm.

In fact, considering the definition of $\Vert \mybeta \Vert_{p,0}$, each block is first represented by its $\ell_{p}$ norm.
%while based on definition, 
So, non-zero representatives indicate active blocks.
Then, by applying the $\ell_{0}$ pseudo-norm on the blocks' representatives, 
%non-zero representatives or indeed 
the active blocks are counted.
When $d_1 \seq \cdots \seq d_K \seq 1$, the $\myBSuppTxt$ and $\myBCardTxt$ will be equivalent to the conventional $\mySuppTxt$ and $\myCardTxt$ (see Section \ref{sec:Conv-Characterization}), respectively.
\begin{Mydefinition}[\myhl{Dictionary $\boldsymbol{\myKerTxt}$}]
\leftbar
The \emph {Block-Kernel} or \emph {block null space} of a dictionary $\myPhi \ssin \mathbb{R}^{m \stimes n}$, is equivalent to its usual $\myKerTxt$: 
\begin{gather*}
\label{eq:Block Kernel}
\begin{aligned}
\myBKerMath &\myeq \mybrace{\boldsymbol{x} \in \mathbb{R}^n, \displaystyle\sum_{k=1}^{K} \myPhi\mybracket{k}\boldsymbol{x}\mybracket{k} = \myPhi \boldsymbol{x} = \boldsymbol{0}} \\ 
&= \myKerMath.
\end{aligned}
\end{gather*}
\endleftbar
\end{Mydefinition}
Actually, the $\myBKerTxt$ is the block-wise definition of the conventional $\myKerTxt$ of a dictionary.
Because of their equivalency and for the sake of simplicity, we use the notation of conventional $\myKerTxt$.

Regarding the uniqueness of the solution of $P_{p,0}$ problem, an interesting question is 
"What is the minimal value of $\Vert \boldsymbol{x} \Vert_{p,0}$ for $\boldsymbol{x} \ssin \myKerMath \backslash\{\boldsymbol{0}\}$, and $p \sg 0$?".
To elucidate the significance of the mentioned question, assume that $\mybetaz$ and $\mybetao$ are both solutions to problem $P_{p,0}$, i.e., $\boldsymbol{y} \seq \myPhi \mybetaz \seq \myPhi \mybetao$.
Then $\myPhi (\mybetaz \sm \mybetao) \seq \boldsymbol{0}$, so $\boldsymbol{x} \seq \mybetaz \sm \mybetao \ssin \myKerMath$. 
Clearly, $\Vert \boldsymbol{x} \Vert_{p,0} \seq \Vert \mybetaz \sm \mybetao \Vert_{p,0} \sleq \Vert \mybetaz \Vert_{p,0} \spl \Vert \mybetao \Vert_{p,0} \seq 2 \Vert\mybetaz \Vert_{p,0}$ since both $\Vert \mybetaz \Vert_{p,0}$ and $\Vert \mybetao \Vert_{p,0}$ attain the minimum $\ell_{p,0}$ pseudo-mixed-norm.
Consequently, assuming two solutions we reach to $\Vert \boldsymbol{x} \Vert_{p,0} / 2 \sleq \Vert \mybetaz \Vert_{p,0}$. 

Hence, if $\Vert \mybetaz \Vert_{p,0} \sless \min_{\boldsymbol{x} \ssin \myKerMath \backslash\{\boldsymbol{0}\}} \Vert \boldsymbol{x} \Vert_{p,0}/2$, then no other solution $\mybetao$ exists for which $\Vert \mybetaz \Vert_{p,0} \seq \Vert \mybetao \Vert_{p,0}$. We stress the importance of the latter through the following definition which is a straightforward generalisation of the conventional $\mySpkTxt$ \cite{Gribonval2003a,Donoho2003}:
\begin{Mydefinition}[\myhl{$\boldsymbol{\myBSpkTxt}$}]
\label{def:Block Spark}
\leftbar
The \emph{Block-Spark} of a dictionary can be defined based on $\ell_{p,0}$ pseudo-mixed-norm:
%or $\ell_{p,0}^{\boldsymbol{w}}$ weighted pseudo-mixed-norm:
\begin{equation*}
\begin{aligned}
\label{eq:Block Spark}
\myBSpkMath &\myeq \min_{\boldsymbol{x} \in \myKerMath\backslash\left\{\boldsymbol{0}\right\}} \mynorm{\boldsymbol x}_{p,0} \qquad \forall p \geq 0. \\
%&= \min_{\boldsymbol{x} \in \myKerMath \backslash\left\{\boldsymbol{0}\right\}} \mynorm{\boldsymbol x}_{\boldsymbol{w};p,0} \qquad \forall p \sg 0.
\end{aligned}
\end{equation*}
\endleftbar
\end{Mydefinition}
$\myBSpkTxt$ characterises the block null space of a dictionary using the $\ell_{p,0}$ pseudo-mixed-norm.
%or the $\ell_{p,0}^{\boldsymbol{w}}$ weighted pseudo-mixed-norm. 
%Notice that, in definition of $\myBSpkTxt$ using the $\ell_{p,0}^{\boldsymbol{w}}$ weighted pseudo-mixed-norm, $p$ cannot have zero value.

By definition, any vector $\boldsymbol{x}$ in the block null space of the dictionary, i.e., $\sum_{k=1}^{K} \myPhi[k]\boldsymbol{x}[k] \seq \boldsymbol{0}$, must satisfy $\forall p \sgeq 0, \Vert \boldsymbol{x} \Vert_{p,0} \sgeq \myBSpkMath$. 
%or $\forall p \sg 0, \Vert \boldsymbol{x} \Vert_{\boldsymbol{w};p,0} \sgeq \myBSpkMath$.
This important block-wise characterisation of the dictionary is the extension of the conventional $\mySpkTxt$ defined in equation (\ref{eq:Conventional Spark}) page \pageref{eq:Conventional Spark}.

Because of the fact that the proposed characterisation is a block-wise generalisation of the conventional element-wise case, in a unit block size setting, i.e., $\forall k, d_k \seq 1$, our proposed $\myBSpkTxt$ is equivalent to the conventional $\mySpkTxt$, i.e., $\myBSpkMath {\equiv}\mySpkMath$.

The following property shows the relationship between the proposed $\myBSpkTxt$ and the conventional $\mySpkTxt$, which will be used in Section \ref{sec:BERC-BS} to demonstrate the improvement of the recovery conditions based on $\myBSpkTxt$ compared to the ones based on the conventional $\mySpkTxt$:
\begin{property}[$\boldsymbol{\myBSpkTxt}$ v.s. $\boldsymbol{\mySpkTxt}$]
\label{prp:BS-S}
Let $d_k$ be the block length of the $k^{th}$ block in $\myPhi$, and denoting $\overbar{d} \seq \sum_{k \ssin S_b(\boldsymbol{x}^\star_b)}d_k / \myabs{S_b(\boldsymbol x^\star_b)}$ the average block length, with $\boldsymbol{x}^\star_b \seq \arg \min_{\boldsymbol{x} \ssin \myKerMath \backslash\{\boldsymbol{0}\}} \Vert \boldsymbol{x} \Vert_{p,0}$, $\forall p \sgeq 0$, then we have:
\begin{equation*}
\overbar{d} \, \myBSpkMath \geq \mySpkMath.
\end{equation*}
%where $\begin{cases}\overbar{d} &=\sum_{k \ssin S_b(\boldsymbol{x}^\star_b)}d_k / \myabs{S_b(\boldsymbol x^\star_b)}\\
%\boldsymbol{x}^\star_b &= \arg \min_{\boldsymbol{x} \ssin \myKer(\myPhi) \backslash\{\boldsymbol{0}\}} \Vert \boldsymbol{x} \Vert_{\boldsymbol{w};p,0}, \quad \forall p \sg 0\end{cases}$.
\end{property}
\begin{proof}
Let $\boldsymbol{x}^\star \ssin \arg \min_{\boldsymbol{x} \ssin \myKerMath \backslash\{\boldsymbol{0}\}} \Vert \boldsymbol{x} \Vert_{0}$ and $\boldsymbol{x}^\star_b \ssin \arg \min_{\boldsymbol{x}\ssin \myKerMath \backslash\{\boldsymbol{0}\}} \Vert \boldsymbol{x} \Vert_{p,0}$. 
Obviously, $\Vert \boldsymbol{x}^\star \Vert_0 \sleq \Vert\boldsymbol{x}^\star_b \Vert_{0}$, and $\Vert \boldsymbol{x}^\star_b \Vert_{0}\sleq \sum_{k\in S_b(\boldsymbol{x}^\star_b)}d_k$, indeed the whole block is activated even if only a single element is needed. 
Now letting $\overbar{d} \seq \myabs{S_b(\boldsymbol{x}^\star_b)}^{-1}\sum_{k\ssin S_b(\boldsymbol{x}^\star_b)}d_k \seq \Vert \boldsymbol{x}^\star_b\Vert_{p,0}^{-1}\sum_{k\in S_b(\boldsymbol{x}^\star_b)}d_k$, the transitivity of the inequalities yields $\Vert \boldsymbol{x}^\star \Vert_0 \sleq \overbar{d} \, \Vert \boldsymbol{x}^\star_b \Vert_{p,0}$, which is exactly $\overbar{d} \, \myBSpkMath \sgeq \mySpkMath$.
\end{proof}
Due to non-convexity of the 
%weighted pseudo-mixed-norm ${\Vert \cdot \Vert_{\boldsymbol{w};p,0}}$ and 
pseudo-mixed-norm $\Vert \cdot \Vert_{p,0}$, $\myBSpkTxt$ is not computationally tractable.
Therefore, the following 
%tractable 
block coherence measure, which is the block-wise extension of the conventional MCC\footnote{\emph{Mutual Coherence Constant}} is proposed:
\newpage
\begin{Mydefinition}[Block coherence]
\label{def:BMIC}
\leftbar
The \emph{$({q,p})$-Block Mutual Coherence Constant (Block-MCC$_{q,p}$)} of a dictionary is defined 
%$\forall q \sg 0$ and $\forall p \sgeq 1$ 
$\forall (q , p) \ssin \mathbb{R}^2_{>0}$ as:	
\begin{equation*}
\begin{aligned}
\label{eq:BMIC}
M_{q,p}\myparanthese{\myPhi} 
%&\myeq \max_{\substack{k,k' \neq k  \\ \boldsymbol{x} \neq \boldsymbol{0}}} \frac{d_{k}^{-\frac1p} \, d_{k'}^{\frac1q} \, \mynorm{\myPhi^\dagger\mybracket{k} \myPhi \mybracket{k'}\boldsymbol{x}\mybracket{k'}}_{p}}{d_{max} \, \mynorm{\boldsymbol{x}\mybracket{k'}}_{q}} \\
\myeq \max_{k,k' \neq k} \frac{d_{k}^{-\frac1p} \, d_{k'}^{\frac1q}}{d_{max}} \mynorm{\myPhi^\dagger\mybracket{k} \myPhi \mybracket{k'}}_{q \to p},
\end{aligned}
\end{equation*}
where, $\myPhi^\dagger[k]$ is Moore-Penrose pseudo-inverse of the \myhl{full column rank} block $\myPhi[k] \ssin \mathbb{R}^{m \stimes d_k}$, and $\Vert \myPhi \Vert_{q \to p}$ is the $\ell_{q {\to} p}$ operator-norm of dictionary $\myPhi$, which can be computed through any of the following expressions \cite{Tropp2004b,Golub2013}:
\begin{gather*}
\begin{aligned}
\label{eq:Operator-Norm}
\mynorm{\boldsymbol{A}}_{q \to p} & \myeq 
\max_{\boldsymbol{a} \neq \boldsymbol{0}} \frac{\mynorm{\boldsymbol{A} \boldsymbol{a}}_{p}}{\mynorm{\boldsymbol{a}}_{q}} 
= \max_{\mynorm{\boldsymbol{a}}_q = 1} \mynorm{\boldsymbol{A} \boldsymbol{a}}_{p} = \max_{\mynorm{\boldsymbol{a}}_q \leq 1} \mynorm{\boldsymbol{A} \boldsymbol{a}}_{p}.
\end{aligned}
\end{gather*}
\endleftbar
\end{Mydefinition}
%The reason for choosing the coefficients $d_{k}^{-\frac1p} \, d_{k'}^{\frac1q}$ and $d_{max}^{\sm 1}$ in Definition \ref{def:BMIC}, is mentioned in the proof of corollaries \ref{crl:Block Spark Inequality} and \ref{prp:BMIC-MIC}, respectively.
%In figure \ref{fig:SchematicBMIC} a schematic representation of computation of Block-MCC$_{q,p}$ for a low-dimensional case is shown.
%In this case, the dictionary is composed of two blocks, so to compute Block-MCC$_{q,p}$ maximisation is applied over two cases, as shown as two lines of computations.
%Take the first line of computations, the columns of $\myPhi^\dagger[1]$ and $\myPhi [2]$ are shown in solid blue and orange colors, respectively.
%Then, the $\myPhi^\dagger[1] \myPhi [2]$ matrix is represented numerically and graphically.
%Afterwards, in order to compute the $q$ to $p$ operator, the $\ell_p$-norm of $\myPhi^\dagger[1] \myPhi [2] \boldsymbol{x}$ should be computed for three tractable norms of $\ell_1$, $\ell_2$, and $\ell_{\infty}$, where $\Vert \boldsymbol{x} \Vert_1$, $\Vert \boldsymbol{x} \Vert_2$, and $\Vert \boldsymbol{x} \Vert_{\infty}$ are less than or equal to one, respectively.
%Then, by multiplying to the proper coefficient based on the block size, $q$, and $p$, and finally by taking the maximum value among all cases (here two), Block-MCC$_{q,p}$ is obtained.
%\begin{figure}[!b]
%\centering
%\includegraphics[width=1\textwidth,keepaspectratio]{images/BMIC-Schematic.png}
%\centering
%\caption{Schematic representation of computation of Block-MCC$_{q,p}$ for a low-dimensional case.}
%\label{fig:SchematicBMIC}
%\end{figure}

For a unit block size setting, $d_1 \seq \cdots \seq d_K \seq 1$, and $\ell_2$-normalized columns of $\myPhi$, as expected, Block-MCC$_{q,p}$ is equivalent to conventional MCC (page \pageref{eq:MIC}), i.e., $M_{q,p}(\myPhi) {\equiv} M(\myPhi)$, because each block of $d_k$ columns $\myPhi[k]$ will be a single column, i.e., $\myPhi[k] \seq \myphi_k$, and for the vector $\myphi_k$, we have $\myphi^\dagger_k \seq \boldsymbol{\varphi}^T_k$ and for the scalar $\boldsymbol{\varphi}^T_k \boldsymbol{\varphi}^{ }_{k'}$, we have $\Vert \boldsymbol{\varphi}^T_k \boldsymbol{\varphi}^{ }_{k'} \Vert_{q \to p} \seq \vert \boldsymbol{\varphi}^T_k \boldsymbol{\varphi}^{ }_{k'} \vert$ for $\forall (q , p) \ssin  \mathbb{R}^2_{\sgeq 0}$, so $M_{q,p}(\myPhi) \seq \max_{k , k' {\neq} k} \vert \boldsymbol{\varphi}^T_k \boldsymbol{\varphi}^{ }_{k'} \vert \seq M(\myPhi)$.

Although Block-MCC$_{q,p}$ is valid for $\forall \myparanthese{q , p} \ssin \mathbb{R}^2_{>0}$, in practice, because of the computational complexity, only some basic operator-norms can be calculated \cite{Tropp2004b}.
Table \ref{table:OperatorNorm} explains these basic operator-norms.
\begin{table*}[bp]
\begin{adjustbox}{width=\textwidth} % ,totalheight=\textheight,.5
\centering
%\tiny
\begin{tabular}{cccc}
\toprule
%\cline{2-4}
                              & \multicolumn{1}{c}{$p=1$}  & \multicolumn{1}{c}{$p=2$} & \multicolumn{1}{c}{$p=\infty$} \\ \midrule %\hline
\multicolumn{1}{l}{$q=1$} & \multicolumn{1}{c}{Maximum $\ell_1$ norm of a column} & \multicolumn{1}{c}{Maximum $\ell_2$ norm of a column} & Maximum absolute entry of matrix \\ %\cline{1-1}%\hline
\multicolumn{1}{l}{$q=2$} & \multicolumn{1}{c}{NP-hard} & \multicolumn{1}{c}{Maximum singular value} & Maximum $\ell_2$ norm of a row    \\ %\cline{1-1} %\hline
\multicolumn{1}{l}{$q=\infty$}   & \multicolumn{1}{c}{NP-hard} & \multicolumn{1}{c}{NP-hard} & Maximum $\ell_1$ norm of a row    \\ \bottomrule %\hline
\end{tabular}
\end{adjustbox}
\caption{Computational complexity of $\ell_{q \to p}$ operator-norm for different basic $(q,p)$ pairs \cite{Tropp2004b}.}
\label{table:OperatorNorm}
\end{table*}

In order to investigate the bounds and relationship of the proposed Block-MCC$_{q,p}$ to the existing dictionary characterisations, we need to study the properties of the $\ell_{q {\to} p}$ operator-norm of a matrix, which in turn requires to establish the bounds of division of norm of two vectors:
\begin{property}[Bounds of two vector norms division]
\label{prp:VectorDivisionBound}
$\forall (q , p) \ssin \mathbb{R}^2_{>0}$, and $\forall \boldsymbol{a} \ssin \mathbb{R}^d$, we have:
\begin{gather*}
\min \mybrace{1 , d^{\frac1p - \frac1q}} \leq \frac{\mynorm{\boldsymbol{a}}_p}{\mynorm{\boldsymbol{a}}_q} \leq \max \mybrace{1 , d^{\frac1p - \frac1q}}.
\end{gather*}
\end{property}
The proof of Property \ref{prp:VectorDivisionBound} is provided in Section \ref{prf:VectorDivisionBound} (page \pageref{prf:VectorDivisionBound}).
The general bounds of Property \ref{prp:VectorDivisionBound} include the following special cases $\forall \boldsymbol{a} \ssin \mathbb{R}^d$ \cite{Golub2013}:
\begin{gather*}
\begin{aligned}
1 &\leq \frac{\mynorm{\boldsymbol{a}}_1}{\mynorm{\boldsymbol{a}}_2} \leq \sqrt{d}, \\
1 &\leq \frac{\mynorm{\boldsymbol{a}}_2}{\mynorm{\boldsymbol{a}}_{\infty}} \leq \sqrt{d}, \\
1 &\leq \frac{\mynorm{\boldsymbol{a}}_1}{\mynorm{\boldsymbol{a}}_{\infty}} \leq d.
\end{aligned}
\end{gather*}
\begin{property}[Bounds of two (pseudo-)mixed-norms division]
\label{lm:FractionBound} %$0 \sless q \sless p$, 
\myhl{$\forall (q , p) \ssin \mathbb{R}^2_{>0}$,} for $\boldsymbol{a} \seq \mybracket{\boldsymbol{a}\mybracket{1},\ldots,\boldsymbol{a}\mybracket{K}}$ and $\forall k ,\, \boldsymbol{a}\mybracket{k} \ssin \mathbb{R}^{d_k}$, and for weighted (pseudo-)mixed-norms we have: % \cite{Golub2013}
\begin{gather*}
\begin{aligned}
&\frac{\mynorm{\boldsymbol{a}}_{\boldsymbol{w};p,1}}{\mynorm{\boldsymbol{a}}_{\boldsymbol{w};q,1}} \geq
\min_k \, \min \mybrace{1 , d_k^{\frac1q - \frac1p}} = 
\begin{cases}
\begin{aligned}
  &1, \quad &&\text{if } 0 < q \leq p \\
  &d_{max}^{\frac1q - \frac1p}, \quad &&\text{if } 0 < p < q
\end{aligned}
\end{cases}, \\
&\frac{\mynorm{\boldsymbol{a}}_{\boldsymbol{w};p,1}}{\mynorm{\boldsymbol{a}}_{\boldsymbol{w};q,1}} \leq
\max_k \, \max \mybrace{1 , d_k^{\frac1q - \frac1p}} = 
\begin{cases}
\begin{aligned}
  &d_{max}^{\frac1q - \frac1p}, \quad &&\text{if } 0 < q \leq p \\
  &1, \quad &&\text{if } 0 < p < q
\end{aligned}
\end{cases}.
\end{aligned}
\end{gather*}
Similarly, for non-weighted (pseudo-)mixed-norms we have: % \cite{Golub2013}
\begin{gather*}
\begin{aligned}
&\frac{\mynorm{\boldsymbol{a}}_{p,1}}{\mynorm{\boldsymbol{a}}_{q,1}} \geq
\min_k \, \min \mybrace{1 , d_k^{\frac1p - \frac1q}} = 
\begin{cases}
\begin{aligned}
  &d_{max}^{\frac1p - \frac1q}, \quad &&\text{if } 0 < q \leq p \\
  &1, \quad &&\text{if } 0 < p < q
\end{aligned}
\end{cases}, \\
&\frac{\mynorm{\boldsymbol{a}}_{p,1}}{\mynorm{\boldsymbol{a}}_{q,1}} \leq
\max_k \, \max \mybrace{1 , d_k^{\frac1p - \frac1q}} = 
\begin{cases}
\begin{aligned}
  &1, \quad &&\text{if } 0 < q \leq p \\
  &d_{max}^{\frac1p - \frac1q}, \quad &&\text{if } 0 < p < q
\end{aligned}
\end{cases}.
\end{aligned}
\end{gather*}
\end{property}
\begin{proof}
For any real fractions $x_1/y_1,\,\ldots,\,x_K/y_K$ with positive denominators, we have \cite{pahio2005}:
\begin{gather*}
\min\mybrace{\frac{x_1}{y_1},\,\ldots,\,\frac{x_K}{y_K}} \leq \frac{x_1+\ldots+x_K}{y_1+\ldots+y_K} \leq \max\mybrace{\frac{x_1}{y_1},\,\ldots,\,\frac{x_K}{y_K}},
\end{gather*}
where, equality happens if and only if all fractions $x_1/y_1,\,\ldots,\,x_K/y_K$ are equal.
Then, for any $k$, considering $x_k \seq \Vert \boldsymbol{a}[k] \Vert_p / d_k^{1/p}$ and $y_k \seq \Vert \boldsymbol{a}[k] \Vert_q / d_k^{1/q}$, we have $x_k / y_k \seq d_k^{1/q-1/p} \Vert \boldsymbol{a}[k] \Vert_p {/} \Vert \boldsymbol{a}[k] \Vert_q$.
On the other hand, from Property \ref{prp:VectorDivisionBound} (bounds of two vector norms division), we have $\forall (q , p) \ssin \mathbb{R}^2_{\sg 0} , \forall \boldsymbol{b} \ssin \mathbb{R}^d : \min \{1 , d^{1/p - 1/q} \} \sleq \Vert \boldsymbol{b} \Vert_p / \Vert \boldsymbol{b} \Vert_q \sleq \max \{1 , d^{1/p - 1/q} \}$. 
%for $0 \sless q \sless p$, and $\boldsymbol{a} \ssin \mathbb{R}^{d}$ we have $1 \sleq \Vert\boldsymbol{a} \Vert_q {/} \Vert \boldsymbol{a} \Vert_p \sleq d^{1/q \sm 1/p}$ \cite{Golub2013}. 
%Then, similarly, we have $d^{1/p \sm 1/q} \sleq \Vert \boldsymbol{a} \Vert_p {/} \Vert \boldsymbol{a} \Vert_q \sleq 1$.
Consequently, $\min \{1 , d_k^{1/q - 1/p} \} \sleq x_k / y_k \sleq \max \{1 , d_k^{1/q - 1/p} \}$, which by considering the above-mentioned inequality, we have:
\begin{gather*}
\min_k \, \min \mybrace{1 , d_k^{\frac1q - \frac1p}} 
\leq \frac{\displaystyle\sum_k x_k}{\displaystyle\sum_k y_k}
\leq \max_k \, \max \mybrace{1 , d_k^{\frac1q - \frac1p}}.
\end{gather*}
On the other hand, from Definition \ref{def:Weighted mixed norm} (weighted (pseudo-)mixed-norm) we have $\sum_k x_k \seq \Vert \boldsymbol{a} \Vert_{\boldsymbol{w};p,1}$ and $\sum_k y_k \seq \Vert \boldsymbol{a} \Vert_{\boldsymbol{w};q,1}$, which proves the bounds.
Similarly, for any $k$, considering $x_k \seq \Vert \boldsymbol{a}[k] \Vert_p$ and $y_k \seq \Vert \boldsymbol{a}[k] \Vert_q$, the bounds for division of two non-weighted (pseudo-)mixed-norms can be obtained.
\end{proof}

\begin{property}[$\ell_{q {\to} p}$ operator-norm properties]
\label{prp:OperatorProperties}
Assuming \myhl{$\boldsymbol{A} \ssin \mathbb{R}^{m \stimes n}$, $\boldsymbol{B} \ssin \mathbb{R}^{m \stimes n}$, and $\boldsymbol{C} \ssin \mathbb{R}^{n \stimes l}$,} the $\ell_{q {\to} p}$ operator-norm of a matrix satisfies the following properties:
%$\ell_{q {\to} p}$ operator-norm of a matrix $\boldsymbol{A}$ is a matrix norm as it satisfies the following properties:
\begin{itemize}
\item Nonnegativity: $\forall (q , p) \ssin \mathbb{R}^2_{\sgeq 0} : \Vert \boldsymbol{A} \Vert_{q {\to} p} \sgeq 0$.
\item Positivity: $\forall (q , p) \ssin \mathbb{R}^2_{\sgeq 0} : \Vert \boldsymbol{A} \Vert_{q {\to} p} \seq 0$ if and only if $\boldsymbol{A} \seq \boldsymbol{0}$.
\item Homogeneity: $\forall q \ssin \mathbb{R}_{\sgeq 0} , \forall p \ssin \mathbb{R}_{\sg 0}$, \myhl{$\forall \alpha \ssin \mathbb{R}$} $: \Vert \alpha \boldsymbol{A} \Vert_{q {\to} p} \seq \myabs{\alpha} \Vert \boldsymbol{A} \Vert_{q {\to} p}$.
\item Triangle inequality: $\forall q \ssin \mathbb{R}_{\sgeq 0} , \forall p \ssin \mathbb{R}_{\sgeq 1}$, for $p \seq 0 : \Vert \boldsymbol{A} \spl \boldsymbol{B} \Vert_{q {\to} p} \sleq \Vert \boldsymbol{A} \Vert_{q {\to} p} \spl \Vert \boldsymbol{B} \Vert_{q {\to} p}$.
\end{itemize}
\begin{remark}[Generalised matrix norm]
\label{rmrk:Generalized matrix norm} 
Any matrix norm definition that satisfies the above four properties is called \emph{generalised matrix norm} \cite{HornR.A.2012}.
All the above-mentioned properties hold true $\forall q \ssin \mathbb{R}_{\sgeq 0}$, in contrast to $p$.
Table \ref{table:OperatorNormProperties} summarises the ranges of $p$ in which different properties hold true.
As it can be seen in table \ref{table:OperatorNormProperties}, for $\forall q \ssin \mathbb{R}_{\sgeq 0}$ and $\forall p \ssin \mathbb{R}_{\sgeq 1}$, the $\ell_{q \to p}$ operator-norm satisfies all the above four properties, hence it is a generalised matrix norm in the mentioned range for $q$ and $p$.
\begin{table*}[bp]
%\begin{adjustbox}{width=\textwidth} % ,totalheight=\textheight,.5
\centering
%\tiny
\begin{tabular}{cccc}
\toprule
$\forall q \ssin \mathbb{R}_{\sgeq 0}$ & \multicolumn{1}{c}{$p=0$}  & \multicolumn{1}{c}{$0 < p < 1$} & \multicolumn{1}{c}{$p \geq 1$} \\ \midrule 
\multicolumn{1}{l}{$\ell_{q \to p}$ properties} & \multicolumn{1}{c}{N, P, T} & \multicolumn{1}{c}{N, P, H} & \multicolumn{1}{c}{N, P, H, T} \\ 
\bottomrule 
\end{tabular}
%\end{adjustbox}
\caption{Properties of $\ell_{q \to p}$ operator-norm for different ranges of $p$, while $\forall q \ssin \mathbb{R}_{\sgeq 0}$, where N, P, H, and T stand for the existence of nonnegativity, positivity, homogeneity, and triangle inequality properties, respectively.}
\label{table:OperatorNormProperties}
\end{table*}
%In the following we continue to investigate the other properties of the $\ell_{q \to p}$ operator-norm.
\end{remark}
\begin{itemize}
\item Submultiplicativity: In general, we have:
\begin{gather*}
\forall (q , p) \in \mathbb{R}^2_{\sg 0}, \qquad \mynorm{\boldsymbol{A} \, \boldsymbol{C}}_{q \to p} \leq \mynorm{\boldsymbol{A}}_{q \to p} \, \mynorm{\boldsymbol{C}}_{q \to p} \max \mybrace{1 , n^{\frac1q - \frac1p}}. 
\end{gather*}
For $q \sgeq p$ we have $\max \{1 , n^{1/q \sm 1/p} \} \seq 1$, and then there exists submultiplicativity property, i.e., for $q \sgeq p \sg 0$, we have $\Vert \boldsymbol{A} \, \boldsymbol{C} \Vert_{q {\to} p} \sleq \Vert \boldsymbol{A} \Vert_{q {\to} p} \, \Vert \boldsymbol{C} \Vert_{q {\to} p}$.
\end{itemize}
\begin{remark}[Matrix norm]
\label{rmrk:Oprtr-nrm-matrx}
Any norm definition that satisfies the above five properties is called \emph{matrix norm} \cite{HornR.A.2012}.
The $\ell_{q \to p}$ operator-norm $\forall q \ssin \mathbb{R}_{\sgeq 0}$ and $\forall p \ssin \mathbb{R}_{\sgeq 1}$ satisfies the first four properties (generalised matrix norm), whereas for $q \sgeq p \sg 0$ satisfies the fifth property.
Then, $\ell_{q \to p}$ operator-norm for $q \sgeq p \sgeq 1$ satisfies all the above five properties, hence it is a matrix norm.
\end{remark}
\begin{itemize}
\item Bounds: We define the following four types of bounds and inequalities for the $\ell_{q {\to} p}$ operator-norm: \\
1) $\forall \myparanthese{q , p , q' , p'} \ssin \mathbb{R}^4_{>0}$, we have:
\begin{gather*}
\begin{aligned}
&\mynorm{\boldsymbol{A}}_{q \to p} \geq \max \mybrace{\min \mybrace{1 , m^{\frac1p - \frac{1}{p'}}} \mynorm{\boldsymbol{A}}_{q \to p'} , \min \mybrace{1 , n^{\frac{1}{q'} - \frac1q}} \mynorm{\boldsymbol{A}}_{q' \to p}}, \\
&\mynorm{\boldsymbol{A}}_{q \to p} \leq \min \mybrace{\max \mybrace{1 , m^{\frac1p - \frac{1}{p'}}} \mynorm{\boldsymbol{A}}_{q \to p'} , \max \mybrace{1 , n^{\frac{1}{q'} - \frac1q}} \mynorm{\boldsymbol{A}}_{q' \to p}},
\end{aligned}
\end{gather*}
which the bounds are based on the operator norm having one of the original domains, either $q$, or $p$.

2) In addition, the bounds can be based on the operator norm having totally new domains, e.g., $q'$ and $p'$, i.e., $\forall \myparanthese{q , p , q' , p'} \ssin \mathbb{R}^4_{>0}$ we have:
\begin{gather*}
\begin{aligned}
&\mynorm{\boldsymbol{A}}_{q \to p} \geq \min \mybrace{1 , m^{\frac1p - \frac{1}{p'}}} \min \mybrace{1 , n^{\frac{1}{q'} - \frac1q}} \mynorm{\boldsymbol{A}}_{q' \to p'}, \\
&\mynorm{\boldsymbol{A}}_{q \to p} \leq \max \mybrace{1 , m^{\frac1p - \frac{1}{p'}}} \max \mybrace{1 , n^{\frac{1}{q'} - \frac1q}} \mynorm{\boldsymbol{A}}_{q' \to p'}.
\end{aligned}
\end{gather*}
\begin{remark}[$\ell_{q {\to} p}$ operator-norm inequalities]
\label{rmrk:operator-norm inequalities} 
The above-mentioned general inequalities for basic tractable $\ell_{q \to p}$ operator-norms based on table \ref{table:OperatorNorm} (page \pageref{table:OperatorNorm}) is shown in table \ref{table:OperatorNormRelations}, which includes the following standard inequalities \cite{Golub2013}:
\begin{gather*}
\begin{aligned}
\mynorm{\boldsymbol{A}}_{1 \to \infty} &\leq \mynorm{\boldsymbol{A}}_{2 \to 2} \leq \sqrt{mn} \mynorm{\boldsymbol{A}}_{1 \to \infty}, \\
\frac{1}{\sqrt{n}}\mynorm{\boldsymbol{A}}_{\infty \to \infty} &\leq \mynorm{\boldsymbol{A}}_{2 \to 2} \leq \sqrt{m} \mynorm{\boldsymbol{A}}_{\infty \to \infty}, \\
\frac{1}{\sqrt{m}}\mynorm{\boldsymbol{A}}_{1 \to 1} &\leq \mynorm{\boldsymbol{A}}_{2 \to 2} \leq \sqrt{n} \mynorm{\boldsymbol{A}}_{1 \to 1}.
\end{aligned}
\end{gather*}
\end{remark}
\begin{table*}[tp]
\begin{adjustbox}{width=\textwidth} % ,totalheight=\textheight,.5
\centering
\begin{tabular}{cccc}
\toprule
$\begin{aligned}
1 &\leq \frac{\mynorm{\boldsymbol{A}}_{1 \to 1}}{\mynorm{\boldsymbol{A}}_{1 \to 2}} \leq m ^{\frac12}, \\
n ^{-1} &\leq \frac{\mynorm{\boldsymbol{A}}_{1 \to 1}}{\mynorm{\boldsymbol{A}}_{\infty \to \infty}} \leq m, \\
n ^{-\frac12} &\leq \frac{\mynorm{\boldsymbol{A}}_{2 \to 2}}{\mynorm{\boldsymbol{A}}_{\infty \to \infty}} \leq m ^{\frac12}, \\
1 &\leq \frac{\mynorm{\boldsymbol{A}}_{1 \to 2}}{\mynorm{\boldsymbol{A}}_{1 \to \infty}} \leq m ^{\frac12}, 
\end{aligned}$ &
$\begin{aligned}
1 &\leq \frac{\mynorm{\boldsymbol{A}}_{1 \to 1}}{\mynorm{\boldsymbol{A}}_{1 \to \infty}} \leq m, \\
1 &\leq \frac{\mynorm{\boldsymbol{A}}_{2 \to 2}}{\mynorm{\boldsymbol{A}}_{1 \to 2}} \leq n ^{\frac12}, \\
m ^{-\frac12} &\leq \frac{\mynorm{\boldsymbol{A}}_{\infty \to \infty}}{\mynorm{\boldsymbol{A}}_{1 \to 2}} \leq n, \\
n ^{-\frac12} &\leq \frac{\mynorm{\boldsymbol{A}}_{1 \to 2}}{\mynorm{\boldsymbol{A}}_{2 \to \infty}} \leq m ^{\frac12},
\end{aligned}$ &
$\begin{aligned}
n ^{-\frac12} &\leq \frac{\mynorm{\boldsymbol{A}}_{1 \to 1}}{\mynorm{\boldsymbol{A}}_{2 \to 2}} \leq m ^{\frac12}, \\
1 &\leq \frac{\mynorm{\boldsymbol{A}}_{2 \to 2}}{\mynorm{\boldsymbol{A}}_{1 \to \infty}} \leq \myparanthese{m \, n} ^{\frac12}, \\
1 &\leq \frac{\mynorm{\boldsymbol{A}}_{\infty \to \infty}}{\mynorm{\boldsymbol{A}}_{1 \to \infty}} \leq n, \\
n ^{-\frac12} &\leq \frac{\mynorm{\boldsymbol{A}}_{1 \to \infty}}{\mynorm{\boldsymbol{A}}_{2 \to \infty}} \leq 1.
\end{aligned}$ &
$\begin{aligned}
n ^{-\frac12} &\leq \frac{\mynorm{\boldsymbol{A}}_{1 \to 1}}{\mynorm{\boldsymbol{A}}_{2 \to \infty}} \leq m, \\
1 &\leq \frac{\mynorm{\boldsymbol{A}}_{2 \to 2}}{\mynorm{\boldsymbol{A}}_{2\to \infty}} \leq m ^{\frac12}, \\
1 &\leq \frac{\mynorm{\boldsymbol{A}}_{\infty \to \infty}}{\mynorm{\boldsymbol{A}}_{2 \to \infty}} \leq n ^{\frac12}, \\
\color{white} n ^{-1} &\color{white}\leq \frac{\mynorm{\boldsymbol{A}}_{1 \to 2}}{\mynorm{\boldsymbol{A}}_{\infty \to \infty}} \color{white}\leq m ^{\frac12}, 
\end{aligned}$
\\ \bottomrule 
\end{tabular}
\end{adjustbox}
\caption{Inequalities of basic tractable $\ell_{q \to p}$ operator-norms based on table \ref{table:OperatorNorm} (page \pageref{table:OperatorNorm}) for a matrix $\boldsymbol{A} \ssin \mathbb{R}^{m \stimes n}$.} % , and $q,p \ssin \{1 , 2 , \infty \}$ corresponding to the tractable 
\label{table:OperatorNormRelations}
\end{table*}
\begin{figure}[!b]
\centering
\includegraphics[width=.4\textwidth,keepaspectratio]{images/OperatorNorm-Inequalities.png} 
\centering
\caption{$\ell_{q {\to} p}$ operator-norm inequalities for common $\ell_{q {\to} p}$ operator-norms according to table \ref{table:OperatorNorm} (page \pageref{table:OperatorNorm}).}
\label{fig:OperatorNorm-Inequalities}
\end{figure}
In addition, in general (tractable and intractable), the $\ell_{q {\to} p}$ operator-norm inequalities for a fixed $p$ or $q$ is shown schematically in figure \ref{fig:OperatorNorm-Inequalities}.

3) Another useful lower- and upper-bound $\forall \myparanthese{q , p , q' , p'} \ssin \mathbb{R}^4_{>0}$ are:
\begin{gather*}
\begin{aligned}
&\mynorm{\boldsymbol{A}}_{q \to p} \geq \frac{\min \mybrace{1 , m^{\frac1p - \frac{1}{p'}}} \min \mybrace{1 , n^{\frac{1}{q'} - \frac1q}} \min \mybrace{1 , m^{\frac{1}{p'} -\frac12}} \min \mybrace{1 , n^{\frac{1}{2} - \frac{1}{q'}}}}{\sqrt{\min \mybrace{m , n}}} \mynorm{\boldsymbol{A}}_F, \\
&\mynorm{\boldsymbol{A}}_{q \to p} \leq \max \mybrace{1 , m^{\frac1p - \frac{1}{p'}}} \max \mybrace{1 , n^{\frac{1}{q'} - \frac1q}} \max \mybrace{1 , m^{\frac{1}{p'} - \frac12}} \max \mybrace{1 , n^{\frac12 - \frac{1}{q'}}} \mynorm{\boldsymbol{A}}_F,
\end{aligned}
\end{gather*}
where, the Frobenius norm is defined as $\Vert \boldsymbol{A} \Vert_F \seq \sqrt{\sum_{i=1}^m \sum_{j=1}^n \vert a_{i,j} \vert^2}$.

\myhl{4) In addition, the $\ell_{q {\to} p}$ operator-norm of any matrix $\boldsymbol{A}$ is less than or equal to the $\ell_{q {\to} p}$ operator-norm of another matrix $\boldsymbol{B}$, in which all the elements are the maximum absolute value of the elements of first matrix.
It also holds true, when all the on-diagonal entries of $\boldsymbol{A}$ and $\boldsymbol{B}$ are set to zero:}
%to compare the $\ell_{q {\to} p}$ operator-norm of two matrices $\boldsymbol{A}$ and $\boldsymbol{B}$ with the same size, where 
%\newpage
\begin{gather*}
\mycolor{\forall i , j, \forall \myparanthese{q , p} \in \mathbb{R}^2_{>0}, \qquad \textrm{if } \myabs{a_{i,j}} \leq b_{i,j} = \max_{i,j} \myabs{a_{i,j}} \Rightarrow \mynorm{\boldsymbol{A}}_{q \to p} \leq \mynorm{\boldsymbol{B}}_{q \to p},}
\end{gather*}
\myhl{and}
\begin{gather*}
\mycolor{\forall i , j, \forall \myparanthese{q , p} \in \mathbb{R}^2_{>0}, \qquad \textrm{if } 
\begin{cases}
\begin{aligned}
  &\myabs{a_{i,j}} \leq b_{i,j} = \max_{i,j} \myabs{a_{i,j}}, \quad &&i \neq j \\
  &a_{i,j} = b_{i,j} = 0, \quad &&i = j
\end{aligned}
\end{cases}
\Rightarrow \mynorm{\boldsymbol{A}}_{q \to p} \leq \mynorm{\boldsymbol{B}}_{q \to p}.}
\end{gather*}
%The above inequality also holds true, when all the on-diagonal entries of $\boldsymbol{A}$ and $\boldsymbol{B}$ are set to zero.
\end{itemize}
\end{property}
See Section \ref{prf:OperatorProperties} (page \pageref{prf:OperatorProperties}) for the proof of Property \ref{prp:OperatorProperties}.

Now, based on the above-mentioned properties of the $\ell_{q {\to} p}$ operator-norm introduced in Property \ref{prp:OperatorProperties} ($\ell_{q {\to} p}$ operator-norm properties), in the following property we investigate the possible relationship between different Block-MCC$_{q,p}$ characterisations (Definition \ref{def:BMIC}, page \pageref{def:BMIC}) with basic ($q,p$) pairs according to table \ref{table:OperatorNorm} (page \pageref{table:OperatorNorm}):
\begin{property}[Block-MCC$_{q,p}$ inequalities]
\label{prp:BMCC-relationships}
The different Block-MCC$_{q,p}$ characterisations calculated for basic tractable $\ell_{q {\to} p}$ operator-norms of table \ref{table:OperatorNorm} (page \pageref{table:OperatorNorm}) have the following relationships:
\begin{equation*}
\begin{aligned}
M_{1,1}\myparanthese{\myPhi} &\leq M_{1,2}\myparanthese{\myPhi} \leq M_{1,\infty}\myparanthese{\myPhi}, \\
M_{2,2}\myparanthese{\myPhi} &\leq M_{1,2}\myparanthese{\myPhi} \leq M_{1,\infty}\myparanthese{\myPhi}, \\
M_{2,2}\myparanthese{\myPhi} &\leq M_{2,\infty}\myparanthese{\myPhi} \leq M_{1,\infty}\myparanthese{\myPhi}, \\
M_{\infty,\infty}\myparanthese{\myPhi} &\leq M_{2,\infty}\myparanthese{\myPhi} \leq M_{1,\infty}\myparanthese{\myPhi},
\end{aligned}
\end{equation*}
while their general relationship is represented schematically in figure \ref{fig:BMCC_Inequalities}. 
\begin{figure}[!t]
\centering
\includegraphics[width=.4\textwidth,keepaspectratio]{images/BMCC_Inequalities.png} 
\centering
\caption{Block-MCC$_{q,p}$ inequalities for common $\ell_{q {\to} p}$ operator-norms according to table \ref{table:OperatorNorm} (page \pageref{table:OperatorNorm}).}
\label{fig:BMCC_Inequalities}
\end{figure}
%\FloatBarrier
\end{property}
The proof of Property \ref{prp:BMCC-relationships} is provided in Section \ref{prf:BMCC-relationships} (page \pageref{prf:BMCC-relationships}).

Since the recovery conditions started with orthonormal bases, the following property shows Definition \ref{def:BMIC} (block coherence) in special case of intra-block orthonormality:
\begin{property}[Block-MCC$_{q,p}$ for intra-block orthonormality]
\label{prp:IntraBlkO}
If the dictionary $\myPhi$ has intra-block orthonormality, i.e., for $1 \sleq k \sleq K$, \myhl{$\myPhi^T[k] \myPhi[k] \seq \boldsymbol{I}_{d_k}$}, then 
%$\forall q \sg 0$ and $\forall p \sgeq 1$ 
$\forall (q , p) \ssin \mathbb{R}^2_{>0}$ we have:
\begin{equation*}
M_{q,p}\myparanthese{\myPhi} = \max_{k,k' \neq k} \frac{d_{k}^{-\frac1p} \, d_{k'}^{\frac1q}}{d_{max}} \mynorm{\myPhi^T \mybracket{k} \myPhi \mybracket{k'}}_{q \to p}.
\end{equation*}
\end{property}
\begin{proof}
Taking into account that, for an orthonormal matrix we have $\boldsymbol{A}^\dagger \seq \boldsymbol{A}^T$, the result is immediately obtained from the Definition \ref{def:BMIC} (block coherence).
\end{proof}
Block-MCC$_{q,p}$ is a measure of similarity between different blocks of the columns of the dictionary, which is the block-wise extension of the conventional MCC (page \pageref{eq:MIC}).
Since the dictionary coherence has an inverse effect on the theoretical recovery conditions, the lower bounds for coherence is desired.
When a dictionary has intra-block orthonormality, Block-MCC$_{q,p}$ can be bounded, as shown in the following property:
\begin{property}[Block-MCC$_{q,p}$ upper-bound with intra-block orthonormality I]
\label{prp:BMIC-orth-bound} 
For a dictionary $\myPhi$ with intra-block orthonormality, i.e., for $1 \sleq k \sleq K$, \myhl{$\myPhi^T[k] \myPhi[k] \seq \boldsymbol{I}_{d_k}$}, the upper-bound of Block-MCC$_{q,p}$ is shown in table \ref{table:BMIC-orth-bound}. 
\begin{table*}[bp]
%\begin{adjustbox}{width=1\textwidth} % ,totalheight=\textheight,.5
\centering
%\tiny
\begin{tabular}{ccccccc}
\toprule
%\cline{2-4}
\multicolumn{1}{c}{$(q,p)$} &\multicolumn{1}{c}{$(1,1)$} & \multicolumn{1}{c}{$(1,2)$}  & \multicolumn{1}{c}{$(1,\infty)$} & \multicolumn{1}{c}{$(2,2)$} & \multicolumn{1}{c}{$(2,\infty)$} & \multicolumn{1}{c}{$(\infty,\infty)$}\\ \midrule %\hline
\multicolumn{1}{l}{$\max M_{q,p}(\myPhi)$} &\multicolumn{1}{l}{\myhl{$d_{min}^{-\frac12} \, m^{\frac12}$}} & \multicolumn{1}{c}{\myhl{$d_{min}^{-\frac12} \, m^{\frac12}$}} & \multicolumn{1}{c}{\myhl{$m$}} &\multicolumn{1}{c}{$d_{min}^{-\frac12} \, d_{max}^{-\frac12}$}  &\multicolumn{1}{c}{\myhl{$d_{max}^{-\frac12} \, m^{\frac12}$}}  &\multicolumn{1}{c}{\myhl{$d_{max}^{-\frac12} \, m^{\frac12}$}} \\ %\hline
\bottomrule
\end{tabular}
%\end{adjustbox}
\caption{Upper-bound of Block-MCC$_{q,p}$ obtained based on the relationship with unit $\ell_{2 {\to} 2}$ operator-norm, for different basic $(q,p)$ pairs and for a dictionary $\myPhi \ssin \mathbb{R}^{m \stimes n}$ with intra-block orthonormality.}
\label{table:BMIC-orth-bound}
\end{table*}
\end{property}
The proof of Property \ref{prp:BMIC-orth-bound} is provided in Section \ref{prf:BMIC-bounds} (page \pageref{prf:BMIC-bounds}).
Later, in Property \ref{prp:BMIC-orth-bound2}, we will show another upper-bounds for Block-MCC$_{q,p}$ based on the relationship with conventional MCC.
Property \ref{prp:BMIC-MIC} investigates the bounds of Block-MCC$_{q,p}$ in terms of the conventional MCC (page \pageref{eq:MIC}).
\begin{property}[Block-MCC$_{q,p}$ bounds]
\label{prp:BMIC-MIC} 
%$\forall q \sgeq 0$ and $\forall p \sgeq 1$, 
Block-MCC$_{q,p}$ is bounded based on the conventional MCC, i.e., $M(\myPhi) {\myeq} \max_{k,k' \neq k} | \left\langle \boldsymbol{\varphi}_k , \boldsymbol{\varphi}_{k'} \right\rangle|$.\\
1) Suppose for a dictionary \myhl{with full column rank blocks,} we have $M(\myPhi) \sless d_{max}^{1/q \sm 2} (d_{max} - 1)^{-1/2}$ and $q \sgeq p \sgeq 1$, then:
%1) For a general dictionary $\myPhi$, suppose $M(\myPhi) < 1/(d-1)$. 
%Then the BMIC$_{q,p}$ is bounded based on the MIC:
\begin{equation*}
0 \leq M_{q,p}\myparanthese{\myPhi} \leq \frac{d_{max}^{\frac32 - \frac1p} \, M\myparanthese{\myPhi}}{1- d_{max}^{2 - \frac1q} \myparanthese{d_{max} - 1}^\frac12 \, M \myparanthese{\myPhi}}.
\end{equation*}
2) For a dictionary with intra-block orthonormality, $\forall (q , p , q' , p') \ssin \mathbb{R}^4_{\sg 0}$ we have:
\begin{gather*}
\begin{aligned}
0 \leq M_{q,p}\myparanthese{\myPhi} \leq 
\frac{M\myparanthese{\myPhi}}{d_{max}} \, \max_{k,k' \neq k} d_{k}^{\frac12 - \frac1p} \, d_{k'}^{\frac1q + \frac12} \, &\max \mybrace{1 , d_k^{\frac1p - \frac{1}{p'}}} \, \max \mybrace{1 , d_{k'}^{\frac{1}{q'} - \frac1q}} \times \\
&\max \mybrace{1 , d_k^{\frac{1}{p'} -\frac12}} \, \max \mybrace{1 , d_{k'}^{\frac{1}{2} - \frac{1}{q'}}}.
%\begin{cases}
%d_{max}^{\frac32 - \frac1p} \, M\myparanthese{\myPhi}, & \qquad \text{for \ }q \geq 1\\
%d_{max}^{\frac1q - \frac1p + \frac12} \, M\myparanthese{\myPhi}, & \qquad \text{for \ }q < 1\\
%\end{cases}.
\end{aligned}
\end{gather*}
\end{property}
%\newpage
The proof of Property \ref{prp:BMIC-MIC} is provided in Section \ref{prf:BMIC-MIC} (page \pageref{prf:BMIC-MIC}).
%Therefore, based on second part of Property \ref{prp:BMIC-MIC}, for $q \sgeq p$ if $d_{min}^{-1/p} \, d_{max}^{1/q} \sleq 1$, and for $p/2 \sless q \sless p$ if $d_{min}^{1/q - 2/p} \, d_{max}^{1/q} \sleq 1$, we have $M_{q,p}(\myPhi) \sleq M(\myPhi)$.
In addition, we have:
%using the second part of Property \ref{prp:BMIC-MIC} and knowing the upper-bound of $M(\myPhi)$, we can again, as in Property \ref{prp:BMIC-orth-bound}, find the upper-bounds of the Block-MCC$_{q,p}$.
\begin{property}[Block-MCC$_{q,p}$ upper-bound with intra-block orthonormality II]
\label{prp:BMIC-orth-bound2} 
For a dictionary with orthonormal blocks, Block-MCC$_{q,p}$ is upper-bounded based on its relationship with MCC, i.e., $\forall (q , p , q' , p') \ssin \mathbb{R}^4_{\sg 0}$, we have: % $d_{min}^{-1/p} d_{max}^{1/q}$.
\begin{gather*}
\begin{aligned}
0 \leq M_{q,p}\myparanthese{\myPhi} \leq 
d_{max}^{-1} \, \max_{k,k' \neq k} d_{k}^{\frac12 - \frac1p} \, d_{k'}^{\frac1q + \frac12} \, &\max \mybrace{1 , d_k^{\frac1p - \frac{1}{p'}}} \, \max \mybrace{1 , d_{k'}^{\frac{1}{q'} - \frac1q}} \times \\
&\max \mybrace{1 , d_k^{\frac{1}{p'} -\frac12}} \, \max \mybrace{1 , d_{k'}^{\frac{1}{2} - \frac{1}{q'}}}.
%\begin{cases}
%d_{max}^{\frac32 - \frac1p}, & \qquad \text{for \ }q \geq 1\\
%d_{max}^{\frac1q - \frac1p + \frac12}, & \qquad \text{for \ }q < 1\\
%\end{cases}.
\end{aligned}
\end{gather*}
The mentioned upper-bound of Block-MCC$_{q,p}$ for basic tractable $(q,p)$ pairs is shown in table \ref{table:BMIC-orth-upper-bound}.
\begin{table*}[tp]
%\begin{adjustbox}{width=1\textwidth} % ,totalheight=\textheight,.5
\centering
%\tiny
%\resizebox{\textwidth}{!}{
\begin{tabular}{ccccccc}
\toprule
%\cline{2-4}
\multicolumn{1}{c}{$(q = q',p = p')$} &\multicolumn{1}{c}{$(1,1)$} & \multicolumn{1}{c}{$(1,2)$}  & \multicolumn{1}{c}{$(1,\infty)$} & \multicolumn{1}{c}{$(2,2)$} & \multicolumn{1}{c}{$(2,\infty)$} & \multicolumn{1}{c}{$(\infty,\infty)$}\\ \midrule %\hline
\multicolumn{1}{c}{$\max M_{q,p}(\myPhi)$} &\multicolumn{1}{l}{$d_{max}^\frac12$} & \multicolumn{1}{c}{$d_{max}^\frac12$} & \multicolumn{1}{c}{$d_{max}$} &\multicolumn{1}{c}{$1$} &\multicolumn{1}{c}{$d_{max}^\frac12$} & \multicolumn{1}{c}{$d_{max}^\frac12$}   \\ \bottomrule %\hline
\end{tabular}
%}
%\end{adjustbox}
\caption{Upper-bound of Block-MCC$_{q,p}$ obtained based on the relationship with MCC, for different basic values of $(q,p)$ pairs and for a dictionary with intra-block orthonormality.}
\label{table:BMIC-orth-upper-bound}
\end{table*}
\end{property}
\begin{proof}
Since MCC is upper-bounded by one, from Property \ref{prp:BMIC-MIC} the proof is done.
\end{proof}

%\FloatBarrier
{
\label{cmmnt:28} 
Notice that in Property \ref{prp:BMIC-orth-bound} we introduced the bounds of Block-MCC$_{q,p}$ based on the bounds of the operator-norms in terms of the unit $\ell_{2 {\to} 2}$ operator-norm, whereas now in Property \ref{prp:BMIC-orth-bound2} we introduced another upper-bound based on its relationship with MCC and considering the upper-bound of MCC.
\myhl{By minimising upper-bounds of Block-MCC$_{q,p}$ in properties {\ref{prp:BMIC-orth-bound}} (table {\ref{table:BMIC-orth-bound}}) and {\ref{prp:BMIC-orth-bound2}} (table {\ref{table:BMIC-orth-upper-bound}}), the final resulted upper-bounds are shown in table {\ref{table:BMIC-orth-bound-min}}.}
}
%we see that all the smallest values belong to table \ref{table:BMIC-orth-bound}, therefore, we select table \ref{table:BMIC-orth-bound} as the upper-bound of Block-MCC$_{q,p}$.

%By selecting the smallest upper-bounds of Block-MCC$_{q,p}$ in properties \ref{prp:BMIC-orth-bound} (table \ref{table:BMIC-orth-bound}) and \ref{prp:BMIC-orth-bound2} (table \ref{table:BMIC-orth-upper-bound}), the final upper-bounds are shown in table \ref{table:BMIC-orth-bound3}.
%As it can be seen in table \ref{table:BMIC-orth-bound3}, the maximum Block-MCC$_{2,2}$ and Block-MCC$_{\infty,\infty}$ are resulted from Property \ref{prp:BMIC-orth-bound}, and \ref{prp:BMIC-orth-bound2}, respectively, whereas for other $(q,p)$ pairs the minimum value between results of the mentioned two properties should be calculated.
%Property \ref{prp:BMIC-orth-bound2} only for Block-MCC$_{1,1}$ and Block-MCC$_{\infty,\infty}$ affects the upper-bound of Block-MCC$_{q,p}$ in Property \ref{prp:BMIC-orth-bound}.

\begin{table*}[tp]
\begin{adjustbox}{width=1\textwidth} % ,totalheight=\textheight,.5
\centering
%\tiny
\begin{tabular}{ccccc}
\toprule
%\cline{2-4}
\multicolumn{1}{c}{$(q,p)$} &\multicolumn{1}{c}{$(1,1)$ and $(1,2)$} & \multicolumn{1}{c}{$(1,\infty)$} & \multicolumn{1}{c}{$(2,2)$} & \multicolumn{1}{c}{$(2,\infty)$ and $(\infty,\infty)$} \\ \midrule %\hline
\multicolumn{1}{l}{$\max M_{q,p}(\myPhi)$} &\multicolumn{1}{l}{$\min \mybrace{d_{min}^{-\frac12} \, m^{\frac12} , d_{max}^{\frac12}}$} & \multicolumn{1}{c}{$\min \mybrace{m , d_{max}}$} &\multicolumn{1}{c}{$d_{min}^{-\frac12} \, d_{max}^{-\frac12}$}  &\multicolumn{1}{c}{$\min \mybrace{d_{max}^{-\frac12} \, m^{\frac12} , d_{max}^{\frac12}}$} \\ %\hline
\bottomrule
\end{tabular}
\end{adjustbox}
\caption{Upper-bound of Block-MCC$_{q,p}$ obtained based on properties \ref{prp:BMIC-orth-bound} (table \ref{table:BMIC-orth-bound}) and \ref{prp:BMIC-orth-bound2} (table \ref{table:BMIC-orth-upper-bound}), for different basic values of $(q,p)$ pairs and for a dictionary $\myPhi \ssin \mathbb{R}^{m \stimes n}$ with intra-block orthonormality.}
\label{table:BMIC-orth-bound-min}
\end{table*}

\begin{remark}
\label{cmmnt:26} 
By comparing the bounds of conventional MCC, i.e., $1/\sqrt{m} \sleq M(\myPhi) \sleq 1$ (equation (\ref{eq:M-bounds}), page \pageref{eq:M-bounds}), with the upper-bounds of the proposed dictionary coherence in table \ref{table:BMIC-orth-bound-min}, where, there is intra-block orthonormality and considering $d_{min} \seq d_{max} \seq m$ (orthonormal complete bases), it can be seen that, \myhl{although the obtained upper-bound for Block-MCC$_{1,\infty}$ is greater than that of conventional MCC ($m$ v.s. $1$), but both characterisations have the same upper-bound for ($q,p$) pairs of ($1,1$), ($1,2$), ($2,\infty$), and ($\infty,\infty$), while the upper-bound for ($2,2$) is $1/m$, which is much less than the conventional unit upper-bound, and even less than the lower-bound of conventional MCC ($1/m$ v.s. $1/\sqrt{m}$).}
%all the upper bounds are significantly less than to the upper-bound in (\ref{eq:M-bounds}).
\end{remark}
%, whereas the upper bounds for $(2,2)$ is less than, for $(1,1)$, $(1,2)$, $(2,\infty)$, and $(\infty,\infty)$ are equal and finally for $(1,\infty)$ is higher than to the upper bound in (\ref{eq:M-bounds}).

If MCC is small enough, the dictionary is said to be \emph{incoherent}, similarly if Block-MCC$_{q,p}$ is small enough, the dictionary will be \emph{block-incoherent}.
The term block-incoherent is previously used in some studies, of course with a different definition for the characterisation of the coherence of the dictionary which is called \emph{block-coherence} \cite{Peotta2007,Eldar2009b,Eldar2010b}. % ,Eldar2010

Next, we demonstrate the relationship between the proposed general Block-MCC$_{q,p}$ and the conventional characterisation of Donoho and Huo, i.e., \myhl{$\tilde{M}(\myPhiOne,\myPhiTwo) {\myeq} \max\{ \max_{i,j} |\myPhiOne^{-1} \myPhiTwo|_{i,j} , \max_{i,j} |\myPhiTwo^{-1} \myPhiOne|_{i,j}\}$ explained on page {\pageref{Def:M-tilda}}} \cite{Donoho2001}, in sparsity, and block-coherence of Eldar et al., i.e., \myhl{$M^{Eldar}_{Inter}(\myPhi) {\myeq} \max_{k,k' \neq k} \Vert\myPhi ^T [k] \myPhi [k']\Vert_{2 \to 2}/d$ explained on page {\pageref{eq:Eldar-Block-coherence}} equation ({\ref{eq:Eldar-Block-coherence}})} \cite{Eldar2010}, in block-sparsity domain.
%show the relationship between Eldar's block-coherence in \cite{Eldar2010}, and our proposed Block-MCC$_{q,p}$:
\begin{property}[Block-MCC$_{q,p}$ v.s. coherence of Donoho and Huo]
\label{prp:BMIC-MDonoho} 
For a dictionary $\myPhi \seq [\myPhiOne,\myPhiTwo]$ with two \myhl{invertible} square blocks of same size, $d_1 \seq d_2 \seq m$, we have:
\begin{equation*}
M_{1,\infty}\myparanthese{\myPhi} = \tilde{M}\myparanthese{\myPhiOne,\myPhiTwo},
\end{equation*}
and
\begin{equation*}
\mybrace{M_{1,1}\myparanthese{\myPhi} , M_{1,2}\myparanthese{\myPhi} , M_{2,2}\myparanthese{\myPhi} , M_{2,\infty}\myparanthese{\myPhi} , M_{\infty,\infty}\myparanthese{\myPhi}} \leq \tilde{M}\myparanthese{\myPhiOne,\myPhiTwo},
\end{equation*}
where, $\tilde{M}(\myPhiOne,\myPhiTwo)$ is the dictionary characterisation of Donoho and Huo, i.e., $\tilde{M}(\myPhiOne,\myPhiTwo) {\myeq} \max\{ \max_{i,j} |\myPhiOne^{-1} \myPhiTwo|_{i,j} , \max_{i,j} |\myPhiTwo^{-1} \myPhiOne|_{i,j}\}$, and $M_{q,p}(\myPhi)$ is the proposed Block-MCC$_{q,p}$, i.e., $M_{q,p}(\myPhi) {\myeq} d_{max}^{-1} \max_{k,k' \neq k} d_{k}^{-1/p} d_{k'}^{1/q} \Vert\myPhi^\dagger [k] \myPhi [k'] \Vert_{q \to p}$.
\end{property}
\begin{proof}
\myhl{Considering that $\ell_{1 {\to}\infty}$ operator-norm represents the maximum absolute value of matrix, i.e., $\Vert \boldsymbol{A} \Vert_{1 {\to}\infty} \seq \max_{i,j} | a_{i,j} |$ (table {\ref{table:OperatorNorm}}, page {\pageref{table:OperatorNorm}}), and the fact that here $\myPhi[k] \seq \myPhi_{\boldsymbol{k}}$, the definition $\tilde{M}(\myPhiOne,\myPhiTwo) {\myeq} \max\{ \max_{i,j} |\myPhiOne^{-1} \myPhiTwo|_{i,j} , \max_{i,j} |\myPhiTwo^{-1} \myPhiOne|_{i,j}\}$ (page {\pageref{Def:M-tilda}}) can be rewritten as $\tilde{M}(\myPhiOne,\myPhiTwo) \seq \max_{k,k' \neq k} \Vert \myPhi^{-1}[k] \myPhi[k'] \Vert_{1 {\to}\infty}$.
On the other hand for the invertible block $\myPhi[k]$, we have $\myPhi^{-1}[k] \seq \myPhi^{\dagger}[k]$, so $\tilde{M}(\myPhiOne,\myPhiTwo) \seq \max_{k,k' \neq k} \Vert \myPhi^{\dagger}[k] \myPhi[k'] \Vert_{1 {\to}\infty} $.
On the other hand, $M_{1,\infty}(\myPhi) {\seq} d_{max}^{-1} \max_{k,k' \neq k} d_{k'} \Vert\myPhi^\dagger [k] \myPhi [k'] \Vert_{1 \to \infty} \seq \max_{k,k' \neq k} \Vert\myPhi^\dagger [k] \myPhi [k'] \Vert_{1 \to \infty}$, because $\forall k, d_k \seq d_{max}$ ($d_1 \seq d_2 \seq m$).
% \sleq d_{max}^{-1} \max_{k,k' \neq k} d_{k'} \max_{k,k' \neq k} \Vert\myPhi^\dagger [k] \myPhi [k'] \Vert_{1 \to \infty}
Then, $M_{1,\infty}(\myPhi) \seq \tilde{M}(\myPhiOne,\myPhiTwo)$.
In addition from Property {\ref{prp:BMCC-relationships}} (Block-MCC$_{q,p}$ inequalities), we know that $M_{1,\infty}(\myPhi)$ upper-bounds the other basic tractable ($q,p$) pairs of table {\ref{table:OperatorNorm}} (page {\pageref{table:OperatorNorm}}), hence, the proof for the second part is done.} 
\end{proof}
\begin{property}[Block-MCC$_{q,p}$ v.s. block-coherence of Eldar et al.]
\label{prp:BMIC-MEldar} 
For a dictionary with equally-sized blocks of length $d$ and intra-block orthonormality, $\forall (q , p) \ssin \mathbb{R}^2_{\sg 0}$ we have:
\begin{gather*}
\begin{aligned}
&M_{q,p}(\myPhi) \geq d^{\frac1q - \frac1p} \min \mybrace{1 , d^{\frac1p - \frac{1}{2}}} \min \mybrace{1 , d^{\frac{1}{2} - \frac1q}} M^{Eldar}_{Inter}\myparanthese{\myPhi}, \\
&M_{q,p}(\myPhi) \leq d^{\frac1q - \frac1p} \max \mybrace{1 , d^{\frac1p - \frac{1}{2}}} \max \mybrace{1 , d^{\frac{1}{2} - \frac1q}} M^{Eldar}_{Inter}\myparanthese{\myPhi},
\end{aligned}
\end{gather*}
where, \myhl{$M^{Eldar}_{Inter} (\myPhi)$ is the block-coherence of Eldar et al., i.e., $M^{Eldar}_{Inter}(\myPhi) {\myeq} \max_{k,k' \neq k} \Vert\myPhi ^T [k] \myPhi [k']\Vert_{2 \to 2}/d$,} 
%presented on page \pageref{eq:Eldar-Block-coherence} equation (\ref{eq:Eldar-Block-coherence}), 
and $M_{q,p}(\myPhi)$ is the proposed Block-MCC$_{q,p}$, i.e., $M_{q,p}(\myPhi) {\myeq} d_{max}^{-1} \max_{k,k' \neq k} d_{k}^{-1/p} d_{k'}^{1/q} \Vert\myPhi^\dagger [k] \myPhi [k'] \Vert_{q \to p}$.
%, defined on page \pageref{def:BMIC}, Definition \ref{def:BMIC}.
Then, for $q \seq p \seq 2$, we have $M_{2,2}(\myPhi) \seq M^{Eldar}_{Inter}(\myPhi)$.
%, both having the same upper-bound of one (table \ref{table:BMIC-orth-upper-bound}).
\end{property}
\begin{proof}
From Property \ref{prp:IntraBlkO} (Block-MCC$_{q,p}$ for intra-block orthonormality; page \pageref{prp:IntraBlkO}) in a special setting of $d_1 \seq \cdots \seq d_K \seq d$, we have $\forall (q , p) \ssin \mathbb{R}^2_{\sg 0} : M_{q,p}(\myPhi) {\seq} d^{1/q \sm 1/p \sm 1} \max_{k,k' \neq k} \Vert \myPhi^T [k] \myPhi [k'] \Vert_{q \to p}$.
\iffalse
\begin{equation*}
M_{q,p}(\myPhi) = d^{\frac1q - \frac1p -1} \max_{k,k' \neq k} \mynorm{\myPhi^T \mybracket{k} \myPhi \mybracket{k'}}_{q \to p}.
\end{equation*}
\fi
Then, from Property \ref{prp:OperatorProperties} ($\ell_{q {\to} p}$ operator-norm properties; second set of bounds), we can determine the bounds of $\ell_{q {\to} p}$ operator-norm based on $\ell_{q' {\to} p'}$ operator-norm, i.e., $\forall \boldsymbol{A} \ssin \mathbb{R}^{m \stimes n}, \forall \myparanthese{q , p , q' , p'} \ssin \mathbb{R}^4_{>0} : \Vert \boldsymbol{A} \Vert_{q \to p} \sgeq (\sleq) \min (\max) \{1 , m^{1/p \sm 1/{p'}} \} \min (\max) \{1 , n^{1/{q'} \sm 1/q} \} \Vert \boldsymbol{A} \Vert_{q' \to p'}$, and by selecting $q' \seq p' \seq 2$, we have:
\begin{gather*}
\begin{aligned}
&M_{q,p}(\myPhi) \geq d^{\frac1q - \frac1p -1} \min \mybrace{1 , d^{\frac1p - \frac{1}{2}}} \min \mybrace{1 , d^{\frac{1}{2} - \frac1q}} \max_{k,k' \neq k} \mynorm{\myPhi^T \mybracket{k} \myPhi \mybracket{k'}}_{2 \to 2}, \\
&M_{q,p}(\myPhi) \leq d^{\frac1q - \frac1p -1} \max \mybrace{1 , d^{\frac1p - \frac{1}{2}}} \max \mybrace{1 , d^{\frac{1}{2} - \frac1q}} \max_{k,k' \neq k} \mynorm{\myPhi^T \mybracket{k} \myPhi \mybracket{k'}}_{2 \to 2}.
\end{aligned}
\end{gather*}
Next, considering the definition $M^{Eldar}_{Inter}(\myPhi) {\myeq} \max_{k,k' \neq k} \Vert\myPhi ^T [k] \myPhi [k']\Vert_{2 \to 2}/d$ presented on page \pageref{eq:Eldar-Block-coherence} equation (\ref{eq:Eldar-Block-coherence}), the proof is done.
\end{proof}
The above introduced coherence constant Block-MCC$_{q,p}$ characterises the dictionary by means of the maximum coherence, but with the expense of heavier computations it is possible to better characterise it using a cumulative coherence.
\begin{Mydefinition}[Cumulative Block-MCC$_{q,p}$]
\label{def:CBMIC}
\leftbar
The \emph{cumulative Block-MCC$_{q,p}$} of a dictionary \myhl{with full column rank blocks,} is defined for all integers $1 \sleq k \sleq K \sm 1$ and $\forall (q , p) \ssin \mathbb{R}^2_{\sg 0}$ as:
\begin{equation*}
%\label{eq:CBMIC}
M_{q,p}\myparanthese{\myPhi , k} \myeq
\max_{\myabs{\Lambda}=k} \max_{j \notin \Lambda} \sum_{i \in \Lambda}
\frac{d_{i}^{-\frac1p} \, d_{j}^{\frac1q}}{d_{max}} \mynorm{\myPhi^\dagger\mybracket{i} \myPhi \mybracket{j}}_{q \to p},
\end{equation*}
where, $\Lambda \ssubset \{1, \cdots, K\}$ of $\myCardTxt$ $k$.
\endleftbar
\end{Mydefinition}

Cumulative Block-MCC$_{q,p}$ measures the maximum total Block-MCC$_{q,p}$ between a fixed block and a collection of other blocks, which is the block-wise generalisation of the conventional cumulative MCC defined in (\ref{eq:CMIC}), i.e., $M(\myPhi , k) {\myeq} \max_{\myabs{\Lambda}=k} \max_{j \notin \Lambda} \sum_{i \in \Lambda} | \left\langle \boldsymbol{\varphi}_i , \boldsymbol{\varphi}_{j} \right\rangle|$.

As expected, for a unit block size scenario, i.e., $d_1 \seq \cdots \seq d_K \seq 1$, cumulative Block-MCC$_{q,p}$ in Definition \ref{def:CBMIC} is equivalent to the conventional cumulative MCC in (\ref{eq:CMIC}) presented on page \pageref{eq:CMIC}.
The following property shows the cumulative Block-MCC$_{q,p}$ properties, which are similar to the cumulative MCC's one presented on page \pageref{txt:CMCC-properties}:
\begin{property}[Cumulative Block-MCC$_{q,p}$ properties]
\label{prp:CBMIC-BMIC}
For any dictionary $\myPhi$ \myhl{with full column rank blocks,} and Block-MCC$_{q,p}$ $M_{q,p}(\myPhi)$, $\forall (q , p) \ssin \mathbb{R}^2_{\sg 0}$ we have:
\begin{equation*}
M_{q,p}\myparanthese{\myPhi , 1} = M_{q,p}\myparanthese{\myPhi},
\end{equation*}
and
\begin{equation*}
M_{q,p}\myparanthese{\myPhi , k} \leq k \, M_{q,p}\myparanthese{\myPhi}.
\end{equation*}
\end{property}
\begin{proof}
An investigation of the formula of cumulative Block-MCC$_{q,p}$, shows that $\forall (q , p) \ssin \mathbb{R}^2_{\sg 0}$ and for $k \seq 1$, it reduces to Block-MCC$_{q,p}$, i.e.,:
\begin{gather*}
\begin{aligned}
M_{q,p}\myparanthese{\myPhi , 1} 
= \max_{\myabs{\Lambda}=1} \max_{j \notin \Lambda} \sum_{i \in \Lambda} \frac{d_{i}^{-\frac1p} \, d_{j}^{\frac1q}}{d_{max}} \mynorm{\myPhi^\dagger\mybracket{i} \myPhi \mybracket{j}}_{q \to p} 
=\max_{i} \max_{j \neq i} \frac{d_{i}^{-\frac1p} \, d_{j}^{\frac1q}}{d_{max}} \mynorm{\myPhi^\dagger\mybracket{i} \myPhi \mybracket{j}}_{q \to p} 
= M_{q,p}\myparanthese{\myPhi}.
\end{aligned}
\end{gather*} 
To prove the second part, similar to the proof of Proposition 2.1 in \cite{Tropp2004}, $\forall (q , p) \ssin \mathbb{R}^2_{\sg 0}$ we have:
\begin{equation*}
\begin{aligned}
M_{q,p}\myparanthese{\myPhi , k} 
= \max_{\myabs{\Lambda}=k} \max_{j \notin \Lambda} \sum_{i \in \Lambda} \frac{d_{i}^{-\frac1p} \, d_{j}^{\frac1q}}{d_{max}} \mynorm{\myPhi^\dagger\mybracket{i} \myPhi \mybracket{j}}_{q \to p} 
\leq \max_{\myabs{\Lambda}=k} \sum_{i \in \Lambda} M_{q,p}\myparanthese{\myPhi}
= k \, M_{q,p}\myparanthese{\myPhi}.
\end{aligned}
\end{equation*}
\end{proof}
According to the inter-block coherence defined by Eldar et al. in (\ref{eq:Eldar-Block-coherence}) page \pageref{eq:Eldar-Block-coherence}, i.e., $M^{Eldar}_{Inter}(\myPhi) {\myeq} \max_{k,k' \neq k} \Vert\myPhi ^T [k] \myPhi [k']\Vert_{2 \to 2}/d$,  \cite{Eldar2009b,Eldar2010b,Eldar2010}, and the concept of cumulative coherence introduced by Tropp \cite{Tropp2004} (page \pageref{txt:CMCC-properties}), we extend the inter-block coherence and introduce the following definition:
\begin{Mydefinition}[Eldar's cumulative coherence]
\label{def:CIIC}
\leftbar
The cumulative inter-block coherence constant of a general dictionary $\myPhi$ is defined for all integers $1 \sleq k \sleq K \sm 1$ as:
\begin{equation*}
M_{Inter}^{Eldar}\myparanthese{\myPhi , k} \myeq
\max_{\myabs{\Lambda}=k} \max_{j \notin \Lambda} \sum_{i \in \Lambda} \frac{1}{d} \mynorm{\myPhi^T\mybracket{i} \myPhi\mybracket{j}}_{2 \to 2},
\end{equation*}
where, $\Lambda \ssubset \{1, \cdots, K\}$ is of $\myCardTxt$ $k$ and $d_1 \seq \cdots \seq d_K \seq d$.
%$d$ is the size of all equally-sized blocks.% and $d \sgeq 1$ is the block length.
\endleftbar
\end{Mydefinition}

As expected, for $d \seq 1$, cumulative inter-block coherence constant in Definition \ref{def:CIIC} is equivalent to the conventional cumulative MCC defined in (\ref{eq:CMIC}) presented on page \pageref{eq:CMIC}.
The following property shows the properties of cumulative inter-block coherence constant, which are similar to the properties of cumulative MCC presented on page \pageref{txt:CMCC-properties}:
\begin{property}[Eldar's cumulative coherence properties]
\label{prp:CIIC-Eldar}
For any general dictionary $\myPhi$ with equally-sized block structure, i.e., $d_1 \seq \cdots \seq d_K \seq d$, we have:
\begin{equation*}
M_{Inter}^{Eldar}\myparanthese{\myPhi , 1} = M_{Inter}^{Eldar}\myparanthese{\myPhi},
\end{equation*}
and
\begin{equation*}
M_{Inter}^{Eldar}\myparanthese{\myPhi , k} \leq k \, M_{Inter}^{Eldar}\myparanthese{\myPhi}.
\end{equation*}
\end{property}
\begin{proof}
It is similar to the proof of Property \ref{prp:CBMIC-BMIC} (cumulative Block-MCC$_{q,p}$ properties).
\end{proof}
%In the following property, the relationship between our proposed two cumulative coherences is stated:
In the following property, the relationship between our two proposed cumulative coherences, i.e., cumulative Block-MCC$_{q,p}$ (Definition \ref{def:CBMIC}) and Eldar's cumulative coherence (Definition \ref{def:CIIC}), is investigated:
\begin{property}[Cumulative Block-MCC$_{q,p}$ v.s. Eldar's cumulative coherence]
\label{prp:BMIC-CIIC} 
For a dictionary $\myPhi$ with orthonormal equally-sized blocks, we have:
\begin{equation*}
M_{2,2}\myparanthese{\myPhi,k} = M_{Inter}^{Eldar}\myparanthese{\myPhi , k}
\end{equation*}
\end{property}
\begin{proof}
From Definition \ref{def:CBMIC} (cumulative Block-MCC$_{q,p}$), we have $\forall (q , p) \ssin \mathbb{R}^2_{\sg 0} : M_{q,p} (\myPhi , k) {\myeq} d_{max}^{-1} \max_{\vert \Lambda \vert =k} \max_{j \notin \Lambda} \sum_{i \in \Lambda} d_{i}^{-1/p} d_{j}^{1/q} \Vert \myPhi^\dagger [i] \myPhi [j] \Vert_{q \to p}$.
On the other hand, for orthonormal blocks we have $\myPhi^\dagger [i] \seq \myPhi^T [i]$, and by selecting $q \seq p\seq 2$ for an equally-sized block structure, i.e., $d_1 \seq \cdots \seq d_K \seq d$, we get $M_{q,p} (\myPhi , k) \seq d^{-1} \max_{\vert \Lambda \vert =k} \max_{j \notin \Lambda} \sum_{i \in \Lambda} \Vert \myPhi^T [i] \myPhi [j] \Vert_{2 \to 2} \seq M_{Inter}^{Eldar} (\myPhi , k)$. % in a special setting of
\end{proof}
