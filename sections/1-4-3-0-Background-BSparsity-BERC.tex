Recently, there has been a huge surge of interest in developing recovery conditions, which ensure the uniqueness or robustness of the block-sparse representation of the USLE.
In addition, block-sparsity can be used in dictionary learning, where there is a joint block-sparse representation of signals \cite{Zelnik-Manor2012}.

In addition to the aforementioned practical interest of the block-sparse representation, i.e., compatibility with some real world problems, from a mathematical point of view, assuming the block-wise structure for the representation leads to weakened recovery conditions \cite{Eldar2010,Ziaei2010,Ben-Haim2011}.
%\cite{Eldar2009b,Eldar2010b,Eldar2010,Ziaei2010,Ben-Haim2011}.
By weakened recovery conditions we mean that for the same number of non-zero elements in the representation, assuming the block structure guarantees the uniqueness of the representation with a higher sparsity level.

Similarly, conditions for guaranteeing the uniqueness or faithful approximation of the solution with the models of (\ref{eq:Model_Noiseless}), i.e., $\boldsymbol{y} \seq \myPhi \mybetaz$, and (\ref{eq:Model_Noisy}), i.e., $\boldsymbol{y} \seq \myPhi \mybetaz \spl \boldsymbol{e}$, are called \emph{Block-sparse Exact Recovery Condition (Block-ERC)} and \emph{block-sparse stable recovery condition}.

{
\label{txt:BlockSL} 
Sparsity level in the block-wise world is called \emph{Block-Sparsity Level (Block-SL)} or \emph{block-sparsity bound}, \myhl{which is represented by $\myBSLMath$.}
As it can be easily derived, supposing equally-sized blocks of length $d$, i.e., $d_1 \seq \cdots \seq d_K \seq d$, to improve the conventional ERC and stable recovery conditions, the $d$ times block-sparsity level have to be greater than the sparsity level, i.e., $d \stimes \myBSLMath \sg \mySLMath$.%Block{-}SL
}

Such sparse representations whose non-zero entries appear in a few blocks are referred to as \emph{block-sparse} or \emph{block $k$-sparse representation} \cite{Eldar2009b,Stojnic2009a,Ben-Haim2011,Elhamifar2012b}
%\cite{Eldar2009b,Eldar2009c,Eldar2009d,Stojnic2009a,Eldar2010b,Eldar2010,Ben-Haim2011,Elhamifar2011,Elhamifar2012b} 
, which $k$ is the maximum number of active blocks.

Similarly, for a block k-sparse representation, for all $p \sgeq 0$ we have $\Vert \mybetaz \Vert_{p,0} \sleq k \sless \myBSLMath$.
From conventional element-wise sparsity point of view, for equally-sized blocks, block $k$-sparse representation is equivalent to \emph{kd-sparse} representation, i.e., $\Vert \mybetaz \Vert_{0} \sleq kd$.

In this section, we review the following main block-sparse exact recovery conditions:
%, which can be divided into the following four classes: Block-ERC based on (1) $\mySpkTxt$, (2) null space property, (3) mutual coherence constant, and (4) cumulative mutual coherence constant.
\begin{itemize}
\item Block-ERC based on $\mySpkTxt$,
\item Block-ERC based on null space property,
\item Block-ERC based on mutual coherence constant, and
\item Block-ERC based on cumulative mutual coherence constant.
\end{itemize}
\newpage
%------------------------------------------------------
\paragraph{1) Block-ERC based on $\boldsymbol{\mySpkTxt}$}
From algorithmic point of view, Ganesh and his colleagues proved that supposing a $k$-subspace sparse measurement vector $\boldsymbol{y}$, if $k \sless \mySpk_S(\myPhi) / 2$ then the sparse decomposition is necessarily unique in their proposed algorithms of subspace matching pursuit and subspace base pursuit \cite{Ganesh2009}.
%------------------------------------------------------
\paragraph{2) Block-ERC based on null space property}
In literature, this type of Block-ERC can be divided into the following groups:
\subparagraph{Based on block null space property:}
\cite{Stojnic2009a} demonstrated that the conventional NSP can be generalised to block-sparse representation. 
By proposing the following condition, they showed the equivalence of the optimisation problems $P_{2,1}$ and $P_{2,0}$:
\begin{gather*}
Q_{2,1}\myparanthese{S_b\myparanthese{\mybeta},\myPhi} \myeq \max_{\boldsymbol{x} \in \myKerMath \backslash\left\{\boldsymbol{0}\right\}} \frac{\displaystyle\sum_{k \in S_b\myparanthese{\mybeta}} \myabs{\displaystyle\sum_{j=1}^{d} \myabs{x_{j}[k]} ^{2}} ^{\frac{1}{2}}}{\displaystyle\sum_{k=1}^{K} \myabs{\displaystyle\sum_{j=1}^{d} \myabs{x_{j}[k]} ^{2}} ^{\frac{1}{2}}} < \frac12,
\end{gather*}
where, $S_b(\mybeta)$ is subset of a set with all subset of size $k$ of $\{1, \cdots , K\}$, i.e., block $k$-sparse representation, and $d$ is the length of equally-sized blocks.
Also, he mentioned that his result can be generalised to $P_{2,p}$, $0 \sless p \sleq 1$.
\subparagraph{Based on fusion null space property:}
In addition, the fusion NSP defined in the fusion frame framework proposed by Boufounos et al., in a special case in which all the subspaces share the same dimension, becomes similar to the NSP proposed by Stojnic et al. \cite{Boufounos2011}.

Moreover, the notions of exact, stable and robust NSP for fusion frames are presented in \cite{Ayaz2014,Ayaz2016}.
%------------------------------------------------------
\paragraph{3) Block-ERC based on mutual coherence constant}
In literature, this type of Block-ERC can be divided into the following groups:
\subparagraph{Based on block-coherence:}
Eldar and her colleagues extended the basic uncertainty principle to the block-sparse representations \cite{Eldar2009b,Eldar2010b,Eldar2010}.
In other words, there is a limit on the block-sparsity level of the representations $\mybetao$ and $\mybetaTwo$:
\begin{gather}
\label{eq:BUP-Eldar} 
\mynorm{\mybetao}_{2,0} + \mynorm{\mybetaTwo}_{2,0} 
\geq\frac{2}{d \, \overbar{M}^{Eldar}_{Inter} \myparanthese{\myPhiOne,\myPhiTwo}},
\end{gather}
where, the basic block-coherence of Eldar et al., i.e., $\overbar{M}^{Eldar}_{Inter}(\myPhiOne,\myPhiTwo)$, is the maximal coherence between the blocks of the orthonormal bases $\myPhiOne$ and $\myPhiTwo$, i.e.,:
\begin{equation*}
\overbar{M}^{Eldar}_{Inter}\myparanthese{\myPhiOne,\myPhiTwo} = \max_{k,k'} \frac1d \mynorm{\myPhiOne^T \mybracket{k} \myPhiTwo \mybracket{k'}}_{2 \to 2},
\end{equation*}
where, $d$ is the length of the equally-sized blocks.

Notice that for orthonormal bases $\myPhiOne$, $\myPhiTwo$ and their concatenation $[\myPhiOne , \myPhiTwo]$, we have $\overbar{M}^{Eldar}_{Inter}(\myPhiOne,\myPhiTwo) \seq M^{Eldar}_{Inter}([\myPhiOne,\myPhiTwo])$.

{
\label{txt:BasicEldarCCBound} 
Again, supposing the dimension of two orthonormal bases $\myPhiOne$ and $\myPhiTwo$ is $m$ by $m$, and $m \seq R \, d$ ($R$ is an integer), it is proved that $\overbar{M}^{Eldar}_{Inter}(\myPhiOne,\myPhiTwo) \sgeq 1/\sqrt{d \, m}$ \cite{Eldar2009b,Eldar2010b,Eldar2010}.
}

Notice that for $d \seq 1$, all the formulation reduces to its conventional one described in Section \ref{sec:Sparsity-ERC-MIC}.

Then, they proposed Block-ERC for ensuring the uniqueness of the solution of their proposed recovery algorithms of $\ell_2/\ell_1$-optimisation program \cite{Eldar2009c}, block matching pursuit and block orthogonal matching pursuit \cite{Eldar2009b,Eldar2010b,Eldar2010} based on their proposed block version of the conventional MCC.

Their Block-ERC for recovering a block $k$-sparse representation through their proposed recovery algorithms and for equally-sized blocks of length $d$ is \cite{Eldar2009b,Eldar2010b,Eldar2010}:
\begin{gather}
\label{BERC-Eldar} 
\mynorm{\mybetaz }_{2,0} < \frac{1+\myparanthese{d \, M^{Eldar}_{Inter}\myparanthese{\myPhi}}^{-1} \myparanthese{1-\myparanthese{d-1}M^{Eldar}_{Intra}\myparanthese{\myPhi}}}{2}.
\end{gather}

In a similar work, Ziaei et al. proved nearly the same Block-ERC of (\ref{BERC-Eldar}), for the different block size setting, where, all the $d$-s in (\ref{BERC-Eldar}) and $M^{Eldar}_{Inter}(\myPhi)$ have been replaced by $\max_{i , j {\neq} i} \sqrt{d_i d_j}$ \cite{Ziaei2010}.

Ziaei et al. proved that their proposed modification of the $\ell_2/\ell_1$-optimisation program for different-size blocks outperforms the $\ell_2/\ell_1$-optimisation program \cite{Eldar2009c} and basis pursuit \cite{Chen2001}, in terms of the error rate \cite{Ziaei2010}.

In the presence of noise, Ben-Haim and Eldar proposed block-sparse stable recovery conditions for block orthogonal matching pursuit and block-thresholding algorithms \cite{Ben-Haim2011}.
\subparagraph{Based on fusion coherence:}
In addition, in the context of fusion frame theory, Boufounos et al. have proved that if $k \sless (1 \spl M_F^{-1}(\myPhi))/2$, then the unique recovery of his $ \ell_0$ and $ \ell_1$ norm optimisation problems is guaranteed \cite{Boufounos2011}.

Later, Ayaz et al. improves some of the results of Boufounos et al. in the fusion frame setup \cite{Ayaz2014,Ayaz2016}.

The recovery condition based on the fusion coherence in the special setting, where, all the subspaces share the same dimension, reduces to (\ref{BERC-Eldar}) \cite{Boufounos2011}.
\newpage
\subparagraph{Based on mutual subspace coherence:}
There is a relationship between subspace $\mySpkTxt$ and mutual subspace coherence, i.e., $\mySpk_S(\myPhi) \sgeq 1 \spl M^{-1}_S(\myPhi)$, which is similar to its conventional relationship \cite{Ganesh2009}.

From algorithmic point of view, Ganesh and his colleagues proved similar conditions in recovering the representation. 
Supposing that the measurement vector $\boldsymbol{y}$ is \emph{$k$-subspace sparse}, they demonstrated that if $k \sless (1 \spl M_S^{-1}(\myPhi))/2$, then subspace matching pursuit and subspace base pursuit algorithms are guaranteed to find the $k$ subspaces exactly \cite{Ganesh2009}.

In a related work based on $M_S(\myPhi)$, Elhamifar and Vidal proved that if the measurement vector $\boldsymbol{y}$ has a unique block $k$-sparse representation vector $\mybetaz$, and the subspaces spanned by the columns of the blocks are disjoint, then under the condition
\begin{equation*}
k < \frac{1 + \myparanthese{\frac{1+\epsilon_p}{1-\epsilon_p} M_S\myparanthese{\myPhi}}^{-1}}{1+\sqrt{\frac{1+\sigma_p}{1+\epsilon_p}}},
\end{equation*}
the optimisation problems $P_{p,1}$ and $P_{p,0}$ are equivalent, and also under the condition
\begin{equation*}
k < \frac{1 + \myparanthese{\frac{1+\epsilon'_p}{1-\epsilon'_p} M_S\myparanthese{\myPhi}}^{-1}}{2},
\end{equation*}
the optimisation problems $P_{(p,1),q,0}(\mybeta_{\myPhi},\boldsymbol{r})$ and $P_{(p,0),q,0}(\mybeta_{\myPhi},\boldsymbol{r})$ are equivalent, where, $\sigma_p$, $\epsilon_p$, and $\epsilon'_p$ are some intra-block $p$-restricted isometry constants which are defined based on the restricted isometry properties introduced in the paper \cite{Elhamifar2011,Elhamifar2012b}.
%------------------------------------------------------
\paragraph{4) Block-ERC based on cumulative mutual coherence}
The mutual subspace coherence of Ganesh et al. \cite{Ganesh2009} represents only the most extreme correlation between the subspaces of the dictionary and does not offer a comprehensive description of the dictionary. 

In order to extract more information and to better characterise the dictionary, Elhamifar and Vidal focused on the cumulative version of mutual subspace coherence.

In the block-wise world, Elhamifar and Vidal proved that if the measurement vector $\boldsymbol{y}$ has a unique block $k$-sparse representation vector $\mybetaz$, and the subspaces spanned by the columns of the blocks are disjoint, then under 
\begin{equation*}
\sqrt{\frac{1+\sigma_p}{1+\epsilon_p}}M_S\myparanthese{\myPhi , k} + M_S\myparanthese{\myPhi , k-1} < \frac{1-\epsilon_p}{1+\epsilon_p},
\end{equation*}
the optimisation problems $P_{p,1}$ and $P_{p,0}$ are equivalent, and also under
\begin{equation*}
M_S\myparanthese{\myPhi , k} + M_S\myparanthese{\myPhi , k-1} < \frac{1-\epsilon'_p}{1+\epsilon'_p},
\end{equation*}
the optimisation problems $P_{(p,1),q,0}(\mybeta_{\myPhi},\boldsymbol{r})$ and $P_{(p,0),q,0}(\mybeta_{\myPhi},\boldsymbol{r})$ are equivalent, where, $\sigma_p$, $\epsilon_p$, and $\epsilon'_p$ are some intra-block $p$-restricted isometry constants which are defined based on the restricted isometry properties introduced in the paper \cite{Elhamifar2012b}.