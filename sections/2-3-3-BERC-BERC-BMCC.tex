Eldar et al. extended the basic uncertainty principle defined in (\ref{eq:UP-basic}) on page \pageref{eq:UP-basic} to block-sparse domain defined in (\ref{eq:BUP-Eldar}) on page \pageref{eq:BUP-Eldar} \cite{Eldar2009b,Eldar2010b,Eldar2010}.
Now we generalise the results of Eldar et al., which will be also considered as a generalisation of the basic uncertainty principle \cite{Donoho1989,Donoho2001,Elad2001,Elad2002a} and name it \emph{basic block-sparse uncertainty principle}.
\begin{lemma}[Basic Block-UP\footnote{\emph{Block-sparse Uncertainty Principle}}]
\label{lm:BBUP} 
Supposing $\mybetao$ and $\mybetaTwo$ are two distinct block-structured representations of the non-zero signal $\boldsymbol{y}$ in two \myhl{orthonormal} $m \stimes m$ matrices $\myPhiOne$ and $\myPhiTwo$ \myhl{with similar block structure (same number of blocks and blocks' length vector $\boldsymbol{d}$),} respectively, i.e.,:
\begin{equation*}
\boldsymbol{y} = 
\displaystyle\sum_{k=1}^K \myPhiOne \mybracket{k} \mybetao \mybracket{k} =
\displaystyle\sum_{k=1}^K \myPhiTwo \mybracket{k} \mybetaTwo \mybracket{k},
\end{equation*}
then \myhl{$\forall (q , p) \ssin \mathbb{R}^2_{\sgeq 1}$,} and $\forall r \ssin \mathbb{R}_{\sgeq 0}$:
\begin{gather*}
\mynorm{\mybetao}_{r,0} + \mynorm{\mybetaTwo}_{r,0} 
\geq\frac{2}{\mycolor{d_{min}^{-\frac12} d_{max}^{\frac32}} \, \overbar{M}_{q,\frac{p}{p-1}} \myparanthese{\myPhiOne , \myPhiTwo}},
\end{gather*}
where, $\overbar{M}_{q,p/(p \sm 1)} (\myPhiOne , \myPhiTwo)$ is defined based on the parameter changing of $p {\to} p/(p \sm 1)$ in $\overbar{M}_{q,p} (\myPhiOne , \myPhiTwo)$, which is called \emph{basic Block-MCC$_{q,p}$}.
Basic Block-MCC$_{q,p}$ is derived from the special case of Property \ref{prp:IntraBlkO} (Block-MCC$_{q,p}$ for intra-block orthonormality, page \pageref{prp:IntraBlkO}), when the characterisation is extracted from two separated matrices instead of one matrix, and is defined as follows:
\iffalse
\begin{equation*}
\overbar{M}_{q,\frac{p}{p-1}}\myparanthese{\myPhiOne , \myPhiTwo} = 
\max_{k,k'} \frac{d_{k}^{-\frac{p-1}{p}} \, d_{k'}^{\frac1q}}{d_{max}}\mynorm{\myPhiOne^T \mybracket{k} \myPhiTwo^{ } \mybracket{k'}}_{q \to \frac{p}{p-1}}.
\end{equation*}
\fi
\begin{equation*}
\mycolor{\forall (q , p) \in \mathbb{R}^2_{>0}, \qquad
\overbar{M}_{q,p}\myparanthese{\myPhiOne , \myPhiTwo} = 
\max_{k,k'} \frac{d_{k}^{-\frac{1}{p}} \, d_{k'}^{\frac1q}}{d_{max}}\mynorm{\myPhiOne^T \mybracket{k} \myPhiTwo^{ } \mybracket{k'}}_{q \to p}.}
\end{equation*}
\end{lemma}
The proof of Lemma \ref{lm:BBUP} is provided in Section \ref{prf:BBUP} (page \pageref{prf:BBUP}).

{
\label{cmmnt:2} 
\myhl{From the notation point of view, notice that the basic Block-MCC$_{q,p}$ of two orthonormal matrices is equal to the Block-MCC$_{q,p}$ of the dictionary built from the concatenation of those previous orthonormal matrices, i.e., $\overbar{M}_{q,p/(p \sm 1)} (\myPhiOne , \myPhiTwo) \seq M_{q,p/(p \sm 1)} ([\myPhiOne , \myPhiTwo])$.}
}

%Notice that for orthonormal matrices $\myPhiOne$ and $\myPhiTwo$, we have $\overbar{M}_{q,p/(p \sm 1)} (\myPhiOne , \myPhiTwo) \seq M_{q,p/(p \sm 1)} ([\myPhiOne , \myPhiTwo])$.

As expected, for $d_1 \seq \cdots \seq d_K$ equal to $1$, Lemma \ref{lm:BBUP} reduces to the basic uncertainty principle explained in (\ref{eq:UP-basic}), i.e., $(\Vert \mybetao \Vert_0 \spl \Vert \mybetaTwo \Vert_0) \sgeq 2/\overbar{M} (\myPhiOne,\myPhiTwo)$.

In comparison of different uncertainty principles, it should be taken into account that a typical uncertainty principle I is weaker than uncertainty principle II, when minimal sum of $\ell_0$ pseudo-norm or $\ell_{r,0}$ pseudo-mixed-norm of two candidate solutions $\mybetao$ and $\mybetaTwo$ in I is higher than II.
To see the effect of block-sparsity on the uncertainty principle, using $\Vert \mybetao \Vert_{0} \sleq d_{max} \, \Vert \mybetao \Vert_{r,0}$ and $\Vert \mybetaTwo \Vert_{0} \sleq d_{max} \, \Vert \mybetaTwo \Vert_{r,0}$, $\forall r \sgeq 0$, in (\ref{eq:UP-basic}), we get $(\Vert \mybetao \Vert_{r,0} \spl \Vert \mybetaTwo \Vert_{r,0}) \sgeq 2/(d_{max} \overbar{M} (\myPhiOne,\myPhiTwo))$.
Then, comparing with Lemma \ref{lm:BBUP}, we need to demonstrate the relationship between $d_{max} \overbar{M}(\myPhiOne,\myPhiTwo)$ and $d_{min}^{-1/2} d_{max}^{3/2} \overbar{M}_{q,p/(p-1)} (\myPhiOne , \myPhiTwo)$.   
%To this aim, first we investigate the general relationship between $\overbar{M}(\myPhiOne,\myPhiTwo)$ and $\overbar{M}_{q,p/(p-1)} (\myPhiOne , \myPhiTwo)$ in the following property:  
\begin{property}[Basic Block-MCC$_{q,p}$ v.s. basic MCC]
\label{prp:DontKnow2} 
Supposing $\myPhiOne$ and $\myPhiTwo$ are two orthonormal matrices, $\forall (q , p) \ssin \mathbb{R}^2_{\sg 0}$ we have:
\begin{gather*}
\begin{aligned}
0 \leq 
\frac{d_{min}^{-\frac12} d_{max}^{\frac32} \, \overbar{M}_{q,\frac{p}{p-1}}\myparanthese{\myPhiOne , \myPhiTwo}}{d_{max} \, \overbar{M}\myparanthese{\myPhiOne , \myPhiTwo}} \leq 
d_{min}^{-\frac12} \, d_{max}^{-\frac12} \, \max_{k,k'} d_{k}^{\frac1p - \frac12} \, d_{k'}^{\frac1q + \frac12} \, &\max \mybrace{1 , d_k^{\frac12 - \frac1p}} \, \max \mybrace{1 , d_{k'}^{\frac12 - \frac1q}}.
\end{aligned}
\end{gather*}
\iffalse
\begin{equation*}
\overbar{M}_{q,\frac{p}{p-1}}\myparanthese{\myPhiOne , \myPhiTwo} \leq 
\begin{cases}
d_{max}^{\frac1p + \frac12} \, \overbar{M}\myparanthese{\myPhiOne , \myPhiTwo}, & \qquad \text{for \ }q \geq 1\\
d_{max}^{\frac1p + \frac1q - \frac12} \, \overbar{M}\myparanthese{\myPhiOne , \myPhiTwo}, & \qquad \text{for \ }q < 1\\
\end{cases}.
\end{equation*}
\fi
The above upper-bound for different values of $q$ and $p$ are shown in table \ref{table:BasicBMCC-BasicMCC}.
\begin{table*}[bp]
\begin{adjustbox}{width=1\textwidth} % ,totalheight=\textheight,.5
\centering
%\tiny
\begin{tabular}{ccccc}
\toprule
%\cline{2-4}
\multicolumn{1}{c}{} &\multicolumn{1}{c}{${0 < q\, \& \, p \leq 2}$} & \multicolumn{1}{c}{${q\, \& \, p \geq 2}$}  & \multicolumn{1}{c}{${0 < q \leq 2\, \& \, p \geq 2}$} & \multicolumn{1}{c}{${q \geq 2\, \& \, 0 < p \leq 2}$} \\ \midrule %\hline
\multicolumn{1}{r}{${\frac{d_{min}^{-\frac12} d_{max}^{\frac32} \, \overbar{M}_{q,\frac{p}{p-1}}\myparanthese{\myPhiOne , \myPhiTwo}}{d_{max} \, \overbar{M}\myparanthese{\myPhiOne , \myPhiTwo}} \leq}$} &\multicolumn{1}{c}{$d_{min}^{-\frac12} \, d_{max}^{\frac1p + \frac1q - \frac12}$} & \multicolumn{1}{c}{$d_{min}^{-\frac12} \, d_{max}^{\frac12}$} & \multicolumn{1}{c}{$d_{min}^{-\frac12} \, d_{max}^{\frac1q}$} &\multicolumn{1}{c}{$d_{min}^{-\frac12} \, d_{max}^{\frac1p}$}    \\ 
\bottomrule %\hline
\end{tabular}
\end{adjustbox}
\caption{Upper-bound of $d_{min}^{-1/2} d_{max}^{3/2} \overbar{M}_{q,p/(p-1)} (\myPhiOne , \myPhiTwo) / d_{max} \overbar{M}(\myPhiOne,\myPhiTwo)$ for different values of $q$ and $p$.}
\label{table:BasicBMCC-BasicMCC}
\end{table*}
\end{property}
\begin{proof}
From second part of Property \ref{prp:BMIC-MIC} (Block-MCC$_{q,p}$ bounds, page \pageref{prp:BMIC-MIC}), which the dictionary has intra-block orthonormality, in a special case of two distinct orthonormal bases instead of one dictionary, $\forall (q , p , q' , p') \ssin \mathbb{R}^4_{\sg 0}$ we have:
\begin{gather*}
\begin{aligned}
0 \leq \overbar{M}_{q,p}\myparanthese{\myPhiOne , \myPhiTwo} \leq 
\frac{\overbar{M}\myparanthese{\myPhiOne , \myPhiTwo}}{d_{max}} \, \max_{k,k'} d_{k}^{\frac12 - \frac1p} \, d_{k'}^{\frac1q + \frac12} \, &\max \mybrace{1 , d_k^{\frac1p - \frac{1}{p'}}} \, \max \mybrace{1 , d_{k'}^{\frac{1}{q'} - \frac1q}} \times \\
&\max \mybrace{1 , d_k^{\frac{1}{p'} -\frac12}} \, \max \mybrace{1 , d_{k'}^{\frac{1}{2} - \frac{1}{q'}}}.
\end{aligned}
\end{gather*}
Then, by parameter changing of $p {\to} p/(p \sm 1)$, and then selecting $q' \seq p' \seq 2$, we get:
\begin{gather*}
\begin{aligned}
0 \leq \overbar{M}_{q,\frac{p}{p-1}}\myparanthese{\myPhiOne , \myPhiTwo} &\leq 
\frac{\overbar{M}\myparanthese{\myPhiOne , \myPhiTwo}}{d_{max}} \, &&\max_{k,k'} d_{k}^{\frac1p - \frac12} \, d_{k'}^{\frac1q + \frac12} \, \max \mybrace{1 , d_k^{1 - \frac1p - \frac{1}{p'}}} \, \max \mybrace{1 , d_{k'}^{\frac{1}{q'} - \frac1q}} \times \\
& && \qquad \qquad \qquad \ \ \ \, \max \mybrace{1 , d_k^{\frac{1}{p'} -\frac12}} \, \max \mybrace{1 , d_{k'}^{\frac{1}{2} - \frac{1}{q'}}} \\
&= 
\frac{\overbar{M}\myparanthese{\myPhiOne , \myPhiTwo}}{d_{max}} \, &&\max_{k,k'} d_{k}^{\frac1p - \frac12} \, d_{k'}^{\frac1q + \frac12} \, \max \mybrace{1 , d_k^{\frac12 - \frac1p}} \, \max \mybrace{1 , d_{k'}^{\frac12 - \frac1q}}.
\end{aligned}
\end{gather*}
\iffalse
%Next, by substituting $q' \seq p' \seq 1$, because the threshold on $q$ and $p$ is on one, i.e., $\forall (q , p) \ssin \mathbb{R}^2_{\sgeq 1}$, we have:
\begin{gather*}
\begin{aligned}
0 \leq \overbar{M}_{q,\frac{p}{p-1}}\myparanthese{\myPhiOne , \myPhiTwo} &\leq 
\frac{\overbar{M}\myparanthese{\myPhiOne , \myPhiTwo}}{d_{max}} \, \max_{k,k'} d_{k}^{\frac1p - \frac12} \, d_{k'}^{\frac1q + \frac12} \, d_{k'}^{1 - \frac1q} d_k^{\frac12} \\
&= \frac{\overbar{M}\myparanthese{\myPhiOne , \myPhiTwo}}{d_{max}} \, \max_{k,k'} d_{k}^{\frac1p} \, d_{k'}^{\frac32} \\
&\leq d_{max}^{\frac1p + \frac12} \, \overbar{M}\myparanthese{\myPhiOne , \myPhiTwo}.
\end{aligned}
\end{gather*}
\fi
Next, by multiplying by $d_{min}^{-1/2} d_{max}^{1/2} \sg 0$, the proof is done.
\end{proof}
%\begin{remark}
\iffalse
Since the threshold on the range of $q$ and $p$ in Lemma \ref{lm:BBUP} is one, to determine the relation between $d_{max} \overbar{M}(\myPhiOne,\myPhiTwo)$ and $d_{min}^{-1/2} d_{max}^{3/2} \overbar{M}_{q,p/(p-1)} (\myPhiOne , \myPhiTwo)$, we select $q' \seq p' \seq 1$ in Property \ref{prp:DontKnow2}, i.e., $0 \sleq \overbar{M}_{q,p/(p-1)} (\myPhiOne , \myPhiTwo) / \overbar{M}(\myPhiOne,\myPhiTwo) \sleq d_{max}^{-1} \max_{k,k'} d_{k}^{\frac1p} d_{k'}^{\frac1q + \frac12} \max \{ 1 , d_{k'}^{1 - \frac1q} \}$.
Therefore, the upper-bound is independent of the relation of $p$ to one, and it only depends on that of $q$, then we have:
\begin{equation*}
\forall p \ssin \mathbb{R}_{\sg 0}, \qquad
0 \leq  \frac{d_{min}^{-\frac12} \, d_{max}^{\frac32} \, \overbar{M}_{q,\frac{p}{p-1}}\myparanthese{\myPhiOne , \myPhiTwo}}{d_{max} \, \overbar{M}\myparanthese{\myPhiOne , \myPhiTwo}} \leq 
\begin{cases}
d_{min}^{- \frac12} \, d_{max}^{\frac1p + \frac1q}, & \qquad \text{for \ }0 < q < 1 \\
d_{min}^{- \frac12} \, d_{max}^{\frac1p + 1}, & \qquad \text{for \ }q \geq 1\\
\end{cases}.
\end{equation*}
%\end{remark}
\fi
\begin{remark}
\label{rmrk:BasicBMCC-BasicMCC} 
\myhl{As it is mentioned before, in order to compare the proposed basic block-sparse uncertainty principle in Lemma {\ref{lm:BBUP}} (Basic Block-UP, page {\pageref{lm:BBUP}}), i.e., $(\Vert \mybetao \Vert_{r,0} \spl \Vert \mybetaTwo \Vert_{r,0}) \sgeq 2/(d_{min}^{-1/2} d_{max}^{3/2} \overbar{M}_{q,p/(p-1)} (\myPhiOne,\myPhiTwo))$, with the conventional basic uncertainty principle explained in ({\ref{eq:UP-basic}}) on page {\pageref{eq:UP-basic}}, i.e., $(\Vert \mybetao \Vert_0 \spl \Vert \mybetaTwo \Vert_0) \sgeq 2/\overbar{M} (\myPhiOne,\myPhiTwo)$, the relationship between $d_{min}^{-1/2} d_{max}^{3/2} \overbar{M}_{q,p/(p-1)} (\myPhiOne , \myPhiTwo)$ and $d_{max} \overbar{M}(\myPhiOne,\myPhiTwo)$ should be investigated.
The mentioned comparison is shown in table {\ref{table:BasicBMCC-BasicMCC}}.
Since the two terms are in the denominator of the right-hand side of the uncertainty principle, the smaller values lead to more weakened uncertainty principles, hence, improve the conditions.
Therefore, the values in table {\ref{table:BasicBMCC-BasicMCC}}, which are less than or equal to one, correspond to weakened uncertainty principles due to using block-sparsity.
From table {\ref{table:BasicBMCC-BasicMCC}}, it can be seen that for $\forall (q , p) \ssin \mathbb{R}^2_{\sgeq 2}$ and $d_{min} \seq d_{max}$, the proposed basic block-sparse uncertainty principle in Lemma {\ref{lm:BBUP}} (Basic Block-UP) improves the conventional basic uncertainty principle explained in ({\ref{eq:UP-basic}}).}
\end{remark}

To compare the proposed basic block-sparse uncertainty principle in Lemma \ref{lm:BBUP} (Basic Block-UP, page \pageref{lm:BBUP}, with $d_{min} \seq d_{max} \seq d$) with the block-sparse uncertainty principle proposed by Eldar et al. in (\ref{eq:BUP-Eldar}) on page \pageref{eq:BUP-Eldar}, i.e., $\Vert \mybetao \Vert_{2,0} \spl \Vert\mybetaTwo \Vert_{2,0} 
\sgeq 2/(d \, \overbar{M}^{Eldar}_{Inter} (\myPhi_{\boldsymbol{1}},\myPhi_{\boldsymbol{2}}))$, where $\overbar{M}^{Eldar}_{Inter}(\myPhi_{\boldsymbol{1}},\myPhi_{\boldsymbol{2}}) \seq \max_{k,k'} \Vert \myPhi_{\boldsymbol{1}}^T [k] \myPhi_{\boldsymbol{2}} [k'] \Vert_{2 \to 2} / d$, we need to investigate the relationship between $\overbar{M}_{q,p/(p \sm 1)} (\myPhiOne , \myPhiTwo)$ and $\overbar{M}^{Eldar}_{Inter}(\myPhiOne,\myPhiTwo)$.
\begin{property}[Basic Block-MCC$_{q,p}$ v.s. basic block-coherence of Eldar et al.]
\label{prp:DontKnow3}
Supposing $\myPhiOne$ and $\myPhiTwo$ are two orthonormal matrices with equally-sized blocks of length $d$, $\forall (q , p) \ssin \mathbb{R}^2_{\sg 0}$ we have:
\begin{gather*}
\begin{aligned}
&\frac{\overbar{M}_{q,\frac{p}{p-1}}\myparanthese{\myPhiOne , \myPhiTwo}}{\overbar{M}^{Eldar}_{Inter}\myparanthese{\myPhiOne , \myPhiTwo}} \geq 
d^{\frac1q + \frac1p - 1} \min \mybrace{1 , d^{\frac{1}{2} -\frac1p}} \min \mybrace{1 , d^{\frac{1}{2} - \frac1q}}, \\
&\frac{\overbar{M}_{q,\frac{p}{p-1}}\myparanthese{\myPhiOne , \myPhiTwo}}{\overbar{M}^{Eldar}_{Inter}\myparanthese{\myPhiOne , \myPhiTwo}} \leq 
d^{\frac1q + \frac1p - 1} \max \mybrace{1 , d^{\frac{1}{2} -\frac1p}} \max \mybrace{1 , d^{\frac{1}{2} - \frac1q}}.
\end{aligned}
\end{gather*}
Then, for $q \seq p \seq 2$ the lower- and upper-bound are both equal to one, then $\overbar{M}_{2,2}(\myPhiOne,\myPhiTwo) \seq \overbar{M}^{Eldar}_{Inter}(\myPhiOne,\myPhiTwo)$.
The above bounds for different values of $q$ and $p$ are shown in table \ref{table:BasicBMCC-BasicEldar}.
\begin{table}[bp]
%\begin{adjustbox}{width=0.5\textwidth} % ,totalheight=\textheight,.5
\centering
%\tiny
\begin{tabular}{ccccc}
\toprule
%\cline{2-4}
\multicolumn{1}{c}{} &\multicolumn{1}{c}{${0 < q\, \& \, p \leq 2}$} & \multicolumn{1}{c}{${q\, \& \, p \geq 2}$}  & \multicolumn{1}{c}{${0 < q \leq 2\, \& \, p \geq 2}$} & \multicolumn{1}{c}{${q \geq 2\, \& \, 0 < p \leq 2}$} \\ \midrule %\hline
\multicolumn{1}{r}{${\frac{\overbar{M}_{q,\frac{p}{p-1}}\myparanthese{\myPhiOne , \myPhiTwo}}{\overbar{M}^{Eldar}_{Inter}\myparanthese{\myPhiOne , \myPhiTwo}} \geq}$} &\multicolumn{1}{c}{$1$} & \multicolumn{1}{c}{$d^{\frac1q + \frac1p -1}$} & \multicolumn{1}{c}{$d^{\frac1p - \frac12}$} &\multicolumn{1}{c}{$d^{\frac1q - \frac12}$}    \\
\multicolumn{1}{r}{${\frac{\overbar{M}_{q,\frac{p}{p-1}}\myparanthese{\myPhiOne , \myPhiTwo}}{\overbar{M}^{Eldar}_{Inter}\myparanthese{\myPhiOne , \myPhiTwo}} \leq}$} &\multicolumn{1}{c}{$d^{\frac1q + \frac1p -1}$} & \multicolumn{1}{c}{$1$} & \multicolumn{1}{c}{$d^{\frac1q - \frac12}$} &\multicolumn{1}{c}{$d^{\frac1p - \frac12}$}    \\ 
\bottomrule %\hline
\end{tabular}
%\end{adjustbox}
\caption{Bounds of $\overbar{M}_{q,p/(p \sm 1)} (\myPhiOne , \myPhiTwo) / \overbar{M}^{Eldar}_{Inter}(\myPhiOne,\myPhiTwo)$ for different values of $q$ and $p$.}
\label{table:BasicBMCC-BasicEldar}
\end{table}
\end{property}
\begin{proof}
It follows from Property \ref{prp:BMIC-MEldar} (Block-MCC$_{q,p}$ v.s. block-coherence of Eldar et al., page \pageref{prp:BMIC-MEldar}) in a special case of two orthonormal matrices $\forall (q , p) \ssin \mathbb{R}^2_{\sg 0}$:
\begin{gather*}
\begin{aligned}
&\overbar{M}_{q,p}\myparanthese{\myPhiOne , \myPhiTwo} \geq d^{\frac1q - \frac1p} \min \mybrace{1 , d^{\frac1p - \frac{1}{2}}} \min \mybrace{1 , d^{\frac{1}{2} - \frac1q}} \overbar{M}^{Eldar}_{Inter}\myparanthese{\myPhiOne , \myPhiTwo}, \\
&\overbar{M}_{q,p}\myparanthese{\myPhiOne , \myPhiTwo} \leq d^{\frac1q - \frac1p} \max \mybrace{1 , d^{\frac1p - \frac{1}{2}}} \max \mybrace{1 , d^{\frac{1}{2} - \frac1q}} \overbar{M}^{Eldar}_{Inter}\myparanthese{\myPhiOne , \myPhiTwo}.
\end{aligned}
\end{gather*}

Then, by parameter changing of $p {\to} p/(p \sm 1)$, the proof is done.
\end{proof}
\begin{remark}
\label{Rmrk:Basic BMIC II} 
\myhl{To compare the proposed basic block-sparse uncertainty principle in Lemma {\ref{lm:BBUP}} (Basic Block-UP, page {\pageref{lm:BBUP}}, with $d_{min} \seq d_{max} \seq d$), i.e., $(\Vert \mybetao \Vert_{r,0} \spl \Vert \mybetaTwo \Vert_{r,0}) \sgeq 2/(d \overbar{M}_{q,p/(p-1)} (\myPhiOne,\myPhiTwo))$, with the block-sparse uncertainty principle proposed by Eldar et al. in ({\ref{eq:BUP-Eldar}}), i.e., $\Vert \mybetao \Vert_{2,0} \spl \Vert\mybetaTwo \Vert_{2,0} 
\sgeq 2/(d \, \overbar{M}^{Eldar}_{Inter} (\myPhi_{\boldsymbol{1}},\myPhi_{\boldsymbol{2}}))$, the relationship between $\overbar{M}_{q,p/(p \sm 1)} (\myPhiOne , \myPhiTwo)$ and $\overbar{M}^{Eldar}_{Inter}(\myPhiOne,\myPhiTwo)$ should be investigated.
With the same reasoning as the one used in Remark {\ref{rmrk:BasicBMCC-BasicMCC}} (page {\pageref{rmrk:BasicBMCC-BasicMCC}}), the upper-bound values in table {\ref{table:BasicBMCC-BasicEldar}}, which are less than or equal to one, correspond to weakened uncertainty principles.
From table {\ref{table:BasicBMCC-BasicEldar}}, it can be seen that for $\forall (q , p) \ssin \mathbb{R}^2_{\sgeq 2}$, the proposed basic block-sparse uncertainty principle in Lemma {\ref{lm:BBUP}} improves the basic block-sparse uncertainty principle proposed by Eldar et al. in ({\ref{eq:BUP-Eldar}}) on page {\pageref{eq:BUP-Eldar}}.}
\iffalse
According to the properties \ref{prp:DontKnow2} and \ref{prp:DontKnow3}, the proposed basic block-sparse uncertainty principle in Lemma \ref{lm:BBUP}, 
%for $\forall q \sgeq p'$, $d_{min}^{-1/p'} d_{max}^{1/q} \sleq 1$, $\forall {p'}/2 \sless q \sless p'$, $d_{min}^{1/q - 2/{p'}} d_{max}^{1/q} \sleq 1$, and $\forall q \sleq {p'}/2$, $d_{max}^{2/q - 2/{p'}} \sleq 1$, where $p' \seq p/(p \sm 1)$, is weaker 
can be stronger than the basic uncertainty principle of (\ref{eq:UP-basic}), i.e., $(\Vert \mybetao \Vert_0 \spl \Vert \mybetaTwo \Vert_0) \sgeq 2/\overbar{M} (\myPhiOne,\myPhiTwo)$, whereas for $q \seq p \seq 2$ and equally-sized blocks, reduces to the principle of Eldar et al. in (\ref{eq:BUP-Eldar}), i.e., $\Vert \mybetao \Vert_{2,0} \spl \Vert\mybetaTwo \Vert_{2,0} \sgeq 2/(d \, \overbar{M}^{Eldar}_{Inter} (\myPhi_{\boldsymbol{1}},\myPhi_{\boldsymbol{2}}))$. 
\fi
\end{remark}

As mentioned before, we had lower-bound for the basic MCC, i.e., $\overbar{M}(\myPhiOne,\myPhiTwo) \sgeq 1/\sqrt{m}$ (page \pageref{txt:BasicMCCBounds}), and for basic block-coherence of Eldar et al., i.e., $\overbar{M}^{Eldar}_{Inter}(\myPhiOne,\myPhiTwo) \sgeq 1/\sqrt{d \, m}$ (page \pageref{txt:BasicEldarCCBound}).
Next, we show the corresponding lower-bound for the proposed basic Block-MCC$_{q,p}$ in a special case of $q \seq p \seq 2$, i.e., $\overbar{M}_{2,2} (\myPhiOne , \myPhiTwo)$.
\begin{property}[Basic Block-MCC$_{2,2}$ lower-bound]
\label{prp:BMIC-LB} 
Supposing $\myPhiOne$ and $\myPhiTwo$ are two $m \stimes m$ orthonormal matrices, we have:
\begin{equation}
\overbar{M}_{2,2} \myparanthese{\myPhiOne , \myPhiTwo} \geq
\frac{d_{min} \, d_{max}^{-\frac32}}{\sqrt{m}}.
\end{equation}
\end{property}
The proof of Property \ref{prp:BMIC-LB} is provided in Section \ref{prf:BMIC-LB} (page \pageref{prf:BMIC-LB}).
\begin{remark}[Block-incoherency]
\label{Rmrk:Basic BMIC} 
It can be seen that for special settings of $d_1 \seq \cdots \seq d_K \seq 1$ and equally-sized blocks, i.e., $d_1 \seq \cdots \seq d_K \seq d$, Property \ref{prp:BMIC-LB} is equivalent to the conventional bound, i.e., $\overbar{M}(\myPhiOne,\myPhiTwo) \sgeq 1/\sqrt{m}$, and the bound of Eldar et al., i.e., $\overbar{M}^{Eldar}_{Inter}(\myPhiOne,\myPhiTwo) \sgeq 1/\sqrt{d \, m}$, respectively.
Generally, for settings of $d_{min} \, d_{max}^{-3/2} \sleq 1$, the lower-bound in Property \ref{prp:BMIC-LB} is less than or equal to the lower-bound in conventional case, which means that in the proposed block-structured scenario, dictionaries can be more \emph{block-incoherent} compared to the conventional case.
Since there is a direct relationship between the sparsity level of recovery condition and block-incoherency, hence improved recovery conditions are obtained in block-structured scenario.
In addition, notice that in the proof of all above basic two orthonormal bases we used the relationships of a dictionary with intra-block orthonormality, because for an orthonormal base $\myPhi \ssin \mathbb{R}^{m \stimes m}$, with $\myPhi^T \myPhi \seq \boldsymbol{I}_{m}$, we have $\myPhi^T [k] \myPhi[k] \seq \boldsymbol{I}_{d_k}$.
\end{remark}
Returning back from basic two orthonormal bases $\myPhiOne$ and $\myPhiTwo$ to the dictionaries $\myPhi$, the corresponding relations based on Block-MCC$_{q,p}$ (Definition \ref{def:BMIC}, page \pageref{def:BMIC}) should be established.
%and in order to shift the recovery conditions based on $\myBSpkTxt$ to Block-MCC$_{q,p}$, which is the aim of this section, first we need to find their relationship.
Thanks to Block-MCC$_{q,p}$, we can overcome the issue of intractability of the $\myBSpkTxt$ (Definition \ref{def:Block Spark}, page \pageref{def:Block Spark}). 
%\iffalse
%First, by proposing the following Lemma, at the same time we investigate the  relationship between two proposed characterisations of $\myBSpkTxt$ and Block-MCC$_{q,p}$, and also we present a computationally reasonable approximation of the $\myBSpkTxt$:
%\begin{lemma}[$\boldsymbol{\myBSpkTxt}$ tractable lower-bound]
%\label{lm:Block Spark Inequality}
%For any general dictionary $\myPhi$ with Block-MCC$_{q,p}$ $M_{q,p}(\myPhi)$, $\forall p \sgeq 1$, and $\forall q \sgeq 1$, we have:
%\begin{equation*}
%\label{eq:BS-BMIC}
%\forall \boldsymbol{x} \in \myKerMath, \qquad \myBSpkMath \geq 
%1+ \myparanthese{d_{max} M_{q,p}\myparanthese{\myPhi} \frac{\mynorm{\boldsymbol{x}}_{\boldsymbol{w};q,1}}{\mynorm{\boldsymbol{x}}_{\boldsymbol{w};p,1}}}^{-1},
%\end{equation*}
%whereas for equally-sized blocks, i.e., $d_1 \seq \cdots \seq d_K \seq d$, it reduces to:
%\begin{equation*}
%\label{eq:BS-BMIC}
%\forall \boldsymbol{x} \in \myKerMath, \qquad \myBSpkMath \geq 
%1+ \myparanthese{d^{1+\frac1p-\frac1q} M_{q,p}\myparanthese{\myPhi} \frac{\mynorm{\boldsymbol{x}}_{q,1}}{\mynorm{\boldsymbol{x}}_{p,1}}}^{-1}.
%\end{equation*}
%\end{lemma}
%The proof of Lemma \ref{lm:Block Spark Inequality} is provided in Section \ref{prf:Block Spark Inequality}.
%
%For $d_1 \seq \cdots \seq d_K \seq 1$, Lemma \ref{lm:Block Spark Inequality} is equivalent to the conventional element-wise counterpart explained in (\ref{eq:S-M}), i.e., $\mySpkMath \sgeq 1 \spl M^{-1}(\myPhi)$.
%\begin{remark}
%Similar to the element-wise characterisations, the higher the lower bound of the $\myBSpkTxt$, the weaker, the recovery conditions that can be established.
%\end{remark}
%The proposed lower bound for the $\myBSpkTxt$ in Lemma \ref{lm:Block Spark Inequality} has two degrees of freedom of $p$ and $q$ which, according to the problem, can be regulated.
%%In the following, to investigate its behaviour, three cases of $0 \sless q \sless p$, $q \seq p$, and $p \sless q$ are considered, where, in all cases we have $p \sgeq 1$ and $q \sgeq 1$.
%According to the lower-bound of the fractions $\Vert \boldsymbol{x} \Vert_{\boldsymbol{w};p,1}/ \Vert\boldsymbol{x} \Vert_{\boldsymbol{w};q,1}$ and $\Vert \boldsymbol{x} \Vert_{p,1}/ \Vert\boldsymbol{x} \Vert_{q,1}$ in Lemma \ref{lm:Block Spark Inequality}, the most pessimistic bounds are resulted in the following corollary:
%%To better determine the lower bound of the $\myBSpkMath$ in Lemma \ref{lm:Block Spark Inequality}, Lemma \ref{lm:FractionBound} can be used, which results in the following corollary:
%\begin{corollary}
%\label{crl:Block Spark Inequality} 
%%For $0 \sless q \sless p$, 
%The bound of inequality of Lemma \ref{lm:Block Spark Inequality} is obtained as follows:
%%The inequality of Lemma \ref{lm:Block Spark Inequality} in the most pessimistic and optimistic cases can be rewritten as:
%\begin{equation*}
%\begin{aligned}
%\myBSpkMath &\geq 
%\min_{\boldsymbol{x} \in \myKerMath} 1+ \myparanthese{d_{max} M_{q,p}\myparanthese{\myPhi} \frac{\mynorm{\boldsymbol{x}}_{\boldsymbol{w};q,1}}{\mynorm{\boldsymbol{x}}_{\boldsymbol{w};p,1}}}^{-1} \\
%&\geq \min_{\boldsymbol{x}} 1+ \myparanthese{d_{max} M_{q,p}\myparanthese{\myPhi} \frac{\mynorm{\boldsymbol{x}}_{\boldsymbol{w};q,1}}{\mynorm{\boldsymbol{x}}_{\boldsymbol{w};p,1}}}^{-1} \\
%&= 1+ \myparanthese{d_{max} M_{q,p}\myparanthese{\myPhi} \max_k d_k^{\frac1p - \frac1q} \max \mybrace{1 , d_k^{\frac1q - \frac1p}}}^{-1}, \\
%%\text{Optimistic} &: \quad \myBSpkMath \geq 
%%1+ \myparanthese{d_{max} M_{q,p}\myparanthese{\myPhi} \min_k d_k^{\frac1p - \frac1q} \min \mybrace{1 , d_k^{\frac1q - \frac1p}}}^{-1}.
%\end{aligned}
%\end{equation*}
%%whereas in its most optimistic case:
%%\begin{equation*}
%%\myBSpkMath \geq 
%%1+ \myparanthese{d_{max}^{1+\frac1p-\frac1q} M_{q,p}\myparanthese{\myPhi}}^{-1}.
%%\end{equation*}
%%For $0 \sless q \seq p$, two cases are equivalent.
%whereas for equally-sized blocks, i.e., $d_1 \seq \cdots \seq d_K \seq d$, we have:
%%the most pessimistic and optimistic cases reduce to:
%\begin{equation*}
%\begin{aligned}
%\myBSpkMath &\geq
%\min_{\boldsymbol{x} \in \myKerMath} 1+ \myparanthese{d^{1+\frac1p-\frac1q} M_{q,p}\myparanthese{\myPhi} \frac{\mynorm{\boldsymbol{x}}_{q,1}}{\mynorm{\boldsymbol{x}}_{p,1}}}^{-1} \\
%&\geq \min_{\boldsymbol{x}} 1+ \myparanthese{d^{1+\frac1p-\frac1q} M_{q,p}\myparanthese{\myPhi} \frac{\mynorm{\boldsymbol{x}}_{q,1}}{\mynorm{\boldsymbol{x}}_{p,1}}}^{-1} \\
%&= 1+ \myparanthese{d^{1+\frac1p-\frac1q} M_{q,p}\myparanthese{\myPhi} \max \mybrace{1 , d^{\frac1q - \frac1p}}}^{-1}.
%%\text{Pessimistic} &: \quad \myBSpkMath \geq 
%%1+ \myparanthese{d^{1+\frac1p-\frac1q} M_{q,p}\myparanthese{\myPhi} \max \mybrace{1 , d^{\frac1q - \frac1p}}}^{-1}, \\
%%\text{Optimistic} &: \quad \myBSpkMath \geq 
%%1+ \myparanthese{d^{1+\frac1p-\frac1q} M_{q,p}\myparanthese{\myPhi} \min \mybrace{1 , d^{\frac1q - \frac1p}}}^{-1}.
%\end{aligned}
%\end{equation*}
%\end{corollary}
%\begin{proof}
%The closed-form lower-bound is obtained using the lower-bound of the fractions $\Vert \boldsymbol{x} \Vert_{\boldsymbol{w};p,1}/ \Vert\boldsymbol{x} \Vert_{\boldsymbol{w};q,1}$ and $\Vert \boldsymbol{x} \Vert_{p,1}/ \Vert\boldsymbol{x} \Vert_{q,1}$ in Lemma \ref{lm:Block Spark Inequality}.
%%The most pessimistic and optimistic cases are derived using the bounds of the fraction $\Vert \boldsymbol{x} \Vert_{\boldsymbol{w};p,1}/ \Vert\boldsymbol{x} \Vert_{\boldsymbol{w};q,1}$ in Lemma \ref{lm:Block Spark Inequality}.
%%These bounds are provided in Lemma \ref{lm:FractionBound}.
%The mentioned lower-bound of a division of two mixed-norms is provided in the following property:
%\end{proof}
%%\begin{remark}
%%Notice that throughout this study for $q \seq p$, the pessimistic and optimistic cases are equivalent.
%%\end{remark}
%\begin{property}[Bounds of division of two mixed-norms]
%\label{lm:FractionBound} %$0 \sless q \sless p$, 
%For $\boldsymbol{x} \seq \mybracket{\boldsymbol{x}\mybracket{1},\ldots,\boldsymbol{x}\mybracket{K}}$ and $\forall k ,\, \boldsymbol{x}\mybracket{k} \ssin \mathbb{R}^{d_k}$, and for weighted mixed-norms we have: % \cite{Golub2013}
%\begin{gather*}
%\begin{aligned}
%&\frac{\mynorm{\boldsymbol{x}}_{\boldsymbol{w};p,1}}{\mynorm{\boldsymbol{x}}_{\boldsymbol{w};q,1}} \geq
%\min_k \, \min \mybrace{1 , d_k^{\frac1q - \frac1p}} = 
%\begin{cases}
%\begin{aligned}
%  &1, \quad &&\text{if } q \leq p \\
%  &d_{max}^{\frac1q - \frac1p}, \quad &&\text{if } q > p
%\end{aligned}
%\end{cases}, \\
%&\frac{\mynorm{\boldsymbol{x}}_{\boldsymbol{w};p,1}}{\mynorm{\boldsymbol{x}}_{\boldsymbol{w};q,1}} \leq
%\max_k \, \max \mybrace{1 , d_k^{\frac1q - \frac1p}} = 
%\begin{cases}
%\begin{aligned}
%  &d_{max}^{\frac1q - \frac1p}, \quad &&\text{if } q \leq p \\
%  &1, \quad &&\text{if } q > p
%\end{aligned}
%\end{cases}.
%\end{aligned}
%\end{gather*}
%Similarly, for non-weighted mixed-norms we have: % \cite{Golub2013}
%\begin{gather*}
%\begin{aligned}
%&\frac{\mynorm{\boldsymbol{x}}_{p,1}}{\mynorm{\boldsymbol{x}}_{q,1}} \geq
%\min_k \, \min \mybrace{1 , d_k^{\frac1p - \frac1q}} = 
%\begin{cases}
%\begin{aligned}
%  &d_{max}^{\frac1p - \frac1q}, \quad &&\text{if } q \leq p \\
%  &1, \quad &&\text{if } q > p
%\end{aligned}
%\end{cases}, \\
%&\frac{\mynorm{\boldsymbol{x}}_{p,1}}{\mynorm{\boldsymbol{x}}_{q,1}} \leq
%\max_k \, \max \mybrace{1 , d_k^{\frac1p - \frac1q}} = 
%\begin{cases}
%\begin{aligned}
%  &1, \quad &&\text{if } q \leq p \\
%  &d_{max}^{\frac1p - \frac1q}, \quad &&\text{if } q > p
%\end{aligned}
%\end{cases}.
%\end{aligned}
%\end{gather*}
%\end{property}
%\begin{proof}
%For any real fractions $a_1/b_1,\,\ldots,\,a_K/b_K$ with positive denominators, we have \cite{pahio2005}:
%\begin{gather*}
%\min\mybrace{\frac{a_1}{b_1},\,\ldots,\,\frac{a_K}{b_K}} \leq \frac{a_1+\ldots+a_K}{b_1+\ldots+b_K} \leq \max\mybrace{\frac{a_1}{b_1},\,\ldots,\,\frac{a_K}{b_K}},
%\end{gather*}
%where, equality happens if and only if all fractions $a_1/b_1,\,\ldots,\,a_K/b_K$ are equal.
%Then, for any $k$, considering $a_k \seq \Vert \boldsymbol{x}[k] \Vert_p / d_k^{1/p}$ and $b_k \seq \Vert \boldsymbol{x}[k] \Vert_q / d_k^{1/q}$, we have $\Vert \boldsymbol{x} \Vert_{\boldsymbol{w};p,1} \seq \sum_k a_k$, $\Vert \boldsymbol{x} \Vert_{\boldsymbol{w};q,1} \seq \sum_k b_k$, and $a_k / b_k \seq d_k^{1/q-1/p} \Vert \boldsymbol{x}[k] \Vert_p {/} \Vert \boldsymbol{x}[k] \Vert_q$.
%On the other hand, from Property \ref{prp:VectorDivisionBound}, we have $\forall \boldsymbol{c} \ssin \mathbb{R}^d, \min \{1 , d^{1/p - 1/q} \} \sleq \Vert \boldsymbol{c} \Vert_p / \Vert \boldsymbol{c} \Vert_q \sleq \max \{1 , d^{1/p - 1/q} \}$. 
%%for $0 \sless q \sless p$, and $\boldsymbol{a} \ssin \mathbb{R}^{d}$ we have $1 \sleq \Vert\boldsymbol{a} \Vert_q {/} \Vert \boldsymbol{a} \Vert_p \sleq d^{1/q \sm 1/p}$ \cite{Golub2013}. 
%%Then, similarly, we have $d^{1/p \sm 1/q} \sleq \Vert \boldsymbol{a} \Vert_p {/} \Vert \boldsymbol{a} \Vert_q \sleq 1$.
%Consequently, $\min \{1 , d_k^{1/q - 1/p} \} \sleq a_k / b_k \sleq \max \{1 , d_k^{1/q - 1/p} \}$, which
%proves the bounds.
%Similarly, for any $k$, considering $a_k \seq \Vert \boldsymbol{x}[k] \Vert_p$ and $b_k \seq \Vert \boldsymbol{x}[k] \Vert_q$, the bounds for division of two non-weighted mixed-norms can be obtained.
%\end{proof}
%%The upper-bound explains the choice of $d_k^{-1/p} d_{k'}^{1/q}$ in Definition \ref{def:BMIC}.
%%For $1 \sleq q \seq p$, the proof is straightforward.
%%As described in the table \ref{table:OperatorNorm} on page \pageref{table:OperatorNorm}, for $0 \sless q \sleq p$ the operator-norm of the Block-MCC$_{q,p}$ is computationally feasible.
%For the case $1 \sleq p \sless q$, minimum value of the fraction $\Vert \boldsymbol{x} \Vert_{\boldsymbol{w};p,1} / \Vert \boldsymbol{x} \Vert_{\boldsymbol{w};q,1}$ in Lemma \ref{lm:Block Spark Inequality} is achieved through $p \seq 1$, and $q\seq \infty$, which is $d_{max}^{-1}$ according to Property \ref{lm:FractionBound}, but as described in the table \ref{table:OperatorNorm} on page \pageref{table:OperatorNorm}, the corresponding operator-norm in Block-MCC$_{q,p}$ is not computationally feasible.
%On the other hand, for the case $1 \sleq q \sleq p$, minimum value is achieved through $p \seq \infty$, and $q \seq 1$, which is $1$ according to Property \ref{lm:FractionBound}, whereas the corresponding operator-norm in Block-MCC$_{q,p}$ is computationally feasible.
%
%Now we can restate the block-sparse uncertainty principle based on $\myBSpkTxt$ proposed in Lemma \ref{lm:BUP-BS} in terms of the Block-MCC$_{q,p}$.
%The required relationship between $\myBSpkTxt$ and Block-MCC$_{q,p}$ is investigated in Lemma \ref{lm:Block Spark Inequality}.
%The following lemma introduces \emph{block-sparse uncertainty principle based on Block-MCC$_{q,p}$}:
%\begin{lemma}[Block-UP\footnote{\emph{Block-sparse Uncertainty Principle}} based on Block-MCC$_{q,p}$]
%\label{lm:BUP-BMIC}
%For any general dictionary $\myPhi$ with Block-MCC$_{q,p}$ $M_{q,p}(\myPhi)$ and for any arbitrary non-zero signal $\boldsymbol{y}$ with two distinct representations $\mybetaz$ and $\mybetao$, the following inequalities hold true $\forall p \sgeq 1$, $\forall q \sgeq 1$, and $\forall r \sgeq 0$:
%\begin{equation*}
%\forall \boldsymbol{x} \ssin \myKerMath, \qquad \mynorm{\mybetaz}_{r,0}+\mynorm{\mybetao}_{r,0} \geq 
%1+ \myparanthese{d_{max} M_{q,p}\myparanthese{\myPhi} \frac{\mynorm{\boldsymbol{x}}_{\boldsymbol{w};q,1}}{\mynorm{\boldsymbol{x}}_{\boldsymbol{w};p,1}}}^{-1},
%\end{equation*}
%whereas for equally-sized blocks, i.e., $d_1 \seq \cdots \seq d_K \seq d$, we have:
%\begin{equation*}
%\forall \boldsymbol{x} \ssin \myKerMath, \qquad \mynorm{\mybetaz}_{r,0}+\mynorm{\mybetao}_{r,0} \geq 
%1+ \myparanthese{d^{1+\frac1p-\frac1q} M_{q,p}\myparanthese{\myPhi} \frac{\mynorm{\boldsymbol{x}}_{q,1}}{\mynorm{\boldsymbol{x}}_{p,1}}}^{-1}.
%\end{equation*}
%\end{lemma}
%\begin{proof}
%This follows from Lemma \ref{lm:BUP-BS} and Lemma \ref{lm:Block Spark Inequality}.
%\end{proof}
%For $d_1 \seq \cdots \seq d_K \seq 1$, Lemma \ref{lm:BUP-BMIC} converges to its conventional element-wise counterpart explained in (\ref{eq:S-M}), i.e., $\Vert \mybetaz \Vert_0 + \Vert \mybetao \Vert_0 \sgeq 1 \spl M^{-1}(\myPhi)$.
%Again, according to the lower-bound of the fractions $\Vert \boldsymbol{x} \Vert_{\boldsymbol{w};p,1}/ \Vert\boldsymbol{x} \Vert_{\boldsymbol{w};q,1}$ and $\Vert \boldsymbol{x} \Vert_{p,1}/ \Vert\boldsymbol{x} \Vert_{q,1}$ we can restate Lemma \ref{lm:BUP-BMIC} as follows:
%\begin{corollary}
%\label{crl:BUP-BMIC}
%%For $0 \sless q \sless p$, 
%%The inequality of Lemma \ref{lm:BUP-BMIC} in the most pessimistic and optimistic cases can be rewritten as:
%The bound of inequality of Lemma \ref{lm:BUP-BMIC} is obtained as follows:
%\begin{equation*}
%\begin{aligned}
%\mynorm{\mybetaz}_{r,0} + \mynorm{\mybetao}_{r,0} &\geq 
%\min_{\boldsymbol{x} \in \myKerMath} 1+ \myparanthese{d_{max} M_{q,p}\myparanthese{\myPhi} \frac{\mynorm{\boldsymbol{x}}_{\boldsymbol{w};q,1}}{\mynorm{\boldsymbol{x}}_{\boldsymbol{w};p,1}}}^{-1} \\
%&\geq \min_{\boldsymbol{x}} 1+ \myparanthese{d_{max} M_{q,p}\myparanthese{\myPhi} \frac{\mynorm{\boldsymbol{x}}_{\boldsymbol{w};q,1}}{\mynorm{\boldsymbol{x}}_{\boldsymbol{w};p,1}}}^{-1} \\
%&= 1+ \myparanthese{d_{max} M_{q,p}\myparanthese{\myPhi} \max_k d_k^{\frac1p - \frac1q} \max \mybrace{1 , d_k^{\frac1q - \frac1p}}}^{-1}, \\
%\end{aligned}
%\end{equation*}
%whereas for equally-sized blocks, i.e., $d_1 \seq \cdots \seq d_K \seq d$, we have:
%\begin{equation*}
%\begin{aligned}
%\mynorm{\mybetaz}_{r,0}+\mynorm{\mybetao}_{r,0} &\geq
%\min_{\boldsymbol{x} \in \myKerMath} 1+ \myparanthese{d^{1+\frac1p-\frac1q} M_{q,p}\myparanthese{\myPhi} \frac{\mynorm{\boldsymbol{x}}_{q,1}}{\mynorm{\boldsymbol{x}}_{p,1}}}^{-1} \\
%&\geq \min_{\boldsymbol{x}} 1+ \myparanthese{d^{1+\frac1p-\frac1q} M_{q,p}\myparanthese{\myPhi} \frac{\mynorm{\boldsymbol{x}}_{q,1}}{\mynorm{\boldsymbol{x}}_{p,1}}}^{-1} \\
%&= 1+ \myparanthese{d^{1+\frac1p-\frac1q} M_{q,p}\myparanthese{\myPhi} \max \mybrace{1 , d^{\frac1q - \frac1p}}}^{-1}.
%\end{aligned}
%\end{equation*}
%\iffalse
%\begin{equation*}
%\begin{aligned}
%\text{Pessimistic} &: \quad \mynorm{\mybetaz}_{r,0}+\mynorm{\mybetao}_{r,0} \geq 
%1+ \myparanthese{d_{max} M_{q,p}\myparanthese{\myPhi} \max_k d_k^{\frac1p - \frac1q} \max \mybrace{1 , d_k^{\frac1q - \frac1p}}}^{-1}, \\
%\text{Optimistic} &: \quad \mynorm{\mybetaz}_{r,0}+\mynorm{\mybetao}_{r,0} \geq 
%1+ \myparanthese{d_{max} M_{q,p}\myparanthese{\myPhi} \min_k d_k^{\frac1p - \frac1q} \min \mybrace{1 , d_k^{\frac1q - \frac1p}}}^{-1}.
%\end{aligned}
%\end{equation*}
%whereas for equally-sized blocks, i.e., $d_1 \seq \cdots \seq d_K \seq d$, the most pessimistic and optimistic cases reduce to:
%\begin{equation*}
%\begin{aligned}
%\text{Pessimistic} &: \quad \mynorm{\mybetaz}_{r,0}+\mynorm{\mybetao}_{r,0} \geq
%1+ \myparanthese{d^{1+\frac1p-\frac1q} M_{q,p}\myparanthese{\myPhi} \max \mybrace{1 , d^{\frac1q - \frac1p}}}^{-1}, \\
%\text{Optimistic} &: \quad \mynorm{\mybetaz}_{r,0}+\mynorm{\mybetao}_{r,0} \geq 
%1+ \myparanthese{d^{1+\frac1p-\frac1q} M_{q,p}\myparanthese{\myPhi} \min \mybrace{1 , d^{\frac1q - \frac1p}}}^{-1}.
%\end{aligned}
%\end{equation*}
%\fi
%\end{corollary}
%\begin{proof}
%This follows from Lemma \ref{lm:BUP-BS} and Corollary \ref{crl:Block Spark Inequality}.
%\end{proof}
%\fi
Now, we introduce the following \emph{Block-ERC based on Block-MCC$_{q,p}$}:
\newpage
\begin{tcolorbox}
\begin{theorem}[Block-ERC based on Block-MCC$_{q,p}$]
\label{th:BERC-BMIC}
For any dictionary $\myPhi$ \myhl{with full column rank blocks, let $\boldsymbol{y} \seq \myPhi \mybetaz$,} then supposing Block-MCC$_{q,p}$ $M_{q,p}(\myPhi)$, \myhl{Block-Support relation $S_b(\mybetaz) {\subset} S_b(\mybeta)$,} $\forall (q,p) \ssin \mathbb{R}^2_{\sgeq 1}$, \myhl{and $\forall r \ssin \mathbb{R}_{\sgeq 0}$, if}
\begin{equation*}
\mynorm{\mybetaz}_{r,0} < 
\frac{1 + \myparanthese{d_{max} M_{q,p}\myparanthese{\myPhi} }^{-1} \mycolor{\displaystyle\min_{k} \, \min \mybrace{1 , d_k^{\frac1q - \frac1p}}}}{2},
\end{equation*}
then $\mybetaz$ is the unique solution to the \myhl{$P_{\boldsymbol{w};q,1}$} problem.
\end{theorem}
\end{tcolorbox}
The proof of Theorem \ref{th:BERC-BMIC} is provided in Section \ref{prf:BERC-BMIC} (page \pageref{prf:BERC-BMIC}).
%\iffalse
%\begin{proof}
%Similar to proof of Block-ERC based on $\myBSpkTxt$ in Theorem \ref{th:BERC-BS}, and using block-sparse uncertainty principle based on Block-MCC$_{q,p}$ in Lemma \ref{lm:BUP-BMIC}, instead of block-sparse uncertainty principle based on $\myBSpkTxt$ in Lemma \ref{lm:BUP-BS}.
%\end{proof}
%\fi

For $d_1 \seq \cdots \seq d_K \seq 1$, Theorem \ref{th:BERC-BMIC} converges to its conventional element-wise counterpart explained in (\ref{eq:ERC-M}) on page \pageref{eq:ERC-M}, i.e., $\Vert \mybetaz \Vert_0 \sless (1 \spl M^{-1} (\myPhi))/2$.

As mentioned before (page \pageref{txt:BlockSL}), the right-hand side of the equation in Theorem \ref{th:BERC-BMIC} is called $\myBSLTxt$ and is represented as $\myBSLMath$.
In order to make a comparison of $\myBSLTxt$ of Theorem \ref{th:BERC-BMIC}, for different basic tractable ($q,p$) pairs of table \ref{table:OperatorNorm} (page \pageref{table:OperatorNorm}), let us represent it as $\myBSLqpMath$ and call it $({q,p})$-Block-Sparsity Level ($\myBSLqpTxt$).
In the following property we investigate the possible relationship between different $\myBSLqpTxt$:
\begin{property}[$\boldsymbol{\myBSLqpTxt}$ inequalities]
\label{prp:BSLqp-relationships}
The different $\myBSLqpTxt$ calculated for basic tractable $\ell_{q {\to} p}$ operator-norms of table \ref{table:OperatorNorm} (page \pageref{table:OperatorNorm}) have the following relationships:
\begin{equation*}
\begin{aligned}
Block{-}SL_{1,\infty}\myparanthese{\myPhi} &\leq Block{-}SL_{1,2}\myparanthese{\myPhi} \leq Block{-}SL_{1,1}\myparanthese{\myPhi}, \\
Block{-}SL_{1,\infty}\myparanthese{\myPhi} &\leq Block{-}SL_{1,2}\myparanthese{\myPhi} \leq Block{-}SL_{2,2}\myparanthese{\myPhi}, \\
Block{-}SL_{1,\infty}\myparanthese{\myPhi} &\leq Block{-}SL_{2,\infty}\myparanthese{\myPhi} \leq Block{-}SL_{2,2}\myparanthese{\myPhi}, \\
Block{-}SL_{1,\infty}\myparanthese{\myPhi} &\leq Block{-}SL_{2,\infty}\myparanthese{\myPhi} \leq Block{-}SL_{\infty,\infty}\myparanthese{\myPhi},
\end{aligned}
\end{equation*}
which is shown in figure \ref{fig:BSLqp_Inequalities}(a), while their general relationship for equally-sized case, i.e., $d_1 \seq \cdots \seq d_K \seq d$, is represented schematically in figure \ref{fig:BSLqp_Inequalities}(b). 
\begin{figure}[hb]
\centering
\includegraphics[width=.8\textwidth,keepaspectratio]{images/BSLqp_Inequalities.png} 
\centering
\caption{$\myBSLqpTxt$ inequalities for (a) differently-sized, and (b) equally-sized block structure, for common $\ell_{q {\to} p}$ operator-norms according to table \ref{table:OperatorNorm} (page \pageref{table:OperatorNorm}).}
\label{fig:BSLqp_Inequalities}
\end{figure}
\end{property}
The proof of Property \ref{prp:BSLqp-relationships} is provided in Section \ref{prf:BSLqp-relationships} (page \pageref{prf:BSLqp-relationships}).

Now we need to prove the claim of improved recovery conditions in the block-wise domain through comparing sparsity level with block-sparsity level. 
To this end, supposing that $\mybetaz$ has a block-sparse structure and satisfies the condition of Theorem \ref{th:BERC-BMIC} (Block-ERC based on Block-MCC$_{q,p}$), and considering the fact that $\forall r \sgeq 0$, $\Vert \mybetaz \Vert_{0} \sleq d_{max} \, \Vert \mybetaz \Vert_{r,0}$, we have:
\begin{equation*}
%\label{eq:DontKnow2}
\forall (q,p) \in \mathbb{R}^2_{\geq 1}, \qquad
\mynorm{\mybetaz}_{0} < 
\frac{d_{max} + M_{q,p}^{-1}\myparanthese{\myPhi} \displaystyle\min_{k} \, \min \mybrace{1 , d_k^{\frac1q - \frac1p}}}{2}.
\end{equation*}
Comparing with $\Vert \mybetaz \Vert_0 \sless (1 \spl M^{-1} (\myPhi))/2$ in (\ref{eq:ERC-M}) on page \pageref{eq:ERC-M}, we see that the relationship between $d_{max} \spl M_{q,p}^{-1}(\myPhi) \min_{k} \, \min \{1 , d_k^{1/q \sm 1/p} \}$ and $1 \spl M^{-1}(\myPhi)$ should be investigated. 
We show this relationship in the following property, which the proof is provided in Section \ref{prf:DontKnow1} (page \pageref{prf:DontKnow1}):
%\iffalse
%\begin{property}[SL\footnote{Sparsity Level} v.s. $\boldsymbol{\myBSLTxt}$]
%\label{prp:DontKnow1} % $\forall \boldsymbol{x} \ssin \myKerMath$
%For a dictionary $\myPhi$ with Block-MCC$_{q,p}$ $M_{q,p}(\myPhi)$ and the small enough MCC $M(\myPhi)$, 
%%$\forall p \sgeq 1$, and $\forall q \sgeq 1$
%we have: 
%\begin{gather*}
%%\forall (q,p) \in \mathbb{R}^2_{\geq 1}, \qquad
%d_{max} + M_{q,p}^{-1}\myparanthese{\myPhi} \displaystyle\min_{k} \, \min \mybrace{1 , d_k^{\frac1q - \frac1p}}
%\geq 1 + M^{-1}\myparanthese{\myPhi},
%\end{gather*}
%where, in the case of a dictionary with full column rank blocks, we assume the following conditions:
%\begin{gather*}
%q \geq p \geq 1, \qquad 
%M\myparanthese{\myPhi} \leq
%\min \mybrace{\frac{d_{max} ^{\frac1q - 2}}{\myparanthese{d_{max} - 1}^{\frac12}} , \frac{1 - d_{max} ^{\frac1q - \frac32}}{\myparanthese{d_{max} - 1}^{\frac12} \mybracket{\myparanthese{d_{max} - 1}^{\frac12} - d_{max} ^ \frac12}}},
%\end{gather*}
%whereas in the case of intra-block orthonormality, $\forall (q , p , q' , p') \ssin \mathbb{R}^4_{\sg 0}$
%\iffalse
%\begin{gather*}
%M\myparanthese{\myPhi} \leq
%\frac{1 - d_{max}^{\frac1p - \frac32}  \displaystyle\min_k \, \min \mybrace{1 , d_k^{\frac1q - \frac1p}}}{d_{max} - 1},
%\end{gather*}
%\fi
%\begin{gather*}
%\begin{aligned}
%M\myparanthese{\myPhi} \leq
%\frac{1 - \frac{d_{max} \, \displaystyle\min_{k} \, \min \mybrace{1 , d_k^{\frac1q - \frac1p}}}{\displaystyle\max_{k,k' \neq k} d_{k}^{\frac12 - \frac1p} \, d_{k'}^{\frac1q + \frac12} \, \max \mybrace{1 , d_k^{\frac1p - \frac{1}{p'}}} \, \max \mybrace{1 , d_{k'}^{\frac{1}{q'} - \frac1q}} \, \max \mybrace{1 , d_k^{\frac{1}{p'} -\frac12}} \, \max \mybrace{1 , d_{k'}^{\frac{1}{2} - \frac{1}{q'}}}}}{d_{max} - 1}.
%\end{aligned}
%\end{gather*}
%should hold true.  
%\end{property}
%\fi
\begin{property}[\myhl{SL\footnote{Sparsity Level} v.s. $\boldsymbol{\myBSLqpTxt}$}]
\label{prp:DontKnow1} % $\forall \boldsymbol{x} \ssin \myKerMath$
For a dictionary $\myPhi$ with intra-block orthonormality, and Block-MCC$_{q,p}$ $M_{q,p}(\myPhi)$, $\forall (q , p , q' , p') \ssin \mathbb{R}^4_{\sg 0}$ if the MCC $M(\myPhi)$ is small enough, i.e., 
%$\forall p \sgeq 1$, and $\forall q \sgeq 1$
%\iffalse
%we have: 
%\begin{gather*}
%%\forall (q,p) \in \mathbb{R}^2_{\geq 1}, \qquad
%d_{max} + M_{q,p}^{-1}\myparanthese{\myPhi} \displaystyle\min_{k} \, \min \mybrace{1 , d_k^{\frac1q - \frac1p}}
%\geq 1 + M^{-1}\myparanthese{\myPhi},
%\end{gather*}
%where, in the case of a dictionary with full column rank blocks, we assume the following conditions:
%\begin{gather*}
%q \geq p \geq 1, \qquad 
%M\myparanthese{\myPhi} \leq
%\min \mybrace{\frac{d_{max} ^{\frac1q - 2}}{\myparanthese{d_{max} - 1}^{\frac12}} , \frac{1 - d_{max} ^{\frac1q - \frac32}}{\myparanthese{d_{max} - 1}^{\frac12} \mybracket{\myparanthese{d_{max} - 1}^{\frac12} - d_{max} ^ \frac12}}},
%\end{gather*}
%whereas in the case of intra-block orthonormality, $\forall (q , p , q' , p') \ssin \mathbb{R}^4_{\sg 0}$
%\iffalse
%\begin{gather*}
%M\myparanthese{\myPhi} \leq
%\frac{1 - d_{max}^{\frac1p - \frac32}  \displaystyle\min_k \, \min \mybrace{1 , d_k^{\frac1q - \frac1p}}}{d_{max} - 1},
%\end{gather*}
%\fi
%\fi
\begin{gather*}
\begin{aligned}
M\myparanthese{\myPhi} \leq
\frac{1 - \frac{d_{max} \, \displaystyle\min_{k} \, \min \mybrace{1 , d_k^{\frac1q - \frac1p}}}{\displaystyle\max_{k,k' \neq k} d_{k}^{\frac12 - \frac1p} \, d_{k'}^{\frac1q + \frac12} \, \max \mybrace{1 , d_k^{\frac1p - \frac{1}{p'}}} \, \max \mybrace{1 , d_{k'}^{\frac{1}{q'} - \frac1q}} \, \max \mybrace{1 , d_k^{\frac{1}{p'} -\frac12}} \, \max \mybrace{1 , d_{k'}^{\frac{1}{2} - \frac{1}{q'}}}}}{d_{max} - 1},
\end{aligned}
\end{gather*}
%should hold true.  
then the proposed sparsity level in Theorem \ref{th:BERC-BMIC} (block-ERC based on Block-MCC$_{q,p}$, page \pageref{th:BERC-BMIC}) is higher than the conventional sparsity level, i.e.,
\begin{gather*}
%\forall (q,p) \in \mathbb{R}^2_{\geq 1}, \qquad
d_{max} + M_{q,p}^{-1}\myparanthese{\myPhi} \displaystyle\min_{k} \, \min \mybrace{1 , d_k^{\frac1q - \frac1p}}
\geq 1 + M^{-1}\myparanthese{\myPhi}.
\end{gather*}
\end{property}
%The proof of Property \ref{prp:DontKnow1} is provided in Section \ref{prf:DontKnow1}.

\begin{remark}
\label{Rmrk:DontKnow1}
Therefore, exploiting the block structure information of the representation and using the proposed characterisation of the dictionary, named Block-MCC$_{q,p}$ (Definition \ref{def:BMIC}, page \pageref{def:BMIC}), improves the conventional MCC-based ERC of Donoho et al. \cite{Donoho2001,Donoho2003,Donoho2006a}, Gribonval and Nielsen \cite{Gribonval2003,Gribonval2007}, Tropp \cite{Tropp2004}, and Bruckstein et al. \cite{Bruckstein2009} by increasing the sparsity level, hence weakening the corresponding conditions.
The mentioned improvement is under condition on conventional MCC, which ensures that the dictionary is sufficiently incoherent, i.e., MCC is small enough.
\end{remark}
In order to have a clear sense of the upper-bound on the conventional MCC in Property \ref{prp:DontKnow1} (SL v.s. $\myBSLqpTxt$), this value is calculated in table \ref{table:MaxMToBeBetterMCC} for the basic tractable $(q,p)$ pairs of table \ref{table:OperatorNorm} (page \pageref{table:OperatorNorm}).
Although, for the MCC values less than the upper-bounds in table \ref{table:MaxMToBeBetterMCC}, the supremacy of the proposed sparsity level is ensured, but these bounds are very pessimistic and in practice for even higher values of MCC the supremacy of the proposed sparsity level can be observed.
\begin{table*}[tp]
\begin{adjustbox}{width=1\textwidth} % ,totalheight=\textheight,.5
\centering
%\tiny
%\resizebox{\textwidth}{!}{
\begin{tabular}{ccccccc}
\toprule
%\cline{2-4}
\multicolumn{1}{c}{$(q = q',p = p')$} &\multicolumn{1}{c}{$(1,1)$} & \multicolumn{1}{c}{$(1,2)$}  & \multicolumn{1}{c}{$(1,\infty)$} & \multicolumn{1}{c}{$(2,2)$} & \multicolumn{1}{c}{$(2,\infty)$} & \multicolumn{1}{c}{$(\infty,\infty)$}\\ \midrule %\hline
\multicolumn{1}{c}{$M(\myPhi) \leq$} &\multicolumn{1}{l}{$\frac{1}{d_{max}^\frac12 \myparanthese{d_{max}^\frac12 + 1}}$} & \multicolumn{1}{c}{$\frac{1}{d_{max}^\frac12 \myparanthese{d_{max}^\frac12 + 1}}$} & \multicolumn{1}{c}{$\frac{1}{d_{max}}$} &\multicolumn{1}{c}{$0$} &\multicolumn{1}{c}{$\frac{1}{d_{max}^\frac12 \myparanthese{d_{max}^\frac12 + 1}}$} & \multicolumn{1}{c}{$\frac{1}{d_{max}^\frac12 \myparanthese{d_{max}^\frac12 + 1}}$}   \\ \bottomrule %\hline
\end{tabular}
%}
\end{adjustbox}
\caption{Upper-bound of MCC ensuring the supremacy of the proposed sparsity level, for different basic values of $(q,p)$ pairs and for a dictionary with intra-block orthonormality.}
\label{table:MaxMToBeBetterMCC}
\end{table*}
\begin{remark}
\label{Rmrk:DontKnow2}
Table \ref{table:MaxMToBeBetterMCC} demonstrates that our proposed sparsity level for the $\ell_{2 {\to} 2}$ operator-norm, which is equal to the conventional $\ell_{2}$ matrix norm, requires the most strict upper-bound on $M(\myPhi)$ in order to be higher than the conventional sparsity level, while the dictionaries with characterisations that use operator-norms other than $\ell_{2 {\to} 2}$ can be less incoherent (more coherent, i.e., higher $M(\myPhi)$ is allowed), and especially for $\ell_{1 {\to} \infty}$ operator-norm the most relaxed upper-bound on $M(\myPhi)$ is achieved.
This is one of the advantages of utilising operator-norms, which enable us to have more relaxed conditions by exploiting norms other than the conventional $\ell_{2}$.
\end{remark}
%\iffalse
%Next, we rewrite the bound in Theorem \ref{th:BERC-BMIC}:
%% in two cases of pessimistic and optimistic: 
%%based on the bounds of $\Vert \boldsymbol{x} \Vert_{\overbar{p},1}/ \Vert\boldsymbol{x} \Vert_{\overbar{q},1}$ for $0 \sless q \sless p$:
%\begin{corollary}
%\label{crl:BERC-BMIC}
%%For $0 \sless q \sless p$, 
%The bound of inequality of Block-ERC based on Block-MCC$_{q,p}$ in Theorem \ref{th:BERC-BMIC} is obtained as follows:
%\begin{equation*}
%\begin{aligned}
%\mynorm{\mybetaz}_{r,0} &<
%\min_{\boldsymbol{x}} \frac{1+ \myparanthese{d_{max} M_{q,p}\myparanthese{\myPhi} \frac{\mynorm{\boldsymbol{x}}_{\boldsymbol{w};q,1}}{\mynorm{\boldsymbol{x}}_{\boldsymbol{w};p,1}}}^{-1}}{2} \\
%&= \frac{1+ \myparanthese{d_{max} M_{q,p}\myparanthese{\myPhi} \displaystyle\max_k d_k^{\frac1p - \frac1q} \max \mybrace{1 , d_k^{\frac1q - \frac1p}}}^{-1}}{2} \\
%&\leq \min_{\boldsymbol{x} \in \myKerMath} \frac{1+ \myparanthese{d_{max} M_{q,p}\myparanthese{\myPhi} \frac{\mynorm{\boldsymbol{x}}_{\boldsymbol{w};q,1}}{\mynorm{\boldsymbol{x}}_{\boldsymbol{w};p,1}}}^{-1}}{2}.
%\end{aligned}
%\end{equation*}
%whereas for equally-sized blocks, i.e., $d_1 \seq \cdots \seq d_K \seq d$, we have:
%\begin{equation*}
%\begin{aligned}
%\mynorm{\mybetaz}_{r,0} &<
%\min_{\boldsymbol{x}} \frac{1+ \myparanthese{d^{1 + \frac1p - \frac1q} M_{q,p}\myparanthese{\myPhi} \frac{\mynorm{\boldsymbol{x}}_{q,1}}{\mynorm{\boldsymbol{x}}_{p,1}}}^{-1}}{2} \\
%&= \frac{1+ \myparanthese{d^{1+\frac1p-\frac1q} M_{q,p}\myparanthese{\myPhi} \max \mybrace{1 , d^{\frac1q - \frac1p}}}^{-1}}{2} \\
%&\leq \min_{\boldsymbol{x} \in \myKerMath} \frac{1+ \myparanthese{d^{1 + \frac1p - \frac1q} M_{q,p}\myparanthese{\myPhi} \frac{\mynorm{\boldsymbol{x}}_{q,1}}{\mynorm{\boldsymbol{x}}_{p,1}}}^{-1}}{2}.
%\end{aligned}
%\end{equation*}
%\end{corollary}
%\begin{proof}
%Similar to proof of Block-ERC based on $\myBSpkTxt$ in Theorem \ref{th:BERC-BS}, and using Corollary \ref{crl:BUP-BMIC} instead of block-sparse uncertainty principle based on $\myBSpkTxt$.
%\end{proof}
%
%As it can be seen in Corollary \ref{crl:BERC-BMIC}, the block-sparsity level can be obtained through a minimisation problem, which can be approximated by a closed-form lower-bound, i.e., $(1 \spl (d_{max} M_{q,p}(\myPhi) \max_k d_k^{1/p \sm 1/q} \max \{1 , d_k^{1/q \sm 1/p} \} )^{-1} )/2$ and $(1 \spl (d^{1 \spl 1/p \sm 1/q} M_{q,p}(\myPhi) \max \{1 , d^{1/q \sm 1/p} \} )^{-1} )/ 2$, for differently and equally-sized block structures, respectively.
%\fi
Now Block-ERC of Eldar et al., i.e., $\Vert \mybetaz \Vert_{2,0} \sless (1 \spl (d M^{Eldar}_{Inter}(\myPhi))^{-1} (1 \sm (d \sm 1)M^{Eldar}_{Intra}(\myPhi)))/2$, explained in (\ref{BERC-Eldar}) on page \pageref{BERC-Eldar}, can be compared to our Block-ERC proposed in Theorem \ref{th:BERC-BMIC} (Block-ERC based on Block-MCC$_{q,p}$, page \pageref{th:BERC-BMIC}).
%In order to do that, we consider the proposed most pessimistic closed-form block-sparsity level.
Supposing that all blocks sharing the same block length $d$, it is clear that the relationship between $(M^{Eldar}_{Inter}(\myPhi))^{-1} (1 \sm (d \sm 1)M^{Eldar}_{Intra}(\myPhi))$ and $M_{q,p}^{-1}(\myPhi) \min \{1 , d^{1/q \sm 1/p} \} $ 
%in the most pessimistic case, and $(d^{1/p \sm 1/q} M_{q,p}(\myPhi) \min \{1 , d^{1/q \sm 1/p} \} )^{-1}$ in the most optimistic case 
should be investigated.
%First, in the following lemma, we show the lower-bound of $M_{q,p}^{-1}(\myPhi)$ to $(M^{Eldar}_{Inter}(\myPhi))^{-1} (1 \sm (d \sm 1)M^{Eldar}_{Intra}(\myPhi))$:
\begin{lemma}[\myhl{Eldar et al.\textquotesingle s v.s. proposed $\boldsymbol{\myBSLqpTxt}$}]
\label{lm:Eldar-BMIC} 
%$\forall p \sgeq 1$ 
%\iffalse
%Suppose $M^{Eldar}_{Inter}(\myPhi)$, $M^{Eldar}_{Intra}(\myPhi)$, and $M_{q,p}(\myPhi)$ are inter-block and intra-block coherence of Eldar et al., and our proposed Block-MCC$_{q,p}$, respectively. \\
%1) For a dictionary $\myPhi$ with full column rank blocks, and for $q \sgeq p \sgeq 1$, if: 
%\begin{gather*}
%M^{Eldar}_{Intra}\myparanthese{\myPhi} \leq 
%\frac{1 - \max \mybrace{1 , d^{\frac1p - \frac12}} \max \mybrace{1 , d^{\frac12 - \frac1q}}}{d^{2 - \frac1q} \myparanthese{d - 1}^\frac12 - \myparanthese{d - 1} \max \mybrace{1 , d^{\frac1p - \frac12}} \max \mybrace{1 , d^{\frac12 - \frac1q}}},
%\end{gather*}
%then, the Block-ERC proposed in Theorem \ref{th:BERC-BMIC} improves Block-ERC of Eldar et al., i.e., $d^{1/q \sm 1/p} M_{q,p}^{-1}(\myPhi) \sgeq (M^{Eldar}_{Inter}(\myPhi))^{-1} (1 \sm (d \sm 1)M^{Eldar}_{Intra}(\myPhi))$. \\
%\iffalse
%, whereas if 
%\begin{gather*}
%M^{Eldar}_{Intra}\myparanthese{\myPhi} \leq 
%\min \mybrace{\frac{1}{d^{2 - \frac1q} \myparanthese{d - 1}^\frac12} , \frac{1 - d^{\frac1q - \frac1p} \max \mybrace{1 , d^{\frac1p - \frac12}} \max \mybrace{1 , d^{\frac12 - \frac1q}}}{d^{2 - \frac1q} \myparanthese{d - 1}^\frac12 - \myparanthese{d - 1} d^{\frac1q - \frac1p} \max \mybrace{1 , d^{\frac1p - \frac12}} \max \mybrace{1 , d^{\frac12 - \frac1q}}}}
%\end{gather*}
%then, the Block-ERC proposed in Corollary \ref{crl:BERC-BMIC} in the most optimistic case improves Eldar's Block-ERC, i.e., $M_{q,p}^{-1}(\myPhi) \sgeq (M^{Eldar}_{Inter}(\myPhi))^{-1} (1 \sm (d \sm 1)M^{Eldar}_{Intra}(\myPhi))$.\\
%\fi
%\fi
For a dictionary with intra-block orthonormality, Block-ERC proposed in Theorem \ref{th:BERC-BMIC} (Block-ERC based on Block-MCC$_{q,p}$, page \pageref{th:BERC-BMIC}) for $q \seq p \seq 2$ as the best case is equal to Block-ERC of Eldar et al. (equation (\ref{BERC-Eldar}), page \pageref{BERC-Eldar}). 
\end{lemma}
%\FloatBarrier
The proof of Lemma \ref{lm:Eldar-BMIC} is provided in Section \ref{prf:Eldar-BMIC} (page \pageref{prf:Eldar-BMIC}).
%\iffalse
%\begin{remark}
%\label{Rmrk:Eldar-BMIC-Rmrk} 
%Lemma \ref{lm:Eldar-BMIC} demonstrates that for $q \sgeq p$, if $M^{Eldar}_{Intra}(\myPhi)$ is small enough, then the proposed Block-ERC in Theorem \ref{th:BERC-BMIC} improves Block-ERC of Eldar et al. in (\ref{BERC-Eldar}) through increasing the block-sparsity level.
%\end{remark}
%\fi
\begin{remark}
\label{Rmrk:Eldar-BMIC-equality}
It is worth mentioning that in dictionaries with intra-block orthonormality, by definition of intra-block coherence of Eldar et al., $M^{Eldar}_{Intra}(\myPhi)$ is equal to zero in Block-ERC of Eldar et al. (equation (\ref{BERC-Eldar}), page \pageref{BERC-Eldar}), i.e., $\Vert \mybetaz \Vert_{2,0} \sless (1 \spl (d M^{Eldar}_{Inter}(\myPhi))^{-1} (1 \sm (d \sm 1)M^{Eldar}_{Intra}(\myPhi)))/2 \seq (1 \spl (d M^{Eldar}_{Inter}(\myPhi))^{-1})/2$.
On the other hand, based on lemma \ref{lm:Eldar-BMIC}, the proposed Block-ERC for $q \seq p \seq 2$ is equivalent to the condition of Eldar and her co-workers.
Therefore, our proposed Block-ERC in Theorem \ref{th:BERC-BMIC} (page \pageref{th:BERC-BMIC}) in the following special setting, reduces to Block-ERC of Eldar et al.: 
\begin{itemize}
\item $q \seq p \seq 2$,
\item equally-sized blocks, i.e., $d_1 \seq \cdots \seq d_K \seq d$, and
\item intra-block orthonormality of dictionary, i.e., for $1 \sleq k \sleq K$, $\myPhi^T[k] \myPhi[k] \seq \boldsymbol{I}_d$.
\end{itemize}
%\iffalse
%\begin{itemize}
%\item $q \seq p \seq 2$, 
%\item equally-sized blocks, i.e., $d_1 \seq \cdots \seq d_K \seq d$, and
%\item intra-block orthonormality of dictionary, i.e., for $1 \sleq k \sleq K$, $\myPhi^T[k] \myPhi[k] \seq \boldsymbol{I}_d$.
%\end{itemize}
%\fi
In other words, in the mentioned special setting, \emph{theoretically and independent of the recovery algorithm, the same Block-ERC can be achieved}.
\end{remark}
%\iffalse
%In the above mentioned recovery condition based on the Block-MCC$_{q,p}$ in Theorem \ref{th:BERC-BMIC}, the corresponding optimisation problem $P_{r,0}$ is non-convex.
%Using Block-NSP, we propose the following recovery condition for convex problems:
%%\newpage
%\begin{tcolorbox}
%\begin{theorem}[Block-ERC based on Block-NSP]
%\label{th:BERC-BMIC-Q}
%For any general dictionary $\myPhi$ with Block-MCC$_{q,p}$ $M_{q,p}(\myPhi)$, characterisation $Q_{\boldsymbol{w};p_1,p_2}\myparanthese{S_b\myparanthese{\mybeta},\myPhi}$ defined in Block-NSP, $\forall p \sgeq 1$, $\forall q \sgeq 1$, and $\forall r \sgeq 0$, if 
%\begin{gather*}
%\forall \boldsymbol{x} \in \myKerMath, \qquad \mynorm{\mybeta}_{r,0} < 
%Q_{\boldsymbol{w};q,1}\myparanthese{S_b\myparanthese{\mybeta},\myPhi} +
%\myparanthese{2 \, d_{max} M_{q,p}\myparanthese{\myPhi} \frac{\mynorm{\boldsymbol{x}}_{\boldsymbol{w};q,1}}{\mynorm{\boldsymbol{x}}_{\boldsymbol{w};p,1}}}^{-1},
%\end{gather*}
%then $\mybeta$ is the unique solution to the $P_{\boldsymbol{w};p,1}$ problem.
%Notice that for equally-sized blocks, i.e., $d_1 \seq \cdots \seq d_K \seq d$, if
%\begin{gather*}
%\forall \boldsymbol{x} \in \myKerMath, \qquad \mynorm{\mybeta}_{r,0} < 
%Q_{q,1}\myparanthese{S_b\myparanthese{\mybeta},\myPhi} +
%\myparanthese{2 \, d^{1+\frac1p-\frac1q} M_{q,p}\myparanthese{\myPhi} \frac{\mynorm{\boldsymbol{x}}_{q,1}}{\mynorm{\boldsymbol{x}}_{p,1}}}^{-1},
%\end{gather*}
%then $\mybeta$ is the unique solution to the $P_{p,1}$ problem.
%\end{theorem}
%\end{tcolorbox}
%The proof of Theorem \ref{th:BERC-BMIC-Q} is provided in Section \ref{prf:BERC-BMIC-Q}.
%Again, in the following we rewrite the recovery condition in Theorem \ref{th:BERC-BMIC-Q} in terms of the bound of $\Vert \boldsymbol{x} \Vert_{\boldsymbol{w};p,1}/ \Vert\boldsymbol{x} \Vert_{\boldsymbol{w};q,1}$ and $\Vert \boldsymbol{x} \Vert_{p,1}/ \Vert\boldsymbol{x} \Vert_{q,1}$ introduced in Property \ref{lm:FractionBound}: % for $0 \sless q \sless p$:
%\begin{corollary}
%\label{crl:BERC-BMIC-Q}
%%For $1 \sleq q \sless p$, 
%The bound of recovery condition in Theorem \ref{th:BERC-BMIC-Q} 
%%in the most pessimistic and optimistic cases 
%can be obtained as follows:
%\begin{gather*}
%\begin{aligned}
%\mynorm{\mybetaz}_{r,0} 
%&< \min_{\boldsymbol{x}} Q_{\boldsymbol{w};q,1}\myparanthese{S_b\myparanthese{\mybeta},\myPhi} +
%\myparanthese{2 \, d_{max} M_{q,p}\myparanthese{\myPhi} \frac{\mynorm{\boldsymbol{x}}_{\boldsymbol{w};q,1}}{\mynorm{\boldsymbol{x}}_{\boldsymbol{w};p,1}}}^{-1} \\
%&= Q_{\boldsymbol{w};q,1}\myparanthese{S_b\myparanthese{\mybeta},\myPhi} +
%\myparanthese{2 \, d_{max} M_{q,p}\myparanthese{\myPhi} \max_k d_k^{\frac1p - \frac1q} \max \mybrace{1 , d_k^{\frac1q - \frac1p}}}^{-1} \\
%&\leq \min_{\boldsymbol{x} \in \myKerMath} Q_{\boldsymbol{w};q,1}\myparanthese{S_b\myparanthese{\mybeta},\myPhi} +
%\myparanthese{2 \, d_{max} M_{q,p}\myparanthese{\myPhi} \frac{\mynorm{\boldsymbol{x}}_{\boldsymbol{w};q,1}}{\mynorm{\boldsymbol{x}}_{\boldsymbol{w};p,1}}}^{-1}.
%%\text{Pessimistic} &: \quad \mynorm{\mybetaz}_{r,0} <
%%Q_{\boldsymbol{w};q,1}\myparanthese{S_b\myparanthese{\mybeta},\myPhi} +
%%\myparanthese{2 \, d_{max} M_{q,p}\myparanthese{\myPhi} \max_k d_k^{\frac1p - \frac1q} \max \mybrace{1 , d_k^{\frac1q - \frac1p}}}^{-1}, \\
%%\text{Optimistic} &: \quad  \mynorm{\mybetaz}_{r,0} <
%%Q_{\boldsymbol{w};q,1}\myparanthese{S_b\myparanthese{\mybeta},\myPhi} +
%%\myparanthese{2 \, d_{max} M_{q,p}\myparanthese{\myPhi} \min_k d_k^{\frac1p - \frac1q} \min \mybrace{1 , d_k^{\frac1q - \frac1p}}}^{-1}.
%\end{aligned}
%\end{gather*}
%whereas for equally-sized blocks, i.e., $d_1 \seq \cdots \seq d_K \seq d$, we have:
%%the most pessimistic and optimistic cases reduce to:
%\begin{equation*}
%\begin{aligned}
%\mynorm{\mybetaz}_{r,0} 
%&< \min_{\boldsymbol{x}} Q_{q,1}\myparanthese{S_b\myparanthese{\mybeta},\myPhi} +
%\myparanthese{2 \, d^{1+\frac1p-\frac1q} M_{q,p}\myparanthese{\myPhi} \frac{\mynorm{\boldsymbol{x}}_{q,1}}{\mynorm{\boldsymbol{x}}_{p,1}}}^{-1} \\
%&= Q_{q,1}\myparanthese{S_b\myparanthese{\mybeta},\myPhi} + \myparanthese{2 \, d^{1+\frac1p-\frac1q} M_{q,p}\myparanthese{\myPhi} \max \mybrace{1 , d^{\frac1q - \frac1p}}}^{-1} \\
%&\leq \min_{\boldsymbol{x} \in \myKerMath} Q_{q,1}\myparanthese{S_b\myparanthese{\mybeta},\myPhi} +
%\myparanthese{2 \, d^{1+\frac1p-\frac1q} M_{q,p}\myparanthese{\myPhi} \frac{\mynorm{\boldsymbol{x}}_{q,1}}{\mynorm{\boldsymbol{x}}_{p,1}}}^{-1}
%%\text{Pessimistic} &: \quad \mynorm{\mybetaz}_{r,0} <
%%Q_{q,1}\myparanthese{S_b\myparanthese{\mybeta},\myPhi} +
%%\myparanthese{2 \, d^{1+\frac1p-\frac1q} M_{q,p}\myparanthese{\myPhi} \max \mybrace{1 , d^{\frac1q - \frac1p}}}^{-1}, \\
%%\text{Optimistic} &: \quad \mynorm{\mybetaz}_{r,0} <
%%Q_{q,1}\myparanthese{S_b\myparanthese{\mybeta},\myPhi} +
%%\myparanthese{2 \, d^{1+\frac1p-\frac1q} M_{q,p}\myparanthese{\myPhi} \min \mybrace{1 , d^{\frac1q - \frac1p}}}^{-1}.
%\end{aligned}
%\end{equation*}
%\end{corollary}
%\begin{proof}
%Replacing the bound of $\Vert \boldsymbol{x} \Vert_{\boldsymbol{w};p,1}/\Vert \boldsymbol{x} \Vert_{\boldsymbol{w};q,1}$ and $\Vert \boldsymbol{x} \Vert_{p,1}/\Vert \boldsymbol{x} \Vert_{q,1}$ in Theorem \ref{th:BERC-BMIC-Q} utilising Property \ref{lm:FractionBound}, the proof is done.
%\end{proof}
%\begin{property}
%\label{prp:BERC-BMIC-Q-qq}
%The condition of Theorem \ref{th:BERC-BMIC-Q} in special setting of $q \seq p$ reduces to:
%\begin{gather*}
%\mynorm{\mybeta}_{r,0} < 
%\frac{1 + \myparanthese{d_{max} M_{p,p}\myparanthese{\myPhi}}^{-1}}{2}.
%\end{gather*}
%%whereas for equally-sized blocks, i.e., $d_1 \seq \cdots \seq d_K \seq d$:
%%\begin{gather*}
%%\forall r \geq 0, \quad \mynorm{\mybeta}_{r,0} < 
%%\frac{1 + \myparanthese{d \, M_{p,p}\myparanthese{\myPhi}}^{-1}}{2}.
%%\end{gather*}
%\end{property}
%\begin{proof}
%Continuing from (\ref{eq:DontKnow6}) in the proof of Lemma \ref{lm:Block Spark Inequality}, where, $q \seq p$, we get:
%\begin{gather*}
%\begin{aligned}
%d_{max} M_{p,p}\myparanthese{\myPhi} Q_{\boldsymbol{w};p,1} \mynorm{\boldsymbol{x}}_{\boldsymbol{w};p,1} + Q_{\boldsymbol{w};p,1} \mynorm{\boldsymbol{x}}_{\boldsymbol{w};p,1} &\leq 
% d_{max} M_{p,p}\myparanthese{\myPhi} \mynorm{\boldsymbol{x}}_{\boldsymbol{w};p,1} \mynorm{\mybeta}_{r,0}, \\
%Q_{\boldsymbol{w};p,1} &\leq 
%\frac{d_{max} M_{p,p}\myparanthese{\myPhi} \mynorm{\mybeta}_{r,0}}{d_{max} M_{p,p}\myparanthese{\myPhi} + 1} \\
%&< \frac12.
%\end{aligned}
%\end{gather*}
%The above last inequality comes from the Block-NSP condition in Theorem \ref{th:BNSP} and the proof is done.
%%Similarly, from (\ref{eq:DontKnow7}) the results for equally-sized blocks can be proved.
%\end{proof}
%For $d_1 \seq \cdots \seq d_K \seq 1$, Property \ref{prp:BERC-BMIC-Q-qq} converges to its conventional element-wise counterpart in (\ref{eq:ERC-M}), i.e., $\Vert \mybetaz \Vert_0 \sless (1 \spl M^{-1} (\myPhi))/2$.
%\begin{remark}[Equivalence]
%\label{Rmrk:P1P0-Equivalence} 
%From Corollary \ref{crl:BERC-BMIC} and Property \ref{prp:BERC-BMIC-Q-qq}, it can be deduced that for $1 \sleq q \seq p$ the 
%%both pessimistic and optimistic 
%condition of $\forall r \sgeq 0, \, \Vert \mybeta \Vert_{r,0} \sless (1 \spl d_{max}^{-1} M_{p,p}^{-1}(\myPhi))/2$, guarantees the uniqueness of the solution $\mybeta$ to both non-convex $P_{r,0}$ and convex $P_{\boldsymbol{w};p,1}$ problems.
%\end{remark}
%\fi
