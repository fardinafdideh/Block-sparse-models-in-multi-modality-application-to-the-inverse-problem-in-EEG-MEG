Table \ref{table:OperatorNorm} explains these basic operator-norms.
\begin{table*}[bp]
\begin{adjustbox}{width=\textwidth} % ,totalheight=\textheight,.5
\centering
%\tiny
\begin{tabular}{cccc}
\toprule
%\cline{2-4}
                              & \multicolumn{1}{c}{$p=1$}  & \multicolumn{1}{c}{$p=2$} & \multicolumn{1}{c}{$p=\infty$} \\ \midrule %\hline
\multicolumn{1}{l}{$q=1$} & \multicolumn{1}{c}{Maximum $\ell_1$ norm of a column} & \multicolumn{1}{c}{Maximum $\ell_2$ norm of a column} & Maximum absolute entry of matrix \\ %\cline{1-1}%\hline
\multicolumn{1}{l}{$q=2$} & \multicolumn{1}{c}{NP-hard} & \multicolumn{1}{c}{Maximum singular value} & Maximum $\ell_2$ norm of a row    \\ %\cline{1-1} %\hline
\multicolumn{1}{l}{$q=\infty$}   & \multicolumn{1}{c}{NP-hard} & \multicolumn{1}{c}{NP-hard} & Maximum $\ell_1$ norm of a row    \\ \bottomrule %\hline
\end{tabular}
\end{adjustbox}
\caption{Computational complexity of $\ell_{q \to p}$ operator-norm for different basic $(q,p)$ pairs \cite{Tropp2004b}.}
\label{table:OperatorNorm}
\end{table*}