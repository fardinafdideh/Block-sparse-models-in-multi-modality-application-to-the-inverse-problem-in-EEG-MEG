In 
%cerebroelectromagnetism
bioelectromagnetism, the transmission of electromagnetic fields from a current dipole through the head tissues towards measurement sensors is called volume conduction head model, which contains the electrical conduction properties of the head.
In frequencies below 1 kHz, the quasi-static approximation of Maxwell's equations can be used for modelling the volume conduction head model \cite{Plonsey1967}.
%,Sarvas1987,Munck1991}.
Therefore, the level of realism of the volume conduction head model determines the quality of EEG/MEG source analysis \cite{Haueisen2014}.
%\cite{vandenBroek1998,Wolters2004,Haueisen2014}.

The volume conduction head model can be modelled analytically, 
%quasi-analytical, 
or numerically.
In other words, for the EEG/MEG, the head can be modelled analytically by a single sphere or three or even four concentric spheres corresponding to brain, cerebrospinal fluid, skull, and scalp \cite{Hosek1978}.
% \cite{Frank1952,Geselowitz1967,Sarvas1987,Hosek1978,Huang1999}.
%The attempts for solving quasi-analytically are \cite{deMunck1988,Munck1993,Nolte1999,Nolte2001}. 
The boundary element methods and finite element methods have been developed to numerically model the head and to better represent the realistic shape of the head \cite{Yan1991,Akalin-Acar2004}.
%\cite{Yan1991,Awada1997,Buchner1997,vandenBroek1998,Marin1998,Akalin-Acar2004,Wolters2006}.
Although utilising numerically-modelled head models lead to increased accuracy in the source localization problem, it is computationally heavy.
Therefore, based on the requirements of the problem a trade-off between computational burden and source localization accuracy should be taken into account.

In figure \ref{fig:HeadModel}, the three-layer analytical and numerical volume conduction head models are shown.
The three layers correspond to scalp, skull, and brain, respectively.
The head model in figure \ref{fig:HeadModel}(a) is computed analytically, whereas the ones in (b) and (c) are computed numerically.
The realistic brain layer in figure \ref{fig:HeadModel}(b) is an inflated cortical sheet, whereas in figure \ref{fig:HeadModel}(c) it is a highly-folded cortical sheet.

Naturally, in order to build realistic head models we need to structural information of head, which is provided by MRI\footnote{\emph{Magnetic Resonance Imaging}}.
\begin{figure}[!b]
\centering
\includegraphics[width=1\textwidth]{images/HeadModel.png} % width=0.5\textwidth  scale=0.49
\caption{Three-layer (scalp, skull, and brain) head models modelled (a) analytically, and (b,c) numerically.
In (b), the cortical sheet is inflated, whereas in (c) it is highly-folded.}
\label{fig:HeadModel}
\end{figure}
\FloatBarrier