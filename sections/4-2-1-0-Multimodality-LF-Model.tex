In order to investigate the impact of multi-modality, i.e., the integration of the complementary information of EEG and MEG modalities, we need to think of a strategy to combine the mono-modalities.

Suppose that the odd rows of USLE related to EEG generative model is represented as $\boldsymbol{y}_{\boldsymbol{\mathrm{EEG}}}^{O} \seq \myPhi_{\boldsymbol{\mathrm{EEG}}}^{O} \mybetaz$, and similarly the even rows of USLE related to MEG model as $\boldsymbol{y}_{\boldsymbol{\mathrm{MEG}}}^{E} \seq \myPhi_{\boldsymbol{\mathrm{MEG}}}^{E} \mybetaz$.
Then, by concatenating the odd rows and even rows of EEG and MEG generative models, respectively, the EEG and MEG multi-modal model is as follows:
\begin{equation}
\label{eq:multimodal-model} 
\begin{bmatrix}
\boldsymbol{y}_{\boldsymbol{\mathrm{EEG}}}^{O} \\
\boldsymbol{y}_{\boldsymbol{\mathrm{MEG}}}^{E}
\end{bmatrix} 
=
\begin{bmatrix}
\myPhi_{\boldsymbol{\mathrm{EEG}}}^{O} \\
\myPhi_{\boldsymbol{\mathrm{MEG}}}^{E}
\end{bmatrix}
\mybetaz.
\end{equation}

There are three reasons behind the mentioned strategy of integration of mono-modalities.
\paragraph{1) Fixed number of sensors}
Assuming that the number of sensors in both modalities is \myhl{even}, then each signal $\boldsymbol{y}_{\boldsymbol{\mathrm{EEG}}}^{O} \ssin \mathbb{R}^{m/2}$ or $\boldsymbol{y}_{\boldsymbol{\mathrm{MEG}}}^{E} \ssin \mathbb{R}^{m/2}$ has as half of observations as the original signal $\boldsymbol{y}_{\boldsymbol{\mathrm{EEG}}/\boldsymbol{\mathrm{MEG}}} \ssin \mathbb{R}^{m}$, \myhl{i.e., each modality contributes the same in terms of the number of sensors, so the comparison would be more fair compared to the odd number of one of original modalities' sensors}.
Then, by concatenating them, the number of equations in the resulted USLE in multi-modality would be equal to the the original USLE in mono-modality.

By keeping fixed the number of sensors in mono-modality and multi-modality, we assure that the probable added value is because of the multi-modality itself and not in reason of the increased observation.
Because, in case of utilising all the equations in each of mono-modal USLE corresponding to EEG and MEG, the equations in final USLE  would be more than the original USLE, i.e., $\boldsymbol{y}_{\boldsymbol{\mathrm{EMEG}}} \ssin \mathbb{R}^{2m}$, where, $\boldsymbol{y}_{\boldsymbol{\mathrm{EMEG}}}$ is the multi-modal observation.
Therefore, we would have an increased input information in multi-modality in comparison to mono-modality, which would not lead to a fair comparison.
\paragraph{2) Minimal change in sensors position}
By drawing for example the odd rows of one modality and even rows of the other, i.e., without shared index for sensors position, we again ensure that the possible enhancement is due to the multi-modality itself and not because of major change in sensors position.
Otherwise, if for example we select odd rows from both mono-modalities, the sensors layout in mono-modality would be significantly different from multi-modal layout.
Because we are assuming that each MEG sensor is selected among the whole standard sensors, based on the proximity to an EEG sensor, to minimise the distance between each pair of EEG and MEG sensors.
So, if the index set of selected sensors in one mono-modality does not have any shared sensor's index with the other mono-modality, the difference in sensors position in mono-modality and multi-modality will be minimised.
\paragraph{3) Uniform sub-sampling in sensor space}
By selecting even or odd rows of USLE, we are assuring that the sensors are uniformly distributed all over the head.
Otherwise, by selecting for example the first half of one mono-modal sensors and the second half of the other mono-modal sensors, although the multi-modal sensors cover whole parts of the head, we would lose the information of one part of brain in mono-modality.
%\section{Combined random dictionaries or EEG and MEG leadfields}

Therefore, considering the three mentioned reasons in multi-modality strategy, we are eliminating or reducing the side effects of increased observation, and optimum sensor position, so we can investigate the impact of multi-modality itself.

As mentioned before, the purpose of this chapter is to investigate the impact of multi-modal lead-field matrix within the block structure identification framework introduced in Chapter \ref{sec:Clustering} with application to brain source space segmentation.
Therefore, from the multi-modal USLE in (\ref{eq:multimodal-model}), we only need to lead-field matrices $\myPhi_{\boldsymbol{\mathrm{EEG}}}$ and $\myPhi_{\boldsymbol{\mathrm{MEG}}}$, or precisely $\myPhi_{\boldsymbol{\mathrm{EEG}}}^O$ and $\myPhi_{\boldsymbol{\mathrm{MEG}}}^E$.
In figure \ref{fig:EMEG-LF}(c), the multi-modal lead-field $\myPhi_{\boldsymbol{\mathrm{EMEG}}}$ is built by concatenating the odd rows of $\myPhi_{\boldsymbol{\mathrm{EEG}}}$ (blue lead-field) and the even rows of $\myPhi_{\boldsymbol{\mathrm{MEG}}}$ (green lead-field).
%the proposed multi-modality framework, we only need the leadfield matrices.
% or dictionaries.
%Therefore, only leadfield matrices $\myPhi_{\boldsymbol{\mathrm{EEG}}}$ and $\myPhi_{\boldsymbol{\mathrm{MEG}}}$ are enough to segment the brain source space.
%The key aim in this chapter is to integrate the complementary information of EEG and MEG modalities within the block structure identification framework introduced in Chapter \ref{sec:Clustering} with application to brain source space segmentation.

As represented graphically in figure \ref{fig:EMEG-LF}, number of multi-modal EMEG sensors in figure \ref{fig:EMEG-LF}(c) is equal to the number of mono-modal EEG and MEG sensors in figure \ref{fig:EMEG-LF}(a) and (b), which is equal to the number of rows of the lead-field matrices, i.e. $m$.

As mentioned before, the MEG sensors in figure \ref{fig:EMEG-LF} are the closest sensors to EEG sensors, e.g., MEG sensor $\#1$ is the closest MEG sensor to EEG sensor $\#1$, and so on.

The minimal change in sensors position in figure \ref{fig:EMEG-LF}(c) in comparison to figure \ref{fig:EMEG-LF}(a) and (b), also the uniform distribution of sensors, would be more evident by increasing the density of sensors.

%On the other hand, in order to combine the electromagnetic properties of the head, latent in the EEG and MEG leadfield matrices, 
% which results in improved source reconstruction problem, 
%one can simply replace one in a row the rows of one modality with the other one, while the total number of the sensors in mono-modal and multi-modal leadfields are equal, as shown in figure \ref{fig:EMEG-LF}(b).


\begin{figure}[!b]
\centering
\includegraphics[width=1\textwidth,keepaspectratio]{images/EMEG-LF.png} % width=0.5\textwidth  scale=0.49
\centering
\caption{The rows of (a) EEG lead-field $\myPhi_{\boldsymbol{\mathrm{EEG}}}$ and (b) MEG lead-field $\myPhi_{\boldsymbol{\mathrm{MEG}}}$ can be combined together to form (c) the new multi-modal lead-field $\myPhi_{\boldsymbol{\mathrm{EMEG}}}$.}
\label{fig:EMEG-LF}
\end{figure}
\FloatBarrier
