Many natural phenomena consist of very complex structural and functional properties.
For instance, considering the sophisticated example of human brain as the phenomenon, the electric, magnetic, functional and anatomical properties can be taken into account.

On the other hand, each modality or measurement instrument measures only a specific property of the phenomenon.
Then, due to the rich characteristics of a natural multi-property phenomenon, a comprehensive information cannot be obtained by only a single modality.
Therefore, the research domain of \emph{multi-modality} revealed to jointly analyse the data collected from different modalities which are somehow complementary.
The concept of multi-modality is shown in figure \ref{fig:Multimodality} in a simple language.
As shown in figure \ref{fig:Multimodality}, a natural multi-property phenomenon is represented by a multi-coloured object, where, each colour represents a specific property of the object.
In figure \ref{fig:Multimodality}, modalities are been represented by eyes that are observing the multispectral object.
\begin{figure}[!b]
\centering
\includegraphics[width=1\textwidth]{images/Multimodality.png} % width=0.5\textwidth  scale=0.49
\caption{Mono-modal dataset cannot describe the multispectral phenomenon of interest, due to the monospectral detectability of each modality. 
While by combining mono-modal datasets, an acceptable description of the phenomenon can be recovered.}
\label{fig:Multimodality}
\end{figure}
%\FloatBarrier
Then, the fact of shortcoming of a single modality to observe all the properties of the phenomenon is represented by the blindness of each of the eyes to a range of colours.
Finally, the data observed from each of the eyes can be combined to build a correct image of the multi-coloured object of interest, which is the ultimate goal in multi-modality, i.e., compensating each modality’s relative shortcomings by integrating other modality's dataset, in order to extract the maximum available information.
In multi-modality, the combined dataset is called \emph{multi-modal dataset} in contrast to the dataset of a single modality which is called mono-modal or uni-modal dataset.
The interest in the multi-modality originates from the idea that the information in the multi-modal dataset is more than the sum of the information in each of individual mono-modal datasets.

In our study, the multi-property phenomenon of interest is the human brain activity and we are interested in its electromagnetic properties as represented in figure \ref{fig:Bimodality}.
%\emph{Cerebroelectromagnetism} is a discipline that studies the electric, and magnetic properties of brain which can be seen as a division of the more general concept of bioelectromagnetism which involves bioelectromagnetic measurements and stimulations \cite{Malmivuo1995}. 
In neurophysiology, a wide variety of modalities are used for bioelectric measurements, among which the EEG is the most interesting from the clinical applications point of view, thanks to the following four main advantages: (1) non-invasive, (2) high temporal resolution (in order of milliseconds), (3) relatively cost-effective, and (4) easy to use and portable.  

In addition, for biomagnetic measurements the MEG is commonly used.
As represented graphically in figure \ref{fig:Bimodality}, fusion of EEG and MEG can be used to improve the quality of source estimation, due to the fact that multi-modality can also reduce the ill-posedness of problems and make them better determined \cite{Velmurugan2016}.
%\cite{Cohen1987,Mosher1992,Mosher1999,Pflieger2000,Huizenga2001,Yoshinaga2002,Liu2002,Silva2004,Jun2010,Pitolli2010,Ebersole2010,Mideksa2013,
%\cite{Hong2013,Muthuraman2014,Kuuluvainen2014,
%Aydin2014,
%Aydin2015,Chowdhury2015,Zimozdra2015,Sohrabpour2016,Velmurugan2016}.
In the Section \ref{sec:Physiological_basis}, we review the electrophysiological properties of neuronal populations.
In Section \ref{sec:ForwardInverseProblem}, the required steps to solve the EEG/MEG source reconstruction problem is discussed.
Finally, Section \ref{sec:EMEG_Complementarities} elucidates some complementarities of EEG and MEG.
\begin{figure}[!b]
\centering
\includegraphics[width=1\textwidth]{images/Bimodality.png} % width=0.5\textwidth  scale=0.49
\caption{Electromagnetic properties of brain activity is measured by EEG and MEG modalities and can be used to reconstruct the brain activity.
Source reconstruction using EEG and MEG multi-modality leads to improved results.
% in a multimodal problem.
}
\label{fig:Bimodality}
\end{figure}
\FloatBarrier