This chapter contains the main contribution of the Ph.D. thesis.
In this chapter, a general framework for block-sparse exact signal recovery in an USLE\footnote{\emph{Underdetermined System(s) of Linear Equations}} problem, i.e., Block-ERC\footnote{\emph{Block-sparse Exact Recovery Condition(s)}}, is proposed.
The framework consists of four theoretical and one algorithmic-dependent Block-ERC.
\newpage
The mentioned generality is in terms of 
\begin{itemize}
\item structure of the dictionary, i.e., 
%blocks could be linearly dependent/independent or 
blocks could be orthonormal bases or there may exist intra-block orthonormality, \myhl{however, in one of the proposed Block-ERC the computation of the Moore-Penrose pseudo-inverse of each block is involved, which requires the blocks to be full column rank,}
\item norm in the corresponding optimisation problem,
\item length of blocks,
\item parameters of the proposed block-sparsity measures, e.g., we are not constrained to only Euclidean norm criterion for determining the active blocks, and
\item there is no constraint on the relationship between the dimensions of size of the dictionary $\myPhi \ssin \mathbb{R}^{m \stimes n}$, except $m \sless n$.
\end{itemize}

As mentioned, the recovery conditions are proposed for the exact solution of the optimisation problem, therefore the \emph{uniqueness} of the representation is the center of our focus. 
In addition, the exact recovery conditions ensure the uniqueness of the $\mySuppTxt$ of the solution, i.e., the problem is $\mySuppTxt$ recovery, and not coefficient recovery.

Through the aforementioned generalisations, we can relax some constraints as mentioned before, and improve the results of Eldar et al. \cite{Eldar2009b,Eldar2010b,Eldar2010}, and extend some findings of Donoho et al. \cite{Donoho2001,Donoho2003}, Elad and and Bruckstein \cite{Elad2001,Elad2002a}, and Gribonval and Nielsen \cite{Gribonval2003a,Gribonval2003} from element-wise (scalar-wise) to block-wise (vector-wise) framework \cite{Afdideh2016}.

This chapter is a major extended version of \cite{Afdideh2016}.
Section \ref{sec:Our_Block-sparsity} presents the required terminology and tools to establish the main block-sparse recovery conditions which are introduced in Section \ref{sec:Our_BERC}.
At last, to demonstrate the supremacy of the proposed theoretical recovery conditions in comparison to the state-of-the-art results, some numerical experiments are implemented in Section \ref{sec:Numerical_experiments}, and finally conclusions in Section \ref{sec:Our_Conclusion} terminate this chapter.