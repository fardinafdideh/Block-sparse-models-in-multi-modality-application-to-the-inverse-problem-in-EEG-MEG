\paragraph{1) Structure and model}
In general, the USLE can be formulated as \cite{Cand`es2005b,Donoho2006a}:
%there are two models for formulating the USLE: 
%\begin{itemize}
%\item noiseless case \cite{Donoho2003,Gribonval2003a}:
%\cite{Donoho2003,Donoho2003a,Gribonval2003a,Donoho2004,Fuchs2004a,Tropp2004,Cand`es2005b,Cand`es2006a,Cand`es2006b,Cand`es2006c,Chartrand2007,Cand`es2008a,Davies2008,Foucart2009a,Davies2009,Cai2009,Foucart2010,Cai2010}:
%\begin{equation}
%\label{eq:Model_Noiseless}
%\boldsymbol{y}=\myPhi \mybetaz,
%\end{equation}
%\item  noisy case \cite{Cand`es2005b,Donoho2006a}:%\cite{Cand`es2005b,Cand`es2006,Tropp2006,Donoho2006a,Cand`es2007a,Cand`es2008a,Cai2009,Foucart2009a,Cai2009,Wainwright2009,Wainwright2009a,Fletcher2009,Cai2010,Jin2010}: 
\begin{equation}
\label{eq:Model_Noisy}
%\boldsymbol{y}=\myPhi \mybetaz + \boldsymbol{e} \qquad \textrm{or} \qquad \boldsymbol{y}=\myPhi \myparanthese{\mybetaz + \boldsymbol{e}} ,
\boldsymbol{y}=\myPhi \mybetaz + \boldsymbol{e} ,
\end{equation}
%\end{itemize}
where, $\boldsymbol{y} \ssin \mathbb{R}^{m}$ is the measurement vector, 
%(also called observation, signal, response variable), 
$\myPhi \ssin \mathbb{R}^{m \stimes n}$, $m \sless n$, is the dictionary, $\mybetaz \ssin \mathbb{R}^{n}$ is the true representation vector, 
or input variable, 
and $\boldsymbol{e} \ssin \mathbb{R}^{m}$ is noise, 
%quantization or round-off error, or bounded perturbation or noise (due to the circuit imperfections) 
bounded by a known noise level, e.g., $\Vert \boldsymbol{e} \Vert_2 \sless \epsilon$. 
%Usually $\boldsymbol{e}$ is assumed to have i.i.d. Gaussian distribution.
%\cite{Cand`es2007a,Cai2009,Cai2010}.
%where $\mybetaz \ssin \mathbb{R}^{n}$ is the unknown true representation vector (input variable) of the known measurement vector (observation, signal, response variable) $\boldsymbol{y} \ssin \mathbb{R}^{m}$ in the known dictionary $\myPhi \ssin \mathbb{R}^{m \stimes n}$ and $m \sless n$ since we are dealing with USLE. 
%$\boldsymbol{e} \ssin \mathbb{R}^{m}$ is an unknown quantization or round-off error, or bounded perturbation or noise (due to the circuit imperfections) bounded by a known \emph{noise level}, $\Vert \boldsymbol{e} \Vert_2 \sless \epsilon$. 
%Furthermore, $\boldsymbol{e}$ can be considered as a modelling error when the true solution does not exactly satisfy a priori knowledge.
if $\boldsymbol{e} \seq \boldsymbol{0}$, then the model is noiseless \cite{Donoho2003,Gribonval2003a}.
In this study, we will concentrate on the following noiseless model:
\begin{equation}
\label{eq:Model_Noiseless}
\boldsymbol{y}=\myPhi \mybetaz.
\end{equation}

As mentioned before, the goal is to extract a representation vector $\hat{\mybeta}$ which is the sparsest among all solutions, i.e., a representation with the fewest non-zero elements relative to its dimension.
For example, suppose the following estimated representation vector $\hat{\mybeta}$:
\begin{equation*}
\label{eq:Beta-hat Structure}
\hat{\mybeta} = \mybracket{\hat{\beta}_1, \cdots, \hat{\beta}_k, \cdots, \hat{\beta}_n}^T,
\end{equation*}
and following dictionary:
\begin{equation*}
\label{eq:Phi Structure}
\myPhi = \mybracket{\myphi_1, \cdots, \myphi_{k}, \cdots, \myphi_{n}},
\end{equation*}
with $\myphi_{j} \ssin \mathbb{R}^{m}$ and without loss of generality, it is assumed that atoms have unit Euclidean norm, i.e., $\forall j,  \Vert \myphi_{j} \Vert_2 \seq 1$.
Then, the sparsity structure of the noiseless USLE 
%the estimated representation vector in the noiseless linear model 
is graphically represented in figure \ref{fig:Sparsity-Structure}.
Commonly, this problem is referred to as \emph{atomic decomposition}, since the measurement vector $\boldsymbol{y}$ is decomposed into its fewest active 
%building 
atoms of the dictionary \cite{Chen2001,Donoho2001,Donoho2003a}.
%, i.e., in the noiseless case we have $\boldsymbol{y} \seq \sum_{j {\in S(\hat{\mybeta})}}  \hat{\beta}_j \boldsymbol{\varphi}_j$, where $S(\hat{\mybeta})$ is the support of $\hat{\mybeta}$ \cite{Chen2001,Donoho2001,Donoho2003a}.
\begin{figure}[!bh]
\centering
\includegraphics[width=0.5\textwidth]{images/Sparsity-Structure.png} % width=0.5\textwidth  scale=0.49
\caption{Sparsity structure of the noiseless USLE.}
%the representation vector $\hat{\mybeta}$ in the noiseless linear model.}
%White colour represents zero values.}
\label{fig:Sparsity-Structure}
\end{figure}
\FloatBarrier