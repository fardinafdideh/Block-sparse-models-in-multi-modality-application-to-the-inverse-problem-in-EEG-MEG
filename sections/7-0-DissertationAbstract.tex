\begin{vcenterpage}
%\iffalse
\noindent\rule[2pt]{\textwidth}{0.5pt}
\\
{\large\textbf{Résumé ---}}
De nombreux phénomènes naturels sont trop complexes pour être pleinement reconnus par un seul instrument de mesure ou par une seule modalité. 
    Par conséquent, le domaine de recherche de la multi-modalité a émergé pour mieux identifier les caractéristiques riches du phénomène naturel de la multi-propriété naturelle, en analysant conjointement les données collectées à partir d’uniques modalités, qui sont en quelque sorte complémentaires.
    Dans notre étude, le phénomène d'intérêt multi-propriétés est l'activité du cerveau humain et nous nous intéressons à mieux la localiser au moyen de ses propriétés électromagnétiques, mesurables de manière non invasive.
    En neurophysiologie, l'électroencéphalographie (EEG) et la magnétoencéphalographie (MEG) constituent un moyen courant de mesurer les propriétés électriques et magnétiques de l'activité cérébrale. 
    Notre application dans le monde réel, à savoir le problème de reconstruction de source EEG / MEG, est un problème fondamental en neurosciences, allant des sciences cognitives à la neuropathologie en passant par la planification chirurgicale.
Considérant que le problème de reconstruction de source EEG / MEG peut être reformulé en un système d'équations linéaires sous-déterminé, la solution (l'activité estimée de la source cérébrale) doit être suffisamment parcimonieuse pour pouvoir être récupérée de manière unique. 
La quantité de parcimonie est déterminée par les conditions dites de récupération. 
Cependant, dans les problèmes de grande dimension, les conditions de récupération conventionnelles sont extrêmement strictes. 
En regroupant les colonnes cohérentes d'un dictionnaire, on pourrait obtenir une structure plus incohérente. 
Cette stratégie a été proposée en tant que cadre d’identification de structure de bloc, ce qui aboutit à la segmentation automatique de l’espace source du cerveau, sans utiliser aucune information sur l’activité des sources du cerveau et les signaux EEG / MEG. 
En dépit du dictionnaire structuré en blocs moins cohérent qui en a résulté, la condition de récupération conventionnelle n’est plus en mesure de calculer la caractérisation de la cohérence. 
Afin de relever le défi mentionné, le cadre général des conditions de récupération exactes par bloc-parcimonie, comprenant trois conditions théoriques et une condition dépendante de l'algorithme, a été proposé. 
Enfin, nous avons étudié la multi-modalité EEG et MEG et montré qu'en combinant les deux modalités, des régions cérébrales plus raffinées sont apparues.
\\
\\
{\large\textbf{Mots clés :}}
(q,p)-constante cohérence mutuelle par bloc,
propriété null space par bloc,
bloc-spark,
conditions de reconstruction exacte par bloc-parcimonieux,
Principe d'incertitude par bloc-parcimonieux,
bloc-parcimonieux,
identification de la structure de bloc,
segmentation spatiale de l’activité cérébrale,
(q,p)-constante cohérence mutuelle cumulée par bloc,
multimodalité en EEG et MEG.
\\
\noindent\rule[2pt]{\textwidth}{0.5pt}
\vspace{0.5cm}
%\fi
\newpage
\noindent\rule[2pt]{\textwidth}{0.5pt}
%\begin{center}
%{\large\textbf{Title in english\\}}
%\end{center}
{\large\textbf{Abstract ---}}
Many natural phenomena are too complex to be fully recognised by only a single measurement instrument or mono-modality. Therefore, the research domain of multi-modality has emerged to better identify the rich characteristics of the natural multi-property phenomenon, through jointly analysing the data collected from mono-modalities, which are somehow complementary. 
In our study, the multi-property phenomenon of interest is the human brain activity and we are interested in better localising it by means of its electromagnetic properties which are measurable non-invasively. In neurophysiology, a common way to measure the electric and magnetic properties of the brain activity is ElectroEncephaloGraphy (EEG), and MagnetoEncephaloGraphy (MEG), respectively. Our real-world application, i.e., EEG/MEG source reconstruction problem, is a fundamental problem in neuroscience ranging from cognitive science to neuropathology to surgical planning. 
Considering that the EEG/MEG source reconstruction problem can be reformulated as an underdetermined system of linear equations, the solution (estimated brain source activity) should be sufficiently sparse in order to be recovered uniquely. The amount of sparsity is determined by the so-called recovery conditions. However, in high-dimensional problems, the conventional recovery conditions are extremely strict. By regrouping the coherent columns of a dictionary, the more incoherent structure could be achieved. This strategy was proposed as a block structure identification framework, which results in the automatic segmentation of the brain source space, without using any information about the brain sources activity and EEG/MEG signals. Despite the resulted less coherent block-structured dictionary, the conventional recovery condition is no longer capable of computing the coherence characterisation. To address the mentioned challenge, the general framework of block-sparse exact recovery conditions including three theoretical and one algorithmic-dependent conditions was proposed. Finally, we investigated the EEG and MEG multi-modality and demonstrated that by combining the two modalities, more refined brain regions appeared.
%\iffalse
%\myhl{In many research area such as signal and image processing, bioinformatics, etc., researchers end up with a high-dimensional inverse problem, which can be reformulated as a vastly underdetermined system of linear equations.
%Usually, the measurement vector is not represented uniformly in all blocks of columns of dictionary, and can be represented in a few number of low-dimensional blocks in dictionary through a sparse representation vector.
%Recovering such low-dimensional representative block structure, which is feasible through a proper constrained optimisation problem, helps to reduce the computational burden and memory requirements of algorithms utilising the data.
%Despite the fact that an underdetermined system of linear equations has infinitely many solutions, the mentioned proper constrained optimisation problem under certain conditions is able to recover a unique sparse solution (or representation vector).
%In other words, if a representation vector is sparse enough, then it is theoretically possible to recover it uniquely via a constrained optimisation problem.
%Usually the mentioned sparsity of the representation vector is a function of a characterisation of the dictionary.
%Sometimes, depending on the nature of the dictionary, the sparsity constraint on the representation vector is so strong (tight) that the need to weaken the sparse recovery conditions becomes more pronounced.
%
%
%Three main challenges have been addressed in this thesis, in three chapters.
%First challenge is about the ineffectiveness of classic sparse recovery conditions in high-dimensional problems. 
%This challenge is partially addressed through the idea of clustering the coherent columns (or block of columns) of the dictionary based on the proposed dictionary characterisation, in order to establish more incoherent atomic entities in the dictionary, which is proposed as a \emph{block structure identification framework}. 
%The more incoherent atomic entities, the more improvement in the sparse recovery conditions, i.e., the conditions are more weakened.
%%In addition, we applied the mentioned clustering idea to real-world EEG/MEG leadfields to segment the brain source space, without using any information about the brain sources activity and EEG/MEG signals.
%Second challenge raises when classic sparse recovery conditions cannot be utilised for the new concept of constraint, i.e., block-sparsity.
%Therefore, as the second research orientation, we developed a \emph{general framework for block-sparse exact recovery conditions}, i.e., three theoretical and one algorithmic-dependent conditions, which ensure the uniqueness of the block-sparse solution of corresponding weighted (pseudo-)mixed-norm optimisation problem in an underdetermined system of linear equations.
%%The mentioned generality of the framework is in terms of the properties of the underdetermined system of linear equations, extracted dictionary characterisations, optimisation problems, and ultimately the recovery conditions.
%Finally, the combination of different information of a same phenomenon is the subject of the third challenge, which is addressed in the last part of dissertation with application to brain source space segmentation.
%More precisely, we showed that by combining the EEG and MEG lead-fields in a \emph{EEG/MEG multi-modality framework} and gaining the electromagnetic properties of the head, more refined brain regions appeared.}
%\fi
\\
\\
	
%    First challenge is about the ineffectiveness of some classic methods in high-dimensional problems.
%Second challenge raises when classic recovery conditions cannot be established for a new concept of constraint of corresponding optimisation problem, because of materials shortage in new framework. 
%Finally, the combination of different information of a same phenomenon is the subject of the third challenge.

%    In the first part of thesis, we present general theoretical recovery conditions ensuring the uniqueness of the solution of underdetermined systems of linear equations.
    
%Then, in the second part by utilizing the idea of clustering the coherent parts of the dictionary based on the proposed characterisation, we demonstrate the improvement of the proposed recovery conditions and also by applying the idea on the real-world EEG or MEG leadfiels and without using any information about the brain sources activity and EEG or MEG signals we introduce brain regions and segments which with their activation still the unique recovery is ensured.

%Finally, in the last part of thesis we show that by combining the EEG and MEG leadfields and gaining the electromagnetic properties of the head, more refined and precise brain regions are appeared, hence, EEG and MEG multi-modality advantage will be proved from another point of view.

%In this study, we propose four theoretical (algorithmic-independent) and one algorithmic-dependent Block-sparse Exact Recovery Conditions (BERCs), which guarantee the uniqueness of block-sparse recovery in general dictionaries through a general mixed norm minimisation problem. 
%These conditions are derived using the proposed Block-sparse Uncertainty Principles (BUPs) and Block Null Space Property (BNSP), based on some newly defined characterisations of \emph{block spark}, \emph{$(q,p)$-Block Mutual Incoherence Constant (BMIC$_{q,p}$)}, \emph{$(q,p)$-Cumulative Block Mutual Incoherence Constant (CBMIC$_{q,p}$)}, and \emph{Cumulative Inter-block Incoherence Constant (CIIC)}.
%Some of the proposed characterisations and also the corresponding optimisation problems for theoretical conditions are convex and some are non-convex. 
%Recently, it has been turned out that exploiting the block structure of the representation gives rise to improvement in the conventional recovery condition, however we redemonstrate this benefit.
%We show that there is improvement in the conventional recovery condition when exploiting the block structure of the representation. 
%In addition, the proposed BERCs extend the existing results for block-sparse setting by generalising the criterion for determining the active blocks, weakening the block-sparse recovery conditions, and relaxing some constraints on blocks such as linear independency of the columns.

{\large\textbf{Keywords:}}
$({q,p})$-Block Mutual Coherence Constant (Block-MCC$_{q,p}$),
Block Null Space Property (Block-NSP),
Block-Spark,     
Block-sparse Exact Recovery Conditions (Block-ERC), 
block-sparse uncertainty principle,
Block-sparsity,
block structure identification,
brain source space segmentation,
cumulative Block-MCC$_{q,p}$,  
EEG and MEG multi-modality.    
\\
\noindent\rule[2pt]{\textwidth}{0.5pt}
\begin{center}
  Grenoble Images Parole Signal Automatique laboratoire (GIPSA-Lab), 11 rue des mathématiques, Domaine Universitaire, BP 46 \\
  38402 Saint Martin d'Hères cedex
\end{center}
\end{vcenterpage}

%%% Local Variables: 
%%% mode: latex
%%% TeX-master: "../roque-phdthesis"
%%% End: 
