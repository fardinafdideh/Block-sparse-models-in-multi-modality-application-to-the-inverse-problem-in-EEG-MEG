As mentioned earlier, the EEG/MEG source reconstruction problem can be modelled as an USLE.
In our real-world problem, the dictionary in USLE is called the lead-field matrix.
the whole lead-field matrix, is built by horizontal concatenation of lead-fields of all sources in the source space.
As described before, the lead-field matrix of a single source is a $m$ by 3 matrix, where, $m$ is the number of sensors in the sensor space.
Hence, a whole lead-field matrix is composed of consecutive three-column individual lead-field matrices.

On the other hands, as mentioned before, in the EEG/MEG linear model each of three-column individual lead-field matrices $\myPhi [k]$, $\forall k$, are multiplied to the activity of corresponding dipole in the source vector $\mybeta$ to construct the activity of the sensor space, i.e., $\boldsymbol{y} \seq \sum_{k \seq 1}^K \myPhi [k] \mybeta [k]$.
Then, each block of dictionary, $\myPhi [k]$, $\forall k$, is the direct coefficient of its corresponding source activity.

% each block of length three in the dictionary represents the leadfield matrix of a source in the source space where each source lie into the representation vector $\mybeta$.
Therefore, by clustering the coherent lead-field matrices $\myPhi [k]$ of the whole lead-field matrix $\myPhi$, their corresponding blocks in the source activity vector $\mybeta$ will also be clustered.
Hence, the sources will be grouped and will form some segments in the source space.

To realise the mentioned idea, we propose the general Block-MCC$_{q,p}$ dictionary characterisation, which measures the similarity between the blocks of the dictionary.
Then, Block-MCC$_{q,p}$ can be used in the clustering step to make similarity matrix.

As represented graphically in figure \ref{fig:Source_Segmentation_Schematic}, the six sources in the brain are clustered in different three segments based on the clustering of their corresponding lead-fields in the lead-field matrix.
In other words, clustering of the coherent lead-field matrices $\myPhi [1]$, $\myPhi [4]$, and $\myPhi [6]$ leads to clustering of their corresponding sources $\mybeta [1]$, $\mybeta [2]$, and $\mybeta [3]$, and so on.
\begin{figure}[!b]
\centering
\includegraphics[width=0.8\textwidth,keepaspectratio]{images/Source_Segmentation_Schematic.png} % width=0.5\textwidth  scale=0.49
\centering
\caption{Segmenting the source space of brain using clustering the corresponding lead-fields.}
\label{fig:Source_Segmentation_Schematic}
\end{figure}
\FloatBarrier
