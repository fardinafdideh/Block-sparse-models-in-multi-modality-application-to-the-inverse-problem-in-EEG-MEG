As mentioned before, the goal is to extract a representation vector $\hat{\mybeta}$ which is the block-sparsest among all solutions, i.e., a representation with the fewest active blocks relative to its dimension.
For instance, suppose the following estimated representation vector $\hat{\mybeta}$ which consists of nonoverlapping equally-sized blocks of length $d$, i.e., $\forall k,\hat{\mybeta}[k] \ssin \mathbb{R}^{d}$:
\begin{equation*}
\label{eq:Beta-hat-block Structure}
\hat{\mybeta} = \mybracket{\hat{\mybeta}^T\mybracket{1}, \cdots, \hat{\mybeta}^T\mybracket{k}, \cdots, \hat{\mybeta}^T\mybracket{K}}^T,
\end{equation*}
where, $Kd \seq n$, and the $k^{th}$ block is:
\begin{equation*}
\hat{\mybeta}\mybracket{k} = \mybracket{\hat{\beta}_1\mybracket{k}, \cdots, \hat{\beta}_{d} \mybracket{k}}^T.
\end{equation*}
Then, the dictionary is represented by:
\begin{equation*}
\label{eq:Phi Structure}
\myPhi = \mybracket{\myPhi\mybracket{1}, \cdots, \myPhi\mybracket{k}, \cdots, \myPhi\mybracket{K}},
\end{equation*}
where, the $k^{th}$ blocks of atoms of the dictionary is:
\begin{equation*}
\myPhi \mybracket{k} = \mybracket{\myphi_1\mybracket{k}, \cdots, \myphi_{d}\mybracket{k}},
\end{equation*}
with $\myphi_{j}[k] \ssin \mathbb{R}^{m}$ and without loss of generality, it is assumed that atoms have unit Euclidean norm, i.e., $\forall j,k,  \Vert \myphi_{j}[k] \Vert_2 \seq 1$.
Then, assuming that the true solution $\mybetaz$ is exactly recovered, i.e., $\hat{\mybeta} \seq \mybetaz$, the block-sparsity structure of the estimated representation vector $\hat{\mybeta}$ in the noiseless linear model is represented graphically in figure \ref{fig:Block-Sparsity-Structure}.
The model can be either noiseless, i.e., $\boldsymbol{y} \seq \myPhi \mybetaz$, or in general case, noisy, i.e., $\boldsymbol{y} \seq \myPhi \mybetaz \spl \boldsymbol{e}$, which are defined in (\ref{eq:Model_Noiseless}) and (\ref{eq:Model_Noisy}), respectively.
\begin{figure}[!b]
\centering
\includegraphics[width=0.5\textwidth]{images/Block-Sparsity-Structure.png} % width=0.5\textwidth  scale=0.49
\caption{Block-sparsity structure of the exactly recovered representation vector $\hat{\mybeta}$ in the noiseless linear model of $\boldsymbol{y} \seq \myPhi \mybetaz$.}
\label{fig:Block-Sparsity-Structure}
\end{figure}
\FloatBarrier