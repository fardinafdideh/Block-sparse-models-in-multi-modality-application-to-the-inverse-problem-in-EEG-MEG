Consider the representation vector $\hat{\mybeta}$ can be viewed as a concatenation of $K$ individual blocks:
\begin{equation*}
\label{eq:Beta Structure-whole}
\hat{\mybeta} = \mybracket{\hat{\mybeta}^T\mybracket{1}, \cdots, \hat{\mybeta}^T\mybracket{k}, \cdots, \hat{\mybeta}^T\mybracket{K}}^T,
\end{equation*}
where, the $k^{th}$ block of $\hat{\mybeta}$ can be represented as:
\begin{equation*}
\label{eq:Beta Structure-block}
\hat{\mybeta}\mybracket{k} = \mybracket{\hat{\beta}_1\mybracket{k}, \cdots, \hat{\beta}_{d_k} \mybracket{k}}^T,
\end{equation*}
where, $\hat{\beta}_{d_k}[k] \seq [\hat{\beta}_i \, | \, i \seq \sum_{j \seq 1}^{k \sm 1} d_j \spl d_k]$, and the length of $k^{th}$ block is $d_k$, whereas the vector of the blocks' length is:
\begin{equation*}
\label{eq:d}
\boldsymbol{d} = \mybracket{d_1, \cdots, d_K},
\end{equation*}
where, $\sum_{k=1}^K d_k \seq n$, i.e., the block structure is non-overlapped, and there is no constraining relationship between any $d_k$ and $m$. 
Then, the block structure of the representation vector $\hat{\mybeta}$, which is consecutive $d_k$-length vectors, is assumed to be known a priori.
Similarly, the following block-wise structure is assumed for the dictionary $\myPhi$, which can be viewed as a concatenation of all $K$ individual blocks:
\begin{equation*}
\label{eq:Phi Structure-whole}
\myPhi = \mybracket{\myPhi\mybracket{1},  \cdots, \myPhi\mybracket{k}, \cdots, \myPhi\mybracket{K}},
\end{equation*}
where, $\myPhi[k] \ssin \mathbb{R}^{m \stimes d_k}$, and as it is mentioned there is not any imposed relationship between $m$ and $n$, other than $m \sless n$.   
The $k^{th}$ block is defined as the $d_k$ columns of matrix $\myPhi$:
\begin{equation*}
\label{eq:Phi Structure-block}
\myPhi \mybracket{k} = \mybracket{\myphi_1\mybracket{k}, \cdots, \myphi_{d_k}\mybracket{k}},
\end{equation*}
with $\myphi_{j}[k] \ssin \mathbb{R}^{m}$, and without loss of generality, it is assumed that $\myphi_{j}[k]$ has unit Euclidean norm, i.e., $\forall j, k : \Vert \myphi_{j}[k] \Vert_2 \seq 1$. 
The block structure of $\myPhi$ and $\hat{\mybeta}$ is shown schematically in figure \ref{fig:Model}.
\begin{figure}[!b]
\centering
\includegraphics[width=0.5\textwidth,keepaspectratio]{images/Model.png} % width=0.5\textwidth  scale=0.49
\centering
\caption{Block structure of the dictionary $\myPhi$ and the representation vector $\hat{\mybeta}$ in the noiseless linear model, i.e., $\boldsymbol{y} \seq \myPhi \hat{\mybeta} \seq \sum_{k=1}^{K} \myPhi[k] \hat{\mybeta} [k]$.}
\label{fig:Model}
\end{figure} 
%\FloatBarrier
The matrix multiplication can be decomposed as sum of multiplication of blocks to reconstruct the measurement vector $\boldsymbol{y} \ssin span \{ \myPhi \} \ssin \mathbb{R}^{m}$.
Therefore, as the model we use the noiseless USLE explained in (\ref{eq:Model_Noiseless}), which can be viewed as the block-wise structure, i.e., $\boldsymbol{y} \seq \myPhi \hat{\mybeta} \seq \sum_{k=1}^{K} \myPhi[k] \hat{\mybeta} [k]$.