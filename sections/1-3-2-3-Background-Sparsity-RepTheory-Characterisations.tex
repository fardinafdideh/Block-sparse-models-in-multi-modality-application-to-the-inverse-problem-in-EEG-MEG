\paragraph{3) Characterisations}
\label{sec:Conv-Characterization} 
To define a framework for recovery or identifiability conditions, some properties and characterisations of the dictionary are introduced in literature, among which the widely studied are gathered in table \ref{table:ERC_properties}.
Most of the mentioned properties and characterisations of the dictionary are computationally unrealistic or intractable.
Only the property of mutual coherence property or the characterisation of mutual coherence constant have the great advantage to be simple and tractable, but with the expense of making the
recovery conditions more restrictive.
Next, we recall some basic notations and characterisations which will be used in our work.
\begin{table}[!bp]
\begin{adjustbox}{width=1\textwidth} % ,totalheight=\textheight,.5
\centering
%\tiny
\begin{tabular}{cc}
\toprule
%\cline{2-4}
\multicolumn{1}{c}{Properties} & \multicolumn{1}{c}{References} \\ \midrule %\hline
\multicolumn{1}{l}{Null space property} & \multicolumn{1}{l}{\cite{Donoho2001,Elad2001,Gribonval2003a,Feuer2003,Zhang2005a,Stojnic2008,Cohen2009}}
%{\cite{Donoho2001,Elad2001,Elad2002a,Donoho2003,Gribonval2003a,Gribonval2003,Feuer2003,Gribonval2004a,Zhang2005a,Gribonval2007,Stojnic2008,Cohen2009}} 
 \\% \midrule
\multicolumn{1}{l}{Robust null space property} & \multicolumn{1}{l}{\cite{Davies2009a,Foucart2013}} \\ %\midrule
\multicolumn{1}{l}{Mutual coherence property} & \multicolumn{1}{l}{\cite{Donoho2001,Elad2001,Gribonval2003a,Tropp2004,Fuchs2004a,Tropp2006,Gribonval2007}}
%{\cite{Donoho2001,Elad2001,Elad2002a,Donoho2003,Gribonval2003a,Tropp2004,Fuchs2004a,Fuchs2005,Donoho2006a,Tropp2006,Gribonval2007}}
 \\ %\midrule
\multicolumn{1}{l}{Restricted isometry property} & \multicolumn{1}{l}{\cite{Cand`es2005b,Baraniuk2008,Cand`es2008a}} \\ %\midrule
\multicolumn{1}{l}{Uniform uncertainty principle} & \multicolumn{1}{l}{\cite{Cand`es2005b,Cand`es2006b}} \\ %\midrule
\multicolumn{1}{l}{Exact reconstruction principle} & \multicolumn{1}{l}{\cite{Cand`es2006b}} \\% \midrule
\multicolumn{1}{l}{Neighborliness of the projected polytopes} & \multicolumn{1}{l}{\cite{Vershik1992,Donoho2005b,Donoho2005,Donoho2006d}} \\ \midrule
\multicolumn{1}{l}{Characterisations} & \multicolumn{1}{c}{References}  \\ \midrule %\hline
\multicolumn{1}{l}{$\mySpkTxt$} & \multicolumn{1}{l}{\cite{Gribonval2003a,Donoho2003}} \\ %\midrule
\multicolumn{1}{l}{Cospark} & \multicolumn{1}{l}{\cite{Cand`es2005b}} \\ %\midrule
\multicolumn{1}{l}{Cumulative coherence} & \multicolumn{1}{l}{\cite{Donoho2003,Tropp2004}} \\ %\midrule
\multicolumn{1}{l}{Quantity invariant $\gamma_{2S}$} & \multicolumn{1}{l}{\cite{Foucart2009a}} \\
\multicolumn{1}{l}{Mutual coherence constant} & \multicolumn{1}{l}{\cite{Donoho2001,Elad2001,Gribonval2003a,Tropp2004,Fuchs2004a,Tropp2006,Gribonval2007}} \\ %\midrule
\multicolumn{1}{l}{Restricted isometry constant} & \multicolumn{1}{l}{\cite{Cand`es2005b,Baraniuk2008,Cand`es2008a}} \\ %\midrule
\multicolumn{1}{l}{Restricted orthogonality constant} & \multicolumn{1}{l}{\cite{Cand`es2005b}} \\ %\midrule
\multicolumn{1}{l}{Asymetric restricted isometry constant} & \multicolumn{1}{l}{\cite{Davies2008,Davies2009}} \\ %\midrule
  \bottomrule %\hline
\end{tabular}
\end{adjustbox}
\caption{Properties and dictionary characterisations used in literature to define exact and stable recovery conditions.}
\label{table:ERC_properties}
\end{table}
%\FloatBarrier
\paragraph{$\boldsymbol{\mySuppTxt}$ and $\boldsymbol{\myCardTxt}$:}% Support and Cardinality
The $\mySuppTxt$ (or active set) and $\myCardTxt$ of a vector $\mybeta$ are defined as:
\begin{equation*}
\label{eq:Conventional Support & Cardinality}
\begin{aligned}
\mySuppTxt\textrm{:}&&S\myparanthese{\mybeta} & \myeq \mybrace{k: \beta_{k} \neq 0}, \\
\myCardTxt\textrm{:}&&\myabs{S\myparanthese{\mybeta}} & \myeq \mynorm{\mybeta}_{0}. 
%\myeq \lim_{p \to 0^+} \myparanthese{\sum_{k=1}^n \myabs{\beta_{k}}^p}^\frac1p.
\end{aligned}
\end{equation*}
Indeed, the non-convex pseudo-norm $\Vert \mybeta \Vert_0$, simply counts the non-zero elements of vector $\mybeta$.
\paragraph{Dictionary $\boldsymbol{\myKerTxt}$:} % Dictionary Kernel
The $\myKerTxt$ (or null space) of a dictionary $\myPhi$ is defined as:
\begin{gather*}
\label{eq:Kernel}
\myKerMath \myeq 
\mybrace{\boldsymbol{x} \in \mathbb{R}^n, \myPhi \boldsymbol{x} = \boldsymbol{0}}.
%= \mathcal{N}\myparanthese{\myPhi}.
\end{gather*}
The $\myKerTxt$ of a dictionary plays an important role in establishing two different classes of recovery conditions, which will be investigated later.
\paragraph{$\boldsymbol{\mySpkTxt}$:} % Spark
The smallest number of columns of dictionary $\myPhi$ that are linearly dependent is called \emph{Spark}.

$\mySpkTxt$ has a very important role in guaranteeing the uniqueness of the sparse representation. 
This importance has been already demonstrated by Gorodnitsky and Rao \cite{Gorodnitsky1997}.

The $\mySpkTxt$ of a dictionary $\myPhi$ is defined mathematically as \cite{Gribonval2003a,Donoho2003,Donoho2003a,Bruckstein2009}:
\begin{equation}
\label{eq:Conventional Spark}
\mySpk\myparanthese{\myPhi} \myeq \min_{\boldsymbol{x} \in \myKerMath \backslash\left\{\boldsymbol{0}\right\}} \mynorm{\boldsymbol{x}}_{0}.
\end{equation}

By the definition of the $\mySpk(\myPhi)$, any vector $\boldsymbol{x}$ in the null space of the dictionary, i.e., 
%$\sum_{k=1}^{n} x_k\boldsymbol{\varphi}_k  \seq \boldsymbol{0}$, 
$\myPhi \boldsymbol{x}\seq \boldsymbol{0}$, must satisfy $\Vert\boldsymbol{x} \Vert_{0} \sgeq \mySpk(\myPhi)$. 

As it can be seen, $\mySpkTxt$ characterises the null space of a dictionary using the $\ell_0$ pseudo-norm. 

Despite of the superficial similarity between $\myRnkTxt$ and $\mySpkTxt$, they are entirely different in concept and computational complexity. 

The $\myRnkTxt$ of a dictionary is defined as the largest number of the columns of $\myPhi$ which are linearly independent, and its computation is sequential (easy to compute), whereas $\mySpkTxt$ is the size of the smallest number of linearly dependent columns of $\myPhi$, and its computation is of combinatorial search with exponential complexity. 

In general, without zero columns, the $\mySpkTxt$ of a dictionary $\myPhi \ssin \mathbb{R}^{m \stimes n}$ is bounded by \cite{Donoho2003}:
\begin{equation}
\label{eq:spark-bounds}
%2 \leq \mySpk(\myPhi) 
%\leq \min \mybrace{n , \text{rank}(\myPhi)+1}
%\leq \min \mybrace{n , m + 1}
%= m + 1.
\begin{aligned}
2 \leq \mySpkMath
&\leq \min \mybrace{\myRnkMath + 1 , n} \\
&\leq \min \mybrace{m + 1 ,n} \\
&= m + 1.
\end{aligned}
\end{equation}

A dictionary with maximal $\mySpkTxt$, i.e., $\mySpkMath \seq m \spl 1$, is called \emph{full-Spark}.

For a dictionary with linearly dependent columns, at least two columns are linearly dependent, i.e., $2 \sleq \mySpk(\myPhi)$, and for many randomly generated dictionaries, $\mySpk(\myPhi)$ is equal to $\myRnkMath \spl 1$. 

On the other hand, we have $\myRnkMath \sleq \min\{m,n\}$, and since the problem is underdetermined, i.e., $m \sless n$, $\min\{m,n\}$ is $m$ and also $\min\{m+1,n\} \seq m \spl 1$.

As it can be seen in (\ref{eq:Conventional Spark}), the dictionary characterization $\mySpk(\myPhi)$ is obtained by minimisation of the intractable non-convex $\ell_0$ pseudo-norm, which is not easy to compute.
More precisely, given a matrix $\myPhi$, computing $\mySpk(\myPhi)$ is NP-hard\footnote{\emph{Non-deterministic Polynomial-hard}}.
The computational complexity of $\mySpk(\myPhi)$ is investigated in \cite{Tillmann2013}.
\paragraph{Mutual coherence constant:} % Mutual coherence constant
The maximum pairwise absolute correlation between the normalized atoms of a dictionary can be used as a characterization of the dictionary, which is called Mutual Coherence Constant (MCC) \cite{Donoho2001}.
%The Mutual Coherence Constant (MCC) of a dictionary is the maximum pairwise absolute correlation between the normalized atoms of the dictionary \cite{Donoho2001}.
%\cite{Donoho2001,Elad2001,Elad2002a,Donoho2003,Gribonval2003a,Tropp2004,Fuchs2004a,Fuchs2005,Donoho2006a,Tropp2006,Gribonval2007}. 
From computational point of view, MCC can be considered as the maximum element of off-diagonal absolute Gram (Gramian) matrix $\boldsymbol{G}(\myPhi) \myeq \myPhi^T\myPhi$ \cite{Donoho2003,Donoho2003a}, i.e.,
\begin{equation*}
\begin{aligned}
\label{eq:MIC}
M\myparanthese{\myPhi} \myeq 
\max_{k,k' \neq k} \myabs{ \left\langle \boldsymbol{\varphi}_k , \boldsymbol{\varphi}_{k'} \right\rangle}
&= \max_{k,k' \neq k} \myabs{\boldsymbol{\varphi}^T_k \boldsymbol{\varphi}^{ }_{k'}} \\
&= \max_{k,k' \neq k} \myabs{\boldsymbol{G}_{k,k'}\myparanthese{\myPhi}},
\end{aligned}
\end{equation*}
where, $\left\langle \boldsymbol{a} , \boldsymbol{b} \right\rangle$ computes the inner product of the vectors $\boldsymbol{a}$ and $\boldsymbol{b}$.
Obviously, the main diagonal of $\boldsymbol{G}(\myPhi)$ is composed of $1$s, due to the $\ell_2$-normalization of the atoms of the dictionary.
Since it is assumed that the atoms of the dictionary have unit $\ell_2$ norm, MCC satisfies $M(\myPhi) \sleq 1$, because of the Cauchy-Schwarz inequality $\vert \langle \boldsymbol{\varphi}_k , \boldsymbol{\varphi}_{k'} \rangle \vert \sleq \Vert \boldsymbol{\varphi}_k \Vert_2  \Vert\boldsymbol{\varphi}_{k'} \Vert_2 $.
For a dictionary as an orthonormal basis, $M(\myPhi) \seq 0$, whereas for a dictionary consisting two or $m$ orthonormal bases, $M(\myPhi)$ is bounded by \cite{Heath2006}:
\begin{equation}
\label{eq:M-bounds}
\frac{1}{\sqrt{m}} \leq M\myparanthese{\myPhi} \leq 1.
\end{equation}

For an equiangular tight frame deterministic dictionary, which satisfies the following three conditions \cite{Foucart2013}:
%For a general dictionary the lower-bound of $M(\myPhi)$ is $\sqrt{(n \sm m)/(m(n \sm 1))}$ \cite{Welch1974,Strohmer2003}.
%%\cite{Welch1974,Rosenfeld1997,Strohmer2003}.
%This lower bound is also known as the \emph{Welch bound}, and is achieved when the deterministic dictionary is equiangular tight frame, which satisfies the following three conditions \cite{Foucart2013}:
\begin{equation*}
\begin{aligned}
&&\mynorm{\boldsymbol{\varphi}_k}_2 = 1, \qquad & \text{for } k = 1, \cdots, n, \\
&&\myabs{\left\langle \boldsymbol{\varphi}_k , \boldsymbol{\varphi}^{ }_{k'} \right\rangle}=c, \qquad & \forall k \neq k' \text{ and constant c}, \\
&&\frac{m}{n} \sum_{k=1}^n \left\langle \boldsymbol{\varphi} , \boldsymbol{\varphi}^{ }_{k} \right\rangle \boldsymbol{\varphi}^{ }_{k} = \boldsymbol{\varphi}, \qquad & \forall \boldsymbol{\varphi} \in \mathbb{R}^m.
\end{aligned}
\end{equation*}
the lower-bound of $M(\myPhi)$ or the \emph{Welch bound} is $\sqrt{(n \sm m)/(m(n \sm 1))}$ \cite{Welch1974,Strohmer2003}.
The Welch bound defined for equiangular tight frame deterministic dictionaries is even less than the lower-bound of $M(\myPhi)$ for a general random dictionary in (\ref{eq:M-bounds}), unless $m \seq 1$, \myhl{for} which the two lower-bounds are equal.

MCC was first computed for two orthonormal bases $\myPhiOne$ and $\myPhiTwo$, 
which is called \emph{basic MCC}:
%, i.e., $\overbar{M}(\myPhiOne,\myPhiTwo)$, is the maximal inner product between the columns of the orthonormal bases $\myPhiOne$ and $\myPhiTwo$, i.e.,:
\begin{equation*}
\overbar{M} \myparanthese{\myPhiOne,\myPhiTwo} = \max_{k,k'} \myabs{\boldsymbol{\varphi}_{{\boldsymbol{1}}_k}^T \boldsymbol{\varphi}_{{\boldsymbol{2}}_{k'}}^{ }}.
\end{equation*}

{
\label{txt:BasicMCCBounds} 
Notice that for orthonormal bases $\myPhiOne$ and $\myPhiTwo$ and their concatenation $[\myPhiOne , \myPhiTwo]$, we have $\overbar{M}\myparanthese{\myPhiOne,\myPhiTwo} \seq M\myparanthese{\mybracket{\myPhiOne,\myPhiTwo}}$.
Supposing the dimension of two orthonormal bases $\myPhiOne$ and $\myPhiTwo$ is $m$ by $m$, it is proved that $1/\sqrt{m} \sleq \overbar{M}(\myPhiOne,\myPhiTwo) \sleq 1$ \cite{Donoho2001,Elad2001,Elad2002a}.
The lower-bound is achieved when the pair of orthonormal bases are spikes and sines \cite{Donoho2001} or Identity and Hadamard \cite{Elad2002a} matrices or any other orthonormal matrices corresponding to the orthonormal bases, whereas the upper-bound is achieved when at least one of the columns in each of two orthonormal bases is common.
}
\paragraph{Cumulative mutual coherence constant:} % Cumulative mutual coherence constant
MCC uses the maximum absolute off-diagonal element of the Gram matrix $\boldsymbol{G}(\myPhi)$ as the characterization of the dictionary $\myPhi$, while by summing over any $k$ elements of $\boldsymbol{G}(\myPhi)$, we would better characterize the dictionary. 
This kind of dictionary characterization is called \emph{cumulative MCC}.

The conventional cumulative MCC, (or Babel function \cite{Tropp2004}, or $\ell_1$-coherence function \cite{Foucart2013}) of a dictionary $\myPhi \ssin \mathbb{R}^{m \stimes n}$ is defined as:
\begin{equation}
\label{eq:CMIC}
M\myparanthese{\myPhi , k} \myeq 
\max_{\myabs{\Lambda}=k} \max_{j \notin \Lambda}
\sum_{i \in \Lambda} \myabs{ \left\langle \boldsymbol{\varphi}_i , \boldsymbol{\varphi}_{j} \right\rangle},
\end{equation}
where, $\Lambda$ represents $k$ different indices from $\{1, \cdots, n\}$.
%, and by convention $M(\myPhi , 0) \seq 0$.

Although cumulative MCC is computationally more difficult than MCC characterization, \myhl{it leads to more weakened or relaxed recovery conditions, i.e., the solution of the corresponding optimisation problem is ensured to be unique even for less sparse representation vectors.}

A straightforward extension of $\ell_1$-coherence function would be $\ell_p$-coherence function, defined for any $p \sg 0$ \cite{Foucart2013}:
\begin{equation*}
M_p\myparanthese{\myPhi , k} \myeq 
\max_{\myabs{\Lambda}=k} \max_{j \notin \Lambda}
\myparanthese{\sum_{i \in \Lambda} \myabs{ \left\langle \boldsymbol{\varphi}_i , \boldsymbol{\varphi}_{j} \right\rangle}^p } ^ {\frac1p}.
\end{equation*}

The $p$ parameter in $M_p(\myPhi , k)$ controls the $\ell_p$ norm of $k$ off-diagonal elements of $\boldsymbol{G}(\myPhi)$, which makes the $M_p(\myPhi , k)$ to have the following properties:
\begin{itemize}
\item For $p \seq 1$, $M_p(\myPhi , k)$ reduces to the conventional cumulative MCC, i.e., $M_1(\myPhi , k) {\equiv} M(\myPhi , k)$.
\item For $p \seq \infty$, for any value of $k$, $M_p(\myPhi , k)$ reduces to the conventional MCC, i.e., $M_{\infty}(\myPhi , k) {\equiv} M(\myPhi)$, because $\ell_{\infty}$ norm of any vector is equal to the maximum absolute value of the vector.
\end{itemize}

The Welch bound mentioned for the MCC can be extended to the cumulative MCC, i.e., for $k \sless \sqrt{n \sm 1}$, cumulative MCC is lower-bounded by $k\sqrt{(n \sm m)/(m(n \sm 1))}$, where, again the lower-bound is achieved when the dictionary is an equiangular tight frame \cite{Schnass2008}.

In a similar work for extracting the cumulative coherence of the dictionary, Donoho and Elad introduced $\mu_{1/2}$ and $\mu_1$ of the Gram matrix, which is the smallest $m$ off-diagonal entries in a single row or column of the Gram matrix $\boldsymbol{G}$, which sums at least to $1/2$ and $1$, respectively.

There are the relationships $M^{-1}(\myPhi) \sleq \mu_1(\boldsymbol{G}) \sless \mySpk(\myPhi)$ and $\mu_{1/2}(\boldsymbol{G}) \sleq (1/2) \mu_1(\boldsymbol{G})$ \cite{Donoho2003,Donoho2003a}.
\newpage
In another study, based on the manner of identifying the $\mySuppTxt$ set of a sparse signal in the OMP\footnote{\emph{Orthogonal Matching Pursuit}} algorithm, \emph{union cumulative coherence} is proposed and denoted by $M_U(\myPhi , k)$ \cite{Dossal2005,Zhao2015a}:
\begin{gather*}
\label{eq:M_U} 
M_U\myparanthese{\myPhi , k} \myeq \max_{\myabs{\Lambda}=k} \mybrace{\max_{j \notin \Lambda} \sum_{i \in \Lambda} \myabs{ \left\langle \boldsymbol{\varphi}_i , \boldsymbol{\varphi}_{j} \right\rangle} + 
\max_{l \in \Lambda} \sum_{i \in \Lambda \backslash \{ l \}} \myabs{ \left\langle \boldsymbol{\varphi}_i , \boldsymbol{\varphi}_{l} \right\rangle}}, 
\end{gather*}
where, $\Lambda$ represents $k$ different indices from $\{1, \cdots, n\}$.

$M_U(\myPhi , k)$ even better characterizes the dictionary $\myPhi$ in comparison to conventional cumulative MCC, i.e., $M(\myPhi , k)$.
Because, a part from the first common term in $M_U(\myPhi , k)$, which seeks the maximum of sum of pairwise absolute \emph{inter-set} correlations between the columns $i \ssin \Lambda$ and $j {\notin} \Lambda$, the second term in $M_U(\myPhi , k)$ seeks the same characterization for \emph{intra-set} columns $i \ssin \Lambda$ and $l \ssin \Lambda$. 

Although $M_U(\myPhi , k)$ is computationally more complicated than $M(\myPhi , k)$, it leads to more accurate analysis of the reconstruction capacity of the orthogonal matching pursuit \cite{Zhao2015a}.

Tropp has shown the following properties for cumulative MCC \cite{Tropp2004}:
\begin{itemize}
\label{txt:CMCC-properties} 
\item $M(\myPhi , 0) \seq 0$ (by convention),
\item $M(\myPhi , 1) \seq M(\myPhi)$,
\item $M(\myPhi , k) \sleq k \, M(\myPhi)$,
\item $M(\myPhi , k \spl 1) \sm M(\myPhi , k) \sgeq 0$,
\item $M(\myPhi , k \spl 2) \sm 2M(\myPhi , k \spl 1) + M(\myPhi , k) \sleq 0, \forall k \sgeq 0$, and
\item For orthonormal basis, $M(\myPhi , k) \seq 0, \forall k \sgeq 0$. 
\end{itemize}

In a more general case, for $1 \sleq k_1,k_2 \sleq n \sm 1$ with $k_1 \spl k_2 \sleq n \sm 1$, we have \cite{Foucart2013}:
\begin{gather*}
\begin{aligned}
\max \mybrace{M(\myPhi , k_1) , M(\myPhi , k_2)} &\leq M(\myPhi , k_1+k_2) \\
&\leq M(\myPhi , k_1) + M(\myPhi , k_2).
\end{aligned}
\end{gather*}

As it can be learned from the above-mentioned different formula of cumulative version of the ordinary coherence measure, this type of dictionary characterisation is more informative and general.
Since instead of the first maximum off-diagonal absolute value of the Gram matrix $\boldsymbol{G}(\myPhi)$, $k$ of them characterise the dictionary $\myPhi$.
