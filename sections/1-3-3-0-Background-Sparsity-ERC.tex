One of the fundamental contributions of the overcomplete signal representations community is theoretically-proved necessary and sufficient conditions for exact or stable signal recovery problems.
In other words, if based on some regularity conditions for sparsity assumption the representation vector is sufficiently sparse, then it can be recovered exactly or stably \cite{Vaiter2015}. 
Therefore, ill-posedness issue of the USLE can be solved \cite{Donoho2006a}.

The importance of the \emph{Exact Recovery Condition(s) (ERC)} or \emph{stable recovery condition} is in guaranteeing the uniqueness or faithful approximation of the solution of the model through the sparse optimisation problem, otherwise, different optimisation algorithms may return different or significantly different solutions. 

The theoretical ERC and stable recovery conditions guarantee that the solution can be found independent of the algorithm used, i.e., supposing the existence of a sparse enough solution, it is possible to derive necessary and sufficient conditions for the recovery of the desired solution, regardless of the recovery algorithm used.

As mentioned earlier, in this study we concentrate on the ERC, i.e., the noiseless model in (\ref{eq:Model_Noiseless}), i.e., $\boldsymbol{y} \seq \myPhi \mybetaz$.

The aforementioned sufficient sparsity condition, which sometimes can be applied explicitly on the representation vector, is determined by a so-called \emph{Sparsity Level (SL)} or \emph{sparsity bound}, which is the upper-bound for the number of non-zero entries of the representation vector, and is derived from the dictionary. 
\myhl{The sparsity level of a dictionary is represented by $SL(\myPhi)$.}

In other words, when the representation vector is very sparse, the sparsity level is low and vice versa \cite{Donoho2001,Cand`es2005b,Chartrand2007,Cai2010}.
%\cite{Donoho2001,Elad2001,Elad2002a,Donoho2003,Gribonval2003a,Cand`es2005b,Donoho2006a,Cand`es2006,Chartrand2007,Cand`es2007a,Cand`es2008a,Foucart2009a,Davies2008,Davies2009,Cai2009,Foucart2010,Cai2010}.

A representation is said to be \emph{$k$-sparse} if it has at most $k$ non-zero entries, which can be arbitrarily placed anywhere in the representation.
In other words, $k$ is less than sparsity level, so for a $k$-sparse representation $\mybetaz$, we have $\Vert \mybetaz \Vert_0 \sleq k \sless SL(\myPhi)$.

In this section, we briefly explain the main following ERC:
\begin{itemize}
\item ERC based on $\mySpkTxt$,
\item ERC based on null space property,
\item ERC based on mutual coherence constant, and
\item ERC based on cumulative coherence constant.
\end{itemize}
\newpage
\paragraph{ERC based on $\boldsymbol{\mySpkTxt}$}
%In general, the columns of the dictionary do not need to be linearly independent. 
%Therefore, $\mySpkTxt$ was defined according to the smallest number of columns which are linearly dependent \cite{Donoho2003,Donoho2003a,Bruckstein2009}. 

In literature, to approach the problem of determining the sufficient conditions for unique sparse recovery, i.e., ERC, a different problem inspired by the concept of uncertainty principle is considered \cite{Gorodnitsky1997,Donoho1989}.
%\cite{Gorodnitsky1997,Donoho1989,Donoho2001,Elad2001,Elad2002a,Donoho2003,Donoho2003a,Bruckstein2009}. 

Consider the problem $P_0$ and suppose $\mybetaz$ and $\mybetao$ are two distinct representations of the non-zero signal $\boldsymbol{y}$, in the dictionary $\myPhi$, i.e., $\boldsymbol{y} \seq \myPhi \mybetaz$ and $\boldsymbol{y} \seq \myPhi \mybetao$. 
The uncertainty principle of redundant solutions states that a non-zero signal cannot have multiple highly sparse representations. 
In other words, in a given dictionary $\myPhi$ there is a limit on the sparsity level of the representations $\mybetaz$ and $\mybetao$:
\begin{equation}
\label{eq:UP-S}
\mynorm{\mybetaz}_0 + \mynorm{\mybetao}_0 \geq \mySpk\myparanthese{\myPhi}.
\end{equation}

The mentioned uncertainty principle has been proved based on the definition of $\mySpkTxt$ in (\ref{eq:Conventional Spark}), i.e., $\mySpk(\myPhi) {\myeq} \min_{\boldsymbol{x} \in \myKerMath \backslash\left\{\boldsymbol{0}\right\}} \Vert \boldsymbol{x} \Vert_{0}$, indicating that for $\mybetaz \sm \mybetao$ in the $\myKerTxt$ of the dictionary $\myPhi$, i.e., $\myPhi(\mybetaz \sm \mybetao) \seq \boldsymbol{0}$, we have
$\Vert \mybetaz \sm \mybetao \Vert_0 \sgeq \mySpk(\myPhi)$. 
Then, triangle inequality induces (\ref{eq:UP-S}).

The uncertainty principle in (\ref{eq:UP-S}) has been stated and demonstrated for different cases of dictionaries. 
At first, the dictionary was considered as a concatenation of two orthonormal bases \cite{Donoho2001,Elad2001,Elad2002a}.
Then, this uncertainty principle was generalised to dictionaries which arise from the union of more than two orthonormal bases \cite{Gribonval2003a}. 
Finally, it was generalised to dictionaries which can be the concatenation of less structured blocks 
%, or frames 
\cite{Donoho2003,Donoho2003a,Gribonval2003}.

%The uncertainty results can be represented either in the form of classical multiplicative \cite{Elad2001,Elad2002a,Mallat2008} or current additive, such as (\ref{eq:UP-S}).
%The multiplicative and additive results are simply related to each other by the arithmetic-geometric mean inequality.

Using the aforementioned uncertainty principle in different cases of dictionary, and the simple criterion of $\mySpkTxt$, the uniqueness of the sparse solution can be demonstrated. 

If $\mybetaz$ is a candidate solution of the $P_0$ problem for a general dictionary $\myPhi$, and meets
\begin{equation}
\label{eq:ERC-S}
\mynorm{\mybetaz}_0 < \frac{\mySpk\myparanthese{\myPhi}}{2},
\end{equation}
then according to the mentioned uncertainty principle in (\ref{eq:UP-S}), any other solution must be denser. Therefore the solution which is sufficiently sparse according to (\ref{eq:ERC-S}), is unique and the sparsest.

For a dictionary $\myPhi \ssin \mathbb{R}^{m \stimes n}$, considering the upper bound of $\mySpkTxt$ in (\ref{eq:spark-bounds}), i.e., $m \spl 1$, and the above-mentioned conventional $\mySpkTxt$-based condition in (\ref{eq:ERC-S}), the admitted sparsity level is at most $(1 \spl m)/2$.

Although among different ERC, ERC based on $\mySpkTxt$ is the most relaxed condition, i.e., the highest sparsity level, the computation of the $\mySpkTxt$ characterisation is not tractable \cite{Tillmann2013}.
\newpage
%------------------------------------------------------ 
\paragraph{ERC based on null space property}
\label{txt:NSP} 
Conventional \emph{Null Space Property (NSP)} provides necessary and sufficient conditions for the exact recovery of $k$-sparse representation via $P_1$, in other words the equivalence of $P_0$ and $P_1$, \myhl{i.e., the solution to $P_0$ is necessarily equal to the solution to $P_1$.}

For a $k$-sparse representation $\mybeta \ssin \mathbb{R}^{n}$, i.e., $|S(\mybeta)| \sleq k$, assuming that the $\mySuppTxt$ $S$ of the true solution lies within $S(\mybeta) \ssubset \{1, \cdots , n\}$, i.e., $S(\mybetaz) \ssubset S(\mybeta)$, the mentioned traditional condition states that if 
\begin{equation*}
Q_p \myparanthese{S\myparanthese{\mybeta} , \myPhi} \myeq 
\max_{\boldsymbol{x} \in \myKerMath \backslash\left\{\boldsymbol{0}\right\}}{\frac{\displaystyle\sum_{i \in S\myparanthese{\mybeta}} \myabs{x_i} ^ p}{\displaystyle\sum_{i} \myabs{x_i} ^ p}} < \frac12,
\end{equation*}
%where support $S$ of the true solution lies within $S(\mybeta) \ssubset \{1, \cdots , n\}$, i.e., $S(\mybetaz) \ssubset S(\mybeta)$, and $|S(\mybeta)| \sleq k$, i.e., $k$-sparse representation. 
then $\mybetaz$ is the unique solution to the problem $P_p$.

NSP has been demonstrated by starting to prove that for all $\boldsymbol{x} \ssin \myKerMath$, we have $\Vert \mybetaz \Vert_p^p \sless \Vert \mybetaz \spl \boldsymbol{x} \Vert_p^p$.
Then using a variant of quasi-triangle inequality, i.e., $\vert a \spl b \vert^p \sm \vert a \vert^p \sgeq \sm \vert b \vert^p$, and calculating $\ell_p$ norm over on-$\mySuppTxt$ ($\ssin S(\mybeta)$) and off-$\mySuppTxt$ (${\notin} S(\mybeta)$) parts, $Q_p (S(\mybeta) , \myPhi)$ is achieved.

NSP was first stated for a dictionary which is concatenation of two orthonormal matrices corresponding to orthonormal bases and for $p$ equal to one \cite{Donoho2001,Elad2001,Elad2002a,Feuer2003}, then was proved for arbitrary nonorthogonal dictionaries \cite{Donoho2003,Gribonval2003,Zhang2005a,Stojnic2008,Cohen2009}. 
Later, it was more generalised to $0 \sleq p \sleq 1$ and for dictionaries being a union of orthonormal bases \cite{Gribonval2003,Gribonval2003a} and for general arbitrary dictionaries \cite{Gribonval2004a,Gribonval2007}.

In order to investigate the relationship between the ERC based on NSP and the previously mentioned ERC based on $\mySpkTxt$ in (\ref{eq:ERC-S}), let $p \seq 0$ in $Q_p (S(\mybeta) , \myPhi)$:
\begin{equation*}
\begin{aligned}
Q_0 \myparanthese{S\myparanthese{\mybeta} , \myPhi} \myeq 
\max_{\boldsymbol{x} \in \myKerMath \backslash\left\{\boldsymbol{0}\right\}}{\frac{\myabs{S\myparanthese{\mybeta}}}{\mynorm{\boldsymbol{x}}_0}} 
= \frac{\myabs{S\myparanthese{\mybeta}}}{\displaystyle\min_{\boldsymbol{x} \in \myKerMath \backslash\left\{\boldsymbol{0}\right\}} \mynorm{\boldsymbol{x}}_0} 
& = \frac{\myabs{S\myparanthese{\mybeta}}}{\mySpk\myparanthese{\myPhi}} \\
& < \frac12.
\end{aligned}
\end{equation*}
Then, $|S(\mybeta)| \sless \mySpk(\myPhi) / 2$.

On the other hand, due to the assumption $S(\mybetaz) \ssubset S(\mybeta)$, we have $|S(\mybetaz)| \sleq |S(\mybeta)|$, which leads to the ERC based on $\mySpkTxt$.
Therefore, the ERC based on NSP is a general property, which in a special case of $p \seq 0$ reduces to the ERC based on $\mySpkTxt$ in (\ref{eq:ERC-S}).

Another stable variant of NSP is also introduced in literature, which is called robust NSP \cite{Davies2009a,Foucart2013}. 
From algorithmic point of view, robust NSP is used for stable signal recovery via basis pursuit \cite{Foucart2013}.
\newpage
%------------------------------------------------------
\paragraph{ERC based on mutual coherence constant}
\label{sec:Sparsity-ERC-MIC} 
In general, 
%dictionary characterisation of $\mySpk(\myPhi)$
$\mySpkTxt$ and 
%property of 
NSP are computationally unrealistic, in other words, it is computationally intractable in polynomial time to check the identifiability of the model through the recovery conditions, specially when the number of atoms in the dictionary is high. 
For detailed information about the computational complexity of $\mySpkTxt$ and NSP, the interested reader is referred to \cite{Tillmann2013} and \cite{Tillmann2014}.

To overcome this shortcoming, another characterisation of the dictionary called \emph{Mutual Coherence Constant} (MCC) was exploited in literature with the expense of making the recovery conditions more restrictive, i.e., lowering the sparsity level.
%, e.g., according to the bound of $M(\myPhi)$ in (\ref{eq:M-bounds}) and the following conventional MCC-based condition in (\ref{eq:ERC-M}), the sparsity level is at most $(1 \spl \sqrt{m})/2$, whereas according to the bound of $\mySpk(\myPhi)$ in (\ref{eq:spark-bounds}) and the conventional $\mySpkTxt$-based condition in (\ref{eq:ERC-S}), the sparsity level is at most $(1 \spl m)/2$.

MCC, which is defined on page \pageref{eq:MIC}, i.e., $M(\myPhi) {\myeq} \max_{k,k' \neq k} |\left\langle \boldsymbol{\varphi}_k , \boldsymbol{\varphi}_{k'} \right\rangle|$, is a simple approach for characterising the proximity or similarity between the atoms of the dictionary.

MCC was first introduced by Mallat and Zhang to heuristically evaluate the performance of the MP\footnote{\emph{Matching Pursuit}} algorithm \cite{Mallat1993}. 

Like $\mySpkTxt$, to approach the problem of exact signal recovery based on mutual coherence, uncertainty principle was used in literature.
Suppose $\mybetao$ and $\mybetaTwo$ are two distinct representations of the non-zero signal $\boldsymbol{y}$ in two orthonormal bases $\myPhiOne$ and $\myPhiTwo$, respectively, i.e., $\boldsymbol{y} \seq \myPhiOne \mybetao \seq \myPhiTwo \mybetaTwo$.
The basic or classic uncertainty principle states that a non-zero signal cannot have multiple sparse representations in two distinct orthonormal bases, if both bases are mutually incoherent \cite{Donoho1989,Donoho2001,Elad2001,Elad2002a}. 

Therefore, there is a limit on the sparsity level of the representations $\mybetao$ and $\mybetaTwo$:
\begin{equation}
\label{eq:UP-basic} 
\mynorm{\mybetao}_0 + \mynorm{\mybetaTwo}_0 
\geq\frac{2}{\overbar{M} \myparanthese{\myPhiOne,\myPhiTwo}},
\end{equation}
where, $\overbar{M} (\myPhiOne,\myPhiTwo) \seq \max_{k,k'} |\boldsymbol{\varphi}_{{\boldsymbol{1}}_k}^T \boldsymbol{\varphi}_{{\boldsymbol{2}}_{k'}}^{ }|$ is the basic MCC.

{
\label{Def:M-tilda} 
\myhl{In an attempt to extend the basic uncertainty principle to non-orthonormal bases but still square and non-singular matrices $\myPhiOne$ and $\myPhiTwo$, $\Vert \mybetao \Vert_0 \spl \Vert\mybetaTwo \Vert_0 
\sleq (1/2) \tilde{M} ^{-1} (\myPhiOne,\myPhiTwo)$ is developed as recovery condition for the uniqueness of the solutions of $P_0$ and $P_1$, and their equivalence, where, $\tilde{M}(\myPhiOne,\myPhiTwo) \seq \max\{ \max_{i,j} |\myPhi^{-1}_{\boldsymbol{1}} \myPhiTwo|_{i,j} , \max_{i,j} |\myPhi^{-1}_{\boldsymbol{2}} \myPhiOne|_{i,j}\}$ {\cite{Donoho2001}}.}}

\myhl{In fact, the definition of coherence in $\tilde{M}(\myPhiOne,\myPhiTwo)$, which computes the $\ell_{\infty}$ norm (maximum entry in a vector) of $\ell_{1,\infty}$ norm (maximum absolute entry in a matrix) of matrices $\myPhi^{-1}_{\boldsymbol{1}} \myPhiTwo$ and $\myPhi^{-1}_{\boldsymbol{2}} \myPhiOne$, implicitly indicates to the definition of coherence of blocks in a special case, which will be explained in Chapter {\ref{sec:BERC}}.}

Returning back from basic two orthornomal bases $\myPhiOne$ and $\myPhiTwo$ to the dictionaries $\myPhi$, and in order to shift the recovery conditions based on $\mySpkTxt$ to MCC, first we need to find their relationship.

MCC for an orhtogonal matrix is zero, 
%and the sparsity level approaches to infinity, 
i.e., no constraint.
It should be mentioned that, $\mySpkTxt$ is lower bounded by a function of inverse of MCC, i.e., $\mySpk(\myPhi) \sgeq f(M^{-1}(\myPhi))$. 
Therefore, we can approximate the intractable characterisation $\mySpkTxt$ with a computationally tractable characterisation MCC.

For a general dictionary, the smallest value of the MCC is of interest, in other words the tightest lower bound of the $\mySpkTxt$, because as mentioned before in (\ref{eq:ERC-S}), the sparsity level is defined as half of the $\mySpkTxt$, i.e., $\Vert \mybetaz \Vert_0 \sless \mySpk(\myPhi)/2$.

In addition to theoretic results, from algorithmic point of view, generally the smaller the MCC, the better the performance of recovery algorithms.
\cite{Foucart2013} has justified the claim for OMP\footnote{\emph{Orthogonal Matching Pursuit}}, BP\footnote{\emph{Basis Pursuit}}, and basic thresholding algorithms. 

If the dictionary is the concatenation of two orthonormal bases, 
%Donoho and Huo 
\cite{Donoho2001} proved that for guaranteeing the uniqueness of the solution of $P_0$ and equivalence of $P_1$, it is sufficient for $f(M^{-1}(\myPhi)) \seq 1 \spl M^{-1}(\myPhi)$.

Later, Elad and Bruckstein improved the condition by getting $f(M^{-1}(\myPhi)) \seq 2M^{-1}(\myPhi)$ (also proved in \cite{Donoho2003} and \cite{Donoho2003a} as a special case) and $f(M^{-1}(\myPhi)) \seq (2\sqrt{2} \sm 1)M^{-1}(\myPhi)$ for guaranteeing the uniqueness of the solution of $P_0$ and equivalence of $P_1$, respectively \cite{Elad2001}.
%\cite{Elad2001,Elad2002a}.
Feuer and Nemirovski proved that the latter is the maximum lower bound for $\mySpkTxt$ that can be achieved in the $P_1$ problem, i.e., the bound is tight \cite{Feuer2003}.

Supposing that the dictionary arises from the union of $L$ orthonormal bases ($L \sgeq 2$), Gribonval and Nielsen proved $f(M^{-1}(\myPhi)) \seq L/(L \sm 1)M^{-1}(\myPhi)$ and $f(M^{-1}(\myPhi)) \seq (2\sqrt{2} \sm 2 \spl 1/(L \sm 1))M^{-1}(\myPhi)$ for guaranteeing the uniqueness of the solution of $P_0$ and equivalence of $P_1$, respectively \cite{Gribonval2003a,Gribonval2003}.
Tropp also proved the recovery condition resulted from the latter $f(M^{-1}(\myPhi))$, for the equivalence of OMP and BP algorithms \cite{Tropp2004}.

Later, Donoho, Elad, Gribonval, Nielsen, and Bruckstein demonstrated that the previous results of Donoho and Huo, \myhl{i.e., $f(M^{-1}(\myPhi)) \seq 1 \spl M^{-1}(\myPhi)$,} can be generalised from a union of two orthonormal bases to a dictionary in a general case, which it can be the concatenation of less structured blocks (in addition to orthonormal bases) \cite{Donoho2003,Donoho2003a,Gribonval2003,Bruckstein2009}.

For $\mybetaz$ and $\mybetao$ as two distinct representations of the non-zero signal $\boldsymbol{y}$, in the dictionary $\myPhi$, i.e., $\boldsymbol{y} \seq \myPhi \mybetaz$ and $\boldsymbol{y} \seq \myPhi \mybetao$, and by combining $f(M^{-1}(\myPhi)) \seq 1 \spl M^{-1}(\myPhi)$ with (\ref{eq:UP-S}), we have
\begin{equation}
\label{eq:S-M} 
\mynorm{\mybetaz}_0 + \mynorm{\mybetao}_0 \geq 
\mySpk \myparanthese{\myPhi} \geq 1+M^{-1}\myparanthese{\myPhi}.
\end{equation}
Therefore, for any general dictionary, \myhl{the $k$-sparse representation vector} $\mybetaz$ is the unique solution of the $P_0$ and $P_1$ problems, if
\begin{equation}
\label{eq:ERC-M} 
\mynorm{\mybetaz}_0 \mycolor{\leq k} < \frac{1+M^{-1}\myparanthese{\myPhi}}{2}.
\end{equation}

According to the lower bound of MCC for a general random dictionary $\myPhi$ in (\ref{eq:M-bounds}), i.e., $1/\sqrt{m}$, and the above-mentioned conventional MCC-based condition in (\ref{eq:ERC-M}), the sparsity level is at most $(1 \spl \sqrt{m})/2$, although for equiangular tight frame deterministic dictionaries it can goes up until $(1 \spl\sqrt{(m(n \sm 1))/(n \sm m)})/2$.

Gribonval and Nielsen demonstrated that under the condition in (\ref{eq:ERC-M}), and for an arbitrary dictionary, the problems $P_p$ with $0 \sleq p \sless 1$ and $P_1$ are equivalent \cite{Gribonval2007}. 

From algorithmic point of view, Tropp proved the same condition of (\ref{eq:ERC-M}) for representation recovery through some greedy recovery algorithms, i.e., equivalence of OMP and BP algorithms \cite{Tropp2004}.

In another algorithmic study, Maleki demonstrated slightly stronger sufficient recovery conditions than (\ref{eq:ERC-M}) via iterative thresholding algorithms of iterative thresholding with inversion, namely IHT\footnote{\emph{Iterative Hard Thresholding}} and IST\footnote{\emph{Iterative Soft Thresholding}}.
He showed that, supposing $\mybetaz$ is sorted in descending order of its
absolute values \cite{Maleki2009}:
\begin{enumerate}
\item if $k \sless (1/3)M^{-1}(\myPhi)$, then iterative thresholding with inversion recovers the true solution, i.e., $\mybetaz$, in at most $k$ iterations.
\item if $k \sless (1/3.1)M^{-1}(\myPhi)$ and $\vert \beta_{0_i} \vert/ \vert\beta_{0_{i \spl 1}}\vert \sless 3^{\ell_i \sm 4}, 1 \sleq i \sless k$, then iterative hard thresholding will recover $\mySuppTxt$ of the true solution, i.e., $S(\mybetaz)$, in at most $\sum_{i \seq 1}^k \ell_i \spl k$ iterations, and after this number of iterations, without changing the $\mySuppTxt$, the error of $\mybetaz$ recovery will be eliminated exponentially.
\myhl{$\ell_i$ is the step after which $\beta_{0_{i \spl 1}}$ will get into the $\mySuppTxt$.}
\item if $k \sless (1/4.1)M^{-1}(\myPhi)$ and $\vert \beta_{0_i} \vert/ \vert\beta_{0_{i \spl 1}} \vert \sless 3^{\ell_i \sm 5}, 1 \sleq i \sless k$, then iterative soft thresholding will recover $\mySuppTxt$ of the true solution, i.e., $S(\mybetaz)$, in at most $\sum_{i \seq 1}^k \ell_i \spl k$ iterations, and after this number of iterations, without changing the $\mySuppTxt$, the error of $\mybetaz$ recovery will be eliminated exponentially.
\myhl{$\ell_i$ is the step after which $\beta_{0_{i \spl 1}}$ will get into the $\mySuppTxt$.}
\end{enumerate}
In another algorithm-based condition, it has been shown that for $k \sless M^{-1}(\myPhi)/4$ every $k$-sparse representation can be exactly recovered via iterative hard thresholding \cite{Foucart2013}.

In addition, Donoho et al. demonstrated the same condition of (\ref{eq:ERC-M}) for a stable solution in a stable signal recovery problem of $P_{0,2,\varepsilon}(\mybeta, \boldsymbol{r})$ \cite{Donoho2006a}.
\iffalse
In an attempt to improve the recovery conditions while using the same formula in (\ref{eq:ERC-M}), it is assumed that if we are given infinite number of USLE $\{y_i\}_{i=1}^\infty \seq \{\myPhi \beta_{0_i} \}_{i=1}^\infty $, generated with the same probabilistic law, then the MCC of the ensemble, denoted by $M_{Ensemble}$, would be $M^O(\myPhi)$ for $O \seq \{2,3,\cdots\}$, where, $O$ is the order of the moment used in the statistic analysis \cite{Donoho2003a}.
\fi
\newpage
%------------------------------------------------------
\paragraph{ERC based on cumulative coherence constant}
The coherence characterisation of MCC \cite{Donoho2001} 
%\cite{Donoho2001,Elad2001,Elad2002a,Donoho2003,Gribonval2003a,Tropp2004,Fuchs2004a,Fuchs2005,Donoho2006a,Tropp2006,Gribonval2007}, 
represents only the most extreme correlation between the atoms of the dictionary and does not offer a comprehensive description of the dictionary. 
In order to extract more information and to better characterise the dictionary, researchers focused on cumulative coherence characterisations. 
Although it is computationally more difficult than the conventional coherence characterisation, it leads to weakened recovery conditions; i.e., it provides sharper results for the equivalence of $P_0$ and $P_1$ optimisation problems. 

The characterization $\mu_{1/2}(\boldsymbol{G})$ introduced by Donoho and Elad as a cumulative coherence characterisation is the smallest $m$ off-diagonal entries in a single row or column of the Gram matrix $\boldsymbol{G}$, which sums at least to $1/2$.
Based on $\mu_{1/2}(\boldsymbol{G})$, if $\mynorm{\mybetaz}_0 \sless \mu_{1/2}(\boldsymbol{G})$, then $\mybetaz$ is the unique solution of the $P_1$ optimisation problem \cite{Donoho2003}.

From algorithmic point of view, considering cumulative MCC defined in (\ref{eq:CMIC}), i.e., $M(\myPhi , k) {\myeq} 
\max_{\myabs{\Lambda}=k} \max_{j \notin \Lambda}
\sum_{i \in \Lambda} | \left\langle \boldsymbol{\varphi}_i , \boldsymbol{\varphi}_{j} \right\rangle |$, Tropp demonstrated that, if 
\begin{equation}
\label{eq:ERC-CMIC}
M\myparanthese{\myPhi , k}+M\myparanthese{\myPhi , k-1} < 1, 
\end{equation}
then $k$-sparse representation vector $\mybetaz$ can be recovered correctly from orthogonal matching pursuit and basis pursuit algorithms, which is a sufficient condition \cite{Tropp2004}.

The ERC based on cumulative MCC, i.e., (\ref{eq:ERC-CMIC}), is weaker than the ERC based on MCC, i.e., the inequality of (\ref{eq:ERC-M}).
\myhl{Because 
%for the $k$-sparse representation vector $\mybetaz$, i.e., $\Vert \mybetaz \Vert_0 \sleq k$, using 
by rearranging ({\ref{eq:ERC-M}}) we have 
%$k \sless (1 \spl M^{-1}(\myPhi)) / 2$, so 
$2 \, k \, M(\myPhi) \sm M(\myPhi) \sless 1$}, which can be reformulated such as: $kM(\myPhi) \spl (k \sm 1)M(\myPhi) \sless 1$.
Now we can compare the reformulated condition of (\ref{eq:ERC-M}), i.e., $kM(\myPhi) \spl (k \sm 1)M(\myPhi) \sless 1$, with condition (\ref{eq:ERC-CMIC}).
%On the other hand, 
Considering the $M(\myPhi , k) \sleq k \, M(\myPhi)$ and $M(\myPhi , k \sm 1) \sleq (k \sm 1) \, M(\myPhi)$ properties, we conclude that the left-hand side  of the just obtained reformulated inequality is greater than or equal to the left-hand side of the inequality in (\ref{eq:ERC-CMIC}), i.e., $M(\myPhi , k) \spl M(\myPhi , k \sm 1) \sleq kM(\myPhi) \spl (k \sm 1)M(\myPhi)$.
Then, higher values of $k$ could meet the condition in (\ref{eq:ERC-CMIC}) compared to the reformulated inequality of (\ref{eq:ERC-M}).
Hence, the condition based on cumulative MCC is weaker than the condition based on MCC.

The results were improved more by a characterisation called union cumulative coherence: $M_U(\myPhi , k) {\myeq} \max_{\myabs{\Lambda}=k} \{\max_{j \notin \Lambda} \sum_{i \in \Lambda} | \left\langle \boldsymbol{\varphi}_i , \boldsymbol{\varphi}_{j} \right\rangle| \spl 
\max_{l \in \Lambda} \sum_{i \in \Lambda \backslash \{ l \}} | \left\langle \boldsymbol{\varphi}_i , \boldsymbol{\varphi}_{l} \right\rangle|\}$ \cite{Dossal2005,Zhao2015a}.
If $M_U(\myPhi , k) \sless 1$, then the $k$-sparse representation vector can be exactly recovered by the orthogonal matching pursuit algorithm.  
%the necessary and sufficient condition for uniformly and exactly recovering all the $k$-sparse representation vectors using orthogonal matching pursuit 
%in terms of his proposed union cumulative coherence 
%defined on page \pageref{eq:M_U}, 
%is $M_U(\myPhi , k) \sless 1$.
%, where, $M_U(\myPhi , k) {\myeq} \max_{\myabs{\Lambda}=k} \{\max_{j \notin \Lambda} \sum_{i \in \Lambda} | \left\langle \boldsymbol{\varphi}_i , \boldsymbol{\varphi}_{j} \right\rangle| \spl \max_{l \in \Lambda} \sum_{i \in \Lambda \backslash \{ l \}} | \left\langle \boldsymbol{\varphi}_i , \boldsymbol{\varphi}_{l} \right\rangle|\}$ \cite{Zhao2015a}.
The ERC using $M_U(\myPhi , k)$ is weaker than (\ref{eq:ERC-CMIC}), since $M_U(\myPhi , k) \sless M(\myPhi , k) \spl M(\myPhi , k \sm 1)$, and so higher values of $k$ can be selected in $M_U(\myPhi , k)$-based condition. 
%He also proposes the stable recovery condition 
%for stably recovering 
There is also a stable recovery condition via orthogonal matching pursuit using the characterisation $M_U(\myPhi , k)$ \cite{Zhao2015a}.

In another work, it is been demonstrated that if $M(\myPhi , k) \spl M(\myPhi , k \sm 1) \sless \min_{i \ssin T} \vert \beta_{0_i} \vert / \max_{i \ssin T} \vert\beta_{0_i} \vert$, where, $\vert T \vert \seq k$, then the $k$-sparse representation vector $\mybetaz$ can be recovered exactly via basic thresholding algorithm \cite{Foucart2013}. 
Moreover, it has been shown that if $ 2M(\myPhi , k) \spl M(\myPhi , k \sm 1) \sless 1$ and $ M(\myPhi , 2k) \sless 1/2$, then the $k$-sparse representation is exactly recovered via hard thresholding pursuit and iterative hard thresholding algorithms, respectively \cite{Foucart2013}.