\begin{proof}
%Since the operator norm is non-negative, simply it can be seen that $M_{q,p}(\myPhi) \sgeq 0$.
%We begin the proof by the following property:
%\begin{property}[Submultiplicativity property]
%\label{lm:Submultiplicativity}
%Supposing $\boldsymbol{A}$ and $\boldsymbol{B}$ as two matrices, we have the following inequality for the operator-norm of their product:
%\begin{gather*}
%\begin{aligned}
%\mynorm{\boldsymbol{A} \boldsymbol{B}}_{q \to p} 
%&\leq \max\mybrace{\mynorm{\boldsymbol{A}}_{q \to p} \mynorm{\boldsymbol{B}}_{q \to q} , \mynorm{\boldsymbol{A}}_{p \to p} \mynorm{\boldsymbol{B}}_{q \to %p}} \\
%&\leq \max\mybrace{\mynorm{\boldsymbol{A}}_{q \to p} , \mynorm{\boldsymbol{A}}_{p \to p}} 
%\max\mybrace{\mynorm{\boldsymbol{B}}_{q \to q} ,  \mynorm{\boldsymbol{B}}_{q \to p}}.
%\end{aligned}
%\end{gather*}
%\end{property}
%\begin{proof}%[Submultiplicativity property]
%\begin{equation*}
%\begin{aligned}
%\mynorm{\boldsymbol{A} \boldsymbol{B}}_{q \to p} = \max_{\boldsymbol{x} \neq \boldsymbol{0}} \frac{\mynorm{\boldsymbol{ABx}}_{p}}{\mynorm{\boldsymbol{x}}_{q}} &= 
%\begin{cases}
%\displaystyle\max_{\boldsymbol{Bx} \neq \boldsymbol{0}} \frac{\mynorm{\boldsymbol{ABx}}_{p}}{\mynorm{\boldsymbol{Bx}}_{q}} \frac{\mynorm{\boldsymbol{Bx}}_{q}}{\mynorm{\boldsymbol{x}}_{q}}\\
%and\\
%\displaystyle\max_{\boldsymbol{Bx} \neq \boldsymbol{0}} \frac{\mynorm{\boldsymbol{ABx}}_{p}}{\mynorm{\boldsymbol{Bx}}_{p}} \frac{\mynorm{\boldsymbol{Bx}}_{p}}{\mynorm{\boldsymbol{x}}_{q}} \\
%\end{cases}\\
%&\leq \begin{cases}
%\displaystyle\max_{\boldsymbol{y} \neq \boldsymbol{0}} \frac{\mynorm{\boldsymbol{Ay}}_{p}}{\mynorm{\boldsymbol{y}}_{q}} \displaystyle\max_{\boldsymbol{x} \neq \boldsymbol{0}} \frac{\mynorm{\boldsymbol{Bx}}_{q}}{\mynorm{\boldsymbol{x}}_{q}}\\
%and\\
%\displaystyle\max_{\boldsymbol{y} \neq \boldsymbol{0}} \frac{\mynorm{\boldsymbol{Ay}}_{p}}{\mynorm{\boldsymbol{y}}_{p}} \displaystyle\max_{\boldsymbol{x} \neq \boldsymbol{0}} \frac{\mynorm{\boldsymbol{Bx}}_{p}}{\mynorm{\boldsymbol{x}}_{q}}. \\
%\end{cases}
%\end{aligned}
%\end{equation*}
%\end{proof}
Using the definition of $M_{q,p}(\myPhi)$ in Property \ref{prp:IntraBlkO}, and submultiplicativity property of operator-norm introduced in Property \ref{prp:OperatorProperties} ($\ell_{q {\to} p}$ operator-norm properties), we have:
\begin{gather}
\begin{aligned}
\label{eq:BMCC_UpperBound} 
\forall \myparanthese{q , p} \in \mathbb{R}^2_{>0}, \qquad
M_{q,p}\myparanthese{\myPhi} &=
\max_{k,k' \neq k} \frac{d_{k}^{-\frac1p} \, d_{k'}^{\frac1q}}{d_{max}} \mynorm{\myPhi^T \mybracket{k} \myPhi \mybracket{k'}}_{q \to p} \\
&\leq \max_{k,k' \neq k} \frac{d_{k}^{-\frac1p} \, d_{k'}^{\frac1q}}{d_{max}} 
\mynorm{\myPhi^T \mybracket{k}}_{q \to p} \mynorm{\myPhi[k']}_{q \to p} \max \mybrace{1 , m^{\frac1q - \frac1p}}.
%\max \mybrace{\mynorm{\myPhi^T \mybracket{k}}_{q \to p}, \mynorm{\myPhi^T [k]}_{p \to p}} \max \mybrace{\mynorm{\myPhi \mybracket{k'}}_{q \to q},  \mynorm{\myPhi[k']}_{q \to p}}.
\end{aligned}
\end{gather}
On the other hand, 
%$d_{k}^{-1/p} d_{k'}^{1/q}$ is bounded between $d_{min}^{1/q} d_{max}^{-1/p}$ and $d_{min}^{-1/p} d_{max}^{1/q}$ and 
from the following property, 
%Lemma \ref{lm:qpTO22}, 
$\Vert \cdot \Vert_{q {\to} p}$ can be bounded in terms of $\Vert \cdot \Vert_{2 {\to} 2}$, which is often called the spectral norm:
\begin{property}[Bounds of $\ell_{q {\to} p}$ operator-norm in terms of $\ell_{2 {\to} 2}$]
\label{lm:qpTO22}
The bounds of the $\ell_{q {\to} p}$ operator-norm of a matrix $\boldsymbol{A} \ssin \mathbb{R}^{m \stimes n}$ in terms of its $\ell_{2 {\to} 2}$ operator-norm based on the second set of bounds in Property \ref{prp:OperatorProperties} ($\ell_{q {\to} p}$ operator-norm properties) for $q' \seq p' \seq 2$ is:
\begin{gather*}
\begin{aligned}
&\mynorm{\boldsymbol{A}}_{q \to p} \geq \min \mybrace{1 , m^{\frac1p - \frac{1}{2}}} \min \mybrace{1 , n^{\frac{1}{2} - \frac1q}} \mynorm{\boldsymbol{A}}_{2 \to 2}, \\
&\mynorm{\boldsymbol{A}}_{q \to p} \leq \max \mybrace{1 , m^{\frac1p - \frac{1}{2}}} \max \mybrace{1 , n^{\frac{1}{2} - \frac1q}} \mynorm{\boldsymbol{A}}_{2 \to 2}.
\end{aligned}
\end{gather*}
These bounds for different values of $q$ and $p$ are shown in table \ref{table:qpTO22}.
\begin{table}[bp]
%\begin{adjustbox}{width=0.5\textwidth} % ,totalheight=\textheight,.5
\centering
%\tiny
\begin{tabular}{ccccc}
\toprule
%\cline{2-4}
\multicolumn{1}{c}{} &\multicolumn{1}{c}{${q\, \& \, p \leq 2}$} & \multicolumn{1}{c}{${q\, \& \, p \geq 2}$}  & \multicolumn{1}{c}{${q \leq 2\, \& \, p \geq 2}$} & \multicolumn{1}{c}{${q \geq 2\, \& \, p \leq 2}$} \\ \midrule %\hline
\multicolumn{1}{r}{${\frac{\mynorm{\boldsymbol{A}}_{q \to p}}{\mynorm{\boldsymbol{A}}_{2 \to 2}} \leq}$} &\multicolumn{1}{c}{$m^{\frac1p - \frac12}$} & \multicolumn{1}{c}{$n^{-\frac1q + \frac12}$} & \multicolumn{1}{c}{$1$} &\multicolumn{1}{c}{$m^{\frac1p - \frac12} n^{-\frac1q + \frac12}$}    \\ %\midrule %\hline
\multicolumn{1}{r}{${\frac{\mynorm{\boldsymbol{A}}_{q \to p}}{\mynorm{\boldsymbol{A}}_{2 \to 2}} \geq}$} &\multicolumn{1}{c}{$n^{-\frac1q + \frac12}$} & \multicolumn{1}{c}{$m^{\frac1p - \frac12}$} & \multicolumn{1}{c}{$m^{\frac1p - \frac12} n^{-\frac1q + \frac12}$} &\multicolumn{1}{c}{$1$}    \\
\bottomrule %\hline
\end{tabular}
%\end{adjustbox}
\caption{Bounds of $\Vert \boldsymbol{A} \Vert_{q \to p}/ \Vert \boldsymbol{A} \Vert_{2 \to 2}$ for different values of $q$ and $p$, where, $\boldsymbol{A}$ is a $m$ by $n$ matrix.}
\label{table:qpTO22}
\end{table}
\end{property}
\iffalse
\begin{proof}%[Bounds of $\mynorm{\cdot}_{q \to p}$]
From the definition of the operator-norm \cite{Tropp2004b,Golub2013}, we have $\Vert \boldsymbol{A}\Vert_{q \to p} \seq \max_{\boldsymbol{a} \neq \boldsymbol{0}} \Vert \boldsymbol{Aa}\Vert_p {/} \Vert \boldsymbol{a}\Vert_q$, where, $\boldsymbol{A} \ssin \mathbb{R}^{d_1 \stimes d_2}$, $\boldsymbol{a} \ssin \mathbb{R}^{d_2}$, and hence $\boldsymbol{Aa} \ssin \mathbb{R}^{d_1}$.
On the other hand, based on the bounds introduced in Property \ref{prp:VectorDivisionBound}, we have:


for $0 \sless q' \sless p'$, and $\boldsymbol{b} \ssin \mathbb{R}^{d}$, we have $B1 {:} d^{1/p' \sm 1/q'} \sleq \Vert \boldsymbol{b} \Vert_{p'} {/} \Vert \boldsymbol{b} \Vert_{q'} \sleq 1$ and $B2 {:} 1 \sleq \Vert\boldsymbol{b} \Vert_{q'} {/} \Vert \boldsymbol{b} \Vert_{p'} \sleq d^{1/q' \sm 1/p'}$ \cite{Golub2013}.
To compute the upper-bound of $\Vert \boldsymbol{A}\Vert_{q \to p} \seq \max_{\boldsymbol{a} \neq \boldsymbol{0}} \Vert \boldsymbol{Aa}\Vert_p {/} \Vert \boldsymbol{a}\Vert_q$, we need to find the maximum value of the numerator, i.e., $\Vert \boldsymbol{Aa}\Vert_p$, and the minimum value of the denominator, i.e., $\Vert \boldsymbol{a}\Vert_q$, for each case of pairs of $q$ and $p$.
Similarly, the lower-bound can be computed.
Utilising the two above-mentioned inequalities $B1$ and $B2$, in the following we demonstrate the values of table \ref{table:qpTO22}:
\begin{itemize}
\item ${q\, \& \, p \sless 2}$: To establish the upper-bound of $\Vert \boldsymbol{A}\Vert_{q \to p}$, using upper-bound of $B2$ with $q' \seq p$ and $p' \seq 2$, we have $\Vert \boldsymbol{Aa}\Vert_p \sleq d_1^{1/p \sm 1/2} \Vert \boldsymbol{Aa}\Vert_2$, whereas using lower-bound of $B2$ with $q' \seq q$ and $p' \seq 2$, we have $\Vert \boldsymbol{a}\Vert_q \sgeq \Vert \boldsymbol{a}\Vert_2$, therefore, $\Vert \boldsymbol{A}\Vert_{q \to p} \sleq \max_{\boldsymbol{a} \neq \boldsymbol{0}} d_1^{1/p \sm 1/2} \Vert \boldsymbol{Aa}\Vert_2 {/} \Vert \boldsymbol{a}\Vert_2 \seq d_1^{1/p \sm 1/2} \Vert \boldsymbol{A}\Vert_{2 \to 2}$.
On the other hand, to find the lower-bound of $\Vert \boldsymbol{A}\Vert_{q \to p}$, using lower-bound of $B2$ with $q' \seq p$ and $p' \seq 2$, we have $\Vert \boldsymbol{Aa}\Vert_p \sgeq \Vert \boldsymbol{Aa}\Vert_2$, whereas using upper-bound of $B2$ with $q' \seq q$ and $p' \seq 2$, we have $\Vert \boldsymbol{a}\Vert_q \sleq d_2^{1/q \sm 1/2} \Vert \boldsymbol{a}\Vert_2$, therefore, $\Vert \boldsymbol{A}\Vert_{q \to p} \sgeq \max_{\boldsymbol{a} \neq \boldsymbol{0}} d_2^{-1/q \spl 1/2} \Vert \boldsymbol{Aa}\Vert_2 {/} \Vert \boldsymbol{a}\Vert_2 \seq d_2^{-1/q \spl 1/2} \Vert \boldsymbol{A}\Vert_{2 \to 2}$.
\item ${q\, \& \, p \sg 2}$: To establish the upper-bound of $\Vert \boldsymbol{A}\Vert_{q \to p}$, using upper-bound of $B1$ with $p' \seq p$ and $q' \seq 2$, we have $\Vert \boldsymbol{Aa}\Vert_p \sleq \Vert \boldsymbol{Aa}\Vert_2$, whereas using lower-bound of $B1$ with $p' \seq q$ and $q' \seq 2$, we have $\Vert \boldsymbol{a}\Vert_q \sgeq d_2^{1/q \sm 1/2} \Vert \boldsymbol{a}\Vert_2$, therefore, $\Vert \boldsymbol{A}\Vert_{q \to p} \sleq \max_{\boldsymbol{a} \neq \boldsymbol{0}} d_2^{-1/q \spl 1/2} \Vert \boldsymbol{Aa}\Vert_2 {/} \Vert \boldsymbol{a}\Vert_2 \seq d_2^{-1/q \spl 1/2} \Vert \boldsymbol{A}\Vert_{2 \to 2}$.
On the other hand, to find the lower-bound of $\Vert \boldsymbol{A}\Vert_{q \to p}$, using lower-bound of $B1$ with $p' \seq p$ and $q' \seq 2$, we have $\Vert \boldsymbol{Aa}\Vert_p \sgeq d_1^{1/p \sm 1/2} \Vert \boldsymbol{Aa}\Vert_2$, whereas using upper-bound of $B1$ with $p' \seq q$ and $q' \seq 2$, we have $\Vert \boldsymbol{a}\Vert_q \sleq \Vert \boldsymbol{a}\Vert_2$, therefore, $\Vert \boldsymbol{A}\Vert_{q \to p} \sgeq \max_{\boldsymbol{a} \neq \boldsymbol{0}} d_1^{1/p \sm 1/2} \Vert \boldsymbol{Aa}\Vert_2 {/} \Vert \boldsymbol{a}\Vert_2 \seq d_1^{1/p \sm 1/2} \Vert \boldsymbol{A}\Vert_{2 \to 2}$.
\item ${q \sleq 2\, \& \, p \sgeq 2}$: To establish the upper-bound of $\Vert \boldsymbol{A}\Vert_{q \to p}$, using upper-bound of $B1$ with $p' \seq p$ and $q' \seq 2$, we have $\Vert \boldsymbol{Aa}\Vert_p \sleq \Vert \boldsymbol{Aa}\Vert_2$, whereas using lower-bound of $B2$ with $q' \seq q$ and $p' \seq 2$, we have $\Vert \boldsymbol{a}\Vert_q \sgeq \Vert \boldsymbol{a}\Vert_2$, therefore, $\Vert \boldsymbol{A}\Vert_{q \to p} \sleq \max_{\boldsymbol{a} \neq \boldsymbol{0}} \Vert \boldsymbol{Aa}\Vert_2 {/} \Vert \boldsymbol{a}\Vert_2 \seq \Vert \boldsymbol{A}\Vert_{2 \to 2}$.
On the other hand, to find the lower-bound of $\Vert \boldsymbol{A}\Vert_{q \to p}$, using lower-bound of $B1$ with $p' \seq p$ and $q' \seq 2$, we have $\Vert \boldsymbol{Aa}\Vert_p \sgeq d_1^{1/p \sm 1/2} \Vert \boldsymbol{Aa}\Vert_2$, whereas using upper-bound of $B2$ with $q' \seq q$ and $p' \seq 2$, we have $\Vert \boldsymbol{a}\Vert_q \sleq d_2^{1/q \sm 1/2} \Vert \boldsymbol{a}\Vert_2$, therefore, $\Vert \boldsymbol{A}\Vert_{q \to p} \sgeq \max_{\boldsymbol{a} \neq \boldsymbol{0}} d_1^{1/p \sm 1/2} \, d_2^{-1/q \spl 1/2} \Vert \boldsymbol{Aa}\Vert_2 {/} \Vert \boldsymbol{a}\Vert_2 \seq d_1^{1/p \sm 1/2} \, d_2^{-1/q \spl 1/2} \Vert \boldsymbol{A}\Vert_{2 \to 2}$.
\item ${q \sgeq 2\, \& \, p \sleq 2}$: To establish the upper-bound of $\Vert \boldsymbol{A}\Vert_{q \to p}$, using upper-bound of $B2$ with $q' \seq p$ and $p' \seq 2$, we have $\Vert \boldsymbol{Aa}\Vert_p \sleq d_1^{1/p \sm 1/2} \Vert \boldsymbol{Aa}\Vert_2$, whereas using lower-bound of $B1$ with $p' \seq q$ and $q' \seq 2$, we have $\Vert \boldsymbol{a}\Vert_q \sgeq d_2^{1/q \sm 1/2} \Vert \boldsymbol{a}\Vert_2$, therefore, $\Vert \boldsymbol{A}\Vert_{q \to p} \sleq \max_{\boldsymbol{a} \neq \boldsymbol{0}} d_1^{1/p \sm 1/2} d_2^{-1/q \spl 1/2} \Vert \boldsymbol{Aa}\Vert_2 {/} \Vert \boldsymbol{a}\Vert_2 \seq d_1^{1/p \sm 1/2} d_2^{-1/q \spl 1/2} \Vert \boldsymbol{A}\Vert_{2 \to 2}$.
On the other hand, to find the lower-bound of $\Vert \boldsymbol{A}\Vert_{q \to p}$, using lower-bound of $B2$ with $q' \seq p$ and $p' \seq 2$, we have $\Vert \boldsymbol{Aa}\Vert_p \sgeq \Vert \boldsymbol{Aa}\Vert_2$, whereas using upper-bound of $B1$ with $p' \seq q$ and $q' \seq 2$, we have $\Vert \boldsymbol{a}\Vert_q \sleq \Vert \boldsymbol{a}\Vert_2$, therefore, $\Vert \boldsymbol{A}\Vert_{q \to p} \sgeq \max_{\boldsymbol{a} \neq \boldsymbol{0}}  \Vert \boldsymbol{Aa}\Vert_2 {/} \Vert \boldsymbol{a}\Vert_2 \seq \Vert \boldsymbol{A}\Vert_{2 \to 2}$.
\end{itemize} 
\end{proof}
\fi
Therefore, based on Property \ref{lm:qpTO22}, we upper-bound the $\ell_{q {\to} p}$ operator-norms of the upper-bound of Block-MCC in equation (\ref{eq:BMCC_UpperBound}):
\begin{gather}
\label{eq:BMCC-UB} 
\begin{aligned} 
M_{q,p}\myparanthese{\myPhi} \leq 
\max_{k,k' \neq k} \frac{d_{k}^{-\frac1p} \, d_{k'}^{\frac1q}}{d_{max}} 
&\mycolor{\max} \mybrace{1 , d_{k}^{\frac1p - \frac{1}{2}}} \mycolor{\max} \mybrace{1 , m^{\frac{1}{2} - \frac1q}} \mynorm{\myPhi^T \mybracket{k}}_{2 \to 2} \times\\ 
&\mycolor{\max} \mybrace{1 , m^{\frac1p - \frac{1}{2}}} \mycolor{\max} \mybrace{1 , d_{k'}^{\frac{1}{2} - \frac1q}} \mynorm{\myPhi[k']}_{2 \to 2} 
\max \mybrace{1 , m^{\frac1q - \frac1p}}.
\end{aligned}
\end{gather}

But $\forall k, \Vert \myPhi^T [k] \Vert_{2 {\to} 2} \seq \Vert \myPhi[k] \Vert_{2 {\to} 2} \seq 1$, because for a typical matrix $\boldsymbol{A} \ssin \mathbb{R}^{m \stimes n}$ we have $\Vert \boldsymbol{A} \Vert_{2 {\to} 2} \seq \sigma_{max}(\boldsymbol{A}) \seq \sqrt{\lambda_{max}(\boldsymbol{A}^T \boldsymbol{A})}$, where, $\sigma$ and $\lambda$ are singular value and eigenvalue, respectively, but here $\boldsymbol{A}$ is an  orthonormal matrix, then $\boldsymbol{A}^T \boldsymbol{A} \seq \boldsymbol{I}_n$, and then $\lambda_{max}(\boldsymbol{I}_n) \seq 1$.

Therefore, the inequality (\ref{eq:BMCC-UB}) is developed into table \ref{table:BMCC_UpperBound}.
In each case in the last column of the table \ref{table:BMCC_UpperBound}, the maximum value is achieved for minimum block length $d_{min}$ with negative power, and maximum block length $d_{max}$ with positive power.
The table \ref{table:BMIC-orth-bound} shows the upper-bound of $M_{q,p}(\myPhi)$ based on table \ref{table:BMCC_UpperBound}, for the basic tractable values of $q$ and $p$ based on table \ref{table:OperatorNorm}. 
\begin{table}[tp]
\begin{adjustbox}{width=1\textwidth} % ,totalheight=\textheight,.5
\centering
%\tiny
\begin{tabular}{ccccc}
\toprule
%\cline{2-4}
\multicolumn{1}{c}{} & \multicolumn{1}{c}{$\max M_{q,p}(\myPhi)$} & \multicolumn{1}{c}{ } & \multicolumn{1}{c}{$\max M_{q,p}(\myPhi)$} \\ \midrule %\hline
\multicolumn{1}{r}{${0 < q\, \& \, p \leq 2}$} & \multicolumn{1}{l}{$\displaystyle\max_{k,k' \neq k} d_{k}^{-\frac12} \, d_{k'}^{\frac1q} \, d_{max}^{-1} \, m^{\frac1p - \frac12} \, \max \mybrace{1 , m^{\frac1q - \frac1p}}$} & \multicolumn{1}{l}{$\leq$} & \multicolumn{1}{l}{$d_{min}^{-\frac12} \, d_{max}^{\frac1q - 1} \, m^{\frac1p - \frac12} \, \max \mybrace{1 , m^{\frac1q - \frac1p}}$}  \\ 
\multicolumn{1}{r}{${q\, \& \, p \geq 2}$} & \multicolumn{1}{l}{$\displaystyle\max_{k,k' \neq k} d_{k}^{-\frac1p} \, d_{k'}^{\frac12} \, d_{max}^{-1} \, m^{\frac12 - \frac1q} \, \max \mybrace{1 , m^{\frac1q - \frac1p}}$} & \multicolumn{1}{l}{$\leq$} & \multicolumn{1}{l}{$d_{min}^{-\frac1p} \, d_{max}^{-\frac12} \, m^{\frac12 - \frac1q} \, \max \mybrace{1 , m^{\frac1q - \frac1p}}$}  \\
\multicolumn{1}{r}{${0 < q \leq 2\, \& \, p \geq 2}$} & \multicolumn{1}{l}{$\displaystyle\max_{k,k' \neq k} d_{k}^{-\frac1p} \, d_{k'}^{\frac1q} \, d_{max}^{-1} \, m^{\frac1q - \frac1p}$}  & \multicolumn{1}{l}{$\leq$} & \multicolumn{1}{l}{$d_{min}^{-\frac1p} \, d_{max}^{\frac1q - 1} \, m^{\frac1q - \frac1p}$}  \\ 
\multicolumn{1}{r}{${q \geq 2\, \& \, 0 < p \leq 2}$} & \multicolumn{1}{l}{$\displaystyle\max_{k,k' \neq k} d_{k}^{-\frac12} \, d_{k'}^{\frac12} \, d_{max}^{-1} \, m^{\frac1p - \frac1q}$}  & \multicolumn{1}{l}{$\leq$} & \multicolumn{1}{l}{$d_{min}^{-\frac12} \, d_{max}^{-\frac12} \, m^{\frac1p - \frac1q}$} \\
\bottomrule %\hline
\end{tabular}
\end{adjustbox}
\caption{Upper-bound of Block-MCC$_{q,p}$ for different ranges of $q$ and $p$ and for a dictionary with intra-block orthonormality.}
\label{table:BMCC_UpperBound}
\end{table}
\end{proof}