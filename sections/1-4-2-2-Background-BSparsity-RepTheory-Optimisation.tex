Similar to the conventional element-wise sparse recovery, the problem of finding a solution $\hat{\mybeta}$, in the noiseless case of (\ref{eq:Model_Noiseless}), i.e., $\boldsymbol{y} \seq \myPhi \mybetaz$, is called \emph{block-sparse exact (unique) signal recovery} whereas in the other case of (\ref{eq:Model_Noisy}), i.e., $\boldsymbol{y} \seq \myPhi \mybetaz \spl \boldsymbol{e}$, is called \emph{block-sparse stable (robust) signal recovery}.

Block-sparse exact and stable signal recovery problems can be realised through a proper constrained optimisation problem.
% (program). 
The straightforward optimisation problem for block-sparse stable signal recovery is:
\begin{equation}
\label{eq:Opt-L20} 
\min_{\mybeta} \mynorm{\mybeta}_{2,0} \qquad s.t. \qquad \mynorm{\boldsymbol{y} - \myPhi \mybeta}_2 \leq \varepsilon,
\end{equation}
where, for $\varepsilon \seq 0$, the optimisation problem (\ref{eq:Opt-L20}) solves a block-sparse exact signal recovery problem.
 
In a general framework, consider the following \emph{constrained $\ell_{p_1,p_2}$ (pseudo-)mixed-norm minimisation problem}:
\begin{equation}
\label{eq:Opt-Lp1p2}
P_{\myparanthese{p_1,p_2},q,\varepsilon} \myparanthese{\boldsymbol{a},\boldsymbol{b}}: \min_{\boldsymbol{a}} \mynorm{\boldsymbol{a}}_{p_1,p_2} \quad s.t. \quad \mynorm{\boldsymbol{b}}_q \leq \varepsilon,
\end{equation}
where, $\boldsymbol{a}$ and $\boldsymbol{b}$ are vectors, $\varepsilon$ bounds the noise level, i.e., $0 \sleq \epsilon \sleq \varepsilon$, and $\ell_{p_1,p_2}$ (pseudo-)mixed-norm of block-structured vector $\boldsymbol{a} \seq \mybracket{\boldsymbol{a}^T\mybracket{1}, \cdots, \boldsymbol{a}^T\mybracket{k}, \cdots, \boldsymbol{a}^T\mybracket{K}}^T$, where, $\boldsymbol{a} \mybracket{k} \seq \mybracket{a_1\mybracket{k}, \cdots, a_{d} \mybracket{k}}^T$, is defined:
\begin{equation*}
\label{eq:MixedNormDef} 
\mynorm{\boldsymbol{a}}_{p_1,p_2} \myeq 
\begin{cases}
    \displaystyle\sum_{k}^{} I \myparanthese{\mynorm{\boldsymbol{a} \mybracket{k}}_{p_1}}, & \qquad \text{for \ }p_2= 0\\
    \myparanthese{\displaystyle\sum_{k}^{} \mynorm{\boldsymbol{a} \mybracket{k}}_{p_1}^ {p_2}} ^ {\frac{1}{p_2}},  & \qquad  \text{for \ } 0 < p_2 < +\infty\\
    \displaystyle\max_{k}{\mybrace{\mynorm{\boldsymbol{a} \mybracket{k}}_{p_1}}}, & \qquad \text{for \ } p_2= \infty,
  \end{cases}
\end{equation*}
where, $I(a)$ is the indicator function, i.e., $I(a) \myeq
  \begin{cases}
    1,  & \quad \text{if } \myabs{a} \sg 0\\
    0,  & \quad \text{if } a \seq 0\\
  \end{cases}$.
%The $\ell_{p_1,p_2}$ mixed-norm for different values of $p_1$ and $p_2$ is investigated in Definition \ref{def:Mixed norm}.

According to (\ref{eq:Opt-Lp1p2}) the problem (\ref{eq:Opt-L20}) can be represented as $P_{(2,0),2,\varepsilon} (\mybeta,\boldsymbol{y} \sm \myPhi \mybeta)$.

Although the notation $P_{(p_1,p_2),q,\varepsilon} (\boldsymbol{a},\boldsymbol{b})$ can cover all the previous types of optimisation problems, the widely used case is when the minimisation is over $\mybeta$ and the problem is a block-sparse exact signal recovery, i.e., $P_{(p_1,p_2),q,0} (\mybeta,\boldsymbol{r})$, which for the sake of simplicity we refer to as $P_{p_1,p_2}$.
Because regardless of the choice of $q$, the $\Vert \boldsymbol{r} \Vert _q \seq 0$ directly indicates to the noiseless model, i.e., $\boldsymbol{y} \seq \myPhi \mybetaz$, so the only parameters to be determined are $p_1$ and $p_2$.

Since the optimisation problem in block-sparsity domain is a generalisation of the conventional optimisation problem in sparsity domain, for $d \seq 1$, the two optimisation problems are equivalent, i.e., $P_{p_1,p_2} {\equiv} P_{p_2}$.

The straightforward configuration of the mentioned optimisation problem in a block-sparse exact or stable signal recovery problem is $P_{(p,0),q,\varepsilon}(\mybeta,\boldsymbol{r})$, which simply minimise the number of active blocks, i.e., blocks with non-zero $\ell_p$ norm.

However, the $P_{(p,0),q,\varepsilon}(\mybeta,\boldsymbol{r})$ optimisation problem is in general non-convex and finding its solution is NP-hard\footnote{\emph{Non-deterministic Polynomial-hard}} \cite{Natarajan1995,Amaldi1998}.
Therefore, there is a need for convexification, relaxation or approximation of $P_{(p,0),q,\varepsilon}(\mybeta,\boldsymbol{r})$.
Similar to sparsity domain, in block-sparsity domain the following two main questions are addressed:
\begin{list}{Q\arabic{qcounter}:}{\usecounter{qcounter}}
\item Is the solution to $P_{2,1}$ necessarily the block-sparsest representation (solution to $P_{2,0}$) (\emph{Uniqueness})?\\
\item Is the solution to $P_{2,0}$ necessarily equal to the solution to $P_{2,1}$ (\emph{Equivalence})?
\end{list}
%Fortunately, the mentioned equivalence theorem for two optimisation problems $P_1$ and $P_0$ in exact signal recovery problem, has a correspondence in block case for finding solution in a block-sparse exact signal recovery problem. 
%In other words, 
It has been shown that under sufficient conditions, $P_{2,1}$ and $P_{2,0}$ are equivalent \cite{Eldar2009b,Eldar2009c,Stojnic2009a,Eldar2010b,Eldar2010}, and in a more general case the equivalence of $P_{p,1}$ and $P_{p,0}$ for $p \sg 0$ and for both non-redundant and redundant blocks is investigated \cite{Elhamifar2011,Elhamifar2012b}.

\myhl{Subspace matching pursuit is a greedy algorithm, which is used for a block-sparse exact and stable signal recovery problems of $P_{(2,0),q,0}(\mybeta_{\myPhi},\boldsymbol{r})$ and $P_{(2,0),2,\varepsilon}(\mybeta_{\myPhi},\boldsymbol{r})$, respectively, where, $\mybeta_{\myPhi} \seq \mybracket{(\myPhi[1] \mybeta[1])^T, \cdots, (\myPhi[k] \mybeta[k])^T, \cdots, (\myPhi[K] \mybeta[K])^T}^T$, and the $\myPhi[k]$\textquotesingle s are assumed to be full rank matrices {\cite{Ganesh2009}}.

In addition, subspace base pursuit and its modified version is used for a block-sparse exact and stable signal recovery problems of $P_{(2,1),q,0}(\mybeta_{\myPhi},\boldsymbol{r})$ and $P_{(2,1),2,\varepsilon}(\mybeta_{\myPhi},\boldsymbol{r})$, respectively, and under sufficient conditions, $P_{(2,1),q,0}(\mybeta_{\myPhi},\boldsymbol{r})$ and $P_{(2,0),q,0}(\mybeta_{\myPhi},\boldsymbol{r})$ are equivalent {\cite{Ganesh2009}}.

Moreover, there are theoretical conditions for ensuring the equivalence of $P_{(p,1),q,0}(\mybeta_{\myPhi},\boldsymbol{r})$ and $P_{(p,0),q,0}(\mybeta_{\myPhi},\boldsymbol{r})$ for $p \sg 0$, while the blocks could be non-redundant or redundant {\cite{Elhamifar2011}}; {\cite{Elhamifar2012b}}.
% and for both non-redundant and redundant blocks \cite{Elhamifar2011,Elhamifar2012b}.

In fact, unlike $P_{p,0}$ that extracts a representation with the fewest active blocks $\mybeta[k]$, the $P_{(p,0),q,0}(\mybeta_{\myPhi},\boldsymbol{r})$ extracts a representation with the fewest active reconstructed vectors $\myPhi[k] \mybeta[k]$ {\cite{Elhamifar2012}}.
When the blocks $\myPhi[k]$\textquotesingle s are non-redundant, 
%In case of each block non-redundancy, 
the solution to $P_{(p,0),q,0}(\mybeta_{\myPhi},\boldsymbol{r})$ has also the fewest active blocks {\cite{Elhamifar2012}}.}

There are also some recovery conditions of $P_{2,1}$ via the mixed $\ell_2/\ell_1$-optimisation program \cite{Eldar2009c,Stojnic2009a}, also known in the statistics literature as group LASSO\footnote{\emph{Least Absolute Shrinkage and Selection Operator}} \cite{Yuan2006,Lv2011}, 
%\cite{Yuan2006,Chesneau2008,Nardi2008,Bach2008,Meier2008,Huang2009a,Lv2011}, 
BMP (Block Matching Pursuit) 
%\cite{Eldar2009b,Eldar2010b,Eldar2010} 
and BOMP (Block Orthogonal Matching Pursuit) 
%\cite{Eldar2009b,Swirszcz2009,Eldar2010b,Eldar2010} 
algorithms \cite{Eldar2010}.

In addition, there are block-sparse stable recovery conditions for block orthogonal matching pursuit and block-thresholding algorithms \cite{Ben-Haim2011}. 
Also $\ell_2/\ell_1$-optimisation program \cite{Eldar2009c}, block versions of CoSaMP\footnote{\emph{Compressive Sampling Matched Pursuit}} and IHT\footnote{\emph{Iterative Hard Thresholding}} \cite{Baraniuk2010} algorithms (which are the extensions of iterative sparse coding algorithms to model-based sparse representation) lead to robust recoveries in the presence of noise.