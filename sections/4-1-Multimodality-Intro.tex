As mentioned in Section \ref{sec:Multimodality}, EEG\footnote{\emph{ElectroEncephaloGraphy}} and MEG\footnote{\emph{MagnetoEncephaloGraphy}} source reconstruction problems solve a same brain source activity problem when acquired simultaneously \cite{Molins2008}.
In other words, for EEG or MEG signal $\boldsymbol{y}_{\boldsymbol{\mathrm{EEG}} / \boldsymbol{\mathrm{MEG}}} \ssin \mathbb{R}^m$, EEG or MEG lead-field matrix $\myPhi_{\boldsymbol{\mathrm{EEG}} / \boldsymbol{\mathrm{MEG}}} \ssin \mathbb{R}^{m \stimes n}$, and brain source activity $\mybetaz \ssin \mathbb{R}^n$, we have the following EEG and MEG USLE \footnote{\emph{Underdetermined System(s) of Linear Equations}}:
\begin{equation*}
\boldsymbol{y}_{\boldsymbol{\mathrm{EEG}}} = \myPhi_{\boldsymbol{\mathrm{EEG}}} \mybetaz, \qquad \text{and} \qquad
\boldsymbol{y}_{\boldsymbol{\mathrm{MEG}}} = \myPhi_{\boldsymbol{\mathrm{MEG}}} \mybetaz.
\end{equation*}
%On the other hand, EEG and MEG are complementary in observing the same brain source activity.

The key aim in this chapter is to integrate the complementary information of EEG and MEG modalities within the block structure identification framework introduced in Chapter \ref{sec:Clustering} with application to brain source space segmentation.
%by combining the information of the two modalities, i.e., multi-modal signal processing, one can improve the final results, which is the ultimate aim of this chapter.

In contrast to the brain source space segmentation studies in literature, which are integrating the information of EEG or MEG signals $\boldsymbol{y}_{\boldsymbol{\mathrm{EEG}} / \boldsymbol{\mathrm{MEG}}}$ and lead-fields $\myPhi_{\boldsymbol{\mathrm{EEG}} / \boldsymbol{\mathrm{MEG}}}$ into the problem, the proposed multi-modality strategy uses the block structure identification framework, which requires only the lead-field matrices, in order to demonstrate the benefit of merging EEG and MEG modalities at the same complexity or number of sensors.