The goal in EEG/MEG source analysis is finding the appropriate sources based on the electric/magnetic measurements, which is also called the \emph{inverse problem}.
It is called an inverse problem because it exploits the information at the results stage and then leads to the causes.

Inverse problems are not specific to the EEG/MEG source analysis research field and are considered as one of the most important mathematical problems in science.
The importance of the inverse problems is due to the fact that by solving the problems inversely, it would be possible to discover the latent parameters, which are not measurable directly.

In order to solve inverse problem, one must first solve the so-called \emph{forward problem}, i.e., what would be the electromagnetic fields of a known source?
This is the inverse of a inverse problem, which exploits the information of the causes and then leads to the results.

In figure \ref{fig:ForwardInverse}, the forward problem is to determine EEG/MEG measurements in sensor space resulted from three active dipoles in source space, whereas, the inverse problem is to estimate the activity of dipoles in source space given EEG/MEG measurements in sensor space.
\begin{figure}[!b]
\centering
\includegraphics[width=1\textwidth]{images/ForwardInverse.png} % width=0.5\textwidth  scale=0.49
\caption{The forward and inverse problems in EEG/MEG source analysis.}
\label{fig:ForwardInverse}
\end{figure}
\FloatBarrier