Assuming that the number of sensors in both modalities is \myhl{even}, then each signal $\boldsymbol{y}_{\boldsymbol{\mathrm{EEG}}}^{O} \ssin \mathbb{R}^{m/2}$ or $\boldsymbol{y}_{\boldsymbol{\mathrm{MEG}}}^{E} \ssin \mathbb{R}^{m/2}$ has as half of observations as the original signal $\boldsymbol{y}_{\boldsymbol{\mathrm{EEG}}/\boldsymbol{\mathrm{MEG}}} \ssin \mathbb{R}^{m}$, \myhl{i.e., each modality contributes the same in terms of the number of sensors, so the comparison would be more fair compared to the odd number of one of original modalities' sensors}.
Then, by concatenating them, the number of equations in the resulted USLE in multi-modality would be equal to the the original USLE in mono-modality.

By keeping fixed the number of sensors in mono-modality and multi-modality, we assure that the probable added value is because of the multi-modality itself and not in reason of the increased observation.
Because, in case of utilising all the equations in each of mono-modal USLE corresponding to EEG and MEG, the equations in final USLE  would be more than the original USLE, i.e., $\boldsymbol{y}_{\boldsymbol{\mathrm{EMEG}}} \ssin \mathbb{R}^{2m}$, where, $\boldsymbol{y}_{\boldsymbol{\mathrm{EMEG}}}$ is the multi-modal observation.
Therefore, we would have an increased input information in multi-modality in comparison to mono-modality, which would not lead to a fair comparison.