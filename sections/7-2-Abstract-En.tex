Many natural phenomena are too complex to be fully recognised by only a single measurement instrument or mono-modality. Therefore, the research domain of multi-modality has emerged to better identify the rich characteristics of the natural multi-property phenomenon, through jointly analysing the data collected from mono-modalities, which are somehow complementary. 
In our study, the multi-property phenomenon of interest is the human brain activity and we are interested in better localising it by means of its electromagnetic properties which are measurable non-invasively. In neurophysiology, a common way to measure the electric and magnetic properties of the brain activity is ElectroEncephaloGraphy (EEG), and MagnetoEncephaloGraphy (MEG), respectively. Our real-world application, i.e., EEG/MEG source reconstruction problem, is a fundamental problem in neuroscience ranging from cognitive science to neuropathology to surgical planning. 
Considering that the EEG/MEG source reconstruction problem can be reformulated as an underdetermined system of linear equations, the solution (estimated brain source activity) should be sufficiently sparse in order to be recovered uniquely. The amount of sparsity is determined by the so-called recovery conditions. However, in high-dimensional problems, the conventional recovery conditions are extremely strict. By regrouping the coherent columns of a dictionary, the more incoherent structure could be achieved. This strategy was proposed as a block structure identification framework, which results in the automatic segmentation of the brain source space, without using any information about the brain sources activity and EEG/MEG signals. Despite the resulted less coherent block-structured dictionary, the conventional recovery condition is no longer capable of computing the coherence characterisation. To address the mentioned challenge, the general framework of block-sparse exact recovery conditions including three theoretical and one algorithmic-dependent conditions was proposed. Finally, we investigated the EEG and MEG multi-modality and demonstrated that by combining the two modalities, more refined brain regions appeared.
%\iffalse
%\myhl{In many research area such as signal and image processing, bioinformatics, etc., researchers end up with a high-dimensional inverse problem, which can be reformulated as a vastly underdetermined system of linear equations.
%Usually, the measurement vector is not represented uniformly in all blocks of columns of dictionary, and can be represented in a few number of low-dimensional blocks in dictionary through a sparse representation vector.
%Recovering such low-dimensional representative block structure, which is feasible through a proper constrained optimisation problem, helps to reduce the computational burden and memory requirements of algorithms utilising the data.
%Despite the fact that an underdetermined system of linear equations has infinitely many solutions, the mentioned proper constrained optimisation problem under certain conditions is able to recover a unique sparse solution (or representation vector).
%In other words, if a representation vector is sparse enough, then it is theoretically possible to recover it uniquely via a constrained optimisation problem.
%Usually the mentioned sparsity of the representation vector is a function of a characterisation of the dictionary.
%Sometimes, depending on the nature of the dictionary, the sparsity constraint on the representation vector is so strong (tight) that the need to weaken the sparse recovery conditions becomes more pronounced.
%
%
%Three main challenges have been addressed in this thesis, in three chapters.
%First challenge is about the ineffectiveness of classic sparse recovery conditions in high-dimensional problems. 
%This challenge is partially addressed through the idea of clustering the coherent columns (or block of columns) of the dictionary based on the proposed dictionary characterisation, in order to establish more incoherent atomic entities in the dictionary, which is proposed as a \emph{block structure identification framework}. 
%The more incoherent atomic entities, the more improvement in the sparse recovery conditions, i.e., the conditions are more weakened.
%%In addition, we applied the mentioned clustering idea to real-world EEG/MEG leadfields to segment the brain source space, without using any information about the brain sources activity and EEG/MEG signals.
%Second challenge raises when classic sparse recovery conditions cannot be utilised for the new concept of constraint, i.e., block-sparsity.
%Therefore, as the second research orientation, we developed a \emph{general framework for block-sparse exact recovery conditions}, i.e., three theoretical and one algorithmic-dependent conditions, which ensure the uniqueness of the block-sparse solution of corresponding weighted (pseudo-)mixed-norm optimisation problem in an underdetermined system of linear equations.
%%The mentioned generality of the framework is in terms of the properties of the underdetermined system of linear equations, extracted dictionary characterisations, optimisation problems, and ultimately the recovery conditions.
%Finally, the combination of different information of a same phenomenon is the subject of the third challenge, which is addressed in the last part of dissertation with application to brain source space segmentation.
%More precisely, we showed that by combining the EEG and MEG lead-fields in a \emph{EEG/MEG multi-modality framework} and gaining the electromagnetic properties of the head, more refined brain regions appeared.}
%\fi