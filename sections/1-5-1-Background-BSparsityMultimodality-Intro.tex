In Section \ref{sec:Multimodality}, first the concept of multi-modality is discussed in general.
Then, we showed that due to the fact that a brain neuronal activity have electric and magnetic properties, the brain imaging modality of EEG or MEG individually, cannot completely describe the electromagnetic behaviour of the brain neurons.
On the other hand, EEG and MEG measure the electrical and magnetic activities of the same cerebral currents, respectively.
Since they have complementary information, they can be an appropriate choice as brain imaging modalities in a multi-modality framework. 
Therefore, we took the distributed EEG and MEG source reconstruction problem as a real-world example of multi-modality.

However, the distributed EEG or/and MEG source reconstruction problems are vastly underdetermined, i.e., the number of EEG or/and MEG sensors is significantly less than the number of brain sources, \myhl{e.g., in a typical problem there could be 30 sensors against 3000 sources.}

Hence, in order to recover a unique solution from the USLE of distributed EEG or/and MEG source reconstruction problems, appropriate constraint(s) need(s) to be applied on the corresponding optimisation problem .

In Section \ref{sec:Block-sparsity}, we pointed out that, in certain USLE a block of coefficients need to be penalized and not necessarily a single coefficient.
Hence, the notion of block-sparsity was extended from the conventional notion of sparsity, which was reviewed in Section \ref{sec:Sparsity}.

As a real-world example, we again took the distributed EEG or/and MEG source reconstruction problem.
Since each brain source, which is a dipole, can be represented as a block of size $d \seq 3$, and the source reconstruction problem is consistent with the constraint of block-sparsity.

Although the two mentioned notions of multi-modality and block-sparse representation theory are fundamentally independent, they meet each other in our main real-world problem, i.e., distributed EEG and MEG source reconstruction problem.

In order to recover a unique solution, in addition to imposing the block-sparsity constraint on the corresponding optimisation problem, the number of active blocks should be less than a threshold.
The mentioned restricted number of active blocks is explained through recovery conditions, which are explained in Chapter \ref{sec:BERC}.

Next, in Chapter \ref{sec:Clustering} by following a framework to merge the lead-field matrices of each source position, we show that at level of grouping \myhl{corresponding to maximum inter-group distance and minimum number of groups,} the system satisfies the existing conditions.
 
Finally, in Chapter \ref{chptr:Multimodality}, we demonstrate the impact of EEG and MEG multi-modality and merging framework on brain source space segmentation.