Dictionaries $\myPhi \ssin \mathbb{R}^{m \stimes n}$, with equally-sized blocks of length $d_1 \seq \cdots \seq d_K \seq d$, are generated from independent and identically distributed (i.i.d.) random variables and all columns of the dictionaries are normalized to have unit $\ell_2$ norm.
The idea to generate the whole dictionary is first to start from an initial block, next some representative blocks of clusters are generated, then from each representative block of a cluster, the blocks belonging to the same cluster are produced.
To construct a second block from a first one, the random matrix $\exp((\boldsymbol{C} \sm \boldsymbol{C}^T) \sqrt{2} \varepsilon / \Vert \boldsymbol{C} \sm \boldsymbol{C}^T \Vert_F)$ is multiplied to the first block, where, $\boldsymbol{C}$ is a square random matrix of dimension $m$.
Depending on the role of $\varepsilon$ whether to generate representative blocks of clusters or blocks of dictionary, it is named $\varepsilon_{inter}$ or $\varepsilon_{intra}$, respectively.

%To generate a random dictionary, first an initial block of dimension $m$ by $d$ is considered.
%Then, to create $N$ representatives of clusters $\boldsymbol{\mu}^1 , \cdots \boldsymbol{\mu}^N$,
In figure \ref{fig:Dictionary_generating}, the process of generating a random dictionary $\myPhi$ with $N$ clusters and $K$ blocks is illustrated.
First, from a random initial block $\boldsymbol{\mu}^0 \ssin \mathbb{R}^{m \stimes d}$, $N$ representatives of clusters $\boldsymbol{\mu}^1 , \cdots , \boldsymbol{\mu}^N$ are created, using the mentioned multiplier and parameter $\varepsilon_{inter}$.
Then, from each representative $\boldsymbol{\mu}^i$, $\forall i$, the blocks $\myPhi^i\mybracket{j}$, $\forall j$, belonging to the same cluster $i$ are generated, using the mentioned multiplier and parameter $\varepsilon_{intra}$.
The parameter $\varepsilon$ in the multiplier controls the amount of overlap and similarity between blocks.
In fact, $\varepsilon_{inter}$ defines the overlap between the representative blocks of clusters, whereas $\varepsilon_{intra}$ controls the overlap between the blocks belonging to the same cluster.
%It is possible to control the amount of overlap and similarity between blocks of the dictionary by tuning the parameters $\varepsilon_{inter}$ and $\varepsilon_{intra}$, which the former parameter defines the overlap between the average block of each cluster whereas the latter controls the overlap between the blocks belonging to the same cluster.  
%Finally, length of blocks before clustering, number of rows and number of columns of the dictionary are equal to $2$, $10$, and $80$, respectively.
Finally, the results are shown in terms of average and standard deviation over 100 repetitions of random dictionary $\myPhi \ssin \mathbb{R}^{10 \stimes 80}$ generation with $d_j \seq 2$, $\forall j$.
\begin{figure}[!b]
\centering
\includegraphics[width=1\textwidth,keepaspectratio]{images/Dictionary_generating.png} % width=0.5\textwidth  scale=0.49
\centering
\caption{From initial block $\boldsymbol{\mu}^0$, $N$ representative blocks of clusters are computed. Then, from each representative block $\boldsymbol{\mu}^i$, all blocks $\myPhi^i\mybracket{j}$ belonging to the same cluster $i$ are generated.}
%Assuming $N$ clusters and $K$ blocks, $\boldsymbol{\mu}^i$ is the representative block of cluster $i$, and $\myPhi^i\mybracket{j}$ is the block $j$ belonging to cluster $i$ resulted from multiplication of a n $\varepsilon$-based matrix 
%, where based on the value of $\varepsilon$, resulted matrices can have certain amount of overlap with the one originated from.}
\label{fig:Dictionary_generating}
\end{figure}
\FloatBarrier
%------------------------------------------------------
\paragraph{Block-ERC in clustered representation and the number of clusters}
%\subsubsection{Effect of clustering the coherent blocks of a dictionary on Block-ERC based on Block-MCC$_{q,p}$}
\label{sec:clusteringOFcoherent_BERC-BMIC} 
%In general the recovery conditions can be improved or weakened through increasing the required number of non-zero entities as a threshold which ensures the uniqueness of the solution to an optimisation problem.
%In Corollary \ref{crl:BERC-BMIC} of Section \ref{sec:BERC_BNSP}, the mentioned threshold or $\myBSLTxt$ is expressed as the number of blocks (not the number of elements) in two cases of optimistic and pessimistic.
%In Section \ref{sec:Sparsity level for clustered blocks of a dictionary}, we explained how different sparsity levels can be calculated from $\myBSLTxt$, 
% of a Block-ERC, which upper-bounds the maximum number of active blocks of representation vector $\mybeta$ ensuring the uniqueness of the solution to an optimisation problem. 
%which upper-bounds the maximum number of active blocks in a Block-ERC.

In this experiment, 
%$\myBSLTxt$ resulted from a Block-ERC based on Block-MCC$_{q,p}$ is going to be investigated in clustered representation.
%To this purpose, 
the agglomerative hierarchical clustering algorithm is applied on the blocks of a random dictionary with certain number of clusters, whereas the Block-MCC$_{2,2}$ and complete method are used to measure the inter-blocks coherence and inter-clusters distance, respectively.
The mentioned certain number of clusters is once set to four and once set to eight.

As it can be seen in figure \ref{fig:SL_Hierarchical}, by applying hierarchical clustering on the blocks of the dictionary, i.e., by decreasing the number of clusters, there exists at least one clustering level in which the relative $\myBSLTxt_{2,2}$ in the most pessimistic case, i.e., $\myBSL_{2,2}(\myPhi)[\%]$, increases in comparison to when the clustering is not applied on the dictionary, i.e., the rightmost part of each diagram corresponding to 40 clusters.
Therefore, clustering coherent blocks of the dictionary improves the Block-ERC through increasing the $\myBSLTxt_{2,2}$.

In addition, for $\varepsilon_{inter} \sg 1$ and $\varepsilon_{intra} \sless 0.1$, the $\myBSL_{2,2}(\myPhi)[\%]$ has a peak in a clustering level equal to the number of the clusters in the simulated dictionary.
In fact, in figure \ref{fig:SL_Hierarchical} when there is four clusters in the simulated dictionary (blue curve, square markers), the maximum is at the fourth clustering level and also the consistent results are obtained for eight clusters in the dictionary (red curve, circle markers).
For the clustering level lower than the optimal level, the space is under-sampled and there would be a block spanning the whole space, whereas for the clustering level higher than the optimal level, the space is over-sampled and the over-partitioning leads to high coherence measure.
\begin{figure}[!b]
\centering
\includegraphics[width=1\textwidth,keepaspectratio]{images/SL_Hierarchical.png} % width=0.5\textwidth  scale=0.49
\centering
\caption{$\myBSL_{2,2}(\myPhi)[\%]$ for each level of clustering computed for complete method, Block-MCC$_{2,2}$, $d \seq 2$, $N \seq \{ 4 , 8\}$, and different values of $\varepsilon_{inter}$ and $\varepsilon_{intra}$ for simulating dictionary $\myPhi \ssin \mathbb{R}^{10 \stimes 80}$.}
\label{fig:SL_Hierarchical}
\end{figure}
\FloatBarrier
%------------------------------------------------------
\paragraph{Block-ERC in clustered representation and the conventional ERC}
%\subsubsection{Sparsity level in a clustered blocks compared to the conventional sparsity level}
In this experiment, the goal is to compare the sparsity level obtained in the proposed Block-ERC to the conventional sparsity level which is equal to the half of the number of rows of $\myPhi$.%the dictionary.

First, we need to express the block-sparsity level in terms of the conventional sparsity level as described in Section \ref{sec:Sparsity level for clustered blocks of a dictionary}, e.g., $\mySLTxt_{min}$ and $\mySLTxt_{max}$.
Then, consider one of the cases in figure \ref{fig:SL_Hierarchical}, e.g., $\varepsilon_{inter} \seq 3.5$, $\varepsilon_{intra} \seq 0.1$.
% and complete method.
So, the complete method is used to calculate inter-clusters distance, whereas the Block-MCC$_{2,2}$ measures the inter-blocks coherence of a dictionary $\myPhi \ssin \mathbb{R}^{10 \stimes 80}$ with equally-sized blocks, i.e., $d_1 \seq \cdots \seq d_{40} \seq d \seq 2$.

In figure \ref{fig:SL_Hierarchical_conventional}, it can be seen that by applying hierarchical clustering on the blocks of the dictionary, the increase in the sparsity levels even surpasses the conventional sparsity level which is marked with a black dotted line.
It can be seen in figure \ref{fig:SL_Hierarchical_conventional} that close to the clustering level equal to the number of clusters in the simulated dictionary, even the proposed $\mySLTxt_{min}$ can surpass the conventional sparsity level.
Therefore, clustering the coherent blocks of a dictionary in addition to enhancing the Block-ERC through increasing the $\myBSLTxt_{2,2}$, improves the conventional sparsity level. 
Hence, improves the conventional ERC.

At last, in order to make sure that the clustering structure in a clustering level, where, there is a peak in sparsity level is done properly, the average clustering accuracy over 100 repetitions is computed for fourth (for the case of four clusters in the dictionary) and eighth (for the case of eight clusters in the dictionary) clustering levels, which are equal to \myhl{$99.65\%$} and \myhl{$100\%$}, respectively.
Therefore, the peak in the sparsity level diagram, which is more striking in figure \ref{fig:SL_Hierarchical} in addition to giving an estimation of the number of clusters in the dictionary, also with \myhl{high probability (at least in this example)} corresponds to the correct clustering structure of the dictionary.
\begin{figure}[!b]
\centering
\includegraphics[width=.95\textwidth,keepaspectratio]{images/SL_Hierarchical_conventional.png} % width=0.5\textwidth  scale=0.49
\centering
\caption{$\mySLTxt_{min}[\%]$ and $\mySLTxt_{max}[\%]$ for each level of clustering computed for complete method, Block-MCC$_{2,2}$, $d \seq 2$, $\varepsilon_{inter} \seq 3.5$ and $\varepsilon_{intra} \seq 0.1$ for (a) $N \seq 4$ and (b) $N \seq 8$ in the dictionary $\myPhi \ssin \mathbb{R}^{10 \stimes 80}$.}
\label{fig:SL_Hierarchical_conventional}
\end{figure}
\FloatBarrier