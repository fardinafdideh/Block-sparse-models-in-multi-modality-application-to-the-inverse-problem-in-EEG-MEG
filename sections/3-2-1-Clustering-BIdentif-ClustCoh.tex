Whether the block structure of a given dictionary is assumed to be known beforehand but it is not the optimal structure or it is unknown, by clustering coherent blocks in a bigger block, there would be the possibility of enhancing the structure of the dictionary as a preprocessing step. 


In this chapter, we made use of standard agglomerative hierarchical clustering method, where, the proposed Block-MCC$_{q,p}$ is used as the measure of similarity or coherence.
Hierarchical clustering provides a set of clustering structures in multilevel hierarchy manner, which allows to decide the level of clustering that is most suitable to the application of interest. 
The set of clustering structures is called clustering tree or \emph{dendrogram}.
The dendrogram consists of some nodes indicating the clusters, and usually the clustering level corresponding to the maximum inter-node distance is selected as the desired clustering level.
%In this study we used agglomerative hierarchical cluster analysis.
%In spectral clustering method, firstly a similarity matrix is obtained based on the coherency between the blocks, then the spectral clustering is applied to the similarity matrix.

In figure \ref{fig:Clustered_Representation}, it has been assumed that before blocks clustering all of the blocks of the dictionary $\myPhi \ssin \mathbb{R}^{8 \stimes 12}$ share the same length of two, i.e., $d_1 \seq \cdots \seq d_6 \seq 2$. 
Then, by applying hierarchical clustering on initial blocks of dictionary and choosing the clustering level corresponding to the maximum inter-node distance in the clustering tree, i.e., three clusters, the clustered representation appears.

As it can be seen in figure \ref{fig:Clustered_Representation}, after blocks clustering, blocks $\myPhi \mybracket{1}$, $\myPhi \mybracket{4}$, and $\myPhi \mybracket{6}$ are concatenated to form a new block $\myPhi_{cls} \mybracket{1}$ in new representation, whereas $\myPhi \mybracket{2}$ and $\myPhi \mybracket{3}$ are concatenated to form $\myPhi_{cls} \mybracket{2}$.
Hence, the identified block structure is $\boldsymbol{d} \seq [6,4,2]$.
Consequently, their corresponding blocks in the representation vector $\mybeta$ will be clustered.
The clustered representation of the new model, i.e., $\boldsymbol{y} \seq \myPhi_{cls} \mybeta_{cls}$, leads to more relaxed Block-ERC, i.e., higher sparsity levels, which will be demonstrated in Section \ref{sec:hierarchical_cluster_estim}.
\begin{figure}[!b]
\centering
\includegraphics[width=1\textwidth,keepaspectratio]{images/Clustered_Representation.png} % width=0.5\textwidth  scale=0.49
\centering
\caption{Clustered representation of general dictionary $\myPhi$ and representation vector $\mybeta$.}
\label{fig:Clustered_Representation}
\end{figure}
\FloatBarrier