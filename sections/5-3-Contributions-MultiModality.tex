\addcontentsline{toc}{section}{\protect\numberline{}Multi-modality}
\begin{tcolorbox}
\begin{challenge}
Is joining multiple modalities always beneficial, knowing that each modality provides us with different properties of the same phenomenon? 
How can the added value of multi-modality be demonstrated?
\end{challenge}
\end{tcolorbox}

In order to approach the above-mentioned important and huge challenge, we simplify and limit the challenge to the conditions of our main problem, i.e., distributed EEG and MEG source reconstruction problem.
In other words, we restrict the mentioned general challenge to have only two modalities, and try to partially address the challenge.

In Chapter \ref{chptr:Multimodality}, we propose a multi-modality framework based on the block structure identification framework proposed in Chapter \ref{sec:Clustering} and block mutual coherence constant proposed in Chapter \ref{sec:BERC}.

To this aim, we applied the block structure identification framework on multi-modal lead-field instead of mono-modal one, to segment brain source space.
To investigate the impact of multi-modality, we defined a lead-field combining strategy, which reduces the impact of other factors such as change in the position and number of sensors.

First, we showed that brain regions resulted from clustering the coherent sources of EEG and MEG lead-field matrices separately, are complementary.
Then, we made use of complementarities of EEG and MEG lead-field matrices to generated a combined EEG and MEG multi-modal lead-field matrix, and it turned out that in multi-modality case the number of clusters determined by the largest distance between adjacent nodes in dendrogram is higher than mono-modal cases. 
In addition, for a fixed number of clusters, the under-sensor brain regions in multi-modal lead-field is smaller than mono-modal lead-field clustering.

%Furthermore, for a fixed number of clusters, the brain regions in multi-modal case are smaller than mono-modal cases.
Therefore, it can be deduced that in multi-modality case, more refined and precise regions appear, hence, the resolution of identifying the active regions increases in comparison to the mono-modal cases.

As perspective in this research area, we could mention to the following points:
\begin{itemize}
\item Study on electromagnetic properties in 2D/3D mediums with different number of boundaries as a general case.
\item Designing more scenarios (other than brain segmentation) to investigate the impact of multi-modality.
\item Designing more methods to combine modalities.
\item The impact of combining more modalities (greater than two).
\item The optimum ratio of the number of measurements from each modality (not necessarily $50\%$ in two modalities case).
\end{itemize}    

%%% Local Variables: 
%%% mode: latex
%%% TeX-master: "../roque-phdthesis"
%%% End: