\chapter{Proofs}
\label{chap:Appendices-Proofs}
%------------------------------------------------------
%\section{Appendices}
%------------------------------------------------------
\section{Proof of Property \ref{prp:VectorDivisionBound} (Bounds of division of two vector norms, page \pageref{prp:VectorDivisionBound})}
\label{prf:VectorDivisionBound}
\begin{proof}
\myhl{In order to demonstrate the bounds of $\Vert \boldsymbol{a} \Vert_{p} / \Vert \boldsymbol{a} \Vert_{q}$, first we prove it using the derivative method for $\forall (q,p) \ssin \mathbb{R}_{\sg 1}$, next utilising the H{\"o}lder's inequality we show that the same bounds hold true for the wider range of $q$ and $p$, i.e., $\forall (q,p) \ssin \mathbb{R}_{\sg 0}$.}

1) To compute the critical point of $\Vert \boldsymbol{a} \Vert_{p} / \Vert \boldsymbol{a} \Vert_{q}$ using the derivative method, we need to compute its derivative with respect to the coordinates, knowing that $(f/g)' \seq (f'g \sm g'f)/g^2$, and $(|f|)' \seq f' \, f / |f| $, where $f'$ is derivative of $f$ with respect to $x$, i.e. $d \, f(x) / d \, x$, we have:
\begin{gather}
\label{eq:prf:VectorDivisionBound} 
\forall (q,p) \in \mathbb{R}_{> 1}, \qquad
\frac{\partial}{\partial a_i} \frac{\mynorm{\boldsymbol{a}}_p}{\mynorm{\boldsymbol{a}}_q} =
\frac{\frac{\partial \mynorm{\boldsymbol{a}}_p}{\partial a_i} \mynorm{\boldsymbol{a}}_q - \frac{\partial \mynorm{\boldsymbol{a}}_q}{\partial a_i} \mynorm{\boldsymbol{a}}_p}{\mynorm{\boldsymbol{a}}_q^2}
\end{gather}
Then, we need to compute $\partial \Vert \boldsymbol{a} \Vert_p / \partial a_i$ and $\partial \Vert \boldsymbol{a} \Vert_q / \partial a_i$, so:
\begin{gather*}
\frac{\partial \mynorm{\boldsymbol{a}}_p}{\partial a_i} =
\frac{\partial}{\partial a_i} \myparanthese{\sum_i \myabs{a_i}^p}^\frac{1}{p} = 
\frac{1}{p} \myparanthese{\sum_i \myabs{a_i}^p}^{\frac{1}{p} - 1} \frac{\partial}{\partial a_i} \sum_i \myabs{a_i}^p =
\mynorm{\boldsymbol{a}}_{\mycolor{p}} ^ {1 - p} \myabs{a_i} ^{p - 2} a_i.
\end{gather*}
Therefore, substituting $\partial \Vert \boldsymbol{a} \Vert_p / \partial a_i$ and $\partial \Vert \boldsymbol{a} \Vert_q / \partial a_i$ in (\ref{eq:prf:VectorDivisionBound}) and make it equal to zero, we have:
\begin{gather*}
\begin{aligned}
\frac{\partial}{\partial a_i} \frac{\mynorm{\boldsymbol{a}}_p}{\mynorm{\boldsymbol{a}}_q} &=
\mycolor{\frac{\mynorm{\boldsymbol{a}}_q \mynorm{\boldsymbol{a}}_p ^ {1 - p} \myabs{a_i} ^{p - 2} a_i
- \mynorm{\boldsymbol{a}}_p \mynorm{\boldsymbol{a}}_q ^ {1 - q} \myabs{a_i} ^{q - 2} a_i}{\mynorm{\boldsymbol{a}}_q^2}} \\
&= a_i \frac{\mynorm{\boldsymbol{a}}_p}{\mynorm{\boldsymbol{a}}_q} \myparanthese{\frac{\myabs{a_i}^{p-2}}{\mynorm{\boldsymbol{a}}_p^p} - \frac{\myabs{a_i}^{q-2}}{\mynorm{\boldsymbol{a}}_q^q}} = 0
&\Rightarrow \myabs{a_i} \in \mybrace{0 , \myparanthese{\frac{\mynorm{\boldsymbol{a}}_p^p}{\mynorm{\boldsymbol{a}}_q^q}}^{\frac{1}{p-q}}}.
\end{aligned}
\end{gather*}
Hence, the derivative cancels for all $\boldsymbol{a}$ with $1 \sleq m \sleq d$ non-zero elements that all are equal, i.e., $|a_i| \seq (\Vert \boldsymbol{a} \Vert_p^p / \Vert \boldsymbol{a} \Vert_q^q) ^ {1/(p-q)} \seq C \ssin \mathbb{R}_{>0}$, whereas all other elements are identically zero. 
Then, the fraction in the critical point evaluates to
\begin{gather*}
\forall (q,p) \in \mathbb{R}_{> 1}, \qquad
\frac{\mynorm{\boldsymbol{a}}_p}{\mynorm{\boldsymbol{a}}_q} = 
\frac{\myparanthese{m C^p}^{\frac1p}}{\myparanthese{m C^q}^{\frac1q}} = 
m ^{\frac1p - \frac1q},
\end{gather*}
which is minimised for $m \seq
  \begin{cases}
  \begin{aligned}
    &d^{1/p-1/q},   \quad  && \text{if } q \sleq p\\
    &1,   \quad && \text{if } q \sg p\\
  \end{aligned}
  \end{cases} \seq \min\{1 , d^{1/p-1/q}\}$, and maximised for $m \seq
  \begin{cases}
  \begin{aligned}
    &1,  \quad  && \text{if } q \sleq p\\
    &d^{1/p-1/q},  \quad && \text{if } q \sg p\\
     \end{aligned}
     \end{cases} \seq \max\{1 , d^{1/p-1/q}\}$.
     
     
\myhl{2) Knowing that the $\ell_p$ norm is a decreasing function in $p$, the following lower-bound is resulted for $p \sleq q$:}
\begin{gather*}
\mycolor{\forall \boldsymbol{a} \in \mathbb{R}^d, \forall (q,p) \in \mathbb{R}^2_{> 0}, \qquad
\text{if } p \leq q 
\Rightarrow \mynorm{\boldsymbol{a}}_q \leq \mynorm{\boldsymbol{a}}_p
\Rightarrow 1 \leq \frac{\mynorm{\boldsymbol{a}}_p}{\mynorm{\boldsymbol{a}}_q}.}
\end{gather*}
\myhl{Now to demonstrate the upper-bound, the following H{\"o}lder's inequality is used {\cite{Golub2013}}:}
\begin{gather*}
\mycolor{\forall r \in \mathbb{R}_{\geq 1}, \qquad
\sum_{i = 1}^d \myabs{x_i y_i} \leq \mynorm{\boldsymbol{x}}_r \mynorm{\boldsymbol{y}}_{\frac{r}{r-1}}
= \myparanthese{\displaystyle\sum_{i = 1}^d \myabs{x_i}^ r} ^ {\frac1r} \, \myparanthese{\displaystyle\sum_{i = 1}^d \myabs{y_i}^ {\frac{r}{r-1}}} ^ {\frac{r-1}{r}}.}
\end{gather*}
\myhl{Next, assuming $x_i \seq {a_i}^p$, $y_i \seq 1$, and $r \seq q/p$ (the above condition on $r$ is met, because $p \sleq q$), we have $\sum_{i \seq 1}^d \myabs{x_i y_i} \seq \sum_{i \seq 1}^d \myabs{{a_i}^p} \seq \sum_{i \seq 1}^d \myabs{{a_i}}^p$, and the above equation turns into:}
\begin{gather*}
\begin{aligned}
\mycolor{\forall \boldsymbol{a} \in \mathbb{R}^d, \forall (q,p) \in \mathbb{R}^2_{> 0}, p \leq q, \qquad
\sum_{i = 1}^d \myabs{a_i}^p} 
&\mycolor{\leq \myparanthese{\displaystyle\sum_{i = 1}^d \myparanthese{\myabs{a_i}^p}^ {\frac{q}{p}}} ^ {\frac{p}{q}} \, \myparanthese{\displaystyle\sum_{i = 1}^d 1^ {\frac{q}{q-p}}} ^ {\frac{q-p}{q}}} \\
&\mycolor{= \myparanthese{\displaystyle\sum_{i = 1}^d \myabs{a_i}^q} ^ {\frac{p}{q}} \, d ^ {\frac{q-p}{q}}.}
\end{aligned}
\end{gather*}
\myhl{Then, taking the both sides to the power of $1/p$, we get:}
\begin{gather*}
\mycolor{\begin{aligned}
\forall \boldsymbol{a} \in \mathbb{R}^d, \forall (q,p) \in \mathbb{R}^2_{> 0}, p \leq q, \qquad
&\myparanthese{\sum_{i = 1}^d \myabs{a_i}^p}^\frac1p
\leq \myparanthese{\displaystyle\sum_{i = 1}^d \myabs{a_i}^q} ^ {\frac1q} \, d ^ {\frac1p -\frac1q} \\
&\Rightarrow \frac{\mynorm{\boldsymbol{a}}_p}{\mynorm{\boldsymbol{a}}_q} \leq d ^ {\frac1p -\frac1q}.
\end{aligned}}
\end{gather*}
\myhl{Therefore, for $p \sleq q$, we have $1 \sleq \Vert \boldsymbol{a} \Vert_p / \Vert \boldsymbol{a} \Vert_q \sleq d^{1/p \sm 1/q}$.
Similarly, for $q \sleq p$, we get $d^{1/p \sm 1/q} \sleq \Vert \boldsymbol{a} \Vert_p / \Vert \boldsymbol{a} \Vert_q \sleq 1$, which proves the property.}
\end{proof}
%------------------------------------------------------
\newpage
\section{Proof of Property \ref{prp:OperatorProperties} ($\ell_{q {\to} p}$ operator-norm properties, page \pageref{prp:OperatorProperties})}
\label{prf:OperatorProperties}
\begin{proof}
Based on the definition of the $\Vert \boldsymbol{A} \Vert_{q \to p}$, i.e., $\max_{\boldsymbol{a} \neq \boldsymbol{0}} \Vert \boldsymbol{A} \boldsymbol{a} \Vert_{p} / \Vert \boldsymbol{a} \Vert_{q}$ or $\max_{\Vert \boldsymbol{a} \Vert_q \sleq 1} \Vert \boldsymbol{A} \boldsymbol{a} \Vert_{p}$, we prove the following properties, for $\boldsymbol{A} \ssin \mathbb{R}^{m \stimes n}$, $\boldsymbol{B} \ssin \mathbb{R}^{m \stimes n}$, and $\boldsymbol{C} \ssin \mathbb{R}^{n \stimes l}$:
\begin{itemize}
\item Nonnegativity: In definition $\Vert \boldsymbol{A} \Vert_{q \to p} \seq \max_{\Vert \boldsymbol{a} \Vert_q \sleq 1} \Vert \boldsymbol{A} \boldsymbol{a} \Vert_{p}$, $\Vert \boldsymbol{A} \boldsymbol{a} \Vert_{p}$ is greater than or equal to zero, then the nonnegativity of $\Vert \boldsymbol{A} \Vert_{q \to p}$ is obvious $\forall (q , p) \ssin \mathbb{R}^2_{\sgeq 0}$. 
\item Positivity: $\forall (q , p) \ssin \mathbb{R}^2_{\sgeq 0}$ if $\Vert \boldsymbol{A} \Vert_{q \to p} \seq 0$, then $\Vert \boldsymbol{A} \boldsymbol{a} \Vert_{p} \seq 0$ for each $\boldsymbol{a} \ssin \mathbb{R}^n$, i.e., each column in $\boldsymbol{A}$ is zero. 
Hence, $\boldsymbol{A} \seq \boldsymbol{0}$.
\item Homogeneity: $\forall q \sgeq 0$ and $\forall p \sg 0$, we have
\begin{gather*} 
\begin{aligned}
\mynorm{\alpha \boldsymbol{A}}_{q \to p} = \max_{\boldsymbol{a} \neq \boldsymbol{0}} \frac{\mynorm{\alpha \boldsymbol{A} \boldsymbol{a} }_{p}}{\mynorm{ \boldsymbol{a}}_{q}} 
&= \max_{\boldsymbol{a} \neq \boldsymbol{0}} \myabs{\alpha} \frac{\mynorm{\boldsymbol{A} \boldsymbol{a}}_{p}}{\mynorm{\boldsymbol{a}}_{q}} \\
&= \myabs{\alpha} \mynorm{\boldsymbol{A}}_{q \to p}.
\end{aligned}
\end{gather*}
The above second equality in the first line is resulted from the homogeneity property of norm of vectors, which holds true only for $p \sg 0$ \cite{Elad2010,Golub2013}.
\item Triangle inequality: $\forall q \sgeq 0$, $p \seq 0$ or $\forall p \sgeq 1$, it is obtained as follows
\begin{gather*}
\begin{aligned}
\mynorm{\boldsymbol{A} + \boldsymbol{B}}_{q \to p} = 
\max_{\boldsymbol{a} \neq \boldsymbol{0}} \frac{\mynorm{\myparanthese{\boldsymbol{A} + \boldsymbol{B}} \boldsymbol{a}}_{p}}{\mynorm{\boldsymbol{a}}_{q}} 
&\leq \max_{\boldsymbol{a} \neq \boldsymbol{0}} \frac{\mynorm{\boldsymbol{A} \boldsymbol{a}}_{p} + \mynorm{\boldsymbol{B} \boldsymbol{a}}_{p}}{\mynorm{\boldsymbol{a}}_{q}} \\
&\leq \max_{\boldsymbol{a} \neq \boldsymbol{0}} \frac{\mynorm{\boldsymbol{A} \boldsymbol{a}}_{p}}{\mynorm{\boldsymbol{a}}_{q}} + \max_{\boldsymbol{a} \neq \boldsymbol{0}} \frac{\mynorm{\boldsymbol{B} \boldsymbol{a}}_{p}}{\mynorm{\boldsymbol{a}}_{q}} \\
&= \mynorm{\boldsymbol{A}}_{q \to p} + \mynorm{\boldsymbol{B}}_{q \to p}.
\end{aligned}
\end{gather*}
The above inequality in the first line is resulted from the triangle inequality property of norm of vectors, which holds true only for $p \seq 0$ and $p \sgeq 1$ \cite{Elad2010,Golub2013}.
\item Submultiplicativity: It is obtained as follows: %for $\boldsymbol{B} \ssin \mathbb{R}^{m \stimes n}$
\begin{gather}
\begin{aligned}
\label{eq:prf:OperatorProperties} 
\mynorm{\boldsymbol{A} \boldsymbol{C}}_{q \to p} = 
\max_{\boldsymbol{a} \neq \boldsymbol{0}} \frac{\mynorm{\boldsymbol{A} \boldsymbol{C} \boldsymbol{a}}_{p}}{\mynorm{\boldsymbol{a}}_{q}} &=
\max_{\boldsymbol{a} \neq \boldsymbol{0}} \frac{\mynorm{\boldsymbol{A} \boldsymbol{C} \boldsymbol{a}}_{p}}{\mynorm{\boldsymbol{a}}_{q}} \frac{\mynorm{\boldsymbol{C} \boldsymbol{a}}_{q}}{\mynorm{\boldsymbol{C} \boldsymbol{a}}_{q}} \frac{\mynorm{\boldsymbol{C} \boldsymbol{a}}_{p}}{\mynorm{\boldsymbol{C} \boldsymbol{a}}_{p}} \\
&\leq \max_{\boldsymbol{C} \boldsymbol{a} \neq \boldsymbol{0}} \frac{\mynorm{\boldsymbol{A} \boldsymbol{C} \boldsymbol{a}}_{p}}{\mynorm{\boldsymbol{C} \boldsymbol{a}}_{q}}
\max_{\boldsymbol{a} \neq \boldsymbol{0}} \frac{\mynorm{\boldsymbol{C} \boldsymbol{a}}_{p}}{\mynorm{\boldsymbol{a}}_{q}}
\max_{\boldsymbol{a} \neq \boldsymbol{0}} \frac{\mynorm{\boldsymbol{C} \boldsymbol{a}}_{q}}{\mynorm{\boldsymbol{C} \boldsymbol{a}}_{p}} \\
&= \mynorm{\boldsymbol{A}}_{q \to p} \, \mynorm{\boldsymbol{C}}_{q \to p} \, \max \mybrace{1 , n^{\frac1q - \frac1p}}.
\end{aligned}
\end{gather}
The last term of (\ref{eq:prf:OperatorProperties}) is obtained using the upper-bound in Property \ref{prp:VectorDivisionBound} (bounds of two vector norms division), i.e.,:
\begin{gather*}
\forall \myparanthese{q , p} \in \mathbb{R}^2_{>0} , \forall \boldsymbol{a} \in \mathbb{R}^d, \qquad 
\min \mybrace{1 , d^{\frac1q - \frac1p}} \leq \frac{\mynorm{\boldsymbol{a}}_q}{\mynorm{\boldsymbol{a}}_p} \leq \max \mybrace{1 , d^{\frac1q - \frac1p}}.
\end{gather*}
Therefore, from (\ref{eq:prf:OperatorProperties}) for $q \sgeq p \sg 0$, we see that the submultiplicativity holds true, i.e., $\mynorm{\boldsymbol{A} \boldsymbol{C}}_{q \to p} \sleq \mynorm{\boldsymbol{A}}_{q \to p} \mynorm{\boldsymbol{C}}_{q \to p}$.
\item Bounds: 1) Considering the definition of $\ell_{q {\to} p}$ operator-norm, i.e., $\max_{\boldsymbol{a} \neq \boldsymbol{0}} \Vert \boldsymbol{A} \boldsymbol{a} \Vert_{p} / \Vert \boldsymbol{a} \Vert_{q}$, and the lower-bound of division of two vector norms introduced in Property \ref{prp:VectorDivisionBound} (bounds of two vector norms division), i.e., $\forall (q , p) \ssin \mathbb{R}^2_{\sg 0}, \forall \boldsymbol{a} \ssin \mathbb{R}^d : \Vert \boldsymbol{a} \Vert_q / \Vert \boldsymbol{a} \Vert_p \sgeq \min \mybrace{1 , d^{1/q - 1/p}}$, we conclude the following lower-bounds for $\boldsymbol{A} \ssin \mathbb{R}^{m \stimes n}$, $\boldsymbol{a} \ssin \mathbb{R}^n$, and $\forall \myparanthese{q , p , q' , p'} \ssin \mathbb{R}^4_{>0}$:
\begin{gather*}
\begin{aligned}
\forall \boldsymbol{a} \neq \boldsymbol{0}, \qquad \frac{\mynorm{\boldsymbol{A} \boldsymbol{a}}_p}{\mynorm{\boldsymbol{a}}_q} \geq
&\min \mybrace{1 , m^{\frac1p - \frac{1}{p'}}} \frac{\mynorm{\boldsymbol{A} \boldsymbol{a}}_{p'}}{\mynorm{\boldsymbol{a}}_q}, \\
\forall \boldsymbol{a} \neq \boldsymbol{0}, \qquad \frac{\mynorm{\boldsymbol{A} \boldsymbol{a}}_p}{\mynorm{\boldsymbol{a}}_q} \geq
&\min \mybrace{1 , n^{\frac{1}{q'} - \frac1q}} \frac{\mynorm{\boldsymbol{A} \boldsymbol{a}}_p}{\mynorm{\boldsymbol{a}}_{q'}}.
\end{aligned}
\end{gather*}
Therefore, we have
\begin{gather*}
\forall \boldsymbol{a} \neq \boldsymbol{0}, \qquad \frac{\mynorm{\boldsymbol{A} \boldsymbol{a}}_p}{\mynorm{\boldsymbol{a}}_q} \geq
\max \mybrace{\min \mybrace{1 , m^{\frac1p - \frac{1}{p'}}} \frac{\mynorm{\boldsymbol{A} \boldsymbol{a}}_{p'}}{\mynorm{\boldsymbol{a}}_q}, 
\min \mybrace{1 , n^{\frac{1}{q'} - \frac1q}} \frac{\mynorm{\boldsymbol{A} \boldsymbol{a}}_p}{\mynorm{\boldsymbol{a}}_{q'}}}.
\end{gather*}
Then, taking into account that the max operator is order preserving, i.e., $\forall x$, if $f(x) \sleq g(x)$, then $\max_x f(x) \sleq \max_x g(x)$, the lower-bound of $\Vert \boldsymbol{A} \Vert_{q {\to} p}$ is achieved.

Similarly, utilising the upper-bound of division of two vector norms introduced in Property \ref{prp:VectorDivisionBound} (bounds of two vector norms division), i.e., $\forall (q , p) \ssin \mathbb{R}^2_{\sg 0}, \forall \boldsymbol{a} \ssin \mathbb{R}^d : \Vert \boldsymbol{a} \Vert_q / \Vert \boldsymbol{a} \Vert_p \sleq \max \{1 , d^{1/q - 1/p} \}$, results in the following inequalities, which conclude the upper-bound of $\Vert \boldsymbol{A} \Vert_{q {\to} p}$ for $\boldsymbol{A} \ssin \mathbb{R}^{m \stimes n}$, $\boldsymbol{a} \ssin \mathbb{R}^n$, and $\forall \myparanthese{q , p , q' , p'} \ssin \mathbb{R}^4_{>0}$:
\begin{gather*}
\begin{aligned}
\forall \boldsymbol{a} \neq \boldsymbol{0}, \qquad \frac{\mynorm{\boldsymbol{A} \boldsymbol{a}}_p}{\mynorm{\boldsymbol{a}}_q} \leq
&\max \mybrace{1 , m^{\frac1p - \frac{1}{p'}}} \frac{\mynorm{\boldsymbol{A} \boldsymbol{a}}_{p'}}{\mynorm{\boldsymbol{a}}_q}, \\
\forall \boldsymbol{a} \neq \boldsymbol{0}, \qquad \frac{\mynorm{\boldsymbol{A} \boldsymbol{a}}_p}{\mynorm{\boldsymbol{a}}_q} \leq
&\max \mybrace{1 , n^{\frac{1}{q'} - \frac1q}} \frac{\mynorm{\boldsymbol{A} \boldsymbol{a}}_p}{\mynorm{\boldsymbol{a}}_{q'}}.
\end{aligned}
\end{gather*}
Therefore,
\begin{gather*}
\forall \boldsymbol{a} \neq \boldsymbol{0}, \qquad \frac{\mynorm{\boldsymbol{A} \boldsymbol{a}}_p}{\mynorm{\boldsymbol{a}}_q} \leq
\min \mybrace{\max \mybrace{1 , m^{\frac1p - \frac{1}{p'}}} \frac{\mynorm{\boldsymbol{A} \boldsymbol{a}}_{p'}}{\mynorm{\boldsymbol{a}}_q}, 
\max \mybrace{1 , n^{\frac{1}{q'} - \frac1q}} \frac{\mynorm{\boldsymbol{A} \boldsymbol{a}}_p}{\mynorm{\boldsymbol{a}}_{q'}}}.
\end{gather*}
2) To prove the second set of bounds, from Property \ref{prp:VectorDivisionBound} (bounds of two vector norms division), i.e., $\forall (q , p) \ssin \mathbb{R}^2_{\sg 0}, \forall \boldsymbol{a} \ssin \mathbb{R}^d : \Vert \boldsymbol{a} \Vert_q / \Vert \boldsymbol{a} \Vert_p \sgeq \min \{1 , d^{1/q - 1/p} \}$, we have the following inequalities for $\boldsymbol{A} \ssin \mathbb{R}^{m \stimes n}$, $\boldsymbol{a} \ssin \mathbb{R}^n$, and $\forall \myparanthese{q , p , q' , p'} \ssin \mathbb{R}^4_{>0}$:
\begin{gather*}
\begin{aligned}
&\forall \boldsymbol{a} \neq \boldsymbol{0}, \qquad \mynorm{\boldsymbol{A} \boldsymbol{a}}_p &&\geq
\min \mybrace{1 , m^{\frac1p - \frac{1}{p'}}} \mynorm{\boldsymbol{A} \boldsymbol{a}}_{p'}, \\
&\forall \boldsymbol{a} \neq \boldsymbol{0}, \qquad \frac{1}{\mynorm{\boldsymbol{a}}_q} &&\geq
\min \mybrace{1 , n^{\frac{1}{q'} - \frac1q}} \frac{1}{\mynorm{\boldsymbol{a}}_{q'}}.
\end{aligned}
\end{gather*}
Considering that both sides of the above inequalities are positive, by multiplying the same sides we get:
\begin{gather*}
\forall \boldsymbol{a} \neq \boldsymbol{0}, \qquad \frac{\mynorm{\boldsymbol{A} \boldsymbol{a}}_p}{\mynorm{\boldsymbol{a}}_q} \geq
\min \mybrace{1 , m^{\frac1p - \frac{1}{p'}}} \min \mybrace{1 , n^{\frac{1}{q'} - \frac1q}} \frac{\mynorm{\boldsymbol{A} \boldsymbol{a}}_{p'}}{\mynorm{\boldsymbol{a}}_{q'}}.
\end{gather*}
Then taking into account that the max function is order preserving, the proof is done for lower-bound.

Similarly, for the upper-bound, from Property \ref{prp:VectorDivisionBound} (bounds of two vector norms division), i.e., $\forall (q , p) \ssin \mathbb{R}^2_{\sg 0} , \forall \boldsymbol{a} \ssin \mathbb{R}^d : \Vert \boldsymbol{a} \Vert_q / \Vert \boldsymbol{a} \Vert_p \sleq \max \{1 , d^{1/q - 1/p} \}$, we have the following inequalities for $\boldsymbol{A} \ssin \mathbb{R}^{m \stimes n}$, $\boldsymbol{a} \ssin \mathbb{R}^n$, and $\forall \myparanthese{q , p , q' , p'} \ssin \mathbb{R}^4_{>0}$:
\begin{gather*}
\begin{aligned}
&\forall \boldsymbol{a} \neq \boldsymbol{0}, \qquad \mynorm{\boldsymbol{A} \boldsymbol{a}}_p &&\leq
\max \mybrace{1 , m^{\frac1p - \frac{1}{p'}}} \mynorm{\boldsymbol{A} \boldsymbol{a}}_{p'}, \\
&\forall \boldsymbol{a} \neq \boldsymbol{0}, \qquad \frac{1}{\mynorm{\boldsymbol{a}}_q} &&\leq
\max \mybrace{1 , n^{\frac{1}{q'} - \frac1q}} \frac{1}{\mynorm{\boldsymbol{a}}_{q'}}.
\end{aligned}
\end{gather*}
Considering that both sides of the above inequalities are positive, by multiplying the same sides we get:
\begin{gather*}
\forall \boldsymbol{a} \neq \boldsymbol{0}, \qquad \frac{\mynorm{\boldsymbol{A} \boldsymbol{a}}_p}{\mynorm{\boldsymbol{a}}_q} \leq
\max \mybrace{1 , m^{\frac1p - \frac{1}{p'}}} \, \max \mybrace{1 , n^{\frac{1}{q'} - \frac1q}} \frac{\mynorm{\boldsymbol{A} \boldsymbol{a}}_{p'}}{\mynorm{\boldsymbol{a}}_{q'}}.
\end{gather*}
Then taking into account that the max function is order preserving, the proof is done for upper-bound.

In order to prove the $\ell_{q {\to} p}$ operator-norm inequalities shown schematically in figure \ref{fig:OperatorNorm-Inequalities}, utilising the lower-bound of second set of bounds in Property \ref{prp:OperatorProperties} ($\ell_{q {\to} p}$ operator-norm properties), i.e., $\forall \myparanthese{q , p , q' , p'} \ssin \mathbb{R}^4_{>0} , \forall \boldsymbol{A} \ssin \mathbb{R}^{m \stimes n} : \Vert \boldsymbol{A} \Vert_{q \to p} \sgeq \min \{1 , m^{1/p \sm 1/{p'}} \} \min \{1 , n^{1/{q'} \sm 1/q} \} \Vert \boldsymbol{A} \Vert_{q' \to p'}$, it is straightforward that for a fixed $p \seq p'$ if $q' \sleq q$ (or similarly, for a fixed $q \seq q'$ if $p \sleq p'$) we have $\Vert \boldsymbol{A} \Vert_{q \to p} \sgeq \Vert \boldsymbol{A} \Vert_{q' \to p'}$.

3) In order to prove the lower-bound of the third set of bounds, using the lower-bound of second set of bounds in Property \ref{prp:OperatorProperties} ($\ell_{q {\to} p}$ operator-norm properties), i.e., $\forall \myparanthese{q , p , q' , p'} \ssin \mathbb{R}^4_{>0} , \forall \boldsymbol{A} \ssin \mathbb{R}^{m \stimes n} : \Vert \boldsymbol{A} \Vert_{q \to p} \sgeq \min \{1 , m^{1/p \sm 1/{p'}} \} \min \{1 , n^{1/{q'} \sm 1/q} \} \Vert \boldsymbol{A} \Vert_{q' \to p'}$, we will have the following first line of inequalities.
Then, by utilising the same inequality once again, this time for $q' \seq p' \seq 2$, we get the following second line of inequalities.
The reason that first the relation with $\Vert \boldsymbol{A} \Vert_{q' \to p'}$ is stated and then the relation with $\Vert \boldsymbol{A} \Vert_{2 \to 2}$, is that if in a problem the $q$ and $p$ values are bounded based on some thresholds such as $q'$ and $p'$, respectively, e.g., $q \sleq / \sgeq q'$ and $p \sleq / \sgeq p'$, then there could be a unique solution to the related $\min$ and $\max$ operators.
Whereas, if directly the relation with $\Vert \boldsymbol{A} \Vert_{2 \to 2}$ was stated, and supposing that $p \sgeq 1$, then form instance $\min \{1 , m^{1/p \sm 1/{2}} \}$ would have two values based on the value of $p$ compared to $1$.
Then from \cite{Golub2013}, we have $\Vert \boldsymbol{A} \Vert_{2 \to 2} \sgeq \Vert \boldsymbol{A} \Vert_F / \sqrt{Rank(\boldsymbol{A})}$, where the Frobenius norm is defined as $(\sum_{i \seq 1}^m \sum_{j \seq 1}^n |a_{i,j}|^2)^{1/2}$, which produces the following third line.
But $Rank(\boldsymbol{A})$ is upper-bounded by $\min \{ m , n \}$ \cite{Golub2013}, then  $1/ \sqrt{Rank(\boldsymbol{A})}$ is lower-bounded by $1/ \sqrt{\min \{ m , n \}}$, which produces the following last line:
\begin{gather*}
\begin{aligned}
\mynorm{\boldsymbol{A}}_{q \to p} &\geq 
\min \mybrace{1 , m^{\frac1p - \frac{1}{p'}}} \min \mybrace{1 , n^{\frac{1}{q'} - \frac1q}} \mynorm{\boldsymbol{A}}_{q' \to p'} \\
&\geq \min \mybrace{1 , m^{\frac1p - \frac{1}{p'}}} \min \mybrace{1 , n^{\frac{1}{q'} - \frac1q}} \min \mybrace{1 , m^{\frac{1}{p'} - \frac12}} \min \mybrace{1 , n^{\frac12 - \frac{1}{q'}}} \mynorm{\boldsymbol{A}}_{2 \to 2} \\
&\geq \frac{\min \mybrace{1 , m^{\frac1p - \frac{1}{p'}}} \min \mybrace{1 , n^{\frac{1}{q'} - \frac1q}} \min \mybrace{1 , m^{\frac{1}{p'} -\frac12}} \min \mybrace{1 , n^{\frac{1}{2} - \frac{1}{q'}}}}{\sqrt{Rank(\boldsymbol{A})}} \mynorm{\boldsymbol{A}}_F \\
&\geq \frac{\min \mybrace{1 , m^{\frac1p - \frac{1}{p'}}} \min \mybrace{1 , n^{\frac{1}{q'} - \frac1q}} \min \mybrace{1 , m^{\frac{1}{p'} -\frac12}} \min \mybrace{1 , n^{\frac{1}{2} - \frac{1}{q'}}}}{\sqrt{\min \mybrace{m , n}}} \mynorm{\boldsymbol{A}}_F.
%&= \frac{\min \mybrace{1 , m^{\frac1p - \frac{1}{p'}}} \min \mybrace{1 , n^{\frac{1}{q'} - \frac1q}} \min \mybrace{1 , m^{\frac{1}{p'} -\frac12}} \min \mybrace{1 , n^{\frac{1}{2} - \frac{1}{q'}}}}{\sqrt{\min \mybrace{m , n}}} \sqrt{\displaystyle\sum_{i=1}^m \displaystyle\sum_{j=1}^n \myabs{a_{i,j}}^2}.
\end{aligned}
\end{gather*}

In order to prove the upper-bound of the last set of bounds, using the upper-bound of second set of bounds in Property \ref{prp:OperatorProperties} ($\ell_{q {\to} p}$ operator-norm properties), i.e., $\forall \myparanthese{q , p , q' , p'} \ssin \mathbb{R}^4_{>0} , \forall \boldsymbol{A} \ssin \mathbb{R}^{m \stimes n} : \Vert \boldsymbol{A} \Vert_{q \to p} \sleq \max \{1 , m^{1/p \sm 1/{p'}} \} \max \{1 , n^{1/{q'} \sm 1/q} \} \Vert \boldsymbol{A} \Vert_{q' \to p'}$, we will have the following first line of inequalities.
Then, by utilising the same inequality once again, this time for $q' \seq p' \seq 2$, we get the following second line of inequalities.
%The reason that first the relation with $\Vert \boldsymbol{A} \Vert_{q' \to p'}$ is stated and then the relation with $\Vert \boldsymbol{A} \Vert_{2 \to 2}$, is that if in a problem the $q$ and $p$ values are bounded based on some thresholds such as $q'$ and $p'$, respectively, e.g., $q \sleq / \sgeq q'$ and $p \sleq / \sgeq p'$, then there could be a unique solution to the related $\min$ and $\max$ operators.
%Whereas, if directly the relation with $\Vert \boldsymbol{A} \Vert_{2 \to 2}$ was stated, and supposing that $p \sgeq 1$, then form instance $\min \{1 , m^{1/p \sm 1/{2}} \}$ would have two values based on the value of $p$ compared to $1$.
But from \cite{Golub2013}, we have $\Vert \boldsymbol{A} \Vert_{2 \to 2} \sleq \Vert \boldsymbol{A} \Vert_F$, which produces the following last line:
\begin{gather*}
\begin{aligned}
\mynorm{\boldsymbol{A}}_{q \to p} &\leq 
\max \mybrace{1 , m^{\frac1p - \frac{1}{p'}}} \max \mybrace{1 , n^{\frac{1}{q'} - \frac1q}} \mynorm{\boldsymbol{A}}_{q' \to p'} \\
&\leq \max \mybrace{1 , m^{\frac1p - \frac{1}{p'}}} \max \mybrace{1 , n^{\frac{1}{q'} - \frac1q}} \max \mybrace{1 , m^{\frac{1}{p'} - \frac12}} \max \mybrace{1 , n^{\frac12 - \frac{1}{q'}}} \mynorm{\boldsymbol{A}}_{2 \to 2} \\
&\leq \max \mybrace{1 , m^{\frac1p - \frac{1}{p'}}} \max \mybrace{1 , n^{\frac{1}{q'} - \frac1q}} \max \mybrace{1 , m^{\frac{1}{p'} -\frac12}} \max \mybrace{1 , n^{\frac{1}{2} - \frac{1}{q'}}} \mynorm{\boldsymbol{A}}_F.
%&= \max \mybrace{1 , m^{\frac1p - \frac{1}{p'}}} \max \mybrace{1 , n^{\frac{1}{q'} - \frac1q}} \max \mybrace{1 , m^{\frac{1}{p'} -\frac12}} \max \mybrace{1 , n^{\frac{1}{2} - \frac{1}{q'}}} \sqrt{\displaystyle\sum_{i=1}^m \displaystyle\sum_{j=1}^n \myabs{a_{i,j}}^2}.
\end{aligned}
\end{gather*}
4) In order to prove the relation between the $\ell_{q {\to} p}$ operator-norm of two matrices $\boldsymbol{A} \ssin \mathbb{R}^{m \stimes n}$ and $\boldsymbol{B} \ssin \mathbb{R}^{m \stimes n}$, having the condition $\forall i , j$, $|a_{i,j}| \sleq b_{i,j} \seq \max_{i,j} |a_{i,j}|$, or even when the on-diagonal entries are set to zero, we can utilise the second bounds   proposed in the current property.
First, using the lower-bound of the mentioned set of bounds, i.e., $\forall \myparanthese{q , p , q' , p'} \ssin \mathbb{R}^4_{>0} , \forall \boldsymbol{A} \ssin \mathbb{R}^{m \stimes n} : \Vert \boldsymbol{A} \Vert_{q \to p} \sgeq \min \{1 , m^{1/p \sm 1/{p'}} \} \min \{1 , n^{1/{q'} \sm 1/q} \} \Vert \boldsymbol{A} \Vert_{q' \to p'}$, for $q' \seq 1$ and $p' \seq \infty$, the following first line is produced.
Then considering $\min \{ 1 , m^{1/p} \} \seq 1$ and also $\Vert \boldsymbol{A} \Vert_{1 \to \infty} \seq \max_{i,j} |a_{i,j}| \seq \max_{i,j} |b_{i,j}| \seq \Vert \boldsymbol{B} \Vert_{1 \to \infty}$, we can substitute $\Vert \boldsymbol{A} \Vert_{1 \to \infty}$ by $\Vert \boldsymbol{B} \Vert_{1 \to \infty}$ to have the following second line.
Again, using the same lower-bound, this time for $q \seq 1$, $p \seq \infty$, $q' \seq q$ and $p' \seq p$, we reach the following third line:
\begin{gather*}
\begin{aligned}
\mynorm{\boldsymbol{A}}_{q \to p} &\geq 
\min \mybrace{1 , m^{\frac1p}} \min \mybrace{1 , n^{1 - \frac1q}} \mynorm{\boldsymbol{A}}_{1 \to \infty} \\
&= \min \mybrace{1 , n^{1 - \frac1q}} \mynorm{\boldsymbol{B}}_{1 \to \infty} \\
&\geq \min \mybrace{1 , n^{1 - \frac1q}} \min \mybrace{1 , m^{-\frac1p}} \min \mybrace{1 , n^{\frac1q - 1}} \mynorm{\boldsymbol{B}}_{q \to p} \\
&= m^{-\frac1p} n^{-\myabs{1 - \frac1q}} \mynorm{\boldsymbol{B}}_{q \to p}. \\
\end{aligned}
\end{gather*}
Therefore, 
\begin{gather*}
\begin{aligned}
\frac{\mynorm{\boldsymbol{A}}_{q \to p}}{\mynorm{\boldsymbol{B}}_{q \to p}} \geq m^{-\frac1p} n^{-\myabs{1 - \frac1q}} \xLongrightarrow{m^{-\frac1p} n^{-\myabs{1 - \frac1q}} \leq 1} \mynorm{\boldsymbol{A}}_{q \to p} \leq \mynorm{\boldsymbol{B}}_{q \to p}.
\end{aligned}
\end{gather*}

Similarly, using the upper-bound of the mentioned set of bounds, i.e., $\forall \myparanthese{q , p , q' , p'} \ssin \mathbb{R}^4_{>0} , \forall \boldsymbol{A} \ssin \mathbb{R}^{m \stimes n} : \Vert \boldsymbol{A} \Vert_{q \to p} \sleq \max \{1 , m^{1/p \sm 1/{p'}} \} \max \{1 , n^{1/{q'} \sm 1/q} \} \Vert \boldsymbol{A} \Vert_{q' \to p'}$, and following above steps, we obtain:
\begin{gather*}
\begin{aligned}
\frac{\mynorm{\boldsymbol{B}}_{q \to p}}{\mynorm{\boldsymbol{A}}_{q \to p}} \leq m^{\frac1p} n^{\myabs{1 - \frac1q}} \xLongrightarrow{m^{\frac1p} n^{\myabs{1 - \frac1q}} \geq 1} \mynorm{\boldsymbol{A}}_{q \to p} \leq \mynorm{\boldsymbol{B}}_{q \to p}.
\end{aligned}
\end{gather*}
\end{itemize}
\end{proof}
\newpage
%------------------------------------------------------
\section{Proof of Property \ref{prp:BMCC-relationships} (Block-MCC$_{q,p}$ inequalities, page \pageref{prp:BMCC-relationships})}
\label{prf:BMCC-relationships}
\begin{proof}
From Block-MCC$_{q,p}$ (Definition \ref{def:BMIC}, page \pageref{def:BMIC}) we have:
\begin{equation*}
\begin{aligned}
\forall (q , p) \in \mathbb{R}^2_{>0}, \qquad
M_{q,p}\myparanthese{\myPhi} 
\myeq \max_{k,k' \neq k} \frac{d_{k}^{-\frac1p} \, d_{k'}^{\frac1q}}{d_{max}} \mynorm{\myPhi^\dagger\mybracket{k} \myPhi \mybracket{k'}}_{q \to p}.
\end{aligned}
\end{equation*}
Next, considering the relationship between the $\ell_{1 \to 1}$ and $\ell_{1 \to 2}$, and also $\ell_{1 \to 2}$ and $\ell_{1 \to \infty}$ operator-norms, which are presented in general in Property \ref{prp:OperatorProperties} ($\ell_{q {\to} p}$ operator-norm properties) and for certain ($q,p$) pairs in table \ref{table:OperatorNormRelations}, the proof is straightforward, as follows:
\begin{equation*}
\begin{aligned}
M_{1,1}\myparanthese{\myPhi} 
&\myeq \max_{k,k' \neq k} &&\frac{d_{k}^{-1} \, d_{k'}}{d_{max}} \mynorm{\myPhi^\dagger\mybracket{k} \myPhi \mybracket{k'}}_{1 \to 1} \\
&\leq \max_{k,k' \neq k} &&\frac{d_{k}^{-1} \, d_{k'}}{d_{max}} d_{k}^{\frac12} \mynorm{\myPhi^\dagger\mybracket{k} \myPhi \mybracket{k'}}_{1 \to 2} \\
&= \max_{k,k' \neq k} &&\frac{d_{k}^{-\frac12} \, d_{k'}}{d_{max}} \mynorm{\myPhi^\dagger\mybracket{k} \myPhi \mybracket{k'}}_{1 \to 2} \\
&= M_{1,2}\myparanthese{\myPhi} 
&&\leq \max_{k,k' \neq k} \frac{d_{k}^{-\frac12} \, d_{k'}}{d_{max}} d_{k}^{\frac12} \mynorm{\myPhi^\dagger\mybracket{k} \myPhi \mybracket{k'}}_{1 \to \infty} \\
& &&= \max_{k,k' \neq k} \frac{d_{k'}}{d_{max}} \mynorm{\myPhi^\dagger\mybracket{k} \myPhi \mybracket{k'}}_{1 \to \infty} \\
& &&= M_{1,\infty}\myparanthese{\myPhi}.
\end{aligned}
\end{equation*}

Similarly, for the second inequality taking into account the relationship between the $\ell_{2 \to 2}$ and $\ell_{1 \to 2}$ operator-norms, we have:
\begin{equation*}
\begin{aligned}
M_{2,2}\myparanthese{\myPhi} 
&\myeq \max_{k,k' \neq k} \frac{d_{k}^{-\frac12} \, d_{k'}^{\frac12}}{d_{max}} \mynorm{\myPhi^\dagger\mybracket{k} \myPhi \mybracket{k'}}_{2 \to 2} \\
&\leq \max_{k,k' \neq k} \frac{d_{k}^{-\frac12} \, d_{k'}^{\frac12}}{d_{max}} d_{k'}^{\frac12} \mynorm{\myPhi^\dagger\mybracket{k} \myPhi \mybracket{k'}}_{1 \to 2} \\
&= \max_{k,k' \neq k} \frac{d_{k}^{-\frac12} \, d_{k'}}{d_{max}} \mynorm{\myPhi^\dagger\mybracket{k} \myPhi \mybracket{k'}}_{1 \to 2} \\
&= M_{1,2}\myparanthese{\myPhi}.
\end{aligned}
\end{equation*}

Then, for the third inequality considering the relationship between the $\ell_{2 \to 2}$ and $\ell_{2 \to \infty}$, and also $\ell_{2 \to \infty}$ and $\ell_{1 \to \infty}$ operator-norms, we have:
\begin{equation*}
\begin{aligned}
M_{2,2}\myparanthese{\myPhi} 
&\myeq \max_{k,k' \neq k} &&\frac{d_{k}^{-\frac12} \, d_{k'}^{\frac12}}{d_{max}} \mynorm{\myPhi^\dagger\mybracket{k} \myPhi \mybracket{k'}}_{2 \to 2} \\
&\leq \max_{k,k' \neq k} &&\frac{d_{k}^{-\frac12} \, d_{k'}^{\frac12}}{d_{max}} d_{k}^{\frac12} \mynorm{\myPhi^\dagger\mybracket{k} \myPhi \mybracket{k'}}_{2 \to \infty} \\
&= \max_{k,k' \neq k} &&\frac{d_{k'}^{\frac12}}{d_{max}} \mynorm{\myPhi^\dagger\mybracket{k} \myPhi \mybracket{k'}}_{2 \to \infty} \\
&= M_{2 , \infty}\myparanthese{\myPhi} 
&&\leq \max_{k,k' \neq k} \frac{d_{k'}^{\frac12}}{d_{max}} d_{k'}^{\frac12} \mynorm{\myPhi^\dagger\mybracket{k} \myPhi \mybracket{k'}}_{1 \to \infty} \\
& &&= \max_{k,k' \neq k} \frac{d_{k'}}{d_{max}} \mynorm{\myPhi^\dagger\mybracket{k} \myPhi \mybracket{k'}}_{1 \to \infty} \\
& &&= M_{1,\infty}\myparanthese{\myPhi}.
\end{aligned}
\end{equation*}

Finally, for the last inequality we need to know the relationship between the $\ell_{\infty \to \infty}$ and $\ell_{2 \to \infty}$ operator-norms:
\begin{equation*}
\begin{aligned}
M_{\infty,\infty}\myparanthese{\myPhi} 
&\myeq \max_{k,k' \neq k} \frac{1}{d_{max}} \mynorm{\myPhi^\dagger\mybracket{k} \myPhi \mybracket{k'}}_{\infty \to \infty} \\
&\leq \max_{k,k' \neq k} \frac{1}{d_{max}} d_{k'}^{\frac12} \mynorm{\myPhi^\dagger\mybracket{k} \myPhi \mybracket{k'}}_{2 \to \infty} \\
&= \max_{k,k' \neq k} \frac{d_{k'}^{\frac12}}{d_{max}} \mynorm{\myPhi^\dagger\mybracket{k} \myPhi \mybracket{k'}}_{2 \to \infty} \\
&= M_{2 , \infty}\myparanthese{\myPhi}.
\end{aligned}
\end{equation*}

For the proof of general relationships of figure \ref{fig:BMCC_Inequalities}, utilising the upper-bound of second set of bounds in Property \ref{prp:OperatorProperties} ($\ell_{q {\to} p}$ operator-norm properties), i.e., $\forall \myparanthese{q , p , q' , p'} \ssin \mathbb{R}^4_{>0} , \forall \boldsymbol{A} \ssin \mathbb{R}^{m \stimes n} : \Vert \boldsymbol{A} \Vert_{q \to p} \sleq \max \{1 , m^{1/p \sm 1/{p'}} \} \max \{1 , n^{1/{q'} \sm 1/q} \} \Vert \boldsymbol{A} \Vert_{q' \to p'}$, and for a fixed $p \seq p'$, and $q' \sleq q$, which yields $\Vert \boldsymbol{A} \Vert_{q \to p} \sleq n^{1/{q'} \sm 1/q} \Vert \boldsymbol{A} \Vert_{q' \to p}$, we get the following inequalities:
\begin{equation*}
\begin{aligned}
\forall (q , p , q') \in \mathbb{R}^3_{>0}, q' \leq q, \qquad
M_{q,p}\myparanthese{\myPhi} 
&\myeq \max_{k,k' \neq k} \frac{d_{k}^{-\frac1p} \, d_{k'}^{\frac1q}}{d_{max}} \mynorm{\myPhi^\dagger\mybracket{k} \myPhi \mybracket{k'}}_{q \to p} \\
&\leq \max_{k,k' \neq k} \frac{d_{k}^{-\frac1p} \, d_{k'}^{\frac1q}}{d_{max}} \, d_{k'}^{\frac{1}{q'} - \frac1q} \, \mynorm{\myPhi^\dagger\mybracket{k} \myPhi \mybracket{k'}}_{q' \to p} \\
&= \max_{k,k' \neq k} \frac{d_{k}^{-\frac1p} \, d_{k'}^{\frac{1}{q'}}}{d_{max}} \mynorm{\myPhi^\dagger\mybracket{k} \myPhi \mybracket{k'}}_{q' \to p} \\
&= M_{q',p}\myparanthese{\myPhi}.
\end{aligned}
\end{equation*}
Similarly, using the same upper-bound of Property \ref{prp:OperatorProperties}, this time for a fixed $q \seq q'$, and $p \sleq p'$, which yields $\Vert \boldsymbol{A} \Vert_{q \to p} \sleq m^{1/p \sm 1/{p'}} \Vert \boldsymbol{A} \Vert_{q \to p'}$, we get the following inequalities:
\begin{equation*}
\begin{aligned}
\forall (q , p , p') \in \mathbb{R}^3_{>0}, p \leq p', \qquad
M_{q,p}\myparanthese{\myPhi} 
&\myeq \max_{k,k' \neq k} \frac{d_{k}^{-\frac1p} \, d_{k'}^{\frac1q}}{d_{max}} \mynorm{\myPhi^\dagger\mybracket{k} \myPhi \mybracket{k'}}_{q \to p} \\
&\leq \max_{k,k' \neq k} \frac{d_{k}^{-\frac1p} \, d_{k'}^{\frac1q}}{d_{max}} \, d_{k}^{\frac1p - \frac{1}{p'}} \, \mynorm{\myPhi^\dagger\mybracket{k} \myPhi \mybracket{k'}}_{q \to p'} \\
&= \max_{k,k' \neq k} \frac{d_{k}^{-\frac{1}{p'}} \, d_{k'}^{\frac1q}}{d_{max}} \mynorm{\myPhi^\dagger\mybracket{k} \myPhi \mybracket{k'}}_{q \to p'} \\
&= M_{q,p'}\myparanthese{\myPhi}.
\end{aligned}
\end{equation*}
\end{proof}
%------------------------------------------------------
\newpage
\section{Proof of Property \ref{prp:BMIC-orth-bound} (Block-MCC$_{q,p}$ bounds with intra-block orthonormality I, page \pageref{prp:BMIC-orth-bound})}
\label{prf:BMIC-bounds}
\begin{proof}
%Since the operator norm is non-negative, simply it can be seen that $M_{q,p}(\myPhi) \sgeq 0$.
%We begin the proof by the following property:
%\begin{property}[Submultiplicativity property]
%\label{lm:Submultiplicativity}
%Supposing $\boldsymbol{A}$ and $\boldsymbol{B}$ as two matrices, we have the following inequality for the operator-norm of their product:
%\begin{gather*}
%\begin{aligned}
%\mynorm{\boldsymbol{A} \boldsymbol{B}}_{q \to p} 
%&\leq \max\mybrace{\mynorm{\boldsymbol{A}}_{q \to p} \mynorm{\boldsymbol{B}}_{q \to q} , \mynorm{\boldsymbol{A}}_{p \to p} \mynorm{\boldsymbol{B}}_{q \to %p}} \\
%&\leq \max\mybrace{\mynorm{\boldsymbol{A}}_{q \to p} , \mynorm{\boldsymbol{A}}_{p \to p}} 
%\max\mybrace{\mynorm{\boldsymbol{B}}_{q \to q} ,  \mynorm{\boldsymbol{B}}_{q \to p}}.
%\end{aligned}
%\end{gather*}
%\end{property}
%\begin{proof}%[Submultiplicativity property]
%\begin{equation*}
%\begin{aligned}
%\mynorm{\boldsymbol{A} \boldsymbol{B}}_{q \to p} = \max_{\boldsymbol{x} \neq \boldsymbol{0}} \frac{\mynorm{\boldsymbol{ABx}}_{p}}{\mynorm{\boldsymbol{x}}_{q}} &= 
%\begin{cases}
%\displaystyle\max_{\boldsymbol{Bx} \neq \boldsymbol{0}} \frac{\mynorm{\boldsymbol{ABx}}_{p}}{\mynorm{\boldsymbol{Bx}}_{q}} \frac{\mynorm{\boldsymbol{Bx}}_{q}}{\mynorm{\boldsymbol{x}}_{q}}\\
%and\\
%\displaystyle\max_{\boldsymbol{Bx} \neq \boldsymbol{0}} \frac{\mynorm{\boldsymbol{ABx}}_{p}}{\mynorm{\boldsymbol{Bx}}_{p}} \frac{\mynorm{\boldsymbol{Bx}}_{p}}{\mynorm{\boldsymbol{x}}_{q}} \\
%\end{cases}\\
%&\leq \begin{cases}
%\displaystyle\max_{\boldsymbol{y} \neq \boldsymbol{0}} \frac{\mynorm{\boldsymbol{Ay}}_{p}}{\mynorm{\boldsymbol{y}}_{q}} \displaystyle\max_{\boldsymbol{x} \neq \boldsymbol{0}} \frac{\mynorm{\boldsymbol{Bx}}_{q}}{\mynorm{\boldsymbol{x}}_{q}}\\
%and\\
%\displaystyle\max_{\boldsymbol{y} \neq \boldsymbol{0}} \frac{\mynorm{\boldsymbol{Ay}}_{p}}{\mynorm{\boldsymbol{y}}_{p}} \displaystyle\max_{\boldsymbol{x} \neq \boldsymbol{0}} \frac{\mynorm{\boldsymbol{Bx}}_{p}}{\mynorm{\boldsymbol{x}}_{q}}. \\
%\end{cases}
%\end{aligned}
%\end{equation*}
%\end{proof}
Using the definition of $M_{q,p}(\myPhi)$ in Property \ref{prp:IntraBlkO}, and submultiplicativity property of operator-norm introduced in Property \ref{prp:OperatorProperties} ($\ell_{q {\to} p}$ operator-norm properties), we have:
\begin{gather}
\begin{aligned}
\label{eq:BMCC_UpperBound} 
\forall \myparanthese{q , p} \in \mathbb{R}^2_{>0}, \qquad
M_{q,p}\myparanthese{\myPhi} &=
\max_{k,k' \neq k} \frac{d_{k}^{-\frac1p} \, d_{k'}^{\frac1q}}{d_{max}} \mynorm{\myPhi^T \mybracket{k} \myPhi \mybracket{k'}}_{q \to p} \\
&\leq \max_{k,k' \neq k} \frac{d_{k}^{-\frac1p} \, d_{k'}^{\frac1q}}{d_{max}} 
\mynorm{\myPhi^T \mybracket{k}}_{q \to p} \mynorm{\myPhi[k']}_{q \to p} \max \mybrace{1 , m^{\frac1q - \frac1p}}.
%\max \mybrace{\mynorm{\myPhi^T \mybracket{k}}_{q \to p}, \mynorm{\myPhi^T [k]}_{p \to p}} \max \mybrace{\mynorm{\myPhi \mybracket{k'}}_{q \to q},  \mynorm{\myPhi[k']}_{q \to p}}.
\end{aligned}
\end{gather}
On the other hand, 
%$d_{k}^{-1/p} d_{k'}^{1/q}$ is bounded between $d_{min}^{1/q} d_{max}^{-1/p}$ and $d_{min}^{-1/p} d_{max}^{1/q}$ and 
from the following property, 
%Lemma \ref{lm:qpTO22}, 
$\Vert \cdot \Vert_{q {\to} p}$ can be bounded in terms of $\Vert \cdot \Vert_{2 {\to} 2}$, which is often called the spectral norm:
\begin{property}[Bounds of $\ell_{q {\to} p}$ operator-norm in terms of $\ell_{2 {\to} 2}$]
\label{lm:qpTO22}
The bounds of the $\ell_{q {\to} p}$ operator-norm of a matrix $\boldsymbol{A} \ssin \mathbb{R}^{m \stimes n}$ in terms of its $\ell_{2 {\to} 2}$ operator-norm based on the second set of bounds in Property \ref{prp:OperatorProperties} ($\ell_{q {\to} p}$ operator-norm properties) for $q' \seq p' \seq 2$ is:
\begin{gather*}
\begin{aligned}
&\mynorm{\boldsymbol{A}}_{q \to p} \geq \min \mybrace{1 , m^{\frac1p - \frac{1}{2}}} \min \mybrace{1 , n^{\frac{1}{2} - \frac1q}} \mynorm{\boldsymbol{A}}_{2 \to 2}, \\
&\mynorm{\boldsymbol{A}}_{q \to p} \leq \max \mybrace{1 , m^{\frac1p - \frac{1}{2}}} \max \mybrace{1 , n^{\frac{1}{2} - \frac1q}} \mynorm{\boldsymbol{A}}_{2 \to 2}.
\end{aligned}
\end{gather*}
These bounds for different values of $q$ and $p$ are shown in table \ref{table:qpTO22}.
\begin{table}[bp]
%\begin{adjustbox}{width=0.5\textwidth} % ,totalheight=\textheight,.5
\centering
%\tiny
\begin{tabular}{ccccc}
\toprule
%\cline{2-4}
\multicolumn{1}{c}{} &\multicolumn{1}{c}{${q\, \& \, p \leq 2}$} & \multicolumn{1}{c}{${q\, \& \, p \geq 2}$}  & \multicolumn{1}{c}{${q \leq 2\, \& \, p \geq 2}$} & \multicolumn{1}{c}{${q \geq 2\, \& \, p \leq 2}$} \\ \midrule %\hline
\multicolumn{1}{r}{${\frac{\mynorm{\boldsymbol{A}}_{q \to p}}{\mynorm{\boldsymbol{A}}_{2 \to 2}} \leq}$} &\multicolumn{1}{c}{$m^{\frac1p - \frac12}$} & \multicolumn{1}{c}{$n^{-\frac1q + \frac12}$} & \multicolumn{1}{c}{$1$} &\multicolumn{1}{c}{$m^{\frac1p - \frac12} n^{-\frac1q + \frac12}$}    \\ %\midrule %\hline
\multicolumn{1}{r}{${\frac{\mynorm{\boldsymbol{A}}_{q \to p}}{\mynorm{\boldsymbol{A}}_{2 \to 2}} \geq}$} &\multicolumn{1}{c}{$n^{-\frac1q + \frac12}$} & \multicolumn{1}{c}{$m^{\frac1p - \frac12}$} & \multicolumn{1}{c}{$m^{\frac1p - \frac12} n^{-\frac1q + \frac12}$} &\multicolumn{1}{c}{$1$}    \\
\bottomrule %\hline
\end{tabular}
%\end{adjustbox}
\caption{Bounds of $\Vert \boldsymbol{A} \Vert_{q \to p}/ \Vert \boldsymbol{A} \Vert_{2 \to 2}$ for different values of $q$ and $p$, where, $\boldsymbol{A}$ is a $m$ by $n$ matrix.}
\label{table:qpTO22}
\end{table}
\end{property}
\iffalse
\begin{proof}%[Bounds of $\mynorm{\cdot}_{q \to p}$]
From the definition of the operator-norm \cite{Tropp2004b,Golub2013}, we have $\Vert \boldsymbol{A}\Vert_{q \to p} \seq \max_{\boldsymbol{a} \neq \boldsymbol{0}} \Vert \boldsymbol{Aa}\Vert_p {/} \Vert \boldsymbol{a}\Vert_q$, where, $\boldsymbol{A} \ssin \mathbb{R}^{d_1 \stimes d_2}$, $\boldsymbol{a} \ssin \mathbb{R}^{d_2}$, and hence $\boldsymbol{Aa} \ssin \mathbb{R}^{d_1}$.
On the other hand, based on the bounds introduced in Property \ref{prp:VectorDivisionBound}, we have:


for $0 \sless q' \sless p'$, and $\boldsymbol{b} \ssin \mathbb{R}^{d}$, we have $B1 {:} d^{1/p' \sm 1/q'} \sleq \Vert \boldsymbol{b} \Vert_{p'} {/} \Vert \boldsymbol{b} \Vert_{q'} \sleq 1$ and $B2 {:} 1 \sleq \Vert\boldsymbol{b} \Vert_{q'} {/} \Vert \boldsymbol{b} \Vert_{p'} \sleq d^{1/q' \sm 1/p'}$ \cite{Golub2013}.
To compute the upper-bound of $\Vert \boldsymbol{A}\Vert_{q \to p} \seq \max_{\boldsymbol{a} \neq \boldsymbol{0}} \Vert \boldsymbol{Aa}\Vert_p {/} \Vert \boldsymbol{a}\Vert_q$, we need to find the maximum value of the numerator, i.e., $\Vert \boldsymbol{Aa}\Vert_p$, and the minimum value of the denominator, i.e., $\Vert \boldsymbol{a}\Vert_q$, for each case of pairs of $q$ and $p$.
Similarly, the lower-bound can be computed.
Utilising the two above-mentioned inequalities $B1$ and $B2$, in the following we demonstrate the values of table \ref{table:qpTO22}:
\begin{itemize}
\item ${q\, \& \, p \sless 2}$: To establish the upper-bound of $\Vert \boldsymbol{A}\Vert_{q \to p}$, using upper-bound of $B2$ with $q' \seq p$ and $p' \seq 2$, we have $\Vert \boldsymbol{Aa}\Vert_p \sleq d_1^{1/p \sm 1/2} \Vert \boldsymbol{Aa}\Vert_2$, whereas using lower-bound of $B2$ with $q' \seq q$ and $p' \seq 2$, we have $\Vert \boldsymbol{a}\Vert_q \sgeq \Vert \boldsymbol{a}\Vert_2$, therefore, $\Vert \boldsymbol{A}\Vert_{q \to p} \sleq \max_{\boldsymbol{a} \neq \boldsymbol{0}} d_1^{1/p \sm 1/2} \Vert \boldsymbol{Aa}\Vert_2 {/} \Vert \boldsymbol{a}\Vert_2 \seq d_1^{1/p \sm 1/2} \Vert \boldsymbol{A}\Vert_{2 \to 2}$.
On the other hand, to find the lower-bound of $\Vert \boldsymbol{A}\Vert_{q \to p}$, using lower-bound of $B2$ with $q' \seq p$ and $p' \seq 2$, we have $\Vert \boldsymbol{Aa}\Vert_p \sgeq \Vert \boldsymbol{Aa}\Vert_2$, whereas using upper-bound of $B2$ with $q' \seq q$ and $p' \seq 2$, we have $\Vert \boldsymbol{a}\Vert_q \sleq d_2^{1/q \sm 1/2} \Vert \boldsymbol{a}\Vert_2$, therefore, $\Vert \boldsymbol{A}\Vert_{q \to p} \sgeq \max_{\boldsymbol{a} \neq \boldsymbol{0}} d_2^{-1/q \spl 1/2} \Vert \boldsymbol{Aa}\Vert_2 {/} \Vert \boldsymbol{a}\Vert_2 \seq d_2^{-1/q \spl 1/2} \Vert \boldsymbol{A}\Vert_{2 \to 2}$.
\item ${q\, \& \, p \sg 2}$: To establish the upper-bound of $\Vert \boldsymbol{A}\Vert_{q \to p}$, using upper-bound of $B1$ with $p' \seq p$ and $q' \seq 2$, we have $\Vert \boldsymbol{Aa}\Vert_p \sleq \Vert \boldsymbol{Aa}\Vert_2$, whereas using lower-bound of $B1$ with $p' \seq q$ and $q' \seq 2$, we have $\Vert \boldsymbol{a}\Vert_q \sgeq d_2^{1/q \sm 1/2} \Vert \boldsymbol{a}\Vert_2$, therefore, $\Vert \boldsymbol{A}\Vert_{q \to p} \sleq \max_{\boldsymbol{a} \neq \boldsymbol{0}} d_2^{-1/q \spl 1/2} \Vert \boldsymbol{Aa}\Vert_2 {/} \Vert \boldsymbol{a}\Vert_2 \seq d_2^{-1/q \spl 1/2} \Vert \boldsymbol{A}\Vert_{2 \to 2}$.
On the other hand, to find the lower-bound of $\Vert \boldsymbol{A}\Vert_{q \to p}$, using lower-bound of $B1$ with $p' \seq p$ and $q' \seq 2$, we have $\Vert \boldsymbol{Aa}\Vert_p \sgeq d_1^{1/p \sm 1/2} \Vert \boldsymbol{Aa}\Vert_2$, whereas using upper-bound of $B1$ with $p' \seq q$ and $q' \seq 2$, we have $\Vert \boldsymbol{a}\Vert_q \sleq \Vert \boldsymbol{a}\Vert_2$, therefore, $\Vert \boldsymbol{A}\Vert_{q \to p} \sgeq \max_{\boldsymbol{a} \neq \boldsymbol{0}} d_1^{1/p \sm 1/2} \Vert \boldsymbol{Aa}\Vert_2 {/} \Vert \boldsymbol{a}\Vert_2 \seq d_1^{1/p \sm 1/2} \Vert \boldsymbol{A}\Vert_{2 \to 2}$.
\item ${q \sleq 2\, \& \, p \sgeq 2}$: To establish the upper-bound of $\Vert \boldsymbol{A}\Vert_{q \to p}$, using upper-bound of $B1$ with $p' \seq p$ and $q' \seq 2$, we have $\Vert \boldsymbol{Aa}\Vert_p \sleq \Vert \boldsymbol{Aa}\Vert_2$, whereas using lower-bound of $B2$ with $q' \seq q$ and $p' \seq 2$, we have $\Vert \boldsymbol{a}\Vert_q \sgeq \Vert \boldsymbol{a}\Vert_2$, therefore, $\Vert \boldsymbol{A}\Vert_{q \to p} \sleq \max_{\boldsymbol{a} \neq \boldsymbol{0}} \Vert \boldsymbol{Aa}\Vert_2 {/} \Vert \boldsymbol{a}\Vert_2 \seq \Vert \boldsymbol{A}\Vert_{2 \to 2}$.
On the other hand, to find the lower-bound of $\Vert \boldsymbol{A}\Vert_{q \to p}$, using lower-bound of $B1$ with $p' \seq p$ and $q' \seq 2$, we have $\Vert \boldsymbol{Aa}\Vert_p \sgeq d_1^{1/p \sm 1/2} \Vert \boldsymbol{Aa}\Vert_2$, whereas using upper-bound of $B2$ with $q' \seq q$ and $p' \seq 2$, we have $\Vert \boldsymbol{a}\Vert_q \sleq d_2^{1/q \sm 1/2} \Vert \boldsymbol{a}\Vert_2$, therefore, $\Vert \boldsymbol{A}\Vert_{q \to p} \sgeq \max_{\boldsymbol{a} \neq \boldsymbol{0}} d_1^{1/p \sm 1/2} \, d_2^{-1/q \spl 1/2} \Vert \boldsymbol{Aa}\Vert_2 {/} \Vert \boldsymbol{a}\Vert_2 \seq d_1^{1/p \sm 1/2} \, d_2^{-1/q \spl 1/2} \Vert \boldsymbol{A}\Vert_{2 \to 2}$.
\item ${q \sgeq 2\, \& \, p \sleq 2}$: To establish the upper-bound of $\Vert \boldsymbol{A}\Vert_{q \to p}$, using upper-bound of $B2$ with $q' \seq p$ and $p' \seq 2$, we have $\Vert \boldsymbol{Aa}\Vert_p \sleq d_1^{1/p \sm 1/2} \Vert \boldsymbol{Aa}\Vert_2$, whereas using lower-bound of $B1$ with $p' \seq q$ and $q' \seq 2$, we have $\Vert \boldsymbol{a}\Vert_q \sgeq d_2^{1/q \sm 1/2} \Vert \boldsymbol{a}\Vert_2$, therefore, $\Vert \boldsymbol{A}\Vert_{q \to p} \sleq \max_{\boldsymbol{a} \neq \boldsymbol{0}} d_1^{1/p \sm 1/2} d_2^{-1/q \spl 1/2} \Vert \boldsymbol{Aa}\Vert_2 {/} \Vert \boldsymbol{a}\Vert_2 \seq d_1^{1/p \sm 1/2} d_2^{-1/q \spl 1/2} \Vert \boldsymbol{A}\Vert_{2 \to 2}$.
On the other hand, to find the lower-bound of $\Vert \boldsymbol{A}\Vert_{q \to p}$, using lower-bound of $B2$ with $q' \seq p$ and $p' \seq 2$, we have $\Vert \boldsymbol{Aa}\Vert_p \sgeq \Vert \boldsymbol{Aa}\Vert_2$, whereas using upper-bound of $B1$ with $p' \seq q$ and $q' \seq 2$, we have $\Vert \boldsymbol{a}\Vert_q \sleq \Vert \boldsymbol{a}\Vert_2$, therefore, $\Vert \boldsymbol{A}\Vert_{q \to p} \sgeq \max_{\boldsymbol{a} \neq \boldsymbol{0}}  \Vert \boldsymbol{Aa}\Vert_2 {/} \Vert \boldsymbol{a}\Vert_2 \seq \Vert \boldsymbol{A}\Vert_{2 \to 2}$.
\end{itemize} 
\end{proof}
\fi
Therefore, based on Property \ref{lm:qpTO22}, we upper-bound the $\ell_{q {\to} p}$ operator-norms of the upper-bound of Block-MCC in equation (\ref{eq:BMCC_UpperBound}):
\begin{gather}
\label{eq:BMCC-UB} 
\begin{aligned} 
M_{q,p}\myparanthese{\myPhi} \leq 
\max_{k,k' \neq k} \frac{d_{k}^{-\frac1p} \, d_{k'}^{\frac1q}}{d_{max}} 
&\mycolor{\max} \mybrace{1 , d_{k}^{\frac1p - \frac{1}{2}}} \mycolor{\max} \mybrace{1 , m^{\frac{1}{2} - \frac1q}} \mynorm{\myPhi^T \mybracket{k}}_{2 \to 2} \times\\ 
&\mycolor{\max} \mybrace{1 , m^{\frac1p - \frac{1}{2}}} \mycolor{\max} \mybrace{1 , d_{k'}^{\frac{1}{2} - \frac1q}} \mynorm{\myPhi[k']}_{2 \to 2} 
\max \mybrace{1 , m^{\frac1q - \frac1p}}.
\end{aligned}
\end{gather}

But $\forall k, \Vert \myPhi^T [k] \Vert_{2 {\to} 2} \seq \Vert \myPhi[k] \Vert_{2 {\to} 2} \seq 1$, because for a typical matrix $\boldsymbol{A} \ssin \mathbb{R}^{m \stimes n}$ we have $\Vert \boldsymbol{A} \Vert_{2 {\to} 2} \seq \sigma_{max}(\boldsymbol{A}) \seq \sqrt{\lambda_{max}(\boldsymbol{A}^T \boldsymbol{A})}$, where, $\sigma$ and $\lambda$ are singular value and eigenvalue, respectively, but here $\boldsymbol{A}$ is an  orthonormal matrix, then $\boldsymbol{A}^T \boldsymbol{A} \seq \boldsymbol{I}_n$, and then $\lambda_{max}(\boldsymbol{I}_n) \seq 1$.

Therefore, the inequality (\ref{eq:BMCC-UB}) is developed into table \ref{table:BMCC_UpperBound}.
In each case in the last column of the table \ref{table:BMCC_UpperBound}, the maximum value is achieved for minimum block length $d_{min}$ with negative power, and maximum block length $d_{max}$ with positive power.
The table \ref{table:BMIC-orth-bound} shows the upper-bound of $M_{q,p}(\myPhi)$ based on table \ref{table:BMCC_UpperBound}, for the basic tractable values of $q$ and $p$ based on table \ref{table:OperatorNorm}. 
\begin{table}[tp]
\begin{adjustbox}{width=1\textwidth} % ,totalheight=\textheight,.5
\centering
%\tiny
\begin{tabular}{ccccc}
\toprule
%\cline{2-4}
\multicolumn{1}{c}{} & \multicolumn{1}{c}{$\max M_{q,p}(\myPhi)$} & \multicolumn{1}{c}{ } & \multicolumn{1}{c}{$\max M_{q,p}(\myPhi)$} \\ \midrule %\hline
\multicolumn{1}{r}{${0 < q\, \& \, p \leq 2}$} & \multicolumn{1}{l}{$\displaystyle\max_{k,k' \neq k} d_{k}^{-\frac12} \, d_{k'}^{\frac1q} \, d_{max}^{-1} \, m^{\frac1p - \frac12} \, \max \mybrace{1 , m^{\frac1q - \frac1p}}$} & \multicolumn{1}{l}{$\leq$} & \multicolumn{1}{l}{$d_{min}^{-\frac12} \, d_{max}^{\frac1q - 1} \, m^{\frac1p - \frac12} \, \max \mybrace{1 , m^{\frac1q - \frac1p}}$}  \\ 
\multicolumn{1}{r}{${q\, \& \, p \geq 2}$} & \multicolumn{1}{l}{$\displaystyle\max_{k,k' \neq k} d_{k}^{-\frac1p} \, d_{k'}^{\frac12} \, d_{max}^{-1} \, m^{\frac12 - \frac1q} \, \max \mybrace{1 , m^{\frac1q - \frac1p}}$} & \multicolumn{1}{l}{$\leq$} & \multicolumn{1}{l}{$d_{min}^{-\frac1p} \, d_{max}^{-\frac12} \, m^{\frac12 - \frac1q} \, \max \mybrace{1 , m^{\frac1q - \frac1p}}$}  \\
\multicolumn{1}{r}{${0 < q \leq 2\, \& \, p \geq 2}$} & \multicolumn{1}{l}{$\displaystyle\max_{k,k' \neq k} d_{k}^{-\frac1p} \, d_{k'}^{\frac1q} \, d_{max}^{-1} \, m^{\frac1q - \frac1p}$}  & \multicolumn{1}{l}{$\leq$} & \multicolumn{1}{l}{$d_{min}^{-\frac1p} \, d_{max}^{\frac1q - 1} \, m^{\frac1q - \frac1p}$}  \\ 
\multicolumn{1}{r}{${q \geq 2\, \& \, 0 < p \leq 2}$} & \multicolumn{1}{l}{$\displaystyle\max_{k,k' \neq k} d_{k}^{-\frac12} \, d_{k'}^{\frac12} \, d_{max}^{-1} \, m^{\frac1p - \frac1q}$}  & \multicolumn{1}{l}{$\leq$} & \multicolumn{1}{l}{$d_{min}^{-\frac12} \, d_{max}^{-\frac12} \, m^{\frac1p - \frac1q}$} \\
\bottomrule %\hline
\end{tabular}
\end{adjustbox}
\caption{Upper-bound of Block-MCC$_{q,p}$ for different ranges of $q$ and $p$ and for a dictionary with intra-block orthonormality.}
\label{table:BMCC_UpperBound}
\end{table}
\end{proof}
\newpage
%------------------------------------------------------
\section{Proof of Property \ref{prp:BMIC-MIC} (Block-MCC$_{q,p}$ bounds, page \pageref{prp:BMIC-MIC})}
\label{prf:BMIC-MIC} 
\begin{proof}
1) Since the operator-norm is nonnegative (Property \ref{prp:OperatorProperties}), it can be seen that $\forall k , k' {\neq} k , \forall (q , p) \ssin \mathbb{R}^2_{\sgeq 0} : \Vert \myPhi^\dagger [k] \myPhi [k'] \Vert_{q \to p} \sgeq 0 \Rightarrow \forall (q , p) \ssin \mathbb{R}^2_{\sg 0} : M_{q,p}(\myPhi) \sgeq 0$.
Considering that for all $k$ the blocks $\myPhi [k]$ are full column rank, using the Moore-Penrose pseudo-inverse property of matrices, $\forall (q , p) \ssin \mathbb{R}^2_{\sg 0}$ we have:
\begin{gather*}
\begin{aligned}
M_{q,p}\myparanthese{\myPhi} &= \max_{k,k' \neq k} \frac{d_{k}^{-\frac1p} \, d_{k'}^{\frac1q}}{d_{max}} \mynorm{\myPhi^\dagger \mybracket{k} \myPhi \mybracket{k'}}_{q \to p} \\
&= \max_{k,k' \neq k} \frac{d_{k}^{-\frac1p} \, d_{k'}^{\frac1q}}{d_{max}} \mynorm{\myparanthese{\myPhi^T \mybracket{k} \myPhi\mybracket{k}}^{-1}\myPhi^T \mybracket{k} \myPhi \mybracket{k'}}_{q \to p}.
\end{aligned}
\end{gather*}
Applying the general format of submultiplicativity property of operator-norm of a matrix, which is introduced in the Property \ref{prp:OperatorProperties} ($\ell_{q {\to} p}$ operator-norm properties) $\forall (q , p) \ssin \mathbb{R}^2_{\sg 0}$, on the above last equality, $\forall (q , p) \ssin \mathbb{R}^2_{\sg 0}$ we have:
\begin{gather}
\label{eq:temp0}
\begin{aligned}
M_{q,p}\myparanthese{\myPhi} &\leq 
\max_{k,k' \neq k} \frac{d_{k}^{-\frac1p} \, d_{k'}^{\frac1q}}{d_{max}} 
\mynorm{\myparanthese{\myPhi^T \mybracket{k} \myPhi \mybracket{k}}^{-1}}_{q \to p} \, \mynorm{\myPhi^T \mybracket{k} \myPhi \mybracket{k'}}_{q \to p} \max \mybrace{1 , d_k^{\frac1q - \frac1p}}. \\
%&= \max_{k,k' \neq k} \frac{d_{k}^{-\frac1p} \, d_{k'}^{\frac1q}}{d_{max}} 
%\mynorm{\myparanthese{\myPhi^T \mybracket{k} \myPhi \mybracket{k}}^{-1}}_{q \to p} \, \mynorm{\myPhi^T \mybracket{k} \myPhi \mybracket{k'}}_{q \to p}.
\end{aligned}
\end{gather}
%The above last line is obtained by considering that for $q \sgeq p$, we have $\max \{1 , d_k^{1/q \sm 1/p} \} \seq 1$.

\myhl{Since each element of the $\myPhi^T [k] \myPhi [k']$ is the pairwise correlation between the columns of the $k$ and $k' {\neq} k$ blocks, for any $k$ and $k' {\neq} k$ the $\myPhi^T [k] \myPhi [k']$ can be upper bounded by $M(\myPhi) \, \boldsymbol{1}_{d_k \stimes d_{k'}}$, so it is true for their operator-norms, because using Property {\ref{prp:OperatorProperties}} ($\ell_{q {\to} p}$ operator-norm properties) for comparison of $\ell_{q {\to} p}$ operator-norm of two matrices (the forth property of the part Bounds), we have the following first line (based on table {\ref{table:OperatorNorm}}, the maximum absolute entry of a matrix is represented by its $\ell_{1 {\to} \infty}$ operator-norm), whereas by the homogeneity property of $\ell_{q {\to} p}$ operator-norm defined in Property {\ref{prp:OperatorProperties}} $\forall q \sgeq 0$ and $\forall p \sg 0$, the following second line is obtained $\forall (q , p) \ssin \mathbb{R}^2_{\sg 0}$:}
\begin{gather*}
\mycolor{\begin{aligned}
\forall k , k' \neq k, \qquad \mynorm{\myPhi^T \mybracket{k} \myPhi \mybracket{k'}}_{q \to p} &\leq 
\mynorm{\mynorm{\myPhi^T \mybracket{k} \myPhi \mybracket{k'}}_{1 \to \infty} \, \boldsymbol{1}_{d_k \times d_{k'}}}_{q \to p} \\
&= \mynorm{\myPhi^T \mybracket{k} \myPhi \mybracket{k'}}_{1 \to \infty} \mynorm{\boldsymbol{1}_{d_k \times d_{k'}}}_{q \to p} \\
&\leq M\myparanthese{\myPhi} \mynorm{\boldsymbol{1}_{d_k \times d_{k'}}}_{q \to p}.
\end{aligned}}
\end{gather*}
\myhl{The above last line comes from the fact that $\forall k , k' {\neq} k$, the maximum absolute value of multiplication of blocks $k$ and $k'$ is less than or equal to $M(\myPhi)$, which is by definition the maximum off-diagonal absolute value of multiplication of the whole dictionary to itself, i.e., $M(\myPhi) {\myeq} \max_{k , k' \neq k} \vert \boldsymbol{G}_{k,k'}(\myPhi) \vert$, where $\boldsymbol{G}(\myPhi) {\myeq} \myPhi^T \myPhi$.}

Then, using the upper-bound of the third set of bounds in Property {\ref{prp:OperatorProperties}}, we will have the following second line.
Then, by computing the Frobenius norm in the second line, $\forall k , k' {\neq} k$ and $\forall (q , p , q' , p') \ssin \mathbb{R}^4_{\sg 0}$ we get the following third line:

%, in which by substituting $q' \seq p' \seq 1$ (because $q$ and $p$ are lower-bounded by 1), and Frobenius norm the following third line is achieved.
%Finally, considering that $q \sgeq p \sgeq 1$, the following last line of inequalities will be obtained:
\iffalse
Then, using the upper-bound of second set of bounds in Property \ref{prp:OperatorProperties}, i.e., $\Vert \boldsymbol{A} \Vert_{q \to p} \sleq \max \{1 , m^{1/p \sm 1/{p'}} \} \max \{1 , n^{1/{q'} \sm 1/q} \} \Vert \boldsymbol{A} \Vert_{q' \to p'}$, $\boldsymbol{A} \ssin \mathbb{R}^{m \stimes n}$, and for $q' \seq p' \seq 1$, we will have the following second line of inequalities.
On the other hand, for $q \sgeq p$ and $p \sgeq 1$, we have $\max \{1 , d_k^{1/p - 1} \} \seq 1$ and $\max \{1 , d_{k'}^{1 - 1/q} \} \seq d_{k'}^{1 - 1/q}$, which leads to the following third line of inequalities.
Again, using the the upper-bound of second set of bounds in Property \ref{prp:OperatorProperties}, we have $\boldsymbol{A} \ssin \mathbb{R}^{m \stimes n}$, $\Vert \boldsymbol{A} \Vert_{1 \to 1} \sleq  m^{1/2} \Vert \boldsymbol{A} \Vert_{2 \to 2}$, and from \cite{Golub2013} we have $\Vert \boldsymbol{A} \Vert_{2 \to 2} \sleq \Vert \boldsymbol{A} \Vert_F \seq (\sum_{i \seq 1}^m \sum_{j \seq 1}^n |a_{i,j}|^2)^{1/2}$, which produces the following fourth and fifth lines of inequalities.
\begin{gather*}
%\label{eq:temp2}
\begin{aligned}
\mynorm{\myPhi^T \mybracket{k} \myPhi \mybracket{k'}}_{q \to p} &\leq 
M\myparanthese{\myPhi} \mynorm{\boldsymbol{1}_{d_k \times d_{k'}}}_{q \to p} \\
&\leq M\myparanthese{\myPhi} \max \mybrace{1 , d_k^{\frac1p - \frac{1}{p'}}} \max \mybrace{1 , d_{k'}^{\frac{1}{q'} - \frac1q}} \max \mybrace{1 , d_k^{\frac{1}{p'} -\frac12}} \max \mybrace{1 , d_{k'}^{\frac{1}{2} - \frac{1}{q'}}} \mynorm{\boldsymbol{1}_{d_k \times d_{k'}}}_F \\
&= M\myparanthese{\myPhi} \max \mybrace{1 , d_k^{\frac1p - 1}} \max \mybrace{1 , d_{k'}^{1 - \frac1q}} \max \mybrace{1 , d_k^{1 -\frac12}} \max \mybrace{1 , d_{k'}^{\frac{1}{2} - 1}} \myparanthese{d_k \, d_{k'}}^{\frac12}\\
&= M\myparanthese{\myPhi} d_{k'}^{1 - \frac1q} \, d_k^{\frac12} \, \myparanthese{d_k \, d_{k'}}^{\frac12} \\
&=  M\myparanthese{\myPhi} d_{k'}^{\frac32 - \frac1q} \, d_k.
\end{aligned}
\end{gather*}
\fi
\begin{gather}
\label{eq:UB-kk'}
\begin{aligned}
\mynorm{\myPhi^T \mybracket{k} \myPhi \mybracket{k'}}_{q \to p} &\leq 
M\myparanthese{\myPhi} \mynorm{\boldsymbol{1}_{d_k \times d_{k'}}}_{q \to p} \\
&\leq M\myparanthese{\myPhi} \, \max \mybrace{1 , d_k^{\frac1p - \frac{1}{p'}}} \, \max \mybrace{1 , d_{k'}^{\frac{1}{q'} - \frac1q}} \, \max \mybrace{1 , d_k^{\frac{1}{p'} -\frac12}} \, \max \mybrace{1 , d_{k'}^{\frac{1}{2} - \frac{1}{q'}}} \mynorm{\boldsymbol{1}_{d_k \times d_{k'}}}_F \\
&= \myparanthese{d_k \, d_{k'}}^{\frac12} \, M\myparanthese{\myPhi} \, \max \mybrace{1 , d_k^{\frac1p - \frac{1}{p'}}} \, \max \mybrace{1 , d_{k'}^{\frac{1}{q'} - \frac1q}} \, \max \mybrace{1 , d_k^{\frac{1}{p'} -\frac12}} \, \max \mybrace{1 , d_{k'}^{\frac{1}{2} - \frac{1}{q'}}}. \\
%&= M\myparanthese{\myPhi} d_{k'}^{1 - \frac1q} \, d_k^{\frac12} \, \myparanthese{d_k \, d_{k'}}^{\frac12} \\
%&=  M\myparanthese{\myPhi} d_{k'}^{\frac32 - \frac1q} \, d_k.
\end{aligned}
\end{gather}
%This upper bound explains the choice of $d_{max}^{-1}$ in Definition \ref{def:BMIC}. 
Therefore, substituting $\Vert \myPhi^T [k] \myPhi [k'] \Vert_{q \to p}$ with its upper-bound in the (\ref{eq:temp0}), $\forall (q , p , q' , p') \ssin \mathbb{R}^4_{\sg 0}$ we get: 
\begin{gather}
\begin{aligned}
\label{eq:temp1} 
M_{q,p}\myparanthese{\myPhi} \leq 
M\myparanthese{\myPhi} \displaystyle\max_{k,k' \neq k} &\frac{d_{k}^{\frac12 - \frac1p} \, d_{k'}^{\frac1q + \frac12}}{d_{max}} \mynorm{\myparanthese{\myPhi^T \mybracket{k} \myPhi \mybracket{k}}^{-1}}_{q \to p} \times \\
&\max \mybrace{1 , d_k^{\frac1p - \frac{1}{p'}}} \, \max \mybrace{1 , d_{k'}^{\frac{1}{q'} - \frac1q}} \, \max \mybrace{1 , d_k^{\frac{1}{p'} -\frac12}}  \, \max \mybrace{1 , d_{k'}^{\frac{1}{2} - \frac{1}{q'}}} \, \max \mybrace{1 , d_k^{\frac1q - \frac1p}}.
\end{aligned}
\end{gather}

{
\label{cmmnt:77} 
Now an upper bound for $\Vert (\myPhi^T[k] \myPhi[k])^{-1} \Vert_{q \to p}$ should be developed.
}
As part of the required tools to reach this aim we will utilise the following property:
\begin{property}
\label{lm:Horn} 
For $q \sgeq p \sgeq 1$, if $\Vert\boldsymbol{A} \Vert_{q \to p} \sless 1$, then \cite{HornR.A.2012}
\begin{gather*}
\myparanthese{\boldsymbol{I} + \boldsymbol{A}}^{-1} = \displaystyle\sum_{i=0}^\infty \myparanthese{-\boldsymbol{A}}^i.
\end{gather*}
\end{property}
\begin{proof}
It follows from Corollary 5.6.16 of \cite{HornR.A.2012}. 
Based on the Corollary 5.6.16 of \cite{HornR.A.2012}, the \emph{matrix norm} of $\boldsymbol{A}$ should be less than one.
Therefore, we imposed the constraint $q \sgeq p \sgeq 1$, because according to Remark \ref{rmrk:Oprtr-nrm-matrx} (matrix norm) on page \pageref{rmrk:Oprtr-nrm-matrx} the $\ell_{q {\to} p}$ operator-norm for $q \sgeq p \sgeq 1$ satisfies all the required properties to be conceived as a matrix norm. 
%to make the operator-norm a multiplicative norm according to Property \ref{prp:OperatorProperties}.
\end{proof}
Therefore, according to Property \ref{lm:Horn}, we need to decompose the $\myPhi^T[k] \myPhi[k]$ into a sum of two matrices, which one of them is an identity matrix.
For now we continue with the current constraints on $q$ and $p$, i.e., $\forall (q , p) \ssin \mathbb{R}^2_{\sg 0}$, but as soon as we use the above property the constraints will change to $q \sgeq p \sgeq 1$.

$\myPhi^T[k] \myPhi[k]$ is a $d_k \stimes d_k$ matrix which all of its diagonal elements are one, and its off-diagonal elements measure pairwise correlation between the columns of the same block $k$. 
Therefore, $\myPhi^T[k] \myPhi[k]$ can be decomposed as $\myPhi^T[k] \myPhi[k] \seq \boldsymbol{I}_{d_k} \spl \boldsymbol{F}[k] $, where:
\begin{equation*}
\forall k, \qquad \begin{aligned}
f_{i,j} \mybracket{k} = 
  \begin{cases}
    0,   \quad   &\text{if }i = j\\
    \myphi_i^T[k] \myphi_j[k],   \quad &\text{if }i \neq j.\\
  \end{cases} 
  \end{aligned}
\end{equation*} 

But we are interested in the upper-bound of $\Vert (\myPhi^T[k] \myPhi[k])^{-1} \Vert_{q \to p}$.
By the mentioned decomposition we have the following first line of inequalities.
Then, utilising the Property \ref{lm:Horn}, assuming $q \sgeq p \sgeq 1$ and $\Vert \boldsymbol{F} [k] \Vert_{q \to p} \sless 1$ (its veracity will be investigated) the following second line is obtained.
Next utilising the generalisation (from two matrices to more than two matrices) of the triangle inequality property of the $\ell_{q {\to} p}$ operator-norm introduced in Property \ref{prp:OperatorProperties} ($\ell_{q {\to} p}$ operator-norm properties) $\forall q \ssin \mathbb{R}_{\sgeq 0} , \forall p \ssin \mathbb{R}_{\sgeq 1}$, or for $p \seq 0$, the following third line is obtained.
Then utilising the generalisation (from two matrices to more than two matrices) of the submultiplicativity property of the $\ell_{q {\to} p}$ operator-norm introduced in Property \ref{prp:OperatorProperties} for $q \sgeq p \sg 0$, the following forth line is obtained. 
Finally, utilising the sum of an infinite geometric series, i.e., $\sum_{i \seq 0}^{\infty} r ^i \seq 1 /(1 - r)$, where $|r| \sless 1$, the following last line is obtained, considering the assumption $\Vert\boldsymbol{F}[k] \Vert_{q \to p} \sless 1$ (to be investigated).
\begin{equation}
\label{eq:UB-invPhik} 
\begin{aligned}
\forall k, q \geq p \geq 1, \qquad
\mynorm{\myparanthese{\myPhi^T \mybracket{k} \myPhi \mybracket{k}}^{-1}}_{q \to p} &= 
\mynorm{\myparanthese{\boldsymbol{I}_{d_k} + \boldsymbol{F}\mybracket{k}}^{-1}}_{q \to p} \\
&= \mynorm{\displaystyle\sum_{i=0}^\infty \myparanthese{-\boldsymbol{F}\mybracket{k}}^i}_{q \to p} \\
&\leq \displaystyle\sum_{i=0}^\infty \mynorm{\myparanthese{-\boldsymbol{F}\mybracket{k}}^i}_{q \to p} \\
&\leq \displaystyle\sum_{i=0}^\infty \mynorm{\boldsymbol{F}\mybracket{k}}_{q \to p}^i \\
&= \frac{1}{1-\mynorm{\boldsymbol{F}\mybracket{k}}_{q \to p}}.
%&\leq \frac{1}{1- M \myparanthese{\myPhi} d_{max}^{2 - \frac1q} \myparanthese{d_{max} - 1}^\frac12}.
\end{aligned}
\end{equation}

%or consequently $\Vert (\boldsymbol{I}_{d_k} \spl \boldsymbol{F}[k])^{-1} \Vert_{q \to p}$, $\forall (q , p) \ssin \mathbb{R}^2_{\sg 0}$.
%Now we can utilise the Property \ref{lm:Horn}, i.e., $\Vert (\boldsymbol{I}_{d_k} \spl \boldsymbol{F}[k])^{-1} \Vert_{q \to p} \seq \Vert \sum_{i \seq 0}^{\infty} (-\boldsymbol{F}[k])i \Vert_{q \to p} \seq \Vert \sum_{i \seq 0}^{\infty} (\boldsymbol{F}[k])i \Vert_{q \to p}$.
Now we investigate the veracity of the assumption $\Vert\boldsymbol{F}[k] \Vert_{q \to p} \sless 1$.
Since each off-diagonal entry of $\boldsymbol{F}[k]$ is the pairwise correlation between the columns of the $k^{th}$ block and the on-diagonal values are all zero, it can be upper bounded by $M\myparanthese{\myPhi} \, \myparanthese{\boldsymbol{1}_{d_k} \sm \boldsymbol{I}_{d_k}}$, so it is true for their operator-norms, because using Property \ref{prp:OperatorProperties} ($\ell_{q {\to} p}$ operator-norm properties) for comparison of $\ell_{q {\to} p}$ operator-norm of two matrices $\forall (q , p) \ssin \mathbb{R}^2_{\sg 0}$ (the forth property of the part Bounds), we have:
\begin{gather*}
\begin{aligned}
\forall k, q \geq p \geq 1, \qquad
\mynorm{\boldsymbol{F} \mybracket{k}}_{q \to p} &\leq 
\mynorm{\mynorm{\boldsymbol{F} \mybracket{k}}_{1 \to \infty} \, \myparanthese{\boldsymbol{1}_{d_k} - \boldsymbol{I}_{d_k}}}_{q \to p} \\
&= \mynorm{\boldsymbol{F} \mybracket{k}}_{1 \to \infty} \mynorm{\myparanthese{\boldsymbol{1}_{d_k} - \boldsymbol{I}_{d_k}}}_{q \to p} \\
&\leq M\myparanthese{\myPhi} \mynorm{\myparanthese{\boldsymbol{1}_{d_k} - \boldsymbol{I}_{d_k}}}_{q \to p}.
\end{aligned}
\end{gather*}
The $\ell_{1 {\to} \infty}$ operator-norm of a matrix computes the maximum absolute entry of the matrix (table \ref{table:OperatorNorm}). 
The above second line uses the homogeneity property of $\ell_{q {\to} p}$ operator-norm defined in Property \ref{prp:OperatorProperties} ($\ell_{q {\to} p}$ operator-norm properties) $\forall q \sgeq 0$ and $\forall p \sg 0$.
The above last line comes from the fact that $\forall k$, the maximum off-diagonal absolute value of multiplication of block $k$ to itself is less than or equal to $M(\myPhi)$, because the off-diagonal values of $\boldsymbol{F}[k]$ are a subset of off-diagonal values of the Gram matrix $\boldsymbol{G}(\myPhi) {\myeq} \myPhi^T \myPhi$, which the $M(\myPhi)$ is derived from, i.e., $M(\myPhi) {\myeq}  \max_{k,k' \neq k} \vert \boldsymbol{G}_{k,k'}(\myPhi) \vert \seq \max_{k,k' \neq k} \vert \boldsymbol{\varphi}^T_k \boldsymbol{\varphi}^{ }_{k'} \vert$, hence $\Vert \boldsymbol{F}[k] \Vert_{1 \to \infty} \sleq M(\myPhi)$.

Then, using the upper-bound of the third set of bounds in Property \ref{prp:OperatorProperties} ($\ell_{q {\to} p}$ operator-norm properties), we will have the following second line.
By computing the Frobenius norm in the second line, the third line is obtained.
Considering the obtained constraint on $q$ and $p$, i.e., $q \sgeq p \sgeq 1$, we choose $q' \seq p' \seq 1$ in the following third line, because both $q$ and $p$ are lower-bounded by one.
%, $\forall k$ and $\forall (q , p , q' , p') \ssin \mathbb{R}^4_{\sg 0}$ we get:
%in which by substituting $q' \seq p' \seq 1$ (because $q$ and $p$ are thresholded by 1), and Frobenius norm the following third line is achieved.
%Finally, considering that $q \sgeq p \sgeq 1$, the following last line of inequalities will be obtained:
\begin{gather*}
\begin{aligned}
\mynorm{\boldsymbol{F} \mybracket{k}}_{q \to p} &\leq 
M\myparanthese{\myPhi} \mynorm{\myparanthese{\boldsymbol{1}_{d_k} - \boldsymbol{I}_{d_k}}}_{q \to p} \\
&\leq M\myparanthese{\myPhi} \, \max \mybrace{1 , d_k^{\frac1p - \frac{1}{p'}}} \, \max \mybrace{1 , d_k^{\frac{1}{q'} - \frac1q}} \, \max \mybrace{1 , d_k^{\frac{1}{p'} -\frac12}} \, \max \mybrace{1 , d_k^{\frac{1}{2} - \frac{1}{q'}}} \mynorm{\myparanthese{\boldsymbol{1}_{d_k} - \boldsymbol{I}_{d_k}}}_F \\
&= \myparanthese{d_k^2 - d_k}^\frac12 M\myparanthese{\myPhi} \, \max \mybrace{1 , d_k^{\frac1p - \frac{1}{p'}}} \, \max \mybrace{1 , d_k^{\frac{1}{q'} - \frac1q}} \, \max \mybrace{1 , d_k^{\frac{1}{p'} -\frac12}} \, \max \mybrace{1 , d_k^{\frac{1}{2} - \frac{1}{q'}}} \\
%&= M\myparanthese{\myPhi} d_k^{1 - \frac1q} \, d_k^{\frac12} \, \myparanthese{d_k^2 - d_k}^\frac12 \\
&= \myparanthese{d_k^2 - d_k}^\frac12 \, d_k^{1 - \frac1q} \, d_k^{\frac12} \, M\myparanthese{\myPhi}\\
&= d_k^{2 - \frac1q} \myparanthese{d_k - 1}^\frac12 M\myparanthese{\myPhi}.
\end{aligned}
\end{gather*}
% and we have $\Vert M(\myPhi) (\boldsymbol{1}_{d_k} \sm \boldsymbol{I}_{d_k}) \Vert_{q \to p} \sleq (d_{max} \sm 1) M(\myPhi)$. 
Therefore, for all $k$ we have $\Vert\boldsymbol{F}[k] \Vert_{q \to p} \sleq d_{max}^{2 - 1/q} (d_{max} \sm 1)^{1/2} M(\myPhi)$.
On the other hand, in the Property \ref{prp:BMIC-MIC}, it is assumed that $d_{max}^{2 - 1/q} (d_{max} \sm 1)^{1/2} M(\myPhi) \sless 1$, so the required condition to use the Property \ref{lm:Horn} in (\ref{eq:UB-invPhik}), i.e., $\Vert\boldsymbol{F}[k] \Vert_{q \to p} \sless 1$, 
%Lemma \ref{lm:Horn} 
is satisfied.
Substituting the obtained upper-bound of $\Vert\boldsymbol{F}[k] \Vert_{q \to p}$ in (\ref{eq:UB-invPhik}), we get:

\iffalse
\begin{property}
\label{lm:Horn} 
For $q \sgeq p$, if $\Vert\boldsymbol{A} \Vert_{q \to p} \sless 1$, then \cite{HornR.A.2012}
\begin{gather*}
\myparanthese{\boldsymbol{I} + \boldsymbol{A}}^{-1} = \displaystyle\sum_{i=0}^\infty \myparanthese{-\boldsymbol{A}}^i.
\end{gather*}
\end{property}
\begin{proof}
It follows from Corollary 5.6.16 of \cite{HornR.A.2012}. 
We imposed the constraint $q \sgeq p$, to make the operator-norm a multiplicative norm according to Property \ref{prp:OperatorProperties}.
\end{proof}

Therefore, using the Property \ref{lm:Horn}, and triangle inequality we obtain the following first three lines.
Then, utilising the sum of an infinite geometric series, i.e., $\sum_{i \seq 0}^{\infty} r ^i \seq 1 /(1 - r)$, where $|r| \sless 1$, the following forth line is obtained, because $\Vert\boldsymbol{F}[k] \Vert_{q \to p} \sless 1$.
Finally, the upper bound for $\Vert \boldsymbol{F}[k] \Vert_{q \to p}$, determines the following last upper-bound:
\fi
\begin{equation*}
\label{eq:UB-invPhik2} 
\begin{aligned}
\forall k, q \geq p \geq 1, \qquad
\mynorm{\myparanthese{\myPhi^T \mybracket{k} \myPhi \mybracket{k}}^{-1}}_{q \to p} &\leq 
%\mynorm{\myparanthese{\boldsymbol{I}_{d_k} + \boldsymbol{F}\mybracket{k}}^{-1}}_{q \to p} \\
%&= \mynorm{\displaystyle\sum_{i=0}^\infty \myparanthese{-\boldsymbol{F}\mybracket{k}}^i}_{q \to p} \\
%&\leq \displaystyle\sum_{i=0}^\infty \mynorm{\boldsymbol{F}\mybracket{k}}_{q \to p}^i \\
\frac{1}{1-\mynorm{\boldsymbol{F}\mybracket{k}}_{q \to p}} \\
&\leq \frac{1}{1- d_k^{2 - \frac1q} \myparanthese{d_k - 1}^\frac12 M \myparanthese{\myPhi} }.
\end{aligned}
\end{equation*}
On the other hand, the previously obtained main equation (\ref{eq:temp1}) for new constraints $q \sgeq p \sgeq 1$ ($\max \{1 , d_k^{1/q \sm 1/p} \} \seq 1$) and $q' \seq p' \seq 1$ becomes:
\begin{gather}
\begin{aligned}
\label{eq:main-eq2} 
M_{q,p}\myparanthese{\myPhi} &\leq 
M\myparanthese{\myPhi} \displaystyle\max_{k,k' \neq k} &&\frac{d_{k}^{\frac12 - \frac1p} \, d_{k'}^{\frac1q + \frac12}}{d_{max}} \mynorm{\myparanthese{\myPhi^T \mybracket{k} \myPhi \mybracket{k}}^{-1}}_{q \to p} \times \\
&  &&\max \mybrace{1 , d_k^{\frac1p - \frac{1}{p'}}} \, \max \mybrace{1 , d_{k'}^{\frac{1}{q'} - \frac1q}} \, \max \mybrace{1 , d_k^{\frac{1}{p'} -\frac12}} \, \max \mybrace{1 , d_{k'}^{\frac{1}{2} - \frac{1}{q'}}} \, \max \mybrace{1 , d_k^{\frac1q - \frac1p}} \\
&=
M\myparanthese{\myPhi} \displaystyle\max_{k,k' \neq k} &&\frac{d_{k}^{\frac12 - \frac1p} \, d_{k'}^{\frac1q + \frac12}}{d_{max}} \mynorm{\myparanthese{\myPhi^T \mybracket{k} \myPhi \mybracket{k}}^{-1}}_{q \to p} \times \\
& &&\max \mybrace{1 , d_k^{\frac1p - 1}} \, \max \mybrace{1 , d_{k'}^{1 - \frac1q}} \, \max \mybrace{1 , d_k^{\frac12}} \, \max \mybrace{1 , d_{k'}^{-\frac12}} \\
&=
M\myparanthese{\myPhi} \displaystyle\max_{k,k' \neq k} &&\frac{d_{k}^{1 - \frac1p} \, d_{k'}^{\frac32}}{d_{max}} \mynorm{\myparanthese{\myPhi^T \mybracket{k} \myPhi \mybracket{k}}^{-1}}_{q \to p}.
\end{aligned}
\end{gather}
Substituting the upper-bound of $\Vert (\myPhi^T[k] \myPhi[k])^{-1} \Vert_{q \to p}$ obtained in equation (\ref{eq:UB-invPhik2}) into the updated main equation (\ref{eq:main-eq2}), we get:
%Combining with (\ref{eq:temp1}):
\begin{gather*} 
\begin{aligned}
q \geq p \geq 1, \qquad
M_{q,p}\myparanthese{\myPhi} &\leq 
\frac{M\myparanthese{\myPhi}}{d_{max}} \displaystyle\max_{k,k' \neq k} \frac{d_{k}^{1 - \frac1p} \, d_{k'}^{\frac32}}{1 - d_k^{2 - \frac1q} \myparanthese{d_k - 1}^\frac12 M \myparanthese{\myPhi}} \\
&\leq \frac{M\myparanthese{\myPhi}}{d_{max}} \displaystyle\max_{k} d_{k}^{1 - \frac1p} \, \displaystyle\max_{k' \neq k} d_{k'}^{\frac32} \displaystyle\max_{k} \frac{1}{1 - d_k^{2 - \frac1q} \myparanthese{d_k - 1}^\frac12 M \myparanthese{\myPhi}} \\
&= \frac{d_{max}^{\frac32 - \frac1p} \, M\myparanthese{\myPhi}}{1- d_{max}^{2 - \frac1q} \myparanthese{d_{max} - 1}^\frac12 M \myparanthese{\myPhi}}.
\end{aligned}
\end{gather*}
\myhl{2) Starting from the definition of coherence in Property {\ref{prp:IntraBlkO}} (block-MCC$_{q,p}$ for intra-block orthonormality) we have the following first line for $\forall (q , p) \ssin \mathbb{R}^2_{\sg 0}$.
Then considering the upper-bound of $\Vert \myPhi^T [k] \myPhi [k'] \Vert_{q {\to} p}$ for any $k$ and $k' {\neq} k$, obtained in ({\ref{eq:UB-kk'}}), 
%can be upper-bounded by $M(\myPhi) \, \boldsymbol{1}_{d_k \stimes d_{k'}}$, hence its $\ell_{q {\to} p}$ operator-norm (the following second line), and using the upper-bound of the third set of bounds in Property \ref{prp:OperatorProperties}, 
we will have the following second line $\forall (q , p , q' , p') \ssin \mathbb{R}^4_{\sg 0}$:}
%, in which by substituting $q' \seq p' \seq 1$, and Frobenius norm the following forth line is achieved.
\begin{gather*}
\label{eq:prp-BMIC-MIC}
\mycolor{\begin{aligned}
M_{q,p}\myparanthese{\myPhi} &= \max_{k,k' \neq k} \frac{d_{k}^{-\frac1p} \, d_{k'}^{\frac1q}}{d_{max}} \mynorm{\myPhi^T \mybracket{k} \myPhi \mybracket{k'}}_{q \to p} \\
%&\leq \max_{k,k' \neq k} \frac{d_{k}^{-\frac1p} \, d_{k'}^{\frac1q}}{d_{max}} M\myparanthese{\myPhi} \mynorm{\boldsymbol{1}_{d_k \stimes d_{k'}}}_{q \to p} \\
&\leq \frac{M\myparanthese{\myPhi}}{d_{max}} \, \max_{k,k' \neq k} d_{k}^{\frac12 - \frac1p} \, d_{k'}^{\frac1q + \frac12} \, \max \mybrace{1 , d_k^{\frac1p - \frac{1}{p'}}} \, \max \mybrace{1 , d_{k'}^{\frac{1}{q'} - \frac1q}} \, \max \mybrace{1 , d_k^{\frac{1}{p'} -\frac12}} \, \max \mybrace{1 , d_{k'}^{\frac{1}{2} - \frac{1}{q'}}}.  \\
%&= M\myparanthese{\myPhi} \max_{k,k' \neq k} \frac{d_{k}^{-\frac1p} \, d_{k'}^{\frac1q}}{d_{max}} \max \mybrace{1 , d_k^{\frac1p - 1}} \max \mybrace{1 , d_{k'}^{1 - \frac1q}} \max \mybrace{1 , d_k^{1 -\frac12}} \max \mybrace{1 , d_{k'}^{\frac{1}{2} - 1}} \myparanthese{d_k \, d_{k'}}^{\frac12} \\
%&= \frac{M\myparanthese{\myPhi}}{d_{max}} \max_{k,k' \neq k} d_{k}^{1 - \frac1p} \, d_{k'}^{\frac1q + \frac12} \, \max \mybrace{1 , d_{k'}^{1 - \frac1q}} \\
%&= \begin{cases}
%\frac{M\myparanthese{\myPhi}}{d_{max}} \displaystyle\max_{k,k' \neq k} d_{k}^{1 - \frac1p} \, d_{k'}^{\frac32}, & \qquad \text{for \ }q \geq 1\\
%\frac{M\myparanthese{\myPhi}}{d_{max}} \displaystyle\max_{k,k' \neq k} d_{k}^{1 - \frac1p} \, d_{k'}^{\frac1q + \frac12}, & \qquad \text{for \ }q < 1\\
%\end{cases} \\
%&\leq \begin{cases}
%d_{max}^{\frac32 - \frac1p} M\myparanthese{\myPhi}, & \qquad \text{for \ }q \geq 1\\
%d_{max}^{\frac1q - \frac1p + \frac12} M\myparanthese{\myPhi}, & \qquad \text{for \ }q < 1\\
%\end{cases}.
\end{aligned}}
\end{gather*}
\end{proof}
%------------------------------------------------------
\newpage
\section{Proof of Lemma \ref{lm:BUP-BS} (Block-UP based on Block-Spark, page \pageref{lm:BUP-BS})} % $\boldsymbol{\myBSpkTxt}$
\label{sec:BUP-BS} 
\begin{proof}
Here, we used the following triangle inequality:
\begin{equation*}
\mynorm{\mybetaz}_{p,0}+\mynorm{\mybetao}_{p,0} \geq \mynorm{\mybetaz-\mybetao}_{p,0}.
\end{equation*}
which is the the block-structured generalisation of one of the properties of the $\ell_0$ pseudo-norm operator for vector space.
%, defined in Lemma \ref{lm:Triangle-inequality}.
The above relation simply indicates that, if two representations have no overlapped supports, the sum of their number of active blocks are equal to the number of active blocks of their subtraction (equality), while in case of overlapped supports, the sum of their number of active blocks is greater than the number of active blocks of their subtraction (inequality).

On the other hand, we have $\myPhi(\mybetaz \sm \mybetao) \seq \boldsymbol{0}$. 
Therefore $(\mybetaz \sm \mybetao)$ is in the $\myKerTxt$ of the dictionary, and based on the definition of $\myBSpkTxt$ in Definition \ref{def:Block Spark} (page \pageref{def:Block Spark}), we have:
\begin{equation*}
\mynorm{\mybetaz-\mybetao}_{p,0} \geq \myBSpkMath,
\end{equation*}
which proves the lemma.
%Similarly, it can be proved for $\ell_{p,0}$ mixed-norm.
\end{proof}
\newpage
%------------------------------------------------------
\section{Proof of Theorem \ref{th:BNSP} (Block-NSP, page \pageref{th:BNSP})}
\label{prf:BNSP} 
\begin{proof}
Under the assumption of $Q_{\boldsymbol{w};p_1,p_2}(S_b(\mybeta),\myPhi) \sless 1/2$ and $S_b(\mybetaz) \ssubset S_b(\mybeta)$, to show that $\mybetaz$ is the unique solution to the $P_{\boldsymbol{w};p_1,p_2}$ minimisation problem, we need to prove that:
\begin{equation*}
\forall \boldsymbol{x} \in \myKerMath, \qquad \mynorm{\mybetaz}_{\boldsymbol{w};p_1,p_2}^{p_2} <
\mynorm{\mybetaz+\boldsymbol{x}}_{\boldsymbol{w};p_1,p_2}^{p_2},
\end{equation*}
where as defined in Definition \ref{def:Weighted mixed norm} (weighted (pseudo-)mixed-norm, page \pageref{def:Weighted mixed norm}), the $\ell_{p_1,p_2}^{\boldsymbol{w}}$ weighted (pseudo-)mixed-norm of a vector $\mybetaz$ is:
 \begin{equation*}
\mynorm{\mybetaz}_{\boldsymbol{w};p_1,p_2} \myeq 
\begin{cases}
    \displaystyle\sum_{k}^{} I \myparanthese{\frac{\mynorm{\mybetaz \mybracket{k}}_{p_1}}{d_k^{\frac{1}{p_1}}}}, & \qquad \text{for \ }p_2= 0\\
    \myparanthese{\displaystyle\sum_{k}^{} \frac{\mynorm{\mybetaz \mybracket{k}}_{p_1}^ {p_2}}{d_k^{\frac{p_2}{p_1}}}} ^ {\frac{1}{p_2}},  & \qquad  \text{for \ } 0 < p_2 < +\infty\\
    \displaystyle\max_{k}{\mybrace{\frac{\mynorm{\mybetaz \mybracket{k}}_{p_1}}{d_k^{\frac{1}{p_1}}}}}, & \qquad \text{for \ } p_2= \infty.
  \end{cases}
\end{equation*}
Then, dividing the whole blocks to on-$\myBSuppTxt$ ($\ssin S_b(\mybeta)$) and off-$\myBSuppTxt$ (${\notin} S_b(\mybeta)$), we have
\begin{gather*}
\begin{aligned}
 0 < &\displaystyle\sum_{k \in S_b\myparanthese{\mybeta}} \myparanthese{\myabs{\frac{\displaystyle\sum_{j=1}^{d_k} \myabs{\beta_{0_j}\mybracket{k} + x_{j}\mybracket{k}} ^{p_1}}{d_k}} ^{\frac{p_2}{p_1}} - 
\myabs{\frac{\displaystyle\sum_{j=1}^{d_k} \myabs{\beta_{0_j}\mybracket{k}} ^{p_1}}{d_k}} ^{\frac{p_2}{p_1}}} \\ 
&+ \displaystyle\sum_{k \notin S_b\myparanthese{\mybeta}} \myabs{\frac{\displaystyle\sum_{j=1}^{d_k} \myabs{x_{j}\mybracket{k}} ^{p_1}}{d_k}} ^{\frac{p_2}{p_1}}.
\end{aligned}
\end{gather*}
It can be rewritten in terms of the norm operator, so
\begin{gather*}
 0<\displaystyle\sum_{k \in S_b\myparanthese{\mybeta}} w_k^{p_2} \myparanthese{\mynorm{\mybetaz\mybracket{k}+\boldsymbol{x}\mybracket{k}}_{p_1}^{p_2} - 
\mynorm{\mybetaz\mybracket{k}}_{p_1}^{p_2}} + 
\displaystyle\sum_{k \notin S_b\myparanthese{\mybeta}} w_k^{p_2} \mynorm{\boldsymbol{x}\mybracket{k}}_{p_1}^{p_2}.
\end{gather*}
where, in a block-structured vector $\boldsymbol{a} \seq [\boldsymbol{a}^T[1], \cdots,\boldsymbol{a}^T[K]]^T$ with length vector $\boldsymbol{d} \seq [d_1, \cdots, d_K ]$, $w_k$ in $w_k \Vert \boldsymbol{a}[k] \Vert_p$ is equal to $d_k^{-1/p}$.

From quasi-triangle inequality for scalars, we generalised it to vector space and derived the corresponding triangle inequality $\mynorm{\boldsymbol{a} \spl \boldsymbol{b}}_{p_1}^{p_2} \sm \mynorm{\boldsymbol{a}}_{p_1}^{p_2} \sgeq \sm\mynorm{\boldsymbol{b}}_{p_1}^{p_2}$ for $0 \sleq p_2 \sleq 1 \sleq p_1$.
% (Lemma \ref{lm:Quasi-triangle}). 
Therefore, it is sufficient to prove:
\begin{equation*}
 0 < \displaystyle\sum_{k \notin S_b\myparanthese{\mybeta}} w_k^{p_2} \mynorm{\boldsymbol{x}\mybracket{k}}_{p_1}^{p_2} - \displaystyle\sum_{k \in S_b\myparanthese{\mybeta}} w_k^{p_2} \mynorm{\boldsymbol{x}\mybracket{k}}_{p_1}^{p_2}.
\end{equation*}
Adding $2 \, \sum_{k \in S_b(\mybeta)} w_k^{p_2} \Vert \boldsymbol{x}[k] \Vert_{p_1}^{p_2}$ to both sides, we have
\begin{gather*}
\begin{aligned}
2 \, \displaystyle\sum_{k \in S_b\myparanthese{\mybeta}} w_k^{p_2} \mynorm{\boldsymbol{x}\mybracket{k}}_{p_1}^{p_2} &< 
\displaystyle\sum_{k \notin S_b\myparanthese{\mybeta}} w_k^{p_2} \mynorm{\boldsymbol{x}\mybracket{k}}_{p_1}^{p_2} + \displaystyle\sum_{k \in S_b\myparanthese{\mybeta}} w_k^{p_2} \mynorm{\boldsymbol{x}\mybracket{k}}_{p_1}^{p_2} \\
&= \displaystyle\sum_{k} w_k^{p_2} \mynorm{\boldsymbol{x}\mybracket{k}}_{p_1}^{p_2},
\end{aligned}
\end{gather*}
or equivalently,
\begin{equation*}
\frac{\displaystyle\sum_{k \in S_b\myparanthese{\mybeta}} w_k^{p_2} \mynorm{\boldsymbol{x}\mybracket{k}}_{p_1}^{p_2}}{\displaystyle\sum_{k} w_k^{p_2} \mynorm{\boldsymbol{x}\mybracket{k}}_{p_1}^{p_2}} < \frac12.
\end{equation*}
But the above left-hand side expression is exactly $Q_{\boldsymbol{w};p_1,p_2}(S_b(\mybeta),\myPhi)$ and the inequality is the initial assumption of the proof, i.e., $Q_{\boldsymbol{w};p_1,p_2}(S_b(\mybeta),\myPhi) \sless 1/2$.



Similarly, for the case of equally-sized blocks, i.e., $d_1 \seq \cdots \seq d_K \seq d$, under the assumption of $Q_{p_1,p_2}(S_b(\mybeta),\myPhi) \sless 1/2$ and $S_b(\mybetaz) \ssubset S_b(\mybeta)$, to show that $\mybetaz$ is the unique solution to the $P_{p_1,p_2}$ minimisation problem, we need to prove that:
\begin{equation*}
\forall \boldsymbol{x} \in \myKerMath, \qquad \mynorm{\mybetaz}_{p_1,p_2}^{p_2} <
\mynorm{\mybetaz+\boldsymbol{x}}_{p_1,p_2}^{p_2},
\end{equation*}
where as defined on page \pageref{eq:MixedNormDef}, the $\ell_{p_1,p_2}$ (pseudo-)mixed-norm of a vector $\mybetaz$ is:
\begin{equation*}
\mynorm{\mybetaz}_{p_1,p_2} \myeq 
\begin{cases}
    \displaystyle\sum_{k}^{} I \myparanthese{\mynorm{\mybetaz \mybracket{k}}_{p_1}}, & \qquad \text{for \ }p_2= 0\\
    \myparanthese{\displaystyle\sum_{k}^{} \mynorm{\mybetaz \mybracket{k}}_{p_1}^ {p_2}} ^ {\frac{1}{p_2}},  & \qquad  \text{for \ } 0 < p_2 < +\infty\\
    \displaystyle\max_{k}{\mybrace{\mynorm{\mybetaz \mybracket{k}}_{p_1}}}, & \qquad \text{for \ } p_2= \infty,
  \end{cases}
\end{equation*}
Then, the above-mentioned inequality is equivalent to showing 
\begin{equation*}
\displaystyle\sum_{k=1}^{K} \myabs{\displaystyle\sum_{j=1}^{d_k} \myabs{\beta_{0_j}\mybracket{k}} ^{p_1}} ^{\frac{p_2}{p_1}} 
< \displaystyle\sum_{k=1}^{K} \myabs{\displaystyle\sum_{j=1}^{d_k} \myabs{\beta_{0_j}\mybracket{k} + x_{j}\mybracket{k}} ^{p_1}} ^{\frac{p_2}{p_1}}.
\end{equation*}
Then, dividing the whole blocks to on-$\myBSuppTxt$ ($\ssin S_b(\mybeta)$) and off-$\myBSuppTxt$ (${\notin} S_b(\mybeta)$), we have
\begin{gather*}
\begin{aligned}
 0 < &\displaystyle\sum_{k \in S_b\myparanthese{\mybeta}} \myparanthese{\myabs{\displaystyle\sum_{j=1}^{d_k} \myabs{\beta_{0_j}\mybracket{k} + x_{j}\mybracket{k}} ^{p_1}} ^{\frac{p_2}{p_1}} - 
\myabs{\displaystyle\sum_{j=1}^{d_k} \myabs{\beta_{0_j}\mybracket{k}} ^{p_1}} ^{\frac{p_2}{p_1}}} \\ 
&+ \displaystyle\sum_{k \notin S_b\myparanthese{\mybeta}} \myabs{\displaystyle\sum_{j=1}^{d_k} \myabs{x_{j}\mybracket{k}} ^{p_1}} ^{\frac{p_2}{p_1}}.
\end{aligned}
\end{gather*}
It can be rewritten in terms of the norm operator, so
\begin{gather*}
 0<\displaystyle\sum_{k \in S_b\myparanthese{\mybeta}} \mynorm{\mybetaz\mybracket{k}+\boldsymbol{x}\mybracket{k}}_{p_1}^{p_2} - 
\mynorm{\mybetaz\mybracket{k}}_{p_1}^{p_2} + 
\displaystyle\sum_{k \notin S_b\myparanthese{\mybeta}} \mynorm{\boldsymbol{x}\mybracket{k}}_{p_1}^{p_2}.
\end{gather*}
Utilising $\mynorm{\boldsymbol{a} \spl \boldsymbol{b}}_{p_1}^{p_2} \sm \mynorm{\boldsymbol{a}}_{p_1}^{p_2} \sgeq \sm\mynorm{\boldsymbol{b}}_{p_1}^{p_2}$ for $0 \sleq p_2 \sleq 1 \sleq p_1$, it is sufficient to prove:
\begin{equation*}
 0 < \displaystyle\sum_{k \notin S_b\myparanthese{\mybeta}} \mynorm{\boldsymbol{x}\mybracket{k}}_{p_1}^{p_2} - \displaystyle\sum_{k \in S_b\myparanthese{\mybeta}} \mynorm{\boldsymbol{x}\mybracket{k}}_{p_1}^{p_2}.
\end{equation*}
Adding $2 \, \sum_{k \in S_b(\mybeta)} \Vert \boldsymbol{x}[k] \Vert_{p_1}^{p_2}$ to both sides, we have
\begin{gather*}
\begin{aligned}
2 \, \displaystyle\sum_{k \in S_b\myparanthese{\mybeta}} \mynorm{\boldsymbol{x}\mybracket{k}}_{p_1}^{p_2} &< 
\displaystyle\sum_{k \notin S_b\myparanthese{\mybeta}} \mynorm{\boldsymbol{x}\mybracket{k}}_{p_1}^{p_2} + \displaystyle\sum_{k \in S_b\myparanthese{\mybeta}} \mynorm{\boldsymbol{x}\mybracket{k}}_{p_1}^{p_2} \\
&= \displaystyle\sum_{k} \mynorm{\boldsymbol{x}\mybracket{k}}_{p_1}^{p_2},
\end{aligned}
\end{gather*}
or equivalently,
\begin{equation*}
\frac{\displaystyle\sum_{k \in S_b\myparanthese{\mybeta}} \mynorm{\boldsymbol{x}\mybracket{k}}_{p_1}^{p_2}}{\displaystyle\sum_{k} \mynorm{\boldsymbol{x}\mybracket{k}}_{p_1}^{p_2}} < \frac12.
\end{equation*}
But the above left-hand side expression is exactly $Q_{p_1,p_2}(S_b(\mybeta),\myPhi)$ and the inequality is the initial assumption of the proof, i.e., $Q_{p_1,p_2}(S_b(\mybeta),\myPhi) \sless 1/2$.
\end{proof}
\newpage
%------------------------------------------------------
\section{Proof of Lemma \ref{lm:BBUP} (Basic Block-UP, page \pageref{lm:BBUP})}
\label{prf:BBUP} 
\begin{proof}
The following proof generalises the corresponding proofs in \cite{Elad2001,Elad2002a,Eldar2009b,Eldar2010b} and \cite{Eldar2010}.
Without loss of generality, assume that the non-zero signal $\boldsymbol{y}$ is normalized to have unit squared Euclidean norm, i.e., $\Vert\boldsymbol{y}\Vert_2^2 \seq \boldsymbol{y}^T \boldsymbol{y} \seq 1$, \myhl{and suppose that $p$ is H{\"o}lder conjugate to $p'$, i.e., $1/p \spl 1/p' \seq 1$ {\cite{Golub2013}}.}
Then,
\begin{equation}
\label{eq:DontKnow3} 
\begin{aligned}
\forall (p , p') \in \mathbb{R}^2_{\geq 1}, \forall q \in \mathbb{R}_{> 0}, \forall r \in \mathbb{R}_{\geq 0}, \qquad
1 &= \displaystyle\sum_{k,k'=1}^K \mybetao^T \mybracket{k} \myPhiOne^T\mybracket{k} \myPhiTwo \mybracket{k'} \mybetaTwo \mybracket{k'} \\
&\leq \displaystyle\sum_{k,k'=1}^K \mynorm{\mybetao \mybracket{k}}_p \mynorm{\myPhiOne^T \mybracket{k} \myPhiTwo \mybracket{k'} \mybetaTwo \mybracket{k'}}_{p'} \\
&\leq d_{max} \overbar{M}_{q,p'}\myparanthese{\myPhiOne, \myPhiTwo} \displaystyle\sum_{k,k'=1}^K d_k^{\frac{1}{p'}} \mynorm{\mybetao \mybracket{k}}_p \, d_{k'}^{-\frac{1}{q}}\mynorm{\mybetaTwo \mybracket{k'}}_q \\
&= d_{max} \overbar{M}_{q,p'}\myparanthese{\myPhiOne , \myPhiTwo} \displaystyle\sum_{k=1}^K d_k^{\frac{1}{p'}} \mynorm{\mybetao \mybracket{k}}_p \displaystyle\sum_{k'=1}^K d_{k'}^{-\frac{1}{q}}\mynorm{\mybetaTwo \mybracket{k'}}_q \\
&= d_{max} \overbar{M}_{q,p'}\myparanthese{\myPhiOne , \myPhiTwo} \displaystyle\sum_{k=1}^{\mynorm{\mybetao}_{r,0}} d_k^{\frac{1}{p'}} \mynorm{\mybetao \mybracket{k}}_p \displaystyle\sum_{k'=1}^{\mynorm{\mybetaTwo}_{r,0}} d_{k'}^{-\frac{1}{q}} \mynorm{\mybetaTwo \mybracket{k'}}_q.
\end{aligned}
\end{equation}
%using Remark \ref{Rmrk:Holder-variant}, the above first inequality is deduced.
For vectors $\boldsymbol{a}$ and $\boldsymbol{b}$ we have $\boldsymbol{a}^T \boldsymbol{b} \sleq \sum_i \vert a_i b_i \vert$, and using H{\"o}lder's inequality \cite{Golub2013}, i.e., $\sum_i \myabs{a_i b_i} \sleq \Vert \boldsymbol{a} \Vert_p \Vert \boldsymbol{b} \Vert_{p'}$, where, $\forall (p , p') \ssin \mathbb{R}^2_{\sgeq 1} : 1/p \spl 1/p' \seq 1$, we have $\boldsymbol{a}^T \boldsymbol{b} \sleq \Vert\boldsymbol{a} \Vert_p \Vert \boldsymbol{b} \Vert_{p'}$, which results the above first inequality.
Considering the definition of basic Block-MCC$_{q,p}$, i.e., $\overbar{M}_{q,p'}(\myPhiOne , \myPhiTwo) \seq  
\max_{k,k'} d_{k}^{-1/{p'}} \, d_{k'}^{1/q}/d_{max} \Vert \myPhiOne^T [k] \myPhiTwo^{ } [k'] \Vert_{q \to p'}$, where $p' \seq p / (p \sm 1)$, we have $\Vert \myPhiOne^T [k] \myPhiTwo^{ } [k'] \boldsymbol{x} \Vert_{p'} \sleq \overbar{M}_{q,p'} (\myPhiOne, \myPhiTwo) d_{max} d_k^{1/p'} d_{k'}^{-1/q} \Vert \boldsymbol{x} \Vert_q$, which produces the above third line with second inequality.
The above last equality follows from summation only over non-zero blocks for any $r \sgeq 0$.

Before finding the upper-bound for the above last inequality (\ref{eq:DontKnow3}), it should be taken into account that according to the Parseval's theorem, we have
$\Vert \boldsymbol{y} \Vert_2^2 \seq \Vert\mybetao \Vert_2^2 \seq \Vert \mybetaTwo \Vert_2^2 \seq 1$, which will be used in the following optimisation problem.
Therefore, in order to upper-bound the inequality (\ref{eq:DontKnow3}), it is sufficient to solve the following optimisation problem:
\iffalse
\begin{equation*}
\begin{aligned}
\max_{k,k'} &\displaystyle\sum_{k=1}^{\mynorm{\mybetao}_{r,0}} d_k^{\frac{1}{p'}} \myparanthese{\displaystyle\sum_{j=1}^{d_k} \beta_{1_j}^p \mybracket{k}}^\frac1p
\displaystyle\sum_{k'=1}^{\mynorm{\mybetaTwo}_{r,0}} d_{k'}^{-\frac{1}{q}} \myparanthese{\displaystyle\sum_{j=1}^{d_{k'}} \beta_{2_j}^q \mybracket{k'}}^\frac1q \\
& s.t. \quad
\begin{aligned}
& \displaystyle\sum_{k=1}^{\mynorm{\mybetao}_{r,0}} \displaystyle\sum_{j=1}^{d_k} \beta_{1_j}^2 \mybracket{k} = 
\displaystyle\sum_{k'=1}^{\mynorm{\mybetaTwo}_{r,0}} \displaystyle\sum_{j=1}^{d_{k'}} \beta_{2_j}^2 \mybracket{k'} = 1 \\
&and \quad \beta_{1_j} \mybracket{k},\beta_{2_j}\mybracket{k'} > 0.
\end{aligned}
\end{aligned}
\end{equation*}
\fi
\begin{equation}
\label{eq:main-opt} 
\begin{aligned}
\forall (q , p , p') \in \mathbb{R}^3_{\geq 1}, \forall r \in \mathbb{R}_{\geq 0}, \qquad
\max_{k,k'} &\displaystyle\sum_{k=1}^{\mynorm{\mybetao}_{r,0}} d_k^{\frac{1}{p'}} \myparanthese{\displaystyle\sum_{j=1}^{d_k} \mycolor{\myabs{\beta_{1_j}\mybracket{k}}} ^p }^\frac1p \,
\displaystyle\sum_{k'=1}^{\mynorm{\mybetaTwo}_{r,0}} d_{k'}^{-\frac{1}{q}} \myparanthese{\displaystyle\sum_{j=1}^{d_{k'}} \mycolor{\myabs{\beta_{2_j} \mybracket{k'}}} ^q }^\frac1q \\
& s.t. \quad
\begin{aligned}
& \displaystyle\sum_{k=1}^{\mynorm{\mybetao}_{r,0}} \displaystyle\sum_{j=1}^{d_k} \beta_{1_j}^2 \mybracket{k} = 
\displaystyle\sum_{k'=1}^{\mynorm{\mybetaTwo}_{r,0}} \displaystyle\sum_{j=1}^{d_{k'}} \beta_{2_j}^2 \mybracket{k'} = 1.
%&and \quad \beta_{1_j} \mybracket{k},\beta_{2_j}\mybracket{k'} > 0.
\end{aligned}
\end{aligned}
\end{equation}
Although the equation (\ref{eq:DontKnow3}) holds true for $\forall q \ssin \mathbb{R}_{\sg 0}$, we are going to upper-bound the (\ref{eq:DontKnow3}) for $\forall q \ssin \mathbb{R}_{\sgeq 1}$, as it is mentioned in (\ref{eq:main-opt}).
\myhl{On the other hand, since during the procedure of finding the upper-bound, the derivative of the $q$ and $p$ norms of a vector should be computed, we solve ({\ref{eq:main-opt}}) in two parts: 1) $\forall (q , p) \ssin \mathbb{R}^2_{\sg 1}$ (differentiable $q$ and $p$ norms), and 2) $q \seq p \seq 1$ (following the related proof in {\cite{Elad2002a}}, page 3).}

Above optimisation problem is separable, so we need to maximise $\forall (q , p) \ssin \mathbb{R}^2_{\sg 1}$, $\sum_{k=1}^{\Vert\mybetao \Vert_{r,0}} d_k^{1/p'}(\sum_{j=1}^{d_k} \mycolor{\vert \beta_{1_j} [k] \vert} ^p)^{1/p}$ subject to $\sum_{k=1}^{\Vert\mybetao \Vert_{r,0}} \sum_{j=1}^{d_k} \beta_{1_j}^2 [k] \seq 1$, 
%$\beta_{1_j} [k] \sg 0$ 
and $\sum_{k'=1}^{\Vert\mybetaTwo \Vert_{r,0}} d_{k'}^{-1/q} (\sum_{j=1}^{d_{k'}} \mycolor{\vert \beta_{2_j}[k'] \vert} ^q )^{1/q}$ subject to $\sum_{k'=1}^{\Vert\mybetaTwo \Vert_{r,0}} \sum_{j=1}^{d_{k'}} \beta_{2_j}^2 [k'] \seq 1$, 
%$\beta_{2_j} [k'] \sg 0$, 
separately.
In order to solve two mentioned problems, first we need to form the Lagrangian function.
Then, for the first problem we have:
\iffalse
\begin{gather*}
\mathcal{L} \myparanthese{\mybetao , \lambda} =
\displaystyle\sum_{k=1}^{\mynorm{\mybetao}_{r,0}} d_k^{\frac{1}{p'}} \myparanthese{\displaystyle\sum_{j=1}^{d_k} \beta_{1_j}^p \mybracket{k}}^\frac1p + \lambda \myparanthese{1 - \displaystyle\sum_{k=1}^{\mynorm{\mybetao}_{r,0}} \displaystyle\sum_{j=1}^{d_k} \beta_{1_j}^2 \mybracket{k}},
\end{gather*}
\fi
\begin{gather*}
\mathcal{L} \myparanthese{\mybetao , \lambda} =
\displaystyle\sum_{k=1}^{\mynorm{\mybetao}_{r,0}} d_k^{\frac{1}{p'}} \myparanthese{\displaystyle\sum_{j=1}^{d_k} \mycolor{\myabs{\beta_{1_j} \mybracket{k}}} ^p}^\frac1p + \lambda \myparanthese{1 - \displaystyle\sum_{k=1}^{\mynorm{\mybetao}_{r,0}} \displaystyle\sum_{j=1}^{d_k} \beta_{1_j}^2 \mybracket{k}},
\end{gather*}
then we need to compute its critical point.
Considering that $(|f|)' \seq f' \, f / |f| $, where $f'$ is derivative of $f$ with respect to $x$, i.e. $d \, f(x) / d \, x$, we have:
\begin{gather*}
\begin{aligned}
& \frac{\partial \mathcal{L}}{\partial \beta_{1_j} \mybracket{k}} = 
d_k^{\frac{1}{p'}} \mycolor{\beta_{1_j} \mybracket{k} \myabs{\beta_{1_j} \mybracket{k}}^{p-2}} \myparanthese{\displaystyle\sum_{j=1}^{d_k} \mycolor{\myabs{\beta_{1_j} \mybracket{k}}} ^p}^{\frac1p - 1} - 
2 \lambda \beta_{1_j} \mybracket{k} = 0 \\
&\Rightarrow \mycolor{\myabs{\beta_{1_j} \mybracket{k}} \in
\mybrace{0 , \myparanthese{\frac{2 \lambda}{d_k^{\frac{1}{p'}}\mynorm{\mybetao \mybracket{k}}_p^{1-p}}}^{\frac{1}{p-2}}}.}
\end{aligned}
\end{gather*}
From the above last equality it can be derived that all the absolute value of the coefficients in a block $k$ has the same value. 
On the other hand, all the identical elements cannot be zero, because it leads to $\Vert \mybetao [k]\Vert_{r,0} \seq 0 $, but from the last line of (\ref{eq:DontKnow3}), only non-zero blocks are selected for optimisation.
Therefore, it reduces to:
\begin{equation*}
\mycolor{\myabs{\beta_{1_j} \mybracket{k}}} =
\frac{1}{2 \lambda}.
\end{equation*}
Next, applying the unit-energy constraint of the coefficients, we get:
\begin{equation*}
\begin{aligned}
&1 = \displaystyle\sum_{k=1}^{\mynorm{\mybetao}_{r,0}} \displaystyle\sum_{j=1}^{d_k} \beta_{1_j}^2 \mybracket{k} =
\frac{d_k \mynorm{\mybetao}_{r,0}}{4 \lambda^2} \\
& \Rightarrow \lambda = \frac{d_k^{\frac12} \mynorm{\mybetao}^{\frac12}_{r,0}}{2} 
\Rightarrow \mycolor{\myabs{\beta_{1_j} \mybracket{k}}} = \myparanthese{d_k \mynorm{\mybetao}_{r,0}}^ \frac{-1}{2}.
\end{aligned}
\end{equation*}
%It can be seen that, although the constraint of $\beta_{1_j} [k] \sg 0$ is not explicitly included in the Lagrangian function, but from the final closed form solution this constraint can be verified.
Therefore, $\sum_{k=1}^{\Vert\mybetao \Vert_{r,0}} d_k^{1/p'} (\sum_{j=1}^{d_k} \mycolor{\vert \beta_{1_j} [k] \vert} ^p)^{1/p}$ is upper-bounded by $\Vert\mybetao \Vert_{r,0}^{-1/2} \sum_{k=1}^{\Vert\mybetao \Vert_{r,0}} d_k^{1/2}$, which is again upper-bounded by $\Vert\mybetao \Vert_{r,0}^{1/2} d_{max}^{1/2}$.
Similarly, for the second problem it can be proved that $\sum_{k'=1}^{\Vert\mybetaTwo \Vert_{r,0}} d_{k'}^{-1/q} (\sum_{j=1}^{d_{k'}} \mycolor{\vert \beta_{2_j} [k'] \vert} ^q)^{1/q}$ is upper-bounded by $\Vert\mybetaTwo \Vert_{r,0}^{-1/2} \sum_{k'=1}^{\Vert\mybetaTwo \Vert_{r,0}} d_{k'}^{-1/2}$, which is upper-bounded by $\Vert\mybetaTwo \Vert_{r,0}^{1/2} d_{\mycolor{min}}^{-1/2}$.
Substituting the recently mentioned upper-bounds into (\ref{eq:DontKnow3}), we get:
\begin{equation*}
\begin{aligned}
\forall (q , p) \in \mathbb{R}^2_{> 1}, \forall r \in \mathbb{R}_{\geq 0}, \qquad
1 &\leq 
\mycolor{d_{min}^{-\frac12} d_{max}^{\frac32}}
%\frac{d^{\frac1q + \frac{1}{p}}}{d^{\frac1q - \frac{1}{p'}}} 
\overbar{M}_{q,p'}\myparanthese{\myPhiOne , \myPhiTwo} \myparanthese{\mynorm{\mybetao}_{r,0} \mynorm{\mybetaTwo}_{r,0}}^{\frac12} \\
&\mycolor{\leq
d_{min}^{-\frac12} d_{max}^{\frac32} \overbar{M}_{q,p'}\myparanthese{\myPhiOne , \myPhiTwo} \frac{\mynorm{\mybetao}_{r,0} + \mynorm{\mybetaTwo}_{r,0}}{2}.}
\end{aligned}
\end{equation*}

The proof is completed in the above equation by replacing $p'$ by $p/(p \sm 1)$, according to the condition of the H{\"o}lder's inequality \cite{Golub2013}, and using the inequality of arithmetic-geometric means, i.e., $\sqrt{ab} \sleq (a \spl b)/2$.

\myhl{Now, consider the case where $q \seq p \seq 1$. 
Then, the optimisation problem in ({\ref{eq:main-opt}}) becomes:}
\begin{equation*}
\label{eq:main-opt-qp11} 
\mycolor{\begin{aligned}
\forall r \in \mathbb{R}_{\geq 0}, \qquad
\max_{k,k'}
&\displaystyle\sum_{k=1}^{\mynorm{\mybetao}_{r,0}} \displaystyle\sum_{j=1}^{d_k} \myabs{\beta_{1_j}\mybracket{k}}  \,
\displaystyle\sum_{k'=1}^{\mynorm{\mybetaTwo}_{r,0}} d_{k'}^{-1} \displaystyle\sum_{j=1}^{d_{k'}} \myabs{\beta_{2_j} \mybracket{k'} }  \\
&s.t. \quad
%\begin{aligned}
\displaystyle\sum_{k=1}^{\mynorm{\mybetao}_{r,0}} \displaystyle\sum_{j=1}^{d_k} \beta_{1_j}^2 \mybracket{k} = 
\displaystyle\sum_{k'=1}^{\mynorm{\mybetaTwo}_{r,0}} \displaystyle\sum_{j=1}^{d_{k'}} \beta_{2_j}^2 \mybracket{k'} = 1,
%&and \quad \beta_{1_j} \mybracket{k},\beta_{2_j}\mybracket{k'} > 0.
%\end{aligned}
\end{aligned}}
\end{equation*}
\myhl{which is equivalent to}
\begin{equation*}
\label{eq:main-opt-qp11} 
\mycolor{\begin{aligned}
\forall r \in \mathbb{R}_{\geq 0}, \qquad
\max_{k,k'}
&\displaystyle\sum_{k=1}^{\mynorm{\mybetao}_{r,0}} \displaystyle\sum_{j=1}^{d_k} \beta_{1_j}\mybracket{k}  \,
\displaystyle\sum_{k'=1}^{\mynorm{\mybetaTwo}_{r,0}} d_{k'}^{-1} \displaystyle\sum_{j=1}^{d_{k'}} \beta_{2_j} \mybracket{k'}  \\
&s.t. \quad
\begin{aligned}
& \displaystyle\sum_{k=1}^{\mynorm{\mybetao}_{r,0}} \displaystyle\sum_{j=1}^{d_k} \beta_{1_j}^2 \mybracket{k} = 
\displaystyle\sum_{k'=1}^{\mynorm{\mybetaTwo}_{r,0}} \displaystyle\sum_{j=1}^{d_{k'}} \beta_{2_j}^2 \mybracket{k'} = 1 \\
&and \quad \beta_{1_j} \mybracket{k},\beta_{2_j}\mybracket{k'} > 0.
\end{aligned}
\end{aligned}}
\end{equation*}
\myhl{The above optimisation problem is separable, so we need to maximise $\sum_{k=1}^{\Vert\mybetao \Vert_{r,0}} \sum_{j=1}^{d_k} \beta_{1_j} [k]$ subject to $\sum_{k=1}^{\Vert\mybetao \Vert_{r,0}} \sum_{j=1}^{d_k} \beta_{1_j}^2 [k] \seq 1$, $\beta_{1_j} [k] \sg 0$ 
and $\sum_{k'=1}^{\Vert\mybetaTwo \Vert_{r,0}} d_{k'}^{-1} \sum_{j=1}^{d_{k'}} \beta_{2_j}[k']$ subject to $\sum_{k'=1}^{\Vert\mybetaTwo \Vert_{r,0}} \sum_{j=1}^{d_{k'}} \beta_{2_j}^2 [k'] \seq 1$, $\beta_{2_j} [k'] \sg 0$, separately.
But, in order to solve these problems (following the related proof in {\cite{Elad2002a}}, page 3), let us consider the following Lagrangian function, in which the positivity constraint is not enforced explicitly:}
\begin{gather*}
\mycolor{\begin{aligned}
&\mathcal{L} \myparanthese{\mybetao , \lambda} =
\displaystyle\sum_{k=1}^{\mynorm{\mybetao}_{r,0}} \displaystyle\sum_{j=1}^{d_k} \beta_{1_j} \mybracket{k} + \lambda \myparanthese{1 - \displaystyle\sum_{k=1}^{\mynorm{\mybetao}_{r,0}} \displaystyle\sum_{j=1}^{d_k} \beta_{1_j}^2 \mybracket{k}} \\
& \frac{\partial \mathcal{L}}{\partial \beta_{1_j} \mybracket{k}} = 
1 - 
2 \lambda \beta_{1_j} \mybracket{k} = 0 
\Rightarrow \beta_{1_j} \mybracket{k} = \frac{1}{2\lambda}.
\end{aligned}}
\end{gather*}
\myhl{Next, applying the unit-energy constraint of the coefficients, we get:}
\begin{equation*}
\mycolor{\begin{aligned}
&1 = \displaystyle\sum_{k=1}^{\mynorm{\mybetao}_{r,0}} \displaystyle\sum_{j=1}^{d_k} \beta_{1_j}^2 \mybracket{k} =
\frac{d_k \mynorm{\mybetao}_{r,0}}{4 \lambda^2} \\
& \Rightarrow \lambda = \frac{d_k^{\frac12} \mynorm{\mybetao}^{\frac12}_{r,0}}{2} 
\Rightarrow \beta_{1_j} \mybracket{k} = \myparanthese{d_k \mynorm{\mybetao}_{r,0}}^ \frac{-1}{2}.
\end{aligned}}
\end{equation*}
\myhl{It can be seen that, although the constraint of $\beta_{1_j} [k] \sg 0$ is not explicitly included in the Lagrangian function, but from the final closed form solution this constraint can be verified.
Therefore, $\sum_{k=1}^{\Vert\mybetao \Vert_{r,0}} \sum_{j=1}^{d_k} \beta_{1_j} [k]$ is upper-bounded by $\Vert\mybetao \Vert_{r,0}^{-1/2} \sum_{k=1}^{\Vert\mybetao \Vert_{r,0}} d_k^{1/2}$, which is again upper-bounded by $\Vert\mybetao \Vert_{r,0}^{1/2} d_{max}^{1/2}$.
Similarly, for the second problem it can be proved that $\sum_{k'=1}^{\Vert\mybetaTwo \Vert_{r,0}} d_{k'}^{-1} \sum_{j=1}^{d_{k'}} \beta_{2_j}[k']$ is upper-bounded by $\Vert\mybetaTwo \Vert_{r,0}^{-1/2} \sum_{k'=1}^{\Vert\mybetaTwo \Vert_{r,0}} d_{k'}^{-1/2}$, which is again upper-bounded by $\Vert\mybetaTwo \Vert_{r,0}^{1/2} d_{min}^{-1/2}$.
Substituting the recently mentioned upper-bounds into ({\ref{eq:DontKnow3}}), we get the same results as for the previous case of $\forall (q , p) \ssin \mathbb{R}^2_{> 1}$.
Hence, the Lemma holds true for $\forall (q , p) \ssin \mathbb{R}^2_{\sgeq 1}$.}
\end{proof}
\newpage
%------------------------------------------------------
\section{Proof of Property \ref{prp:BMIC-LB} (Basic Block-MCC$_{2,2}$ lower-bound, page \pageref{prp:BMIC-LB})}
\label{prf:BMIC-LB} 
\begin{proof}
From Property \ref{prp:IntraBlkO} (Block-MCC$_{q,p}$ for intra-block orthonormality, page \pageref{prp:IntraBlkO}), in the special case of two matrices and parameter changing of $p {\to} p/(p \sm 1)$, we have:
\begin{equation*}
\forall \myparanthese{q , p} \in \mathbb{R}^2_{>0}, \qquad
\overbar{M}_{q,\frac{p}{p-1}}\myparanthese{\myPhiOne , \myPhiTwo} = 
\max_{k,k'} \frac{d_k^{-\frac{p-1}{p}} d_{k'}^{\frac1q}}{d_{max}} \mynorm{\myPhiOne ^T \mybracket{k} \myPhiTwo^{ }\mybracket{k'}}_{q \to \frac{p}{p-1}}.
\end{equation*}
Then, squaring both sides and for $q \seq p \seq 2$, we get:
\begin{gather*}
\begin{aligned}
\overbar{M}^2_{2,2}\myparanthese{\myPhiOne , \myPhiTwo} &= 
\max_{k,k'} \frac{d_k^{-1} d_{k'}}{d^2_{max}}\mynorm{\myPhiOne^T \mybracket{k} \myPhiTwo^{ }\mybracket{k'}}^2_{2 \to 2}\\
&= \max_{k,k'} \frac{d_k^{-1} d_{k'}}{d^2_{max}}\mynorm{\myPhiTwo^T \mybracket{k'} \myPhiOne^{ } \mybracket{k} \myPhiOne^T \mybracket{k} \myPhiTwo^{ }\mybracket{k'}}_{2 \to 2}.
\end{aligned}
\end{gather*}
The above second equality follows from a property of operator-norms, 
%defined in Section \ref{sec:operator-norm}, 
i.e., $\Vert \boldsymbol{A} \Vert^2 _{2 \to 2} \seq \Vert \boldsymbol{A}^T \boldsymbol{A} \Vert _{2 \to 2}$.
Summing over $k$ and $k'$, we have:
\begin{gather*}
\begin{aligned}
K^2 \overbar{M}^2_{2,2}\myparanthese{\myPhiOne , \myPhiTwo} &\geq
\displaystyle\sum_{k=1}^K \displaystyle\sum_{k'=1}^K \frac{d_k^{-1} d_{k'}}{d^2_{max}} \mynorm{\myPhiTwo^T \mybracket{k'} \myPhiOne^{ } \mybracket{k} \myPhiOne^T \mybracket{k} \myPhiTwo^{ }\mybracket{k'}}_{2 \to 2} \\
&\geq \frac{1}{d^2_{max}} \mynorm{\displaystyle\sum_{k'=1}^K d_{k'} \myPhiTwo ^T \mybracket{k'}\myparanthese{\displaystyle\sum_{k=1}^K d_k^{-1} \myPhiOne^{ } \mybracket{k} \myPhiOne^T \mybracket{k}} \myPhiTwo^{ }\mybracket{k'}}_{2 \to 2}\\
&\mycolor{\geq \frac{d_{min}}{d^{3}_{max}} \mynorm{\displaystyle\sum_{k'=1}^K \myPhiTwo ^T \mybracket{k'}\myparanthese{\displaystyle\sum_{k=1}^K \myPhiOne^{ } \mybracket{k} \myPhiOne^T \mybracket{k}} \myPhiTwo^{ }\mybracket{k'}}_{2 \to 2}} \\
&= \frac{d_{min}}{d^{3}_{max}} \mynorm{K \boldsymbol{I}_{d_{k'}}}_{2 \to 2}\\
&= \frac{d_{min}}{d^{3}_{max}} K.
\end{aligned}
\end{gather*}
The above second inequality follows from the triangle inequality, whereas the third one results from considering the minimum values for variable block length coefficients in the sum operator.
\myhl{In fact, in this problem, having some finite integer values $d_k \sgeq 1$, $\forall k$, and the corresponding orthonormal bases $\myPhi[k]$, we are utilising the lower-bound $\sum_k d_k \myPhi[k] \sgeq \min_k d_k \sum_k \myPhi[k] \seq d_{min} \sum_k \myPhi[k]$.}
The above first equality results from the fact that for orthonormal matrices $\myPhiOne$ and $\myPhiTwo$ we have, $\sum_{k=1}^K \myPhiOne^{ } [k] \myPhiOne^T [k] \seq \myPhiOne \myPhiOne^T \seq \boldsymbol{I}_m$ and $\forall k'$, $\myPhiTwo^T [k'] \myPhiTwo ^{ } [k'] \seq \boldsymbol{I}_{d_{k'}}$.
The last equality results from the homogeneity property (Property \ref{prp:OperatorProperties}, page \pageref{prp:OperatorProperties}) and unity of operator-norm of the identity matrix.
The proof is then completed by taking square roots from both sides and noticing that for each matrix $\myPhiOne$ and $\myPhiTwo$, we have $m \seq \sum_{k \seq 1}^K d_k \sgeq \min_k d_k \sum_{k \seq 1}^K 1 \seq K \, d_{min}$, then $K \sleq m / d_{min}$, i.e.,:
\begin{gather*}
\overbar{M}_{2,2}\myparanthese{\myPhiOne , \myPhiTwo} 
\geq \sqrt{\frac{d_{min}}{K \, d^{3}_{max}}} 
\geq \sqrt{\frac{d^2_{min}}{m \, d^{3}_{max}}}.
\end{gather*}  
\end{proof}
\newpage
%------------------------------------------------------
%\iffalse
%\section{Proof of Lemma \ref{lm:Block Spark Inequality} (Block-Spark tractable lower-bound)} % $\boldsymbol{\myBSpkTxt}$
%\label{prf:Block Spark Inequality} 
%\begin{proof}
%Because $\boldsymbol{x} \ssin \myKerMath$, we have $\sum_{k'} \myPhi[k']\boldsymbol{x}[k'] \seq \boldsymbol{0}$. 
%Then for all $k$, $-\sum_{k' \neq k} \myPhi[k']\boldsymbol{x}[k'] \seq \myPhi[k]\boldsymbol{x}[k]$. 
%Therefore, $-\sum_{k' \neq k} \myPhi^\dagger[k] \myPhi[k']\boldsymbol{x}[k'] \seq\boldsymbol{x}[k]$.
%Applying $\Vert \cdot \Vert_{p}$ to both sides and using the triangular inequality for $p \sgeq 1$, $ \sum{\Vert \cdot \Vert_p} \sgeq \Vert \sum{\cdot} \Vert_p$, we have $\sum_{k' \neq k} \Vert \myPhi^\dagger[k]\myPhi[k']\boldsymbol{x}[k'] \Vert_{p} \sgeq \Vert \boldsymbol{x}[k] \Vert_{p}$. 
%It follows from Definition \ref{def:BMIC}, that $d_{max} M_{q,p}(\myPhi) \sum_{k' \neq k} \Vert\boldsymbol{x}[k'] \Vert_{q} / d^{1/q}_{k'} \sgeq \Vert \boldsymbol{x}[k] \Vert_{p} / d^{1/p}_k$.
%Adding $d_{max} M_{q,p}(\myPhi) \Vert \boldsymbol{x}[k] \Vert_{q} / d^{1/q}_{k}$ to both sides, we have:
%\begin{gather*}
%%\label{eq:DontKnow4} 
%\begin{aligned}
%d_{max} M_{q,p}\myparanthese{\myPhi} \displaystyle\sum_{k'} \frac{\mynorm{\boldsymbol{x}\mybracket{k'}}_{q}}{d^{\frac1q}_{k'}} \geq 
% d_{max} M_{q,p}\myparanthese{\myPhi} \frac{\mynorm{\boldsymbol{x}\mybracket{k}}_q}{d^{\frac1q}_{k}} 
%+ \frac{\mynorm{\boldsymbol{x}\mybracket{k}}_p}{d^{\frac1p}_{k}}.
%\end{aligned}
%\end{gather*}
%Summing over blocks $k \ssin S_b(\mybeta)$, and using the characterisation $Q_{\boldsymbol{w};p_1,p_2}(S_b(\mybeta),\myPhi)$ defined in Theorem \ref{th:BNSP}, 
%%instead of all non-zero blocks of $\boldsymbol{x}$
%we get:
%\begin{gather}
%\label{eq:DontKnow6} 
%d_{max} M_{q,p}\myparanthese{\myPhi} \mynorm{\boldsymbol{x}}_{\boldsymbol{w};q,1} \mynorm{\mybeta}_{r,0}\geq 
% d_{max} M_{q,p}\myparanthese{\myPhi} Q_{\boldsymbol{w};q,1} \mynorm{\boldsymbol{x}}_{\boldsymbol{w};q,1} + Q_{\boldsymbol{w};p,1} \mynorm{\boldsymbol{x}}_{\boldsymbol{w};p,1},
%\end{gather}
%where, $r \sgeq 0$, $p \sgeq 1$, and $q \sgeq 1$. 
%In this step, for the sake of simplicity $Q_{\boldsymbol{w};p_1,p_2}(S_b(\mybeta),\myPhi)$ is represented by $Q_{\boldsymbol{w};p_1,p_2}$.
%Then, by rearranging the above inequality, we have
%\begin{gather*}
%\begin{aligned}
%Q_{\boldsymbol{w};p,1} &\leq d_{max} M_{q,p}\myparanthese{\myPhi} \mynorm{\boldsymbol{x}}_{\boldsymbol{w};q,1} \frac{\mynorm{\mybeta}_{r,0} - Q_{\boldsymbol{w};q,1}}{\mynorm{\boldsymbol{x}}_{\boldsymbol{w};p,1}} \\
%&< \frac12.
%\end{aligned}
%\end{gather*}
%The above second inequality comes from the Block-NSP condition in Theorem \ref{th:BNSP}, which ensures that $\mybetaz$ is the unique solution to the $P_{\boldsymbol{w};p,1}$, where $S_b(\boldsymbol{\beta_0}) \ssubset S_b(\mybeta)$.
%Then, by rearranging the above inequality, we have
%\begin{gather}
%\label{eq:condition_dk} 
%\begin{aligned}
%\mynorm{\mybeta}_{r,0} &<
%Q_{\boldsymbol{w};q,1} +
%\frac{\myparanthese{d_{max} M_{q,p}\myparanthese{\myPhi} \frac{\mynorm{\boldsymbol{x}}_{\boldsymbol{w};q,1}}{\mynorm{\boldsymbol{x}}_{\boldsymbol{w};p,1}}}^{-1}}{2} \\
%&< 
%\frac12 +
%\frac{\myparanthese{d_{max} M_{q,p}\myparanthese{\myPhi} \frac{\mynorm{\boldsymbol{x}}_{\boldsymbol{w};q,1}}{\mynorm{\boldsymbol{x}}_{\boldsymbol{w};p,1}}}^{-1}}{2} 
%= \frac{1 + \myparanthese{d_{max} M_{q,p}\myparanthese{\myPhi} \frac{\mynorm{\boldsymbol{x}}_{\boldsymbol{w};q,1}}{\mynorm{\boldsymbol{x}}_{\boldsymbol{w};p,1}}}^{-1}}{2}.
%\end{aligned}
%\end{gather}
%Again, the above second inequality comes from the Block-NSP condition in Theorem \ref{th:BNSP}, which ensures that $\mybetaz$ is the unique solution to the $P_{\boldsymbol{w};q,1}$.
%On the other hand, from Corollary \ref{crl:BERC-BNSP-BS-beta}, we have if $| S_b(\mybeta) | \sless \myBSpkMath/2$, then $\mybeta$ is the unique solution of $P_{p,0}$, where $| S_b(\mybeta) | \seq \Vert \mybeta \Vert_{r,0}$ as defined in definition \ref{Def:Block Support}. 
%Therefore, in order to meet the condition of Corollary \ref{crl:BERC-BNSP-BS-beta}, we should have $\myBSpkMath \sgeq 1 \spl (d_{max} M_{q,p}(\myPhi) \mynorm{\boldsymbol{x}}_{\boldsymbol{w};q,1} / \mynorm{\boldsymbol{x}}_{\boldsymbol{w};p,1})^{-1}$.
%
%The proof for equally-sized blocks, i.e., $d_1 \seq \cdots \seq d_K \seq d$, is similar to the differently-sized blocks.
%Starting from $\sum_{k' \neq k} \mynorm{\myPhi^\dagger[k]\myPhi[k']\boldsymbol{x}[k']}_{p} \sgeq \mynorm{\boldsymbol{x}[k]}_{p}$ and considering that for equally-sized blocks we have $M_{q,p}(\myPhi) \seq d^{1/q - 1/p - 1} \max_{k,k' \neq k} \Vert \myPhi^T \mybracket{k} \myPhi \mybracket{k'} \Vert_{q \to p}$ defined in Definition \ref{def:BMIC}, then we get $d^{1 + 1/p - 1/q} M_{q,p}(\myPhi) \sum_{k' \neq k} \Vert \boldsymbol{x}[k'] \Vert_{q} \sgeq \Vert \boldsymbol{x}[k] \Vert_{p}$.
%Adding $d^{1 + 1/p - 1/q} M_{q,p}(\myPhi) \Vert \boldsymbol{x}[k] \Vert_{q}$ to both sides, we have:
%\begin{gather*} 
%%\label{eq:DontKnow4+} 
%\begin{aligned}
%d^{1 + \frac1p - \frac1q} M_{q,p}\myparanthese{\myPhi} \displaystyle\sum_{k'} \mynorm{\boldsymbol{x}\mybracket{k'}}_{q} \geq 
%d^{1 + \frac1p - \frac1q} M_{q,p}\myparanthese{\myPhi} \mynorm{\boldsymbol{x}\mybracket{k}}_q +
%\mynorm{\boldsymbol{x}\mybracket{k}}_p.
%\end{aligned}
%\end{gather*}
%Summing over blocks $k \ssin S_b(\mybeta)$, and using the characterisation $Q_{\boldsymbol{w};p_1,p_2}$ defined in Theorem \ref{th:BNSP}, we obtain:
%\begin{gather}
%\label{eq:DontKnow7} 
%\begin{aligned}
%d^{1 + \frac1p - \frac1q} M_{q,p}\myparanthese{\myPhi} \mynorm{\boldsymbol{x}}_{q,1} \mynorm{\mybeta}_{r,0} \geq 
%d^{1 + \frac1p - \frac1q} M_{q,p}\myparanthese{\myPhi} Q_{q,1} \mynorm{\boldsymbol{x}}_{q,1} 
%+ Q_{p,1} \mynorm{\boldsymbol{x}}_{p,1},
%\end{aligned}
%\end{gather}
%where, $r \sgeq 0$, $p \sgeq 1$, and $q \sgeq 1$. 
%Then, by rearranging the above inequality, we have
%\begin{gather*}
%\begin{aligned}
%Q_{p,1} &\leq d^{1 + \frac1p - \frac1q} M_{q,p}\myparanthese{\myPhi} \mynorm{\boldsymbol{x}}_{q,1} \frac{\mynorm{\mybeta}_{r,0} - Q_{q,1}}{\mynorm{\boldsymbol{x}}_{p,1}} \\
%&< \frac12.
%\end{aligned}
%\end{gather*}
%By the same reasoning as the above weighted case and rearranging the above inequality, we have
%\begin{gather}
%\label{eq:condition_d} 
%\begin{aligned}
%\mynorm{\mybeta}_{r,0} &<
%Q_{q,1} +
%\frac{\myparanthese{d^{1 + \frac1p - \frac1q} M_{q,p}\myparanthese{\myPhi} \frac{\mynorm{\boldsymbol{x}}_{q,1}}{\mynorm{\boldsymbol{x}}_{p,1}}}^{-1}}{2} \\
%&< 
%\frac12 +
%\frac{\myparanthese{d^{1 + \frac1p - \frac1q} M_{q,p}\myparanthese{\myPhi} \frac{\mynorm{\boldsymbol{x}}_{q,1}}{\mynorm{\boldsymbol{x}}_{p,1}}}^{-1}}{2} 
%= \frac{1 + \myparanthese{d^{1 + \frac1p - \frac1q} M_{q,p}\myparanthese{\myPhi} \frac{\mynorm{\boldsymbol{x}}_{q,1}}{\mynorm{\boldsymbol{x}}_{p,1}}}^{-1}}{2}.
%\end{aligned}
%\end{gather}
%Using Corollary \ref{crl:BERC-BNSP-BS-beta}, and the same reasoning as the above weighted case, the proof is done.
%\iffalse
%Summing over non-zero blocks of $\boldsymbol{x}$, we obtain:
%\begin{gather*}
%\begin{aligned}
%d_{max} M_{q,p}\myparanthese{\myPhi} \mynorm{\boldsymbol{x}}_{\boldsymbol{w};q,1} \mynorm{\boldsymbol{x}}_{\boldsymbol{w};r,0}\geq 
% d_{max} M_{q,p}\myparanthese{\myPhi} \mynorm{\boldsymbol{x}}_{\boldsymbol{w};q,1} 
%+ \mynorm{\boldsymbol{x}}_{\boldsymbol{w};p,1},
%\end{aligned}
%\end{gather*}
%where, $r \sg 0$. 
%Then,
%\begin{equation*}
%\mynorm{\boldsymbol{x}}_{\boldsymbol{w};r,0} \geq 
%1+ d^{-1}_{max} M_{q,p}^{-1}\myparanthese{\myPhi}  \frac{\mynorm{\boldsymbol{x}}_{\boldsymbol{w};p,1}}{\mynorm{\boldsymbol{x}}_{\boldsymbol{w};q,1}},
%\end{equation*}
%which using Definition \ref{def:Block Spark} ($\myBSpkTxt$), the proof is completed. 
%The proof for equally-sized blocks, i.e., $d_1 \seq \cdots \seq d_K \seq d$, is similar to the differently-sized blocks.
%Starting from $\sum_{k' \neq k} \mynorm{\myPhi^\dagger[k]\myPhi[k']\boldsymbol{x}[k']}_{p} \sgeq \mynorm{\boldsymbol{x}[k]}_{p}$ and considering that for equally-sized blocks we have $M_{q,p}(\myPhi) \seq d^{1/q - 1/p - 1} \max_{k,k' \neq k} \Vert \myPhi^T \mybracket{k} \myPhi \mybracket{k'} \Vert_{q \to p}$ defined in Definition \ref{def:BMIC}, then we get $d^{1 + 1/p - 1/q} M_{q,p}(\myPhi) \sum_{k' \neq k} \Vert \boldsymbol{x}[k'] \Vert_{q} \sgeq \Vert \boldsymbol{x}[k] \Vert_{p}$.
%Adding $d^{1 + 1/p - 1/q} M_{q,p}(\myPhi) \Vert \boldsymbol{x}[k] \Vert_{q}$ to both sides, we have:
%\begin{gather} 
%\label{eq:DontKnow4+} 
%\begin{aligned}
%d^{1 + 1/p - 1/q} M_{q,p}\myparanthese{\myPhi} \displaystyle\sum_{k'} \mynorm{\boldsymbol{x}\mybracket{k'}}_{q} \geq 
%d^{1 + 1/p - 1/q} M_{q,p}\myparanthese{\myPhi} \mynorm{\boldsymbol{x}\mybracket{k}}_q +
%\mynorm{\boldsymbol{x}\mybracket{k}}_p
%\end{aligned}
%\end{gather}
%Summing over non-zero blocks of $\boldsymbol{x}$, we obtain:
%\begin{gather*}
%\begin{aligned}
%d^{1 + 1/p - 1/q} M_{q,p}\myparanthese{\myPhi} \mynorm{\boldsymbol{x}}_{q,1} \mynorm{\boldsymbol{x}}_{r,0} \geq 
%d^{1 + 1/p - 1/q} M_{q,p}\myparanthese{\myPhi} \mynorm{\boldsymbol{x}}_{q,1} 
%+ \mynorm{\boldsymbol{x}}_{p,1},
%\end{aligned}
%\end{gather*}
%where, $r \sg 0$. 
%Then,
%\begin{equation*}
%\mynorm{\boldsymbol{x}}_{r,0} \geq 
%1+ d^{1/q - 1/p - 1} M_{q,p}^{-1}\myparanthese{\myPhi}  \frac{\mynorm{\boldsymbol{x}}_{p,1}}{\mynorm{\boldsymbol{x}}_{q,1}},
%\end{equation*}
%which using Definition \ref{def:Block Spark} ($\myBSpkTxt$), the proof is completed.
%\fi
%\end{proof}
%\fi
%\newpage
%------------------------------------------------------
\section{Proof of Theorem \ref{th:BERC-BMIC} (Block-ERC based on Block-MCC$_{q,p}$, page \pageref{th:BERC-BMIC})}
\label{prf:BERC-BMIC} % $\boldsymbol{\mySLTxt}$ $\boldsymbol{\myBSLTxt}$
\begin{proof}
Because $\boldsymbol{x} \ssin \myKerMath$, we have $\sum_{k'} \myPhi[k']\boldsymbol{x}[k'] \seq \boldsymbol{0}$. 
Hence, for all $k$, $-\sum_{k' \neq k} \myPhi[k']\boldsymbol{x}[k'] \seq \myPhi[k]\boldsymbol{x}[k]$. 
Therefore, \myhl{since for all $k$ the block $\myPhi[k]$ is full column rank,} we have $-\sum_{k' \neq k} \myPhi^\dagger[k] \myPhi[k']\boldsymbol{x}[k'] \seq\boldsymbol{x}[k]$.
Applying $\Vert \cdot \Vert_{p}$ to both sides and using the triangular inequality for $p \sgeq 1$ and $p \seq 0$, $ \sum{\Vert \cdot \Vert_p} \sgeq \Vert \sum{\cdot} \Vert_p$, we have $\sum_{k' \neq k} \Vert \myPhi^\dagger[k]\myPhi[k']\boldsymbol{x}[k'] \Vert_{p} \sgeq \Vert \boldsymbol{x}[k] \Vert_{p}$. 
On the other hand, from the definition of mutual coherence constant (Definition \ref{def:BMIC}, page \pageref{def:BMIC}), i.e., $\forall (q , p) \ssin \mathbb{R}^2_{>0} : M_{q,p}(\myPhi) \seq \max_{\substack{k,k' \neq k \\ \boldsymbol{x}[k'] \neq \boldsymbol{0}}} d_k^{-1/p} d_{k'}^{1/q} d_{max}^{-1} \Vert \myPhi^\dagger[k]\myPhi[k']\boldsymbol{x}[k'] \Vert_{p} / \Vert \boldsymbol{x}[k'] \Vert_{q}$, we can see that in order to compute $M_{q,p}(\myPhi)$, the value $d_k^{-1/p} d_{k'}^{1/q} d_{max}^{-1} \Vert \myPhi^\dagger[k]\myPhi[k']\boldsymbol{x}[k'] \Vert_{p} / \Vert \boldsymbol{x}[k'] \Vert_{q}$ is calculated for each $k$ and $k' {\neq} k$ and finally the maximum calculated value is considered as $M_{q,p}(\myPhi)$.
Then for any $k$ and $k' {\neq} k$, the $d_k^{-1/p} d_{k'}^{1/q} d_{max}^{-1} \Vert \myPhi^\dagger[k]\myPhi[k']\boldsymbol{x}[k'] \Vert_{p} / \Vert \boldsymbol{x}[k'] \Vert_{q}$ is upper-bounded by $M_{q,p}(\myPhi)$, and since $d_k^{1/p} d_{k'}^{-1/q} d_{max} \Vert \boldsymbol{x}[k'] \Vert_{q}$ is positive, we have $\Vert \myPhi^\dagger[k]\myPhi[k']\boldsymbol{x}[k'] \Vert_{p} \sleq d_k^{1/p} d_{k'}^{-1/q} d_{max} \Vert \boldsymbol{x}[k'] \Vert_{q} M_{q,p}(\myPhi)$.
Hence, returning back to the proof, we have $\sum_{k' \neq k} d_k^{1/p} d_{k'}^{-1/q} d_{max} \Vert \boldsymbol{x}[k'] \Vert_{q} M_{q,p}(\myPhi) \sgeq \sum_{k' \neq k} \Vert \myPhi^\dagger[k]\myPhi[k']\boldsymbol{x}[k'] \Vert_{p} \sgeq \Vert \boldsymbol{x}[k] \Vert_{p}$ or $d_{max} M_{q,p}(\myPhi) \sum_{k' \neq k} d_k^{1/p} d_{k'}^{-1/q} \Vert \boldsymbol{x}[k'] \Vert_{q} \sgeq \Vert \boldsymbol{x}[k] \Vert_{p}$.
Then by rearranging the inequality we get $d_{max} M_{q,p}(\myPhi) \sum_{k' \neq k} \Vert\boldsymbol{x}[k'] \Vert_{q} / d^{1/q}_{k'} \sgeq \Vert \boldsymbol{x}[k] \Vert_{p} / d^{1/p}_k$.
Adding $d_{max} M_{q,p}(\myPhi) \Vert \boldsymbol{x}[k] \Vert_{q} / d^{1/q}_{k}$ to both sides, we have:
\begin{gather*}
%\label{eq:DontKnow4} 
\forall q \in \mathbb{R}_{>0}, \forall p \in \mathbb{R}_{\geq 1}, \qquad
\begin{aligned}
d_{max} M_{q,p}\myparanthese{\myPhi} \displaystyle\sum_{k'} \frac{\mynorm{\boldsymbol{x}\mybracket{k'}}_{q}}{d^{\frac1q}_{k'}} \geq 
 d_{max} M_{q,p}\myparanthese{\myPhi} \frac{\mynorm{\boldsymbol{x}\mybracket{k}}_q}{d^{\frac1q}_{k}} 
+ \frac{\mynorm{\boldsymbol{x}\mybracket{k}}_p}{d^{\frac1p}_{k}},
\end{aligned}
\end{gather*}
\myhl{which using the definition of weighted (pseudo-)mixed-norm $\ell^{\boldsymbol{w}}_{p_1 , p_2}$ (Definition {\ref{def:Weighted mixed norm}}, page {\pageref{def:Weighted mixed norm}}), with $w_k \seq d_{k}^{-1/{q}}$, $p_1 \seq q$, and $p_2 \seq 1$, it is equivalent to}
\begin{gather*} 
\mycolor{\begin{aligned}
\forall q \in \mathbb{R}_{>0}, \forall p \in \mathbb{R}_{\geq 1}, \qquad
d_{max} M_{q,p}\myparanthese{\myPhi} \mynorm{\boldsymbol{x}}_{\boldsymbol{w};q,1} \geq 
 d_{max} M_{q,p}\myparanthese{\myPhi} \frac{\mynorm{\boldsymbol{x}\mybracket{k}}_q}{d^{\frac1q}_{k}} 
+ \frac{\mynorm{\boldsymbol{x}\mybracket{k}}_p}{d^{\frac1p}_{k}}.
\end{aligned}}
\end{gather*}
{
\label{cmmnt:74} 
\myhl{Now take any vector $\mybeta$ with Block-Support $S_b(\mybeta)$ such that $S_b(\mybetaz) {\ssubset} S_b(\mybeta)$, then summing over blocks $k \ssin S_b(\mybeta)$, and knowing that for any constants $\alpha$ and $\forall r \sgeq 0$, we have $\sum_{k \ssin S_b(\mybeta)} \alpha \seq \alpha \Vert \mybeta \Vert_{r , 0}$, we get:}
}
\begin{gather*} 
\mycolor{\begin{aligned}
\forall q \in \mathbb{R}_{>0}, \forall p \in \mathbb{R}_{\geq 1}, \qquad
d_{max} M_{q,p}\myparanthese{\myPhi} \mynorm{\boldsymbol{x}}_{\boldsymbol{w};q,1} \mynorm{\mybeta}_{r,0} &\geq 
\sum_{k \in S_b(\mybeta)} \myparanthese{d_{max} M_{q,p}\myparanthese{\myPhi} \frac{\mynorm{\boldsymbol{x}\mybracket{k}}_q}{d^{\frac1q}_{k}} 
+ \frac{\mynorm{\boldsymbol{x}\mybracket{k}}_p}{d^{\frac1p}_{k}}} \\
&=
d_{max} M_{q,p}\myparanthese{\myPhi} \sum_{k \in S_b(\mybeta)} \frac{\mynorm{\boldsymbol{x}\mybracket{k}}_q}{d^{\frac1q}_{k}}
+ \sum_{k \in S_b(\mybeta)} \frac{\mynorm{\boldsymbol{x}\mybracket{k}}_p}{d^{\frac1p}_{k}}.
\end{aligned}}
\end{gather*}
\myhl{On the other hand from the proof of Property {\ref{lm:FractionBound}} (bounds of two (pseudo-)mixed-norms division, page {\pageref{lm:FractionBound}}), we have $\forall (q , p) \ssin \mathbb{R}^2_{>0}: (\Vert \boldsymbol{x}[k] \Vert_p / d^{1/p}_{k}) / (\Vert \boldsymbol{x}[k] \Vert_q / d^{1/q}_{k}) \sgeq \min \{1 , d_k^{1/q - 1/p} \}$.
Again using the lower-bound on the fraction of two sums of values explained in Property {\ref{lm:FractionBound}}, we have $\forall (q , p) \ssin \mathbb{R}^2_{>0} : (\sum_{k \in S_b(\mybeta)} \Vert \boldsymbol{x}[k] \Vert_p / d^{1/p}_{k}) / (\sum_{k \in S_b(\mybeta)} \Vert \boldsymbol{x}[k] \Vert_q / d^{1/q}_{k}) \sgeq \min_{k \in S_b(\mybeta)} \min \{1 , d_k^{1/q - 1/p} \}$, which is a special case of Property {\ref{lm:FractionBound}}. 
%when the sum is over certain blocks $k \ssin S_b(\mybeta)$ instead of all of the blocks.
Hence, returning back to the proof, $\forall q \ssin \mathbb{R}_{>0}$, $\forall p \ssin \mathbb{R}_{\sgeq 1}$, and $\forall r \ssin \mathbb{R}_{\sgeq 0}$ we get:}
\begin{gather*} 
\mycolor{\begin{aligned}
d_{max} M_{q,p}\myparanthese{\myPhi} \mynorm{\boldsymbol{x}}_{\boldsymbol{w};q,1} \mynorm{\mybeta}_{r,0} &\geq 
d_{max} M_{q,p}\myparanthese{\myPhi} \sum_{k \in S_b(\mybeta)} \frac{\mynorm{\boldsymbol{x}\mybracket{k}}_q}{d^{\frac1q}_{k}} 
+ \myparanthese{\min_{k \in S_b(\mybeta)} \, \min \mybrace{1 , d_k^{\frac1q - \frac1p}}} \sum_{k \in S_b(\mybeta)} \frac{\mynorm{\boldsymbol{x}\mybracket{k}}_q}{d^{\frac1q}_{k}} \\
&= \myparanthese{d_{max} M_{q,p}\myparanthese{\myPhi} + \min_{k \in S_b(\mybeta)} \, \min \mybrace{1 , d_k^{\frac1q - \frac1p}}} \sum_{k \in S_b(\mybeta)} \frac{\mynorm{\boldsymbol{x}\mybracket{k}}_q}{d^{\frac1q}_{k}} \\
&\geq \myparanthese{d_{max} M_{q,p}\myparanthese{\myPhi} + \min_{k} \, \min \mybrace{1 , d_k^{\frac1q - \frac1p}}} \sum_{k \in S_b(\mybeta)} \frac{\mynorm{\boldsymbol{x}\mybracket{k}}_q}{d^{\frac1q}_{k}}.
\end{aligned}}
\end{gather*}
\myhl{The above last line results from $\min_{k\in S} f(k) \sgeq \min_{k} f(k)$.
Then, by dividing the both sides by $\Vert \boldsymbol{x} \Vert_{\boldsymbol{w};q,1}$, $\forall q \ssin \mathbb{R}_{>0}$, $\forall p \ssin \mathbb{R}_{\sgeq 1}$, and $\forall r \ssin \mathbb{R}_{\sgeq 0}$ we have:}
\begin{gather*}  
\mycolor{\begin{aligned}
\forall \boldsymbol{x} \in \myKerMath, \qquad d_{max} M_{q,p}\myparanthese{\myPhi} \mynorm{\mybeta}_{r,0} &\geq 
\myparanthese{d_{max} M_{q,p}\myparanthese{\myPhi} + \min_{k} \, \min \mybrace{1 , d_k^{\frac1q - \frac1p}}} \frac{\displaystyle\sum_{k \in S_b(\mybeta)} \frac{\mynorm{\boldsymbol{x}\mybracket{k}}_q}{d^{\frac1q}_{k}}}{\mynorm{\boldsymbol{x}}_{\boldsymbol{w};q,1}},
\end{aligned}}
\end{gather*}
\myhl{or}
\begin{gather*}  
\mycolor{\begin{aligned}
\forall \boldsymbol{x} \in \myKerMath, \qquad d_{max} M_{q,p}\myparanthese{\myPhi} \mynorm{\mybeta}_{r,0} \myparanthese{d_{max} M_{q,p}\myparanthese{\myPhi} + \min_{k} \, \min \mybrace{1 , d_k^{\frac1q - \frac1p}}}^{-1} &\geq 
 \frac{\displaystyle\sum_{k \in S_b(\mybeta)} \frac{\mynorm{\boldsymbol{x}\mybracket{k}}_q}{d^{\frac1q}_{k}}}{\mynorm{\boldsymbol{x}}_{\boldsymbol{w};q,1}}.
\end{aligned}}
\end{gather*}
\myhl{Since the above inequality holds true for $\forall \boldsymbol{x} \ssin \myKerMath$, it also holds for the maximiser of the right-hand side (because the left-hand side does not depend on $\boldsymbol{x}$).
But that maximiser gives us exactly $Q_{\boldsymbol{w};p_1,p_2}(S_b(\mybeta),\myPhi)$ by its definition:}
\begin{gather*} 
\label{eq:Q-appears} 
\mycolor{\begin{aligned} 
d_{max} M_{q,p}\myparanthese{\myPhi} \mynorm{\mybeta}_{r,0} \myparanthese{d_{max} M_{q,p}\myparanthese{\myPhi} + \min_{k} \, \min \mybrace{1 , d_k^{\frac1q - \frac1p}}}^{-1} &\geq 
\max_{\boldsymbol{x} \in \myKerMath} \frac{\displaystyle\sum_{k \in S_b(\mybeta)} \frac{\mynorm{\boldsymbol{x}\mybracket{k}}_q}{d^{\frac1q}_{k}}}{\mynorm{\boldsymbol{x}}_{\boldsymbol{w};q,1}} \\
&= Q_{\boldsymbol{w};q,1}\myparanthese{S_b\myparanthese{\mybeta},\myPhi}.
\end{aligned}}
\end{gather*}
\myhl{where, $\forall (q,p) \ssin \mathbb{R}^2_{\sgeq 1}$ (constraint on $q$ was imposed by the Block-NSP condition), $\forall r \ssin \mathbb{R}_{\sgeq 0}$, and $w_k \seq d_k^{-1/q}$. 
The above last line is obtained considering the definition of $Q_{\boldsymbol{w};p_1,p_2}(S_b(\mybeta),\myPhi)$ in Block-NSP (Theorem {\ref{th:BNSP}}, page {\pageref{th:BNSP}}).
On the other hand, from the Block-NSP condition in Theorem 2.2, we have, if $Q_{\boldsymbol{w};q,1}(S_b(\mybeta),\myPhi) \sless 1/2$ then a solution $\mybetaz$ to the problem $P_{\boldsymbol{w};q,1}$ is the unique solution whenever $S_b(\mybetaz) {\ssubset} S_b(\mybeta)$.
%Therefore, by meeting the Block-NSP condition, we get:
Hence, we have that if:}
\begin{gather*}  
\mycolor{\forall (q,p) \in \mathbb{R}^2_{\geq 1}, \forall r \in \mathbb{R}_{\geq 0}, \qquad
d_{max} M_{q,p}\myparanthese{\myPhi} \mynorm{\mybeta}_{r,0} \myparanthese{d_{max} M_{q,p}\myparanthese{\myPhi} + \min_{k} \, \min \mybrace{1 , d_k^{\frac1q - \frac1p}}}^{-1} < \frac12,}
\end{gather*}
\myhl{Then, the Block-NSP condition of Theorem 2.2 is met, hence any $\mybetaz$ solution to $\boldsymbol{y} \seq \myPhi \mybeta$ is unique if $S_b(\mybetaz) \ssubset S_b(\mybeta)$.
Now, by rearranging the above inequality, we have}
\begin{gather*} 
\mycolor{\begin{aligned} 
\forall (q,p) \in \mathbb{R}^2_{\geq 1}, \forall r \in \mathbb{R}_{\geq 0}, \qquad
\mynorm{\mybeta}_{r,0} &<
\frac{1 + \myparanthese{d_{max} M_{q,p}\myparanthese{\myPhi}}^{-1} \displaystyle\min_{k} \, \min \mybrace{1 , d_k^{\frac1q - \frac1p}}}{2}, \\
%&= 
%\frac{1 + \myparanthese{d_{max} M_{q,p}\myparanthese{\myPhi} \displaystyle\max_{k} \, \max \mybrace{1 , d_k^{\frac1p - \frac1q}}}^{-1} }{2}.
\end{aligned}}
\end{gather*}
\myhl{and since $S_b(\mybetaz) \ssubset S_b(\mybeta)$, then $\Vert \mybetaz \Vert_{r,0} \sleq \Vert \mybeta \Vert_{r,0}$, so}
\begin{gather*} 
\mycolor{\begin{aligned} 
\forall (q,p) \in \mathbb{R}^2_{\geq 1}, \forall r \in \mathbb{R}_{\geq 0}, \qquad
\mynorm{\mybetaz}_{r,0} &<
\frac{1 + \myparanthese{d_{max} M_{q,p}\myparanthese{\myPhi}}^{-1} \displaystyle\min_{k} \, \min \mybrace{1 , d_k^{\frac1q - \frac1p}}}{2},
\end{aligned}}
\end{gather*}
\myhl{which is exactly the condition of the theorem.
Therefore, the Block-NSP theory under the mentioned condition is actually met and the unique solution of the problem $P_{\boldsymbol{w};q,1}$ is ensured.}
\end{proof}
\newpage
%------------------------------------------------------
\section{Proof of Property \ref{prp:BSLqp-relationships} (Block-SL$_{q,p}$ inequalities, page \pageref{prp:BSLqp-relationships})}
\label{prf:BSLqp-relationships} 
\begin{proof}
First, considering that for all basic tractable ($q,p$) pairs of table \ref{table:OperatorNorm} (page \pageref{table:OperatorNorm}), we have $q \sleq p$, then the term $\min_{k} \, \min \{1 , d_k^{1/q \sm 1/p} \}$ in $\myBSLqpTxt$ of Theorem \ref{th:BERC-BMIC} (page \pageref{th:BERC-BMIC}) would be equal to one.
Hence, $\myBSLqpTxt$ has directly an inverse relationship with Block-MCC$_{q,p}$ (Definition \ref{def:BMIC}, page \pageref{def:BMIC}), i.e., for $1 \sleq q \sleq p$, $\myBSLqpMath \seq (1 \spl (d_{max} M_{q,p}(\myPhi))^{-1}) {/} 2$.
Thus, according to the previously-mentioned relationship between different Block-MCC$_{q,p}$ characterisations with basic ($q,p$) pairs according to table \ref{table:OperatorNorm} (page \pageref{table:OperatorNorm}) demonstrated in Property \ref{prp:BMCC-relationships} (Block-MCC$_{q,p}$ inequalities, page \pageref{prp:BMCC-relationships}), the $\myBSLqpTxt$ inequalities would be in the opposite direction, which proves the inequalities of the Property.
Therefore, for $q \sleq p$ the directions in figure \ref{fig:BSLqp_Inequalities}(a) (page \pageref{fig:BSLqp_Inequalities}) is in the opposite of the ones in figure \ref{fig:BMCC_Inequalities} (page \pageref{fig:BMCC_Inequalities}).

For the proof of general relationships (tractable and intractable) of figure \ref{fig:BSLqp_Inequalities}(b), we have the following $\myBSLqpTxt$ for $q \sg p$, and $d_1 \seq \cdots \seq d_K \seq d$:
\begin{gather}
\label{eq:BSLq>p} 
\begin{aligned}
\forall (q , p) \in \mathbb{R}^2_{\geq 1}, q > p, \qquad
\myBSLqpMath &= \frac{1 + \myparanthese{d \, M_{q,p}\myparanthese{\myPhi} }^{-1} d^{\frac1q - \frac1p}}{2} \\
&= \frac{1 + \myparanthese{d^{1 + \frac1p - \frac1q}  M_{q,p}\myparanthese{\myPhi}}^{-1} }{2} \\
&= \frac{1 + \myparanthese{d^{1 + \frac1p - \frac1q}  \displaystyle\max_{k,k' \neq k} d^{-1 - \frac1p + \frac1q} \mynorm{\myPhi^\dagger\mybracket{k} \myPhi \mybracket{k'}}_{q \to p}}^{-1} }{2} \\
&= \frac{1 + \myparanthese{\displaystyle\max_{k,k' \neq k} \mynorm{\myPhi^\dagger\mybracket{k} \myPhi \mybracket{k'}}_{q \to p}}^{-1} }{2}.
\end{aligned}
\end{gather}

On the other hand, for $q \seq p$, and $d_1 \seq \cdots \seq d_K \seq d$, we get the following $\myBSLqpTxt$:
\begin{gather}
\label{eq:BSLq=p}
\begin{aligned}
\forall q \in \mathbb{R}_{\geq 1}, \qquad
Block{-}SL_{q,q}\myparanthese{\myPhi} &= \frac{1 + \myparanthese{d \, M_{q,q}\myparanthese{\myPhi} }^{-1}}{2} \\
&= \frac{1 + \myparanthese{d \displaystyle\max_{k,k' \neq k} d^{-1} \mynorm{\myPhi^\dagger\mybracket{k} \myPhi \mybracket{k'}}_{q \to q}}^{-1} }{2} \\
&= \frac{1 + \myparanthese{\displaystyle\max_{k,k' \neq k} \mynorm{\myPhi^\dagger\mybracket{k} \myPhi \mybracket{k'}}_{q \to q}}^{-1} }{2}.
\end{aligned}
\end{gather}
By comparing the $\myBSLqpTxt$ in (\ref{eq:BSLq>p}) for $p \sg q$, with (\ref{eq:BSLq=p}) for $q \seq p$, we see that the only difference between the equations is the $\ell_{q {\to} p}$ operator-norm.
On the other hand, from figure \ref{fig:OperatorNorm-Inequalities}, we know that for a fixed $p$ by increasing $q$, and for a fixed $q$ by decreasing $p$, the $\ell_{q {\to} p}$ operator-norm increases, hence, $\myBSLqpTxt$ decreases, which results in the figure \ref{fig:BSLqp_Inequalities}(b).
\end{proof}
\newpage
%------------------------------------------------------
\section{Proof of Property \ref{prp:DontKnow1} (SL v.s. Block-SL$_{q,p}$, page \pageref{prp:DontKnow1})}
\label{prf:DontKnow1} % $\boldsymbol{\mySLTxt}$ $\boldsymbol{\myBSLTxt}$
\begin{proof}
Starting from the relation between $M(\myPhi)$ and $M_{q,p}(\myPhi)$ proposed in Property \ref{prp:BMIC-MIC} (Block-MCC$_{q,p}$ bounds, page \pageref{prp:BMIC-MIC}), the aim is to build the $d_{max}$ times of the numerator of the condition of the Block-ERC based on Block-MCC$_{q,p}$ in Theorem \ref{th:BERC-BMIC}, i.e., $d_{max} \spl M_{q,p}^{-1}(\myPhi) \min_{k} \, \min \{1 , d_k^{1/q \sm 1/p} \}$, which is comparable with the conventional sparsity level.
First, we investigate the required condition on $M(\myPhi)$ for a dictionary with full column rank blocks.
Using the first part of Property \ref{prp:BMIC-MIC} (Block-MCC$_{q,p}$ bounds, page \pageref{prp:BMIC-MIC}) with conditions $M(\myPhi) \sless d_{max}^{1/q \sm 2} (d_{max} \sm 1)^{-1/2}$ and $q \sgeq p \sgeq 1$, and then multiplying by the positive coefficient $\min_{k} \, \min \{1 , d_k^{1/q \sm 1/p} \}$ and summing to $d_{max}$, for $q \sgeq p \sgeq 1$ we get:
\begin{gather*}
\begin{aligned}
M_{q,p}^{-1}\myparanthese{\myPhi} &\geq 
\frac{1- d_{max}^{2 - \frac1q} \myparanthese{d_{max} - 1}^\frac12 M \myparanthese{\myPhi}}{d_{max}^{\frac32 - \frac1p} \, M\myparanthese{\myPhi}}, \\
M_{q,p}^{-1}\myparanthese{\myPhi} \displaystyle\min_{k} \, \min \mybrace{1 , d_k^{\frac1q - \frac1p}} &\geq 
\frac{1- d_{max}^{2 - \frac1q} \myparanthese{d_{max} - 1}^\frac12 M \myparanthese{\myPhi}}{d_{max}^{\frac32 - \frac1p} \, M\myparanthese{\myPhi}} \displaystyle\min_{k} \, \min \mybrace{1 , d_k^{\frac1q - \frac1p}}, \\
d_{max} + M_{q,p}^{-1}\myparanthese{\myPhi} \displaystyle\min_{k} \, \min \mybrace{1 , d_k^{\frac1q - \frac1p}} &\geq 
d_{max} + \frac{1- d_{max}^{2 - \frac1q} \myparanthese{d_{max} - 1}^\frac12 M \myparanthese{\myPhi}}{d_{max}^{\frac32 - \frac1p} \, M\myparanthese{\myPhi}} \displaystyle\min_{k} \, \min \mybrace{1 , d_k^{\frac1q - \frac1p}}.
\end{aligned}
\end{gather*}
But, $\forall q \sgeq p$, we have $\min_{k} \, \min \{1 , d_k^{1/q \sm 1/p} \} \seq d_{max}^{1/q \sm 1/p}$, which leads to the following inequality:
%We used the lower-bound to establish the conditions in the worst case.
\begin{gather*}
\begin{aligned}
q \geq p \geq 1, \qquad
d_{max} + d_{max}^{\frac1q - \frac1p} M_{q,p}^{-1}\myparanthese{\myPhi} &\geq 
d_{max} + \frac{1- d_{max}^{2 - \frac1q} \myparanthese{d_{max} - 1}^\frac12 M \myparanthese{\myPhi}}{d_{max}^{\frac32 - \frac1p} \, M\myparanthese{\myPhi} } d_{max}^{\frac1q - \frac1p} \\
&= d_{max} \mycolor{-} d_{max}^{\frac12} \, \myparanthese{d_{max} - 1}^{\frac12} + d_{max}^{\frac1q - \frac32} \, M^{-1}\myparanthese{\myPhi}.
\end{aligned}
\end{gather*}
\myhl{The above right-hand side expression under the following condition would be greater than or equal to $1 + M^{-1}(\myPhi)$:}
\begin{gather*}
\mycolor{\begin{aligned}
d_{max} - d_{max}^{\frac12} \, \myparanthese{d_{max} - 1}^{\frac12} + d_{max}^{\frac1q - \frac32} \, M^{-1}\myparanthese{\myPhi} \geq 1 + M^{-1}\myparanthese{\myPhi}, \\
\Rightarrow M\myparanthese{\myPhi} \leq \frac{1 - d_{max} ^{\frac1q - \frac32}}{\myparanthese{d_{max} - 1}^{\frac12} \mybracket{\myparanthese{d_{max} - 1}^{\frac12} - d_{max} ^ \frac12}}.
\end{aligned}}
\end{gather*}

\myhl{Therefore, considering another condition of the utilised property, i.e., $M(\myPhi) \sless d_{max}^{1/q \sm 2} (d_{max} \sm 1)^{-1/2}$ in Property {\ref{prp:BMIC-MIC}} (Block-MCC$_{q,p}$ bounds, page {\pageref{prp:BMIC-MIC}}), the following upper-bound condition on $M(\myPhi)$ ensures that the proposed $\myBSLqpTxt$ would be greater than or equal to the conventional SL, i.e., $d_{max} \spl M_{q,p}^{-1}(\myPhi) \min_{k} \, \min \{1 , d_k^{1/q \sm 1/p} \} \sgeq 1 \spl M^{-1}(\myPhi)$:}
\begin{gather*}
\mycolor{\begin{aligned}
q \geq p \geq 1, \qquad M\myparanthese{\myPhi} &\leq
\min \mybrace{\frac{d_{max} ^{\frac1q - 2}}{\myparanthese{d_{max} - 1}^{\frac12}} , \frac{1 - d_{max} ^{\frac1q - \frac32}}{\myparanthese{d_{max} - 1}^{\frac12} \mybracket{\myparanthese{d_{max} - 1}^{\frac12} - d_{max} ^ \frac12}}}.
%\min \mybrace{\frac{d_{max} ^{\frac1q - 2}}{\myparanthese{d_{max} - 1}^{\frac12}} , \frac{1 - d_{max} ^{\frac1q - \frac32}}{d_{max} - d_{max} ^ \frac12 \myparanthese{d_{max} - 1}^{\frac12} - 1}}.
%&= \frac{1 - d_{max} ^{\frac1q - \frac32}}{d_{max} - d_{max} ^ \frac12 \myparanthese{d_{max} - 1}^{\frac12} - 1}.
\end{aligned}}
\end{gather*}
%yields from the condition mentioned in the Property for general dictionaries.
\myhl{But, the above second argument of the $\min$ operator $\forall q, d_{max}$ is negative.
Therefore, although in practice the numerical simulations approve that the proposed sparsity level for general dictionaries with full column rank blocks is higher than the conventional one, in theory due to utilising the most pessimistic bounds, its supremacy cannot be proved.} 

\myhl{Now, suppose that the dictionary $\myPhi$ has intra-block orthonormality, which is the case in the Property {\ref{prp:DontKnow1}} (SL v.s. $\myBSLqpTxt$, page {\pageref{prp:DontKnow1}}).
Using the second part of Property {\ref{prp:BMIC-MIC}} (Block-MCC$_{q,p}$ bounds, page {\pageref{prp:BMIC-MIC}}) as the following first line, multiplying by $\min_{k} \, \min \{1 , d_k^{1/q \sm 1/p} \}$ and summing to $d_{max}$, 
% to make the $d_{max}$ times of the numerator of the condition of the Block-ERC based on Block-MCC$_{q,p}$ in Theorem \ref{th:BERC-BMIC}, 
$\forall (q , p , q' , p') \ssin \mathbb{R}^4_{\sg 0}$ we get:} 
\begin{gather*}
\mycolor{\begin{aligned}
M_{q,p}^{-1}\myparanthese{\myPhi} &\geq 
\frac{d_{max} \, M^{-1}\myparanthese{\myPhi}}{\displaystyle\max_{k,k' \neq k} d_{k}^{\frac12 - \frac1p} \, d_{k'}^{\frac1q + \frac12} \, \max \mybrace{1 , d_k^{\frac1p - \frac{1}{p'}}} \, \max \mybrace{1 , d_{k'}^{\frac{1}{q'} - \frac1q}} \, \max \mybrace{1 , d_k^{\frac{1}{p'} -\frac12}} \, \max \mybrace{1 , d_{k'}^{\frac{1}{2} - \frac{1}{q'}}}}, \\ 
M_{q,p}^{-1}\myparanthese{\myPhi} \displaystyle\min_{k} \, \min \mybrace{1 , d_k^{\frac1q - \frac1p}} &\geq 
\frac{d_{max} \, M^{-1}\myparanthese{\myPhi}}{\displaystyle\max_{k,k' \neq k} d_{k}^{\frac12 - \frac1p} \, d_{k'}^{\frac1q + \frac12} \, \max \mybrace{1 , d_k^{\frac1p - \frac{1}{p'}}} \, \max \mybrace{1 , d_{k'}^{\frac{1}{q'} - \frac1q}} \, \max \mybrace{1 , d_k^{\frac{1}{p'} -\frac12}} \, \max \mybrace{1 , d_{k'}^{\frac{1}{2} - \frac{1}{q'}}}} \displaystyle\min_{k} \, \min \mybrace{1 , d_k^{\frac1q - \frac1p}}, \\ 
d_{max} + M_{q,p}^{-1}\myparanthese{\myPhi} \displaystyle\min_{k} \, \min \mybrace{1 , d_k^{\frac1q - \frac1p}} &\geq 
d_{max} + \frac{d_{max} \, M^{-1}\myparanthese{\myPhi}}{\displaystyle\max_{k,k' \neq k} d_{k}^{\frac12 - \frac1p} \, d_{k'}^{\frac1q + \frac12} \, \max \mybrace{1 , d_k^{\frac1p - \frac{1}{p'}}} \, \max \mybrace{1 , d_{k'}^{\frac{1}{q'} - \frac1q}} \, \max \mybrace{1 , d_k^{\frac{1}{p'} -\frac12}} \, \max \mybrace{1 , d_{k'}^{\frac{1}{2} - \frac{1}{q'}}}} \displaystyle\min_{k} \, \min \mybrace{1 , d_k^{\frac1q - \frac1p}}.
\end{aligned}}
\end{gather*}

\iffalse
From Property \ref{lm:FractionBound}, we know that the fraction $\Vert \boldsymbol{x} \Vert_{\boldsymbol{w};p,1}/ \Vert\boldsymbol{x} \Vert_{\boldsymbol{w};q,1}$ is lower-bounded by $\min_k d_k^{1/q \sm 1/p} \min \{1 , d_k^{1/p \sm 1/q} \}$, 

$\begin{cases}
\begin{aligned}
&\min_k d_k^{1/q \sm 1/p} \min \{1 , d_k^{1/p \sm 1/q} \}, \quad &&\text{if }q \sgeq 1 \\
&1, \quad &&\text{if }q \sless 1
\end{aligned}
\end{cases}
$, 

$\forall p \sgeq 1$, and $\forall q \sgeq 1$, which leads to the following inequality.
%We used the lower-bound to establish the conditions in the worst case.
\begin{gather*}
\begin{aligned}
d_{max} + M_{q,p}^{-1}\myparanthese{\myPhi} \frac{\mynorm{\boldsymbol{x}}_{\boldsymbol{w};p,1}}{\mynorm{\boldsymbol{x}}_{\boldsymbol{w};q,1}} &\geq 
%\begin{cases}
d_{max} + d_{max}^{\frac1p - \frac32} \, M^{-1}\myparanthese{\myPhi} \displaystyle\min_k d_k^{\frac1q - \frac1p} \min \mybrace{1 , d_k^{\frac1p - \frac1q}} \\ %, & \qquad \text{for \ }q \geq 1\\
%d_{max} + d_{max}^{\frac1p - \frac1q - \frac12} \, M^{-1}\myparanthese{\myPhi}, & \qquad \text{for \ }q < 1\\
%\end{cases}, \\
&\geq 1 + M^{-1}\myparanthese{\myPhi}.
\end{aligned}
\end{gather*}

The above last inequality yields from the condition mentioned in the Property for a dictionary with intra-block orthonormality.
\fi
\myhl{The above right-hand side expression under the following condition and $\forall (q , p , q' , p') \ssin \mathbb{R}^4_{\sg 0}$ would be greater than or equal to $1 + M^{-1}(\myPhi)$:} 
\begin{gather*}
\mycolor{\begin{aligned}
d_{max} + \frac{d_{max} \, M^{-1}\myparanthese{\myPhi}}{\displaystyle\max_{k,k' \neq k} d_{k}^{\frac12 - \frac1p} \, d_{k'}^{\frac1q + \frac12} \, \max \mybrace{1 , d_k^{\frac1p - \frac{1}{p'}}} \, \max \mybrace{1 , d_{k'}^{\frac{1}{q'} - \frac1q}} \, \max \mybrace{1 , d_k^{\frac{1}{p'} -\frac12}} \, \max \mybrace{1 , d_{k'}^{\frac{1}{2} - \frac{1}{q'}}}} \displaystyle\min_{k} \, \min \mybrace{1 , d_k^{\frac1q - \frac1p}} 
\geq 1 + M^{-1}\myparanthese{\myPhi}, \\
\Rightarrow 
M\myparanthese{\myPhi} \leq
\frac{1 - \frac{d_{max} \, \displaystyle\min_{k} \, \min \mybrace{1 , d_k^{\frac1q - \frac1p}}}{\displaystyle\max_{k,k' \neq k} d_{k}^{\frac12 - \frac1p} \, d_{k'}^{\frac1q + \frac12} \, \max \mybrace{1 , d_k^{\frac1p - \frac{1}{p'}}} \, \max \mybrace{1 , d_{k'}^{\frac{1}{q'} - \frac1q}} \, \max \mybrace{1 , d_k^{\frac{1}{p'} -\frac12}} \, \max \mybrace{1 , d_{k'}^{\frac{1}{2} - \frac{1}{q'}}}}}{d_{max} - 1}.
\end{aligned}}
\end{gather*}
\end{proof}
\newpage
%------------------------------------------------------
\section{Proof of Lemma \ref{lm:Eldar-BMIC} (Eldar et al.\textquotesingle s v.s. proposed Block-SL$_{q,p}$, page \pageref{lm:Eldar-BMIC})} % $\boldsymbol{\myBSLTxt}$
\label{prf:Eldar-BMIC} 
\begin{proof}
The proof is similar to the proof of Property \ref{prp:BMIC-MIC} (Block-MCC$_{q,p}$ bounds).
First, we investigate the required condition on $M^{Eldar}_{Intra}(\myPhi)$ for a dictionary \myhl{with full column rank blocks}.
Utilising the Block-MCC$_{q,p}$ proposed in Definition \ref{def:BMIC} (\pageref{def:BMIC}), we have the following first line in (\ref{eq:BMCC-Eldar-BSL}). 
Using the pseudo-inverse property of full column rank matrices and submultiplicativity property of operator-norms introduced in Property \ref{prp:OperatorProperties} ($\ell_{q {\to} p}$ operator-norm properties), and for $d_1 \seq \cdots \seq d_K \seq d$, $\forall (q , p) \ssin \mathbb{R}^2_{>0}$ we have:
\begin{gather}
\label{eq:BMCC-Eldar-BSL} 
\begin{aligned}
M_{q,p}\myparanthese{\myPhi} &= d^{\frac1q - \frac1p-1} \max_{k,k' \neq k} \mynorm{\myPhi^\dagger \mybracket{k} \myPhi \mybracket{k'}}_{q \to p} \\
&= d^{\frac1q - \frac1p-1} \max_{k,k' \neq k} \mynorm{\myparanthese{\myPhi^T \mybracket{k} \myPhi\mybracket{k}}^{-1}\myPhi^T \mybracket{k} \myPhi \mybracket{k'}}_{q \to p} \\
&\leq d^{\frac1q - \frac1p-1} \max_{k,k' \neq k} \mynorm{\myparanthese{\myPhi^T \mybracket{k} \myPhi \mybracket{k}}^{-1}}_{q \to p} \, \mynorm{\myPhi^T \mybracket{k} \myPhi \mybracket{k'}}_{q \to p} \max \mybrace{1 , d^{\frac1q - \frac1p}}.
\end{aligned}
\end{gather}

{
\label{cmmnt:77-2} 
\myhl{Similar to the proof of the Property {\ref{prp:BMIC-MIC}} in finding upper-bound of $\Vert (\myPhi^T [k] \myPhi [k])^{-1} \Vert_{q \to p}$, by replacing $M(\myPhi)$ by $M^{Eldar}_{Intra}(\myPhi)$} (because $\forall i,j$, $\boldsymbol{F}_{i,j}[k] \sleq M^{Eldar}_{Intra}(\myPhi) {\myeq} \max_{\substack{i,j \neq i \\ k}} \vert \myphi_i ^T [k] \myphi_j [k] \vert \sleq M(\myPhi)$), \myhl{and using Property {\ref{lm:Horn}} ($q \sgeq p \sgeq 1$, $M^{Eldar}_{Intra}(\myPhi) \sless d^{1/q - 2} (d - 1)^{-1/2}$ to meet the related condition), we have:}
}
\begin{gather*}
\forall k, q \geq p \geq 1, \qquad
\mynorm{\myparanthese{\myPhi^T \mybracket{k} \myPhi \mybracket{k}}^{-1}}_{q \to p} 
\leq \frac{1}{1-  d^{2 - \frac1q} \myparanthese{d - 1}^\frac12 \, M^{Eldar}_{Intra}\myparanthese{\myPhi}}.
\end{gather*}
For $q \sgeq p \sgeq 1$ (condition imposed by Property \ref{lm:Horn}), we have $\max \{1 , d^{1/q \sm 1/p} \} \seq 1$ in (\ref{eq:BMCC-Eldar-BSL}). 
Therefore, by substituting the upper-bound of $\Vert (\myPhi^T [k] \myPhi [k])^{-1} \Vert_{q \to p}$ and the value of $\max \{1 , d^{1/q \sm 1/p} \}$ in (\ref{eq:BMCC-Eldar-BSL}), for $q \sgeq p \sgeq 1$ we get:
\begin{gather*}
\begin{aligned}
M_{q,p}\myparanthese{\myPhi} &\leq 
\frac{d^{\frac1q - \frac1p -1 }}{1- d^{2 - \frac1q} \myparanthese{d - 1}^\frac12 M^{Eldar}_{Intra}\myparanthese{\myPhi}} \max_{k,k' \neq k}  \mynorm{\myPhi^T \mybracket{k} \myPhi \mybracket{k'}}_{q \to p}\\
&\leq \frac{d^{\frac1q - \frac1p -1 } \max \mybrace{1 , d^{\frac1p - \frac{1}{2}}} \max \mybrace{1 , d^{\frac{1}{2} - \frac1q}}}{1-  d^{2 - \frac1q} \myparanthese{d - 1}^\frac12 M^{Eldar}_{Intra}\myparanthese{\myPhi}} \max_{k,k' \neq k} \mynorm{\myPhi^T \mybracket{k} \myPhi \mybracket{k'}}_{2 \to 2}.
\end{aligned}
\end{gather*}
The above last inequality is achieved based on the upper-bound of $\ell_{q \to p}$ operator-norm based on $\ell_{2 \to 2}$ operator-norm explained in Property \ref{lm:qpTO22}.
\myhl{Then by definition, substituting $\max_{k,k' \neq k} \Vert \myPhi ^T [k] \myPhi [k'] \Vert_{2 \to 2}$ by $d \, M^{Eldar}_{Inter}(\myPhi)$ in the above inequality, for $q \sgeq p \sgeq 1$ we have:}
\begin{gather*}
\label{cmmnt:78} 
\mycolor{M_{q,p}\myparanthese{\myPhi} \leq 
\frac{d^{\frac1q - \frac1p} M^{Eldar}_{Inter}\myparanthese{\myPhi} \max \mybrace{1 , d^{\frac1p - \frac{1}{2}}} \max \mybrace{1 , d^{\frac{1}{2} - \frac1q}}}{1-  d^{2 - \frac1q} \myparanthese{d - 1}^\frac12 M^{Eldar}_{Intra}\myparanthese{\myPhi}},}
\end{gather*}
or,
\begin{gather*}
M_{q,p}^{-1}\myparanthese{\myPhi} \geq 
\frac{1-  d^{2 - \frac1q} \myparanthese{d - 1}^\frac12 M^{Eldar}_{Intra}\myparanthese{\myPhi}}{d^{\frac1q - \frac1p} M^{Eldar}_{Inter}\myparanthese{\myPhi} \max \mybrace{1 , d^{\frac1p - \frac{1}{2}}} \max \mybrace{1 , d^{\frac{1}{2} - \frac1q}}}.
\end{gather*}
By comparing Eldar's Block-ERC, i.e., $\Vert \mybetaz \Vert_{2,0} \sless (1 \spl (d M^{Eldar}_{Inter}(\myPhi))^{-1} (1 \sm (d \sm 1)M^{Eldar}_{Intra}(\myPhi)))/2$, explained in (\ref{BERC-Eldar}) and the proposed equally-sized Block-ERC proposed in Theorem \ref{th:BERC-BMIC}, i.e., $\forall (q,p) \ssin \mathbb{R}^2_{\sgeq 1}, \forall r \ssin \mathbb{R}_{\sgeq 0} : \Vert \mybetaz \Vert_{r,0} \sless ( 1 + (dM_{q,p}(\myPhi))^{-1} \min \{1 , d^{1/q \sm 1/p} \}) / 2 $, it is clear that for $q \sgeq p \sgeq 1$ (the constraint imposed up to this step) the relationship between $d^{1/q \sm 1/p} M_{q,p}^{-1}(\myPhi)$ 
%and $M_{q,p}^{-1}(\myPhi)$, respectively, with 
and $(M^{Eldar}_{Inter}(\myPhi))^{-1} (1 \sm (d \sm 1)M^{Eldar}_{Intra}(\myPhi))$ should be investigated.
By multiplying both sides of the above inequality by $d^{1/q \sm 1/p} \sg 0$, for $q \sgeq p \sgeq 1$ we have:
\begin{gather*}
d^{\frac1q - \frac1p} M_{q,p}^{-1}\myparanthese{\myPhi} \geq 
\frac{1-  d^{2 - \frac1q} \myparanthese{d - 1}^\frac12 M^{Eldar}_{Intra}\myparanthese{\myPhi}}{M^{Eldar}_{Inter}\myparanthese{\myPhi} \, \max \mybrace{1 , d^{\frac1p - \frac{1}{2}}} \max \mybrace{1 , d^{\frac{1}{2} - \frac1q}}}.
\end{gather*}
Assuming that the above right-hand side is greater than or equal to the expression that we are interested in to its relation with $d^{1/q \sm 1/p} M_{q,p}^{-1}(\myPhi)$, i.e., $(M^{Eldar}_{Inter}(\myPhi))^{-1} (1 \sm (d \sm 1)M^{Eldar}_{Intra}(\myPhi))$, we extract the required corresponding condition.
In other words, assuming that for $q \sgeq p \sgeq 1$, we have:
\begin{gather*} 
\frac{1-  d^{2 - \frac1q} \myparanthese{d - 1}^\frac12 M^{Eldar}_{Intra}\myparanthese{\myPhi}}{M^{Eldar}_{Inter}\myparanthese{\myPhi} \, \max \mybrace{1 , d^{\frac1p - \frac{1}{2}}} \max \mybrace{1 , d^{\frac{1}{2} - \frac1q}}} 
\geq \frac{1 - \myparanthese{d - 1} M^{Eldar}_{Intra}\myparanthese{\myPhi}}{M^{Eldar}_{Inter}\myparanthese{\myPhi}},
\end{gather*}
we conclude that for
\begin{gather*}
q \geq p \geq 1, \qquad
M^{Eldar}_{Intra}\myparanthese{\myPhi} \leq 
\frac{1 - \max \mybrace{1 , d^{\frac1p - \frac12}} \max \mybrace{1 , d^{\frac12 - \frac1q}}}{d^{2 - \frac1q} \myparanthese{d - 1}^\frac12 - \myparanthese{d - 1} \max \mybrace{1 , d^{\frac1p - \frac12}} \max \mybrace{1 , d^{\frac12 - \frac1q}}},
\end{gather*}
the assumption will hold true.
On the other hand, we had another condition on the upper-bound of $M^{Eldar}_{Intra}(\myPhi)$ to meet the condition of Property \ref{lm:Horn}.
Therefore, the minimum of the obtained upper-bounds on the $M^{Eldar}_{Intra}(\myPhi)$ ensures that $\forall q \sgeq p \sgeq 1$ the Block-ERC proposed in Theorem \ref{th:BERC-BMIC} improves Eldar's Block-ERC, i.e., $d^{1/q \sm 1/p} M_{q,p}^{-1}(\myPhi) \sgeq (M^{Eldar}_{Inter}(\myPhi))^{-1} (1 \sm (d \sm 1)M^{Eldar}_{Intra}(\myPhi))$:
\begin{gather*}
\begin{aligned}
M^{Eldar}_{Intra}\myparanthese{\myPhi} &\leq 
\min \mybrace{\frac{1}{d^{2 - \frac1q} \myparanthese{d - 1}^\frac12} , \frac{1 - \max \mybrace{1 , d^{\frac1p - \frac12}} \max \mybrace{1 , d^{\frac12 - \frac1q}}}{d^{2 - \frac1q} \myparanthese{d - 1}^\frac12 - \myparanthese{d - 1} \max \mybrace{1 , d^{\frac1p - \frac12}} \max \mybrace{1 , d^{\frac12 - \frac1q}}}} \\
&= \frac{1 - \max \mybrace{1 , d^{\frac1p - \frac12}} \max \mybrace{1 , d^{\frac12 - \frac1q}}}{d^{2 - \frac1q} \myparanthese{d - 1}^\frac12 - \myparanthese{d - 1} \max \mybrace{1 , d^{\frac1p - \frac12}} \max \mybrace{1 , d^{\frac12 - \frac1q}}}.
\end{aligned}
\end{gather*}
The above last inequality yields from the fact that a fraction with smaller numerator and denominator is smaller.
But, the resulted upper-bound is negative.
Therefore, although in practice the numerical simulations approve that the proposed block-sparsity level for general dictionaries with full column rank blocks is higher than the block-sparsity level proposed by Eldar et al., in theory due to utilising the most pessimistic bounds, its supremacy cannot be proved.
%then, the Block-ERC proposed in Theorem \ref{th:BERC-BMIC} improves Eldar's Block-ERC, i.e., $d^{1/q \sm 1/p} M_{q,p}^{-1}(\myPhi) \sgeq (M^{Eldar}_{Inter}(\myPhi))^{-1} (1 \sm (d \sm 1)M^{Eldar}_{Intra}(\myPhi))$.
%Under conditions mentioned in the Lemma, we have $(d^{1/p \sm 1/q} M_{q,p}(\myPhi))^{-1} \sgeq (M^{Eldar}_{Inter}(\myPhi))^{-1} (1 \sm (d \sm 1)M^{Eldar}_{Intra}(\myPhi))$. \\
% and $M_{q,p}^{-1}(\myPhi) \sgeq (M^{Eldar}_{Inter}(\myPhi))^{-1} (1 \sm (d \sm 1)M^{Eldar}_{Intra}(\myPhi))$ for the most pessimistic and optimistic cases, respectively. \\

Now let us consider the assumption of the lemma, i.e., intra-block orthonormality.
By definition, $M^{Eldar}_{Intra}(\myPhi)$ is equal to zero in this case.
Therefore, according to Theorem \ref{th:BERC-BMIC}, the relationship between $\forall (q,p) \ssin \mathbb{R}^2_{\sgeq 1} : M_{q,p}^{-1}(\myPhi) \min \{1 , d^{1/q \sm 1/p} \}$ and 
%$(d^{1/p \sm 1/q} M_{q,p}(\myPhi) \min \{1 , d^{1/q \sm 1/p} \})^{-1}$, respectively, with 
$(M^{Eldar}_{Inter}(\myPhi))^{-1}$ should be investigated.
Starting from the relationship between $M_{q,p}(\myPhi)$ and $M^{Eldar}_{Inter}(\myPhi)$ explained in Property \ref{prp:BMIC-MEldar}, we build the required relationship $\forall (q,p) \ssin \mathbb{R}^2_{\sg 0}$:
\begin{gather*}
\begin{aligned}
M_{q,p}^{-1}\myparanthese{\myPhi} &\geq 
\frac{d^{\frac1p - \frac1q} \myparanthese{M^{Eldar}_{Inter}\myparanthese{\myPhi}} ^{-1}}{\max \mybrace{1 , d^{\frac1p - \frac{1}{2}}} \max \mybrace{1 , d^{\frac{1}{2} - \frac1q}}}, \\
M_{q,p}^{-1}\myparanthese{\myPhi} \, \min \mybrace{1 , d^{\frac1q - \frac1p}}  &\geq 
\frac{d^{\frac1p - \frac1q} \myparanthese{M^{Eldar}_{Inter}\myparanthese{\myPhi}} ^{-1} \min \mybrace{1 , d^{\frac1q - \frac1p}}}{\max \mybrace{1 , d^{\frac1p - \frac{1}{2}}} \max \mybrace{1 , d^{\frac{1}{2} - \frac1q}}}, \\
\frac{M_{q,p}^{-1}\myparanthese{\myPhi} \, \min \mybrace{1 , d^{\frac1q - \frac1p}}}{\myparanthese{M^{Eldar}_{Inter}\myparanthese{\myPhi}} ^{-1}}  &\geq 
\frac{\min \mybrace{1 , d^{\frac1p - \frac1q}}}{\max \mybrace{1 , d^{\frac1p - \frac{1}{2}}} \max \mybrace{1 , d^{\frac{1}{2} - \frac1q}}}.
\end{aligned}
\end{gather*}
The numerator and denominator of the above left-hand side are the values that we need to compare with each other.
To enhance Eldar's Block-ERC the above right-hand side should be greater than one.
Those right-hand side values are shown in table \ref{table:BMCC-EldarBMCC} for different basic values of $(q,p)$ pairs.
From table \ref{table:BMCC-EldarBMCC}, it can be seen that for $q \seq p \seq 2$, $M_{q,p}^{-1}(\myPhi) \, \min \{1 , d^{1/q \sm 1/p} \}$ is greater than or equal to $(M^{Eldar}_{Inter}(\myPhi)) ^{-1}$, which is in fact just equal (not greater), because according to Property \ref{prp:BMIC-MEldar} for dictionaries with intra-block orthonormality, we have $M_{2,2}(\myPhi) \seq M^{Eldar}_{Inter}(\myPhi)$.
Therefore, the proposed Block-ERC for $q \seq p \seq 2$ as the best case equals to Eldar's Block-ERC.
\begin{table*}[tp]
%\begin{adjustbox}{width=1\textwidth} % ,totalheight=\textheight,.5
\centering
%\tiny
%\resizebox{\textwidth}{!}{
\begin{tabular}{ccccccc}
\toprule
%\cline{2-4}
\multicolumn{1}{c}{$(q ,p )$} &\multicolumn{1}{c}{$(1,1)$} & \multicolumn{1}{c}{$(1,2)$}  & \multicolumn{1}{c}{$(1,\infty)$} & \multicolumn{1}{c}{$(2,2)$} & \multicolumn{1}{c}{$(2,\infty)$} & \multicolumn{1}{c}{$(\infty,\infty)$}\\ \midrule %\hline
\multicolumn{1}{c}{$\frac{M_{q,p}^{-1}\myparanthese{\myPhi} \, \min \mybrace{1 , d^{\frac1q - \frac1p}}}{\myparanthese{M^{Eldar}_{Inter}\myparanthese{\myPhi}} ^{-1}}  \geq$} &\multicolumn{1}{l}{$d^{-\frac12}$} & \multicolumn{1}{c}{$d^{-\frac12}$} & \multicolumn{1}{c}{$d^{-1}$} &\multicolumn{1}{c}{$1$} &\multicolumn{1}{c}{$d^{-\frac12}$} & \multicolumn{1}{c}{$d^{-\frac12}$}   \\ \bottomrule %\hline
\end{tabular}
%}
%\end{adjustbox}
\caption{Lower-bound of $M_{q,p}^{-1}(\myPhi) \, \min \{1 , d^{1/q \sm 1/p} \} / (M^{Eldar}_{Inter}(\myPhi)) ^{-1}$ ensuring the supremacy of the proposed sparsity level when it is greater than one, for different basic values of $(q,p)$ pairs and for a dictionary with intra-block orthonormality.}
\label{table:BMCC-EldarBMCC}
\end{table*}
%\iffalse
%which it is not possible. 
%
%Similarly, for optimistic case we have:
%\begin{gather*}
%\myparanthese{d^{\frac1p - \frac1q} M_{q,p}\myparanthese{\myPhi} \min \mybrace{1 , d^{\frac1q - \frac1p}}} ^{-1} \geq 
%\frac{\myparanthese{M^{Eldar}_{Inter}\myparanthese{\myPhi}} ^{-1}}{\max \mybrace{1 , d^{\frac1p - \frac{1}{2}}} \max \mybrace{1 , d^{\frac{1}{2} - \frac1q}} \min \mybrace{1 , d^{\frac1q - \frac1p}}}.
%\end{gather*}
%Again, to enhance Eldar's Block-ERC the denominator of right-hand side should be less than one, i.e., $\max \{1 , d^{1/p \sm 1/2} \} \max \{1 , d^{1/2 \sm 1/q} \} \min \{1 , d^{1/q \sm 1/p} \} \sless 1$, which it is not possible. 
%Therefore, the proposed Block-ERC for $q \seq p \seq 2$ as the best case equals to Eldar's Block-ERC.
%\fi
\end{proof}
\newpage
%------------------------------------------------------
%\iffalse
%\section{Proof of Theorem \ref{th:BERC-BMIC-Q} (Block-ERC based on Block-NSP)}
%\label{prf:BERC-BMIC-Q} 
%\begin{proof}%[BERC-BMIC-Q]
%The proof for differently and equally-sized cases is already done in (\ref{eq:condition_dk}) and (\ref{eq:condition_d}), respectively.
%\iffalse
%Continuing from (\ref{eq:DontKnow4}) in the proof of Lemma \ref{lm:Block Spark Inequality} and summing over blocks $k \ssin S_b(\mybeta)$, 
%%instead of all non-zero blocks of $\boldsymbol{x}$
%we get:
%\begin{gather}
%\label{eq:DontKnow5}
%d_{max} M_{q,p}\myparanthese{\myPhi} \mynorm{\boldsymbol{x}}_{\boldsymbol{w};q,1} \mynorm{\mybeta}_{\boldsymbol{w};r,0}\geq 
% d_{max} M_{q,p}\myparanthese{\myPhi} Q_{\boldsymbol{w};q,1} \mynorm{\boldsymbol{x}}_{\boldsymbol{w};q,1} + Q_{\boldsymbol{w};p,1} \mynorm{\boldsymbol{x}}_{\boldsymbol{w};p,1},
%\end{gather}
%where, $r \sg 0$. 
%In this phase, for sake of simplicity $Q_{\boldsymbol{w};p_1,p_2}(S_b(\mybeta),\myPhi)$ is represented by $Q_{\boldsymbol{w};p_1,p_2}$.
%Rearranging (\ref{eq:DontKnow5}), we have
%\begin{gather*}
%\begin{aligned}
%Q_{\boldsymbol{w};p,1} &\leq d_{max} M_{q,p}\myparanthese{\myPhi} \mynorm{\boldsymbol{x}}_{\boldsymbol{w};q,1} \frac{\mynorm{\mybeta}_{\boldsymbol{w};r,0} - Q_{\boldsymbol{w};q,1}}{\mynorm{\boldsymbol{x}}_{\boldsymbol{w};p,1}} \\
%&< \frac12.
%\end{aligned}
%\end{gather*}
%The above last inequality comes from the Block-NSP condition in Theorem \ref{th:BNSP} and the proof is done.
%The proof for equally-sized blocks is straightforward.
%\fi
%\end{proof}
%\fi
%\newpage
%------------------------------------------------------
\section{Proof of Theorem \ref{th:BERC-CIIC} (Block-ERC based on Eldar's cumulative coherence, page \pageref{th:BERC-CIIC})}
\label{prf:BERC-CIIC}
\begin{proof}
The proof is similar to the proof of Theorem 3 in \cite{Eldar2010}.
First, divide the whole matrix $\myPhi$ into two complementary matrices $\myPhi_{opt}$ and $\myPhi_{\overline{opt}}$. 
Suppose $\myPhi_{opt}$ is a \myhl{full column rank} $m$ by $kd$ matrix whose blocks correspond to non-zero blocks of $\mybetaz$, and let $\myPhi_{\overline{opt}}$ be its complementary matrix. 
From Theorem 2 in \cite{Eldar2010}, a sufficient condition for block orthogonal matching pursuit and $\ell_2/\ell_1$-optimisation program algorithms to correctly recover the block $k$-sparse $\mybetaz$, is that $\rho_c(\myPhi_{opt}^\dagger \myPhi_{\overline{opt}}) \sless 1$, where,  $\rho_c(\boldsymbol{A}) \seq \max_j \sum_i \Vert \boldsymbol{A}[i,j] \Vert_{2 {\to} 2}$ and $\boldsymbol{A}[i,j]$ is the $(i,j)^{th}$ $d \stimes d$ block of $\boldsymbol{A}$.
Using Moore-Penrose pseudo-inverse property of matrices and the submultiplicativity property of $\rho_c(\cdot)$ been proved in Lemma 2 in \cite{Eldar2010}, we have:
\begin{gather*}
\begin{aligned}
\rho_c \myparanthese{\myPhi_{opt}^\dagger \myPhi_{\overline{opt}}} &=
\rho_c \myparanthese{\myparanthese{\myPhi_{opt}^T \myPhi_{opt}}^{-1} \myPhi_{opt}^T \myPhi_{\overline{opt}}} \\
&\leq \rho_c \myparanthese{\myparanthese{\myPhi_{opt}^T \myPhi_{opt}}^{-1}} \rho_c \myparanthese{\myPhi_{opt}^T \myPhi_{\overline{opt}}}. \\
\end{aligned}
\end{gather*}
On the other hand, we have:
\begin{gather*}
\begin{aligned}
\rho_c \myparanthese{\myPhi_{opt}^T \myPhi_{\overline{opt}}} 
&= \max_{j} \sum_{i} \mynorm{\myPhi_{opt}^T \myPhi_{\overline{opt}} \mybracket{i,j}}_{2 \to 2} \\
&= \max_{j \notin \Lambda} \sum_{i \in \Lambda} \mynorm{\myPhi^T \myPhi \mybracket{i,j}}_{2 \to 2} \\
&=  \max_{j \notin \Lambda} \sum_{i \in \Lambda} \mynorm{\myPhi^T \mybracket{i} \myPhi \mybracket{j}}_{2 \to 2} \\
&\leq d \, M_{Inter}^{Eldar}\myparanthese{\myPhi , k},
\end{aligned}
\end{gather*}
where, $\Lambda$ is the set of indices of blocks of $\myPhi$ which are in $\myPhi_{opt}$, and by Definition \ref{def:CIIC} $M_{Inter}^{Eldar}(\myPhi , k) {\myeq}
\max_{\vert \Lambda \vert =k} \max_{j \notin \Lambda} \sum_{i \in \Lambda} \Vert \myPhi^T[i] \myPhi[j] \Vert_{2 \to 2} /d$.
Therefore, we have:
\begin{gather*}
\rho_c \myparanthese{\myPhi_{opt}^\dagger \myPhi_{\overline{opt}}} \leq \rho_c \myparanthese{\myparanthese{\myPhi_{opt}^T \myPhi_{opt}}^{-1}} d \, M_{Inter}^{Eldar}\myparanthese{\myPhi , k}.
\end{gather*}
Now it remains to upper-bound $\rho_c((\myPhi_{opt}^T \myPhi_{opt})^{-1})$.
To this aim, we decompose $\myPhi_{opt}^T \myPhi_{opt}$ as $\myPhi_{opt}^T \myPhi_{opt} \seq \boldsymbol{I}_{kd} \spl \boldsymbol{F}$, where, $\boldsymbol{F}$ is a $kd$ by $kd$ matrix with blocks $\boldsymbol{F} [i,j]$ of size $d \stimes d$, that
\begin{equation*}
\begin{aligned}
\boldsymbol{F} \mybracket{i,j} = 
  \begin{cases}

    \myPhi_{opt}^T \mybracket{i} \myPhi_{opt} \mybracket{j}-\boldsymbol{I}_{d},   \quad &\text{if }i = j\\
        \myPhi_{opt}^T \mybracket{i} \myPhi_{opt} \mybracket{j},   \quad   &\text{if }i \neq j.
  \end{cases} 
  \end{aligned}
\end{equation*}
All the main diagonal entries of $\boldsymbol{F}$ are zero.
Then, we have:
\begin{gather*}
\begin{aligned}
\rho_c \myparanthese{\boldsymbol{F}} &=
\max_j \sum_i \mynorm{\boldsymbol{F}\mybracket{i,j}}_{2 \to 2} \\
&\leq \max_j \mynorm{\boldsymbol{F}\mybracket{j,j}}_{2 \to 2} + \max_j \sum_{i \neq j} \mynorm{\boldsymbol{F}\mybracket{i,j}}_{2 \to 2} \\
&\leq \myparanthese{d-1} M^{Eldar}_{Intra}\myparanthese{\myPhi} + d \, M^{Eldar}_{Inter}\myparanthese{\myPhi , k-1},
\end{aligned}
\end{gather*}
where, the first term is obtained by applying Ger\v{s}gorin's disc theorem \cite{HornR.A.2012}, and using the definition of sub-coherence proposed by Eldar \cite{Eldar2010}.
{
\label{cmmnt:79} 
\myhl{Precisely, from the definition of $\boldsymbol{F}[k,k] \ssin \mathbb{R}^{d \stimes d}$, we have $\forall k$, $\boldsymbol{F}_{i,j{\neq}i}[k,k] \sleq M^{Eldar}_{Intra}(\myPhi)$, while $\boldsymbol{F}_{i,i}[k,k] \seq 0$.
On the other hand, from Corollary 6.1.5 in {\cite{HornR.A.2012}} (Ger\v{s}gorin's disc theorem), we have $\forall \boldsymbol{A} \ssin \mathbb{R}^{m \stimes m}$, $\Vert \boldsymbol{A} \Vert_{2 {\to} 2} \sleq \min \{ \Vert \boldsymbol{A} \Vert_{1 {\to} 1} , \Vert \boldsymbol{A} \Vert_{\infty {\to} \infty} \} $.
Therefore, $\max_j \Vert \boldsymbol{F}\mybracket{j,j} \Vert_{2 \to 2} \sleq (d \sm 1) M^{Eldar}_{Intra}(\myPhi)$.}
}
The second term follows from the definition of the cumulative inter-block coherence constant, defined in Definition \ref{def:CIIC}. 
Using Lemma 4 in \cite{Eldar2010} and considering the assumption of Theorem \ref{th:BERC-CIIC} which indicates that $\rho_c (\boldsymbol{F}) \sless 1$, we have:
\begin{gather*}
\begin{aligned}
\rho_c \myparanthese{\myparanthese{\myPhi_{opt}^T \myPhi_{opt}}^{-1}} &=
\rho_c \myparanthese{\sum_{i=0}^\infty \myparanthese{-\boldsymbol{F}}^i} \\
 &\leq \sum_{i=0}^\infty \myparanthese{\rho_c\myparanthese{\boldsymbol{F}}}^i \\
 &= \frac{1}{1-\rho_c \myparanthese{\boldsymbol{F}}} \\
 &\leq \frac{1}{1-\myparanthese{d-1} M^{Eldar}_{Intra}\myparanthese{\myPhi} - d \, M^{Eldar}_{Inter}\myparanthese{\myPhi , k-1}},
\end{aligned}
\end{gather*}
where, the first inequality follows from the triangle inequality and submultiplicativity properties.
Then we have the following inequality, which is a simple rearrangement of the equation in Theorem \ref{th:BERC-CIIC}:
\begin{gather*}
\begin{aligned}
\rho_c \myparanthese{\myPhi_{opt}^\dagger \myPhi_{\overline{opt}}} &\leq \frac{d \, M^{Eldar}_{Inter}\myparanthese{\myPhi , k}}{1-\myparanthese{d-1} M^{Eldar}_{Intra}\myparanthese{\myPhi} - d \, M^{Eldar}_{Inter}\myparanthese{\myPhi , k-1}} \\
&< 1.
\end{aligned}
\end{gather*}
%which is a simple rearrangement of the equation in Theorem \ref{th:BERC-CIIC}.
\end{proof}