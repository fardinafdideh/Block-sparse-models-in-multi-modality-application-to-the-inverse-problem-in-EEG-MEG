The main direct neuronal source of EEG and MEG signals is pyramidal neuron.  
%(cell) 
%which
%is named for its shape and
Task of a pyramidal neuron is to take synaptic inputs and produce patterned output of action potential \cite{Okada1993}.
% cite{Okada1993,Okada1997,Murakami2002,Murakami2003,Murakami2006,Blagoev2007,Jones2007,Cassara2008,Cassara2009,Jones2009}. 
As can be seen in figure \ref{fig:ECD}, the postsynaptic dendrite part is polarized due to the concentration discrepancy of ions.
When the action current reaches the presynaptic axon terminal, the glands of neurotransmitters are released into the synapse and bind to the receptors on the postsynaptic dendrite, and this leads to a flow of ions of $Na^+$ from outside of cell to inside, and producing post synaptic current as shown in figure \ref{fig:ECD}.
As shown in figure \ref{fig:ECD}, by spreading this ions, intracellular current (primary current) and extracellular current (secondary current)
%, or volume current) 
 are produced.
A small patch of brain cortex which comprises thousands of these simultaneously activated parallel pyramidal neurons can be represented by a current dipole \cite{Cohen2003}.%,Oosterom2012}.
This relationship between the macroscopic representation and the corresponding microscopic neuronal activity is important \cite{Okada1997}.
%,LopesdaSilva2010,Andino2011,Ahlfors2015}. 
These pyramidal neurons are oriented perpendicular to the brain cortex, but since the cortex is folded, 'radial', 'tangential', and 'oblique' dipole orientations defined with respect to the local curvature of the skull, must be considered. 
\begin{figure}[htpb]
\centering
\includegraphics[width=1\textwidth]{images/ECD.png} % width=0.5\textwidth  scale=0.49
\caption{Action current leads to releasing neurotransmitters into the synapse and producing post synaptic current inside the cell. A current dipole is a representation of thousands of simultaneously firing neighbouring neurons and can have different 
%types, according to its 
orientations.}
\label{fig:ECD}
\end{figure}
\FloatBarrier