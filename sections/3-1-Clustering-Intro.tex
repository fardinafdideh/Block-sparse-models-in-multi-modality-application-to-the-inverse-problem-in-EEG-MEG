The structure of block-sparsity, which is defined in Section \ref{sec:Structure and the model} had been assumed to be known.
In other words, the block structure of the representation vector $\mybeta$ and dictionary $\myPhi$ is assumed to be stored in $\boldsymbol{d} \seq [d_1, \cdots, d_K]$.
However, in some problems the block structure prior knowledge might be not available.
Therefore, the problem of block structure identification needs to be taken into account \cite{Eksioglu2011}.

In this chapter, the proposed block structure identification method is explained in Section \ref{sec:Block structure identification}.
Next, the advantages of the proposed method including the improved recovery conditions and brain source space segmentation are investigated in sections \ref{sec:Sparsity level for clustered blocks of a dictionary} and \ref{sec:EEG/MEG source reconstruction problem and USLE}, respectively.
Then, some experiments on synthetic dictionary and real lead-field in Section \ref{sec:hierarchical_cluster_estim} demonstrate the advantages of the proposed block structure identification method.