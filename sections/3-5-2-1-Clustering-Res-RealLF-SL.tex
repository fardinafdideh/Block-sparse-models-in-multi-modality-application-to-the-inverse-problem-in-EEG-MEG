As explained in Section \ref{sec:Sparsity level for clustered blocks of a dictionary}, generally, in transforming the $\myBSLqpTxt$, which is determined by Block-ERC, into the conventional sparsity level, different types of sparsity levels can be appeared, e.g., minimum sparsity level 
%in the most pessimistic case 
$\mySLTxt_{min}$, and maximum sparsity level 
%in the most pessimistic case 
$\mySLTxt_{max}$.
%, minimum sparsity level in the most optimistic case $\mySLTxt^{opt}_{min}$, and maximum sparsity level in the most optimistic case $\mySLTxt^{opt}_{max}$.
On the other hand, as described earlier in the Section \ref{sec:hierarchical_cluster_estim}, number of branches in a clustering level with maximum inter-node distance in a clustering tree resulted from a hierarchical clustering algorithm can be used as an estimation of the number of clusters of a dictionary.

Assume the sparsity level computed in the clustering level corresponding to the maximum inter-node distance in the clustering tree is shown by $\mySLTxt_2$, whereas the sparsity level computed for the lead-field without clustering is called $\mySLTxt_1$.
In order to investigate the effect of clustering the coherent blocks of the lead-field on sparsity levels, we compute the relative change quantity, i.e., $(\mySL_2(\myPhi) \sm \mySL_1(\myPhi)) {/} \mySL_1(\myPhi)$.

In figure \ref{fig:EMEG-LF-clustering-SL}, the relative change of sparsity levels $\mySLTxt_{min}$, and $\mySLTxt_{max}$ is shown.
%, $\mySLTxt^{opt}_{min}$, and $\mySLTxt^{opt}_{max}$ is shown.
In addition, the experiment is repeated for two modalities of EEG and MEG, and three head models with spherical, realistic inflated and realistic highly-folded cortical sheets.

As it can be seen in figure \ref{fig:EMEG-LF-clustering-SL}, by clustering coherent blocks of the lead-field (whether EEG/MEG or spherical/realistic cortical sheet) using the proposed Block-MCC$_{2,2}$, all sparsity levels significantly increase, especially $\mySLTxt_{max}$.
% and $\mySLTxt^{opt}_{max}$.
Hence, clustering coherent blocks of lead-field using the proposed Block-MCC$_{q,p}$ coherence measure leads to improved Block-ERC.

%In addition to the segmentation effect mentioned in previous part, the relative improvement of $\mySLTxt^{pes}_{min}$, $\mySLTxt^{pes}_{max}$, $\mySLTxt^{opt}_{min}$, and $\mySLTxt^{opt}_{max}$ in the estimated clustering level in comparison to the last level of clustering where the number of clusters is equal to the number of blocks, is shown in figure \ref{fig:EMEG-LF-clustering-SL} as bar charts, which indicates a significant increase on values, i.e., clustering coherent blocks of leadfield results in weakened recovery conditions.
\begin{figure}[!b]
\centering
\includegraphics[width=0.9\textwidth,keepaspectratio]{images/EMEG-LF-clustering-SL.png} % width=0.5\textwidth  scale=0.49
\centering
\caption{Relative increase of sparsity levels $\mySLTxt_{min}$, $\mySLTxt_{max}$, 
%$\mySLTxt^{opt}_{min}$, and $\mySLTxt^{opt}_{max}$, 
when clustering the coherent blocks of lead-field, repeated for two modalities of EEG and MEG, and three head models with spherical, realistic inflated and realistic highly-folded cortical sheets.}
\label{fig:EMEG-LF-clustering-SL}
\end{figure}
\FloatBarrier