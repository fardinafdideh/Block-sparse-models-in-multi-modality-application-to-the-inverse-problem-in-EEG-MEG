In this part, the set of two complementary random dictionaries $\myPhi_i \ssin \mathbb{R}^{10 \stimes 80}$, $i \seq \{\boldsymbol{1} , \boldsymbol{2} \}$, with equally-sized blocks of length $d_j \seq d \seq 2$, $\forall j$, is generated from independent and identically distributed (i.i.d.) random variables and all columns of the dictionaries are normalized to have unit $\ell_2$ norm.

%In this chapter, the experiments of multi-modality have done on synthetic dictionaries and real EEG/MEG leadfields.
%In the experiment with synthetic data, 
In order to generate two complementary random dictionaries, 
%and to ultimately combine them, 
firstly, as described in Section \ref{sec:Artificial simulated dictionary}, one of the dictionaries is generated by setting the related parameters $\varepsilon_{inter}$ and $\varepsilon_{intra}$.
Then, utilising the same strategy and by setting the parameter $\varepsilon_{inter-dictionary}$, which controls the overlap between the two dictionaries, the second dictionary is produced.
In figure \ref{fig:Multomodal-Phi}(a) and (b), the role of different types of $\varepsilon$ in generation of random dictionaries $\myPhi_{\boldsymbol{1}}$ and $\myPhi_{\boldsymbol{2}}$ is shown.
Then, as shown in figure \ref{fig:Multomodal-Phi}(c) by drawing half of the rows of each of the dictionaries and finally by making a sufficient shift on one of the set of rows belonging to one dictionary, here $\myPhi_{\boldsymbol{2}}$, the combined dictionary $\myPhi_{\boldsymbol{12}}$ is obtained.
All the results are shown in average and standard deviation over 100 repetitions of generation of complementary random dictionaries pairs.
\begin{figure}[!b]
\centering
\includegraphics[width=1\textwidth,keepaspectratio]{images/Multomodal-Phi.png} % width=0.5\textwidth  scale=0.49
\centering
\caption{The role of (a) $\varepsilon_{inter}$ and $\varepsilon_{intra}$ in generation of random dictionary $\myPhi_{\boldsymbol{1}}$, and (b) $\varepsilon_{inter-dictionary}$ in generation of $\myPhi_{\boldsymbol{2}}$ to produce the multi-modal dictionary $\myPhi_{\boldsymbol{12}}$.}
\label{fig:Multomodal-Phi}
\end{figure}
\FloatBarrier
%\newpage
%------------------------------------------------------
\paragraph{Block-ERC in clustered representation for a combined dictionary}
%\subsubsection{Effect of clustering the coherent blocks of a combined dictionary on Block-ERC based on Block-MCC$_{q,p}$} \footnote{\emph{Block-sparse Exact Recovery Condition(s)}} \footnote{\emph{Block-sparse Exact Recovery Condition(s)}}
In Section \ref{sec:clusteringOFcoherent_BERC-BMIC}, we demonstrated that in a random dictionary with certain values of $\varepsilon_{inter}$ and $\varepsilon_{intra}$, the $\myBSLTxt_{2,2}$\footnote{\emph{$(2,2)$-Block-Sparsity Level}} computed in each clustering level of the hierarchical clustering algorithm have a peak value at a clustering level equal to the number of clusters in the generated dictionary.

Therefore, under certain values of $\varepsilon_{inter}$ and $\varepsilon_{intra}$, the argument of the peak in $\myBSL_{2,2}(\myPhi)[\%]$ computed at each level of the hierarchical clustering algorithm is equal to the number of clusters in the dictionary, i.e., $N$.

In this part, in an experiment similar to Section \ref{sec:clusteringOFcoherent_BERC-BMIC}, the block sparsity-level in the pessimistic case, i.e., $\myBSL_{2,2}(\myPhi)[\%]$, is computed in each clustering level of the 
hierarchical clustering analysis, but here the dictionary 
%instead of being a simple random matrix 
is a combination of two complementary dictionaries instead of being a simple mono-modal random matrix.
In Section \ref{sec:Multimodal Structure and the model}, the procedure of combining two complementary dictionaries is described.
Also, the complementarity of two dictionaries is defined based on their overlap, determined by $\varepsilon_{inter-dictionary}$.

In generating two complementary dictionaries, one of them is produced as explained in Section \ref{sec:Artificial simulated dictionary}. 
Then the other dictionary is generated by multiplying the first dictionary to a random matrix $\exp((\boldsymbol{C} \sm \boldsymbol{C}^T) \sqrt{2} \varepsilon_{inter-dictionary} / \Vert \boldsymbol{C} \sm \boldsymbol{C}^T \Vert_F)$, where, $\boldsymbol{C}$ is a square random matrix of dimension $m$.

As it can be seen in figure \ref{fig:SL_Hierarchical_Multimodal}, by applying hierarchical clustering on the blocks of the combined dictionary for $\varepsilon_{inter-dictionary} \seq 1$, $\varepsilon_{inter} \sgeq 3.5$ and $\varepsilon_{intra} \sleq 0.1$, a peak in $\myBSL_{2,2}(\myPhi)[\%]$ appears in an argument equal to the sum of the clusters in each of the two original dictionaries.

For instance, when there are four clusters in each of the two complementary dictionaries i.e., $N \seq 4$ (blue squares), which construct the combined one, the maximum $\myBSL_{2,2}(\myPhi)[\%]$ for combined dictionary is at the eighth clustering level.

Similarly the consistent results are obtained for five clusters in each of the two complementary dictionaries, i.e., $N \seq 5$ (red circles).
In other words, there is a peak in $\myBSL_{2,2}(\myPhi)[\%]$ at the tenth clustering level, when the multi-modal dictionary is composed of two mono-modal dictionaries, each has five clusters.
For the clustering level lower than the optimal level, the space is under-sampled and there would be a block spanning the whole space, whereas for the clustering level higher than the optimal level, the space is over-sampled and the over-partitioning leads to high coherence measure.

On the other hand, when the average blocks of each cluster of blocks are not different enough, i.e., $\varepsilon_{inter} \sless 3.5$, but still the blocks in each cluster of blocks are similar enough, i.e., $\varepsilon_{intra} \sleq 0.1$, we can observe that the number of clusters in the combined dictionary or the argument of the peak in $\myBSL_{2,2}(\myPhi)[\%]$ is less than the sum of clusters in each of initial dictionaries but greater than the number of clusters in each of initial dictionaries.
\newpage

Therefore, the number of clusters in the combined dictionary depending on the overlap between the blocks, which is determined by $\varepsilon$, is greater than the number of clusters in one of the initial dictionaries and at most is equal to the sum of the number of clusters in the initial dictionaries.

In addition, the amplitude of peak of $\myBSL_{2,2}(\myPhi)[\%]$ increases when the number of clusters decreases.
For instance, in figure \ref{fig:SL_Hierarchical_Multimodal}, the amplitude of average peak of $\myBSL_{2,2}(\myPhi)[\%]$ for multi-modal dictionaries composed of two four-clusters dictionaries (blue square) is higher than the corresponding average amplitude for five clusters in each of mono-modal dictionaries (red circle).

Therefore, by reducing the number of clusters in the dictionary, $\myBSL_{2,2}(\myPhi)[\%]$ increases, hence, more weakened Block-ERC\footnote{\emph{Block-sparse Exact Recovery Condition(s)}} are obtained. 
%As it can be seen in figure \ref{fig:SL_Hierarchical_Multimodal}, the average peak of the multi-modal dictionaries with eight clusters (blue squares) is higher than the average peak of the multi-modal dictionaries with ten clusters (red circles)
\begin{figure}[!b]
\centering
\includegraphics[width=1\textwidth,keepaspectratio]{images/SL_Hierarchical_Multimodal.png} % width=0.5\textwidth  scale=0.49
\centering
\caption{$\myBSL_{2,2}(\myPhi)[\%]$ for each level of clustering is computed for the complete method, Block-MCC$_{2,2}$, $d \seq 2$, $N \seq \{4 , 5 \}$, $\varepsilon_{inter-dictionary} \seq 1$, and different values of $\varepsilon_{inter}$ and $\varepsilon_{intra}$ for simulating dictionary $\myPhi \ssin \mathbb{R}^{10 \stimes 80}$.}
\label{fig:SL_Hierarchical_Multimodal}
\end{figure}
\FloatBarrier