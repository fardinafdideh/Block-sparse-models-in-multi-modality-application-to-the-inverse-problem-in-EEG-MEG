As mentioned before, in the conventional element-wise sparse recovery problem, the goal is to extract a representation vector among the set of eligible solutions which is the sparsest one, i.e., a representation with the fewest non-zero elements. 
But in the framework of block-sparsity, the fewest \emph{active blocks} are of interest, and not fewest non-zero elements. 
Any active block $k$ has at least one non-zero element in $\mybeta[k]$ which results in its non-zero $\ell_p$ norm (for any $p$), which can be used for determining the active blocks. 
Usually, $p$ is assigned to two, because the $\ell_2$ norm is a rotational invariant measure.
Here we use the $\ell_p$ norm in more general cases of $0 \sleq p$ and $1 \sleq p$. 

As we said, a representation $\mybeta$ is called block k-sparse if it has at most $k$ active blocks, i.e., $\Vert \mybeta \Vert_{p,0} \sleq k$. 
The $\ell_{p,0}$ pseudo-mixed-norm of $\mybeta$, i.e., $\sum_{k}^{} I (\Vert \mybeta [k] \Vert_p)$, 
%(see table \ref{table:MixedNorm} for its complete definition), 
which can be used for counting the number of active blocks, measures the activity of each block in $\ell_{p}$ norm sense and the sparsity of the active blocks in $\ell_{0}$ pseudo-norm sense.
Therefore, block's $\ell_p$ norm for any $p$ could be a suitable criterion for determining the activity of blocks and then using another sparsity inducing norm applied on the resulted vector containing the $\ell_p$ norm of each block could implement the concept of block-sparsity.

The previously-introduced (equation (\ref{eq:Opt-Lp}), page \pageref{eq:Opt-Lp}) general constrained $\ell_p$ norm optimisation problem $P_{p,q,\varepsilon} (\boldsymbol{a},\boldsymbol{b}) : \min_{\boldsymbol{a}} \Vert \boldsymbol{a} \Vert_p \quad s.t. \quad \Vert \boldsymbol{b} \Vert_q \sleq \varepsilon$, 
%defined in (\ref{eq:Opt-Lp}) 
is defined to recover sparse solution, i.e., solution with the fewest non-zero elements, \myhl{with the widely used problem $P_{p} : \min_{\mybeta} \Vert \mybeta \Vert_p \quad s.t. \quad \boldsymbol{y} \seq \myPhi \mybeta$.}
Then, the general constrained $\ell_{p_1,p_2}$ (pseudo-)mixed-norm optimisation problem $P_{(p_1,p_2),q,\varepsilon} (\boldsymbol{a},\boldsymbol{b}) : \min_{\boldsymbol{a}} \Vert \boldsymbol{a} \Vert_{p_1,p_2} \quad s.t. \quad \Vert \boldsymbol{b} \Vert_q \sleq \varepsilon$, introduced (equation (\ref{eq:Opt-Lp1p2}), page \pageref{eq:Opt-Lp1p2}) to recover equally-sized block-sparse solution, i.e., solution with the fewest active blocks, where, $\forall k$, $d_k \seq d$, \myhl{with the widely used problem $P_{p_1,p_2} : \min_{\mybeta} \Vert \mybeta \Vert_{p_1,p_2} \quad s.t. \quad \boldsymbol{y} \seq \myPhi \mybeta$.}
In this chapter, we introduce the following \emph{constrained $\ell_{p_1,p_2}^{\boldsymbol{w}}$ weighted (pseudo-)mixed-norm optimisation problem} $P_{(\boldsymbol{w};p_1,p_2),q,\varepsilon}(\boldsymbol{a},\boldsymbol{b})$ in a more general case to recover differently-sized block-sparse solution, and cover all previously introduced optimisation problems:
%can be adapted for the block-sparse recovery and redefined in the following more general cases to meet the needs in block-sparse recovery:
%\begin{equation*}
%\label{eq:Opt-Lp1p2}
%P_{\myparanthese{p_1,p_2},q,\varepsilon} \myparanthese{\boldsymbol{a},\boldsymbol{b}}: \min_{\boldsymbol{a}} \mynorm{\boldsymbol{a}}_{p_1,p_2} \quad s.t. \quad \mynorm{\boldsymbol{b}}_q \leq \varepsilon,
%\end{equation*}
%and 
\begin{Mydefinition}[Constrained $\ell_{p_1,p_2}^{\boldsymbol{w}}$ weighted (pseudo-)mixed-norm optimisation problem]
\leftbar
\begin{equation*}
\label{eq:Opt-Lp1barp2}
P_{\myparanthese{\boldsymbol{w};p_1,p_2},q,\varepsilon} \myparanthese{\boldsymbol{a},\boldsymbol{b}}: \min_{\boldsymbol{a}} \mynorm{\boldsymbol{a}}_{\boldsymbol{w};p_1,p_2} \quad s.t. \quad \mynorm{\boldsymbol{b}}_q \leq \varepsilon,
\end{equation*}
\endleftbar
\end{Mydefinition}
%where 
%$\ell_{p_1,p_2}$ mixed norm of $\mybeta$ is defined in table \ref{table:MixedNorm}, and 
$\ell_{p_1,p_2}^{\boldsymbol{w}}$ \emph{weighted (pseudo-)mixed-norm} of a vector $\boldsymbol{a}$, i.e., $\Vert \boldsymbol{a} \Vert_{\boldsymbol{w};p_1,p_2}$, is defined by following definition:
\begin{Mydefinition}[\myhl{Weighted (pseudo-)mixed-norm}]
\label{def:Weighted mixed norm} 
\leftbar
The $\ell_{p_1,p_2}^{\boldsymbol{w}}$ weighted (pseudo-)mixed-norm of a block-structured vector $\boldsymbol{a} \seq [\boldsymbol{a}^T[1], \cdots, \boldsymbol{a}^T[k], \cdots, \boldsymbol{a}^T[K]]^T$, where, $\boldsymbol{a} [k] \seq [a_1[k], \cdots, a_{{d_k}} [k]]^T$, is defined as:
\begin{equation*}
\mynorm{\boldsymbol{a}}_{\boldsymbol{w};p_1,p_2} \myeq 
\begin{cases}
    \mycolor{\displaystyle\sum_{k}^{} I \myparanthese{w_k \mynorm{\boldsymbol{a} \mybracket{k}}_{p_1}},} & \qquad \mycolor{\text{for \ }p_2= 0}\\
    \myparanthese{\displaystyle\sum_{k}^{} w_k^{p_2} \mynorm{\boldsymbol{a} \mybracket{k}}_{p_1}^ {p_2}} ^ {\frac{1}{p_2}},  & \qquad  \text{for \ } 0 < p_2 < +\infty\\
    \displaystyle\max_{k}{\mybrace{w_k \mynorm{\boldsymbol{a} \mybracket{k}}_{p_1}}}, & \qquad \text{for \ } p_2= \infty,
  \end{cases}
\end{equation*}
where, $I(a)$ is the indicator function, i.e., $I(a) \myeq
  \begin{cases}
    1,  & \quad \text{if } \myabs{a} \sg 0\\
    0,  & \quad \text{if } a \seq 0\\
  \end{cases}$, and $\boldsymbol{w} \seq [w_1, \cdots, w_K ]$ is the weight vector. 
%and the weighted norm $\Vert \boldsymbol{a}[k] \Vert_{\overbar{p}}$ is equal to $\Vert \boldsymbol{a}[k] \Vert_p /d_k^{1/p}$.
\endleftbar
\end{Mydefinition}

The $\ell_{p_1,p_2}^{\boldsymbol{w}}$ for either $0 \sleq p_1 \sless 1$ or $0 \sleq p_2 \sless 1$ is called weighted \emph{pseudo}-mixed-norm, consequently, the corresponding problem $P_{(\boldsymbol{w};p_1,p_2),q,\varepsilon}(\boldsymbol{a},\boldsymbol{b})$ is called constrained $\ell_{p_1,p_2}^{\boldsymbol{w}}$ weighted \emph{pseudo}-mixed-norm optimisation problem.

$\ell_{p_1,p_2}^{\boldsymbol{w}}$ weighted (pseudo-)mixed-norm when all the weights are $1$, i.e., $\boldsymbol{w} \seq \boldsymbol{1}_{1 \stimes K}$, is equal to the $\ell_{p_1,p_2}$ (pseudo-)mixed-norm, i.e., $\ell_{p_1,p_2}^{\boldsymbol{1}} {\equiv} \ell_{p_1,p_2}^{}$.
\myhl{In addition, for $p_1 \sg 0$ and $p_2 \seq 0$, the $\ell_{p_1,0}^{\boldsymbol{w}}$ weighted pseudo-mixed-norm is independent of weight vector $\boldsymbol{w}$, so it equals to ordinary pseudo-mixed-norm, i.e., $\ell_{p,0}^{\boldsymbol{w}} {\equiv} \ell_{p,0}^{}$ for $p \sg 0$.}

Although 
%the \emph{constrained $\ell_{p_1,p_2}$ mixed norm minimisation problem} $P_{(p_1,p_2),q,\varepsilon}(\boldsymbol{a},\boldsymbol{b})$ and \emph{constrained $\ell_{p_1,p_2}^{\boldsymbol{w}}$ weighted mixed norm minimisation problem} 
$P_{(\boldsymbol{w};p_1,p_2),q,\varepsilon}(\boldsymbol{a},\boldsymbol{b})$ with carefully selected values for $p_1$ and $p_2$, is defined generally and can cover lots of scenarios, but the proposed conditions in this study concentrate on the following special cases:
\begin{itemize}
\item $\boldsymbol{a}$: $\boldsymbol{a}$ is chosen to be the representation vector, i.e., $\boldsymbol{a} \seq \mybeta$.
\item $\boldsymbol{b}$: $\boldsymbol{b}$ is selected as $\boldsymbol{r}$, i.e., $\boldsymbol{b} \seq \boldsymbol{r} \seq \boldsymbol{y} \sm \myPhi \mybeta$.
\item $\varepsilon$ (and $q$): The representation vector $\mybeta$ is minimised in a block-sparse exact signal recovery problem, hence, $\varepsilon$ should be zero, i.e., 
%$P_{(p_1,p_2),q,0} (\mybeta,\boldsymbol{r})$ and 
$P_{(\boldsymbol{w};p_1,p_2),q,0} (\mybeta,\boldsymbol{r})$.
For the sake of simplicity we refer to $P_{(\boldsymbol{w};p_1,p_2),q,0} (\mybeta,\boldsymbol{r})$
%them 
as %$P_{p_1,p_2}$ and 
$P_{\boldsymbol{w};p_1,p_2}$, because regardless of choice of $q$, the $\Vert \boldsymbol{r} \Vert _q \seq 0$ directly indicates to the noiseless model, so the only parameters to be determined are $\boldsymbol{w}$, $p_1$ and $p_2$.
Therefore, the utilised optimisation problem is $P_{\boldsymbol{w};p_1,p_2} : \min_{\mybeta} \Vert \mybeta \Vert_{\boldsymbol{w};p_1,p_2} \quad s.t. \quad \boldsymbol{y} \seq \myPhi \mybeta$.
%, respectively.
\item $\boldsymbol{w}$: Throughout this thesis, only two cases for $\boldsymbol{w}$ is considered. 
In a block-structured vector $\boldsymbol{a} \seq [\boldsymbol{a}^T[1], \cdots,\boldsymbol{a}^T[K]]^T$ with length vector $\boldsymbol{d} \seq [d_1, \cdots, d_K ]$, $w_k$ in \myhl{$w_k \Vert \boldsymbol{a}[k] \Vert_{p_1}$} is either \myhl{$d_k^{-1/{p_1}}$} or $1$.
Therefore, although in general the weight vector $\boldsymbol{w}$ can consist of any arbitrary values, in this work it is considered as \myhl{$\boldsymbol{d}^{-1/{p_1}}$} and the corresponding problem is represented as $P_{\boldsymbol{w};p_1,p_2}$, except when mentioned explicitly that it is a vector of ones, i.e., $\boldsymbol{1}_{1 \stimes K}$, and it can be represented as $P_{\boldsymbol{1};p_1,p_2}$.
But $P_{\boldsymbol{1};p_1,p_2}$ is equal to $P_{p_1,p_2}$, so we refer this case to $P_{p_1,p_2}$.
Therefore, we have two cases of $P_{p_1,p_2}$ and $P_{\boldsymbol{w};p_1,p_2}$, where the weight vector $\boldsymbol{w}$ is:
\begin{equation*}
\mycolor{\forall k \in \mybrace{1, \cdots, K}, \qquad w_k = d_k^{-\frac{1}{p_1}}.}
\end{equation*}
\item $p_1$ and $p_2$: The main proposed Block-ERC include the cases $0 \seq p_2 \sleq p_1$, and the more general one of $0 \sleq p_2 \sleq 1 \sleq p_1$.
\end{itemize}

%\begin{remark}
It is obvious that, if the size of all the blocks is chosen to be 1, i.e., $\forall k, d_k \seq 1$, then the block-sparse exact signal recovery reduces to the conventional exact signal recovery, i.e., $P_{\boldsymbol{w};p_1,p_2} {\equiv} P_{p_1,p_2} {\equiv} P_{p_2}$.
In addition, the problem $P_{\boldsymbol{w};p_1,p_2}$ for $0 \seq p_2 \sless p_1$ is independent of the weight vector $\boldsymbol{w}$ and equals to ordinary pseudo-mixed-norm problem, i.e., $P_{\boldsymbol{w};p,0} {\equiv} P_{p,0}$ for $p \sg 0$.
%\end{remark}

As it can be seen in Section \ref{sec:Block-sparse_representation_theory}, most of the previous works are based on Euclidean norm as a measure of block activity, i.e., $P_{2,p}$. However, it has been shown that by using norms other than Euclidean, the performance of the selection of group of variables increases \cite{Zhao2009}.
So, we study the impact of other norms.

 %%%
In this chapter, we first introduce Block-ERC based on $P_{p,0}$ in two different cases of $p \sgeq 0$ and $p \sgeq 1$. 
Then, we generalise the results to $P_{p_1,p_2}$ and $P_{\boldsymbol{w};p_1,p_2}$, where, the activity of blocks is measured by $\ell_{p_1}$ norm, $p_1 \sgeq 1$, and the sparsity of the blocks by $\ell_{p_2}$ (pseudo-)norm, $0 \sleq p_2 \sleq 1$.
%, see table \ref{table:MixedNorm} for $\ell_{p_1,p_2}$ mixed norm definition of the vector $\mybeta$ for different values of $p_1$ and $p_2$.
\begin{remark}[$P_{\boldsymbol{w};p_1,p_2}$ v.s. $P_{p_1,p_2}$]
\label{Rmrk:TwoP} 
Notice that in general:
\begin{itemize}
\item For equally-sized blocks, i.e., $\forall k, d_k \seq d$, we propose Block-ERC based on $P_{p_1,p_2}$, where, $p_1$ can have zero value, while for a more general case of differently-sized blocks we propose Block-ERC based on $P_{\boldsymbol{w};p_1,p_2}$ for $p_1 {\neq} 0$, to have finite values for the elements of the weight vector $\boldsymbol{w}$, because $w_k \seq d_k^{-1/{p_1}}$.
\item As mentioned before, if $0 \seq p_2 \sless p_1$ then the $\ell_{p_1,p_2}^{\boldsymbol{w}}$ weighted pseudo-mixed-norm optimisation problem is equivalent to the $\ell_{p_1,p_2}$ pseudo-mixed-norm optimisation problem, i.e., $\forall p \sg 0$, $P_{\boldsymbol{w};p,0} {\equiv} P_{p,0}$.
\end{itemize} 
\end{remark}

The selection of the values of the (pseudo-)mixed-norm coefficients $p_1$ and $p_2$ in the corresponding optimisation problems $P_{p_1,p_2}$ and $P_{\boldsymbol{w};p_1,p_2}$ is very important. 
Since the optimisation problems $P_{p_1,p_2}$ and $P_{\boldsymbol{w};p_1,p_2}$ are presented in a general case, by selecting a certain values of the (pseudo-)mixed-norm coefficients $p_1$ and $p_2$, one can switch the problem between the conventional sparsity and block-sparsity problem.
Map of the values of the (pseudo-)mixed-norm coefficients $p_1$ and $p_2$, which leads to sparsity and block-sparsity problems is shown in figure \ref{fig:MixedNorm-Map}.

Figure \ref{fig:MixedNorm-Map} represents schematically our regions of interest in orange colour for the (pseudo-)mixed-norm coefficients $p_1$ and $p_2$ in a block sparsity problems $P_{p_1,p_2}$ or $P_{\boldsymbol{w};p_1,p_2}$, compared to the traditional sparsity problem, shown in blue colour region. 
As it can be seen in figure \ref{fig:MixedNorm-Map}, considering additional constraint of proposed block sparsity (orange-coloured regions) on the representation vector, gives rise to a wider region of the (pseudo-)mixed-norm coefficients of the corresponding optimisation problem compared to the existing block sparsity (green-coloured regions), and conventional sparsity (blue-coloured regions).
In figure \ref{fig:MixedNorm-Map}, the most frequently studied cases are represented by solid circles. 
For instance, $P_{0,1}$ (${\equiv} P_0$) and $P_{1,1}$ (${\equiv} P_1$), which are represented by blue circles (left and right one, respectively) are the widely used cases in sparse exact recovery problems, whereas $P_{2,0}$ and $P_{2,1}$, which are represented by green circles (down and top one, respectively) are the widely used cases in block-sparse exact recovery problems.
\newpage
\begin{remark}[Sparsity regions in figure \ref{fig:MixedNorm-Map}]
For $0 \sless p_1 \seq p_2 \seq p \sleq 1$, the block-sparse exact signal recovery optimisation problems $P_{p,p}$ and $P_{\boldsymbol{w};p,p}$ reduce to the traditional exact signal recovery problem, i.e., $P_{p,p} {\equiv} P_p$ and $P_{\boldsymbol{w};p,p} {\equiv} d^{-1/p} \, P_p$ ($\forall k, d_k \seq d$).
In other words, in the common region of sparsity and block-sparsity, i.e., $p_1 \seq p_2 \seq 1$ (the right blue circle in figure \ref{fig:MixedNorm-Map}), and under certain conditions, the conventional exact signal recovery problem can recover the block-sparse representation.
In addition, for $p_1 \seq 0$ and $p_2 \seq 1$ (the left blue circle in figure \ref{fig:MixedNorm-Map}), the $P_{0,1}$ problem reduces to the classic exact signal recovery problem, i.e., $P_{0,1} {\equiv} P_0$, but it is still non-convex.
\end{remark}
\begin{remark}[Block-sparsity regions in figure \ref{fig:MixedNorm-Map}]
As shown in figure \ref{fig:MixedNorm-Map} the proposed block-sparsity region (orange region) covers the existing most frequently studied cases in block sparsity problem (green circles) and the widely used tractable case in the traditional sparsity problem (the right blue circle) in literature.
\end{remark}
\begin{figure}[!t]
\centering
\includegraphics[width=0.9\textwidth,keepaspectratio]{images/Weighted-MixedNorm-Map.png}
\centering
\caption{Sparsity and block sparsity regions in terms of $\mynorm{\mybeta}_{p_1,p_2}$, and $\mynorm{\mybeta}_{\boldsymbol{w};p_1,p_2}$, assuming $\forall k, d_k \seq d$. The most commonly studied cases are indicated with solid circles.}
\label{fig:MixedNorm-Map}
\end{figure}
\FloatBarrier
%mixed norm coefficients of the representation vector, weighted mixed norm coefficients of the representation vector,